\chapter{Modular Forms and Higher Genus Phenomena}

\section{Eisenstein Series and Genus One}

At genus one, the universal chiral algebra contribution is:
$$\omega_{1,0}^{\text{univ}}(\tau) = -\frac{c}{24} \frac{E_2(\tau)}{2\pi i} dz$$
where $E_2(\tau) = 1 - 24\sum_{n=1}^{\infty} \sigma_1(n)q^n$ is the weight-2 Eisenstein series.

This generates all genus-one corrections through:
$$\langle T(z) \rangle_{\tau} = -\frac{c}{12} \wp(z|\tau) + \frac{c}{24} E_2(\tau)$$
where $\wp$ is the Weierstrass function.

\section{Siegel Modular Forms at Higher Genus}

At genus $g \geq 2$, correlation functions involve Siegel modular forms on $\mathcal{H}_g$:
$$F^{(g)}(\Omega) = \sum_{n \in \mathbb{Z}^g} a_n \exp(2\pi i \, {}^t n \Omega n)$$

Key examples:
\begin{itemize}
\item Genus 2: Igusa invariants $\psi_4, \psi_6, \chi_{10}, \chi_{12}$
\item Genus 3: Schottky form vanishes on Jacobian locus
\item General $g$: Theta constants $\theta[\delta](\Omega)$ for characteristics $\delta$
\end{itemize}

\section{Bosonization at All Genera}

The bosonization formula extends to:
$$Z_g^{\text{fermion}} = \sum_{\delta} \epsilon(\delta) \, Z_g^{\text{boson}}[\delta]$$
where the sum is over spin structures $\delta$ and $\epsilon(\delta) = \pm 1$ is the Arf invariant.

At genus $g$:
\begin{itemize}
\item Number of spin structures: $2^{2g}$
\item Even (periodic) spin structures: $2^{g-1}(2^g + 1)$
\item Odd (antiperiodic) spin structures: $2^{g-1}(2^g - 1)$
\end{itemize}

\section{Mumford Forms and Belavin-Knizhnik Measure}

The Polyakov measure at genus $g$:
$$d\mu_g^{\text{Pol}} = \prod_{k=1}^{3g-3} d^2\tau_k \left[\det(\text{Im}\Omega)\right]^{-13}$$

This arises from:
\begin{itemize}
\item Ghost determinant: $[\det(\text{Im}\Omega)]^{-26}$
\item Matter contribution: $[\det(\text{Im}\Omega)]^{c/2}$
\item Critical dimension: $c = 26$ cancels anomaly
\end{itemize}

\section{The Genus-Graded Chiral Operad}

The chiral operad extends to:
$$\mathcal{P} = \bigoplus_{g \geq 0} \mathcal{P}^{(g)}$$
with composition:
$$\circ: \mathcal{P}^{(g_1)}(m) \otimes \mathcal{P}^{(g_2)}(n) \to \mathcal{P}^{(g_1+g_2)}(m+n-1)$$

This encodes:
\begin{itemize}
\item Gluing of surfaces along boundaries
\item Factorization at nodes
\item Modular operad structure
\end{itemize}

\section{Swiss-Cheese Structure}

At higher genus, we get a Swiss-cheese operad:
\begin{itemize}
\item Closed sector: Full genus-$g$ surfaces
\item Open sector: Surfaces with boundaries
\item Mixed compositions: Open-closed duality
\end{itemize}

This relates to:
\begin{itemize}
\item D-branes in string theory
\item Boundary CFT
\item Kapustin-Witten equations
\end{itemize}

\section{String Amplitudes}

The genus-$g$ string amplitude:
$$A_g = \int_{\mathcal{M}_g} \langle \prod_i V_i \rangle_g \, d\mu_g^{\text{Pol}}$$

For critical strings ($c=26$ bosonic, $c=15$ superstring):
\begin{itemize}
\item Tree level: Classical scattering
\item One loop: Quantum corrections
\item Higher loops: Quantum gravity
\end{itemize}

\section{Mirror Symmetry}

The genus-$g$ Gromov-Witten invariants:
$$F_g^{\text{GW}} = \sum_{d} N_{g,d} \, Q^d$$
relate to B-model periods:
$$F_g^{\text{B-model}} = \int_{\Gamma_g} \Omega_g$$

The bar-cobar duality provides the mathematical framework:
\begin{itemize}
\item A-model: Holomorphic maps (bar complex)
\item B-model: Period integrals (cobar complex)
\item Mirror map: Bar-cobar duality
\end{itemize}

\section{AGT Correspondence}

The Alday-Gaiotto-Tachikawa correspondence relates:
\begin{itemize}
\item 4D $\mathcal{N}=2$ gauge theory on $\Sigma_g \times S^2$
\item 2D Liouville/Toda CFT on $\Sigma_g$
\end{itemize}

Through bar-cobar:
$$Z_{\text{gauge}}^{(g)} = \langle \text{Bar}^{(g)}(\mathcal{W}) \rangle$$
where $\mathcal{W}$ is the relevant W-algebra.
