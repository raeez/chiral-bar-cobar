\appendix
\chapter{Koszul Duality Across Genera}

\section{Genus-Graded Koszul Duality}

\begin{theorem}[Extended Koszul Duality]
If $(\mathcal{A}, \mathcal{A}^!)$ form a genus-0 Koszul dual pair, then:
$$\left(\bigoplus_{g \geq 0} \mathcal{A}^{(g)}, \bigoplus_{g \geq 0} (\mathcal{A}^!)^{(g)}\right)$$
form a multi-genus Koszul dual pair with pairing:
$$\langle -, - \rangle: \mathcal{A}^{(g)} \otimes (\mathcal{A}^!)^{(g)} \to \mathbb{C}[\![\hbar]\!]$$
where $\hbar$ tracks the genus.
\end{theorem}

\section{Definition and Basic Properties}

\begin{definition}[Genus-Graded Koszul Algebra]
A genus-graded associative algebra $\mathcal{A} = \bigoplus_{g \geq 0} \mathcal{A}^{(g)}$ is \emph{Koszul} if:
$$\text{Ext}_{\mathcal{A}^{(g)}}^{i,j}(\mathbb{k}, \mathbb{k}) = 0 \text{ for } i \neq j$$
where the bigrading is by homological degree and internal degree, and the Koszul property holds at each genus.
\end{definition}

\begin{theorem}[Genus-Graded Koszul Duality Theorem]
If $\mathcal{A}$ is genus-graded Koszul, then:
$$\mathcal{A}^! := \bigoplus_{g \geq 0} \text{Ext}_{\mathcal{A}^{(g)}}^*(\mathbb{k}, \mathbb{k})$$
is also genus-graded Koszul, and $(\mathcal{A}^!)^! \cong \mathcal{A}$.
\end{theorem}

\subsection{Genus-Graded Chiral Koszul Duality}

For chiral algebras across all genera, we need a modified definition:

\begin{definition}[Genus-Graded Chiral Koszul Duality]
Genus-graded chiral algebras $\mathcal{A} = \bigoplus_{g \geq 0} \mathcal{A}^{(g)}$ and $\mathcal{B} = \bigoplus_{g \geq 0} \mathcal{B}^{(g)}$ are Koszul dual if:
$$\text{RHom}_{\mathcal{A}^{(g)} \otimes \mathcal{B}^{(g)}}(\mathbb{C}, \mathbb{C}) \simeq \mathbb{C}$$
in the derived category of chiral modules at each genus $g$, with modular covariance under $\text{Sp}(2g, \mathbb{Z})$ transformations.
\end{definition}

\subsection{Curved and Filtered Generalizations Across Genera}

\begin{definition}[Genus-Graded Curved Koszul Duality]
A genus-graded curved algebra $(\mathcal{A}^{(g)}, d^{(g)}, m_0^{(g)})$ with $(d^{(g)})^2 = m_0^{(g)} \cdot \text{id}$ has curved dual:
$$((\mathcal{A}^{(g)})^!, d^{!(g)}, m_0^{!(g)})$$
where $m_0^{!(g)} = -m_0^{(g)}$ under the genus-graded pairing, with modular corrections from period integrals.
\end{definition}

\subsection{Computational Tools Across Genera}

\begin{lemma}[Genus-Graded Koszul Complex Resolution]
For genus-graded Koszul $\mathcal{A}$, the minimal resolution of $\mathbb{k}$ at genus $g$ is:
$$\cdots \to \mathcal{A}^{(g)} \otimes (\mathcal{A}^!)_{(2)}^{(g)} \to \mathcal{A}^{(g)} \otimes (\mathcal{A}^!)_{(1)}^{(g)} \to \mathcal{A}^{(g)} \to \mathbb{k}$$
where $(\mathcal{A}^!)_{(n)}^{(g)}$ is the degree $n$ part of $\mathcal{A}^!$ at genus $g$, with modular corrections from period integrals.
\end{lemma}

\subsection{Physical Interpretation Across Genera}

In physics, genus-graded Koszul duality appears as:
\begin{itemize}
\item Electric-magnetic duality with genus corrections (abelian case)
\item Open-closed string duality with modular forms (topological strings)  
\item Holographic duality with genus expansion (AdS/CFT)
\item Mirror symmetry with period integrals (A-model/B-model)
\item String amplitudes with genus-graded corrections
\end{itemize}

\subsection{Genus-Graded Maurer-Cartan Elements and Twisting}

\begin{theorem}[Genus-Graded MC Elements Parametrize Deformations]
For a genus-graded chiral algebra $\mathcal{A} = \bigoplus_{g \geq 0} \mathcal{A}^{(g)}$ and its bar complex $\bar{B}(\mathcal{A})$:

\textbf{1. Genus-Graded Maurer-Cartan Equation:}
$$\alpha^{(g)} \in \barBgeom^{(g)}(\mathcal{A}), \quad d^{(g)}\alpha^{(g)} + \frac{1}{2}[\alpha^{(g)}, \alpha^{(g)}] = 0$$
with modular corrections from period integrals.

\textbf{2. Genus-Graded Twisting:}
Each MC element $\alpha^{(g)}$ yields a twisted differential:
$$d_{\alpha^{(g)}}^{(g)} = d^{(g)} + [\alpha^{(g)}, -]$$
with $(d_{\alpha^{(g)}}^{(g)})^2 = 0$ and modular covariance.

\textbf{3. Genus-Graded Deformation:}
MC elements correspond to first-order deformations of $\mathcal{A}^{(g)}$:
$$\mu_{\alpha^{(g)}}^{(g)}(a \otimes b) = \mu^{(g)}(a \otimes b) + \langle \alpha^{(g)}, a \otimes b \rangle$$
with genus corrections.

\textbf{4. Geometric Interpretation Across Genera:}
On configuration spaces, MC elements are:
\begin{itemize}
\item Closed 1-forms on $\overline{C}_2^{(g)}(\Sigma_g)$ with prescribed residues and period integrals
\item Flat connections on the punctured configuration space with modular structure
\item Solutions to the classical Yang-Baxter equation with genus corrections
\end{itemize}

\textbf{5. Genus-Graded Moduli Space:}
$$\mathcal{M}_{\text{MC}}^{(g)}(\mathcal{A}) = \{\text{MC elements at genus } g\}/\text{gauge equivalence}$$
parametrizes deformations of the chiral algebra structure at each genus.
\end{theorem}

\subsection{Koszul Duality at Higher Genus: The Tower Structure}
\label{app:koszul_higher_genus}

The genus 0 Koszul duality:
$$\Omega C_{\bullet}^{(0)}(\mathcal{A}) \simeq \mathcal{A}$$
extends to all genera by the modular operad structure.

\subsubsection{The Genus $g$ Statement}

For each $g \geq 0$, there is a duality:
$$\Omega^{(g)} C_{\bullet}^{(g)}(\mathcal{A}) \simeq \mathcal{A}^{(g)}$$
where:
\begin{itemize}
\item $\Omega^{(g)}$ is the genus $g$ cobar construction
\item $\mathcal{A}^{(g)}$ is the genus $g$ component of $\mathcal{A}$
\end{itemize}

\subsubsection{Compatibility}

The genus stratification satisfies:
$$\partial: C_{\bullet}^{(g)} \to C_{\bullet}^{(g-1)}$$
(boundary/degeneration maps) compatible with:
$$\iota: \mathcal{A}^{(g-1)} \to \mathcal{A}^{(g)}$$
(restriction maps).

This gives a \textbf{tower of Koszul dualities}:
\begin{center}
\begin{tikzcd}[column sep=small]
\cdots \arrow[r] & C_{\bullet}^{(2)}(\mathcal{A}) \arrow[r] \arrow[d, "\Omega^{(2)}"] & 
C_{\bullet}^{(1)}(\mathcal{A}) \arrow[r] \arrow[d, "\Omega^{(1)}"] & 
C_{\bullet}^{(0)}(\mathcal{A}) \arrow[d, "\Omega^{(0)}"] \\
\cdots \arrow[r] & \mathcal{A}^{(2)} \arrow[r] & 
\mathcal{A}^{(1)} \arrow[r] & 
\mathcal{A}^{(0)}
\end{tikzcd}
\end{center}

\subsubsection{The Limit}

Taking the inverse limit:
$$\mathcal{A}_{\text{complete}} = \varprojlim_g \mathcal{A}^{(g)}$$
gives the \textbf{completed chiral algebra}, encoding all genus contributions.

\subsubsection{Modular Invariance}

At each genus $g$, the duality respects the action of the mapping class group $\Gamma_g = \operatorname{MCG}(\Sigma_g)$:
$$\Omega^{(g)}(\sigma^* C_{\bullet}^{(g)}(\mathcal{A})) \simeq \sigma^* \mathcal{A}^{(g)}$$
for $\sigma \in \Gamma_g$.

This ensures that genus $g$ quantum corrections are modular-invariant.

% ================================================================
% PATCH 014: CLASSIFICATION TABLE
% ================================================================

\section{Classification of Chiral Algebras by Koszul Type}
\label{app:koszul-classification}

This appendix provides a complete classification of chiral algebras by their Koszul duality 
properties. See \S\ref{sec:filtered-vs-curved-comprehensive} for detailed theory.

\begin{table}[H]
\centering
\caption{Complete Classification of Chiral Algebras}
\begin{tabular}{|l|c|c|c|c|}
\hline
\textbf{Algebra} & \textbf{Type} & \textbf{$\dim(V)$} & \textbf{Completion?} & \textbf{Koszul Dual Exists?} \\
\hline
\hline
Heisenberg $\mathcal{H}_k$ & Quadratic & 1 & No & Yes \\
\hline
Kac-Moody $\widehat{\mathfrak{g}}_k$ & Quadratic & $\dim(\mathfrak{g})$ & No & Yes \\
\hline
Free fermion $\beta\gamma$ & Quadratic & 2 & No & Yes \\
\hline
Virasoro $\text{Vir}_c$ & Curved & 1 & Yes & Yes (completed) \\
\hline
$W_3$ & Filtered & 2 & Yes & Yes (completed) \\
\hline
$W_N$ ($N \geq 4$) & Filtered & $N-1$ & Yes & Yes (completed) \\
\hline
$W_{1+\infty}$ & Filtered & $\infty$ & Yes & Yes (completed) \\
\hline
$W_\infty$ & General & $\infty$ & N/A & NO \\
\hline
\end{tabular}
\end{table}

\textbf{Key:}
\begin{itemize}
\item \textbf{Type}: Quadratic / Curved / Filtered / General
\item \textbf{$\dim(V)$}: Dimension of generating space
\item \textbf{Completion?}: Whether nilpotent completion is needed
\item \textbf{Koszul Dual Exists?}: Whether $\mathcal{A}^!$ is well-defined
\end{itemize}

%================================================================
% SECTION: ESSENTIAL IMAGE CHARACTERIZATION
%================================================================

\section{Essential Image: When is $\widehat{\mathcal{C}} = \mathcal{A}^!$?}
\label{sec:essential-image-koszul}

\subsection{The Characterization Problem}

\begin{question}[Inverse Problem]\label{q:inverse-problem-koszul}
Given a chiral coalgebra $\widehat{\mathcal{C}}$, when does there exist a chiral 
algebra $\mathcal{A}$ such that:
$$\widehat{\mathcal{C}} \cong \mathcal{A}^!$$
(as Koszul dual)?

In other words: What is the \textbf{essential image} of the Koszul duality functor?
\end{question}

\begin{remark}[Why This Matters]\label{rem:why-essential-image-matters}
This question is important for several reasons:

\textbf{1. Recognition problem:}
Given a coalgebra from geometry or physics, can we identify it as a Koszul dual?

\textbf{2. Completeness:}
Does the Koszul duality correspondence cover ``all'' coalgebras, or only a special class?

\textbf{3. Uniqueness:}
If $\widehat{\mathcal{C}} = \mathcal{A}^!$, is $\mathcal{A}$ unique?

\textbf{4. Construction:}
Can we reconstruct $\mathcal{A}$ from $\widehat{\mathcal{C}}$?
\end{remark}

\subsection{Main Characterization Theorem}

\begin{theorem}[Essential Image of Koszul Duality]\label{thm:essential-image-koszul}
A chiral coalgebra $\widehat{\mathcal{C}}$ is (isomorphic to) the Koszul dual 
$\mathcal{A}^!$ of some chiral algebra $\mathcal{A}$ if and only if:

\begin{enumerate}
\item \textbf{Conilpotency:} $\widehat{\mathcal{C}}$ is conilpotent:
      $$\bigcap_{n=1}^\infty \text{coker}(\Delta^n) = \{0\}$$
      
\item \textbf{Connected:} The counit is surjective onto the ground field:
      $$\epsilon: \widehat{\mathcal{C}} \twoheadrightarrow \mathbb{C}$$
      
\item \textbf{Geometric representability:} $\widehat{\mathcal{C}}$ arises as the 
      bar complex of some factorization algebra on configuration spaces
      
\item \textbf{Curvature centrality:} Any curvature term $\mu_0 \in \widehat{\mathcal{C}}^{\otimes 2}[2]$ 
      is central in the dual algebra
      
\item \textbf{Formal completeness:} $\widehat{\mathcal{C}}$ is complete with respect 
      to its coaugmentation coideal
\end{enumerate}

When these conditions hold, the algebra $\mathcal{A}$ is recovered by:
$$\mathcal{A} = \Omega(\widehat{\mathcal{C}})$$
(cobar construction), and this is unique up to quasi-isomorphism.
\end{theorem}

\begin{proof}[Proof Strategy]
The proof has two directions:

\textbf{($\Rightarrow$) Necessity:} If $\widehat{\mathcal{C}} = \mathcal{A}^!$, 
then conditions (1)-(5) hold.

This follows from properties of the Koszul dual construction:
\begin{itemize}
\item (1) Conilpotency: Automatic for Koszul duals (Theorem \ref{thm:koszul-conilpotent})
\item (2) Connected: Dual to augmentation of $\mathcal{A}$
\item (3) Geometric: Bar complex construction is geometric (Theorem \ref{thm:bar-geometric})
\item (4) Curvature: Central obstructions in $\mathcal{A}$ give central curvature
\item (5) Completeness: Induced by filtration on $\mathcal{A}$
\end{itemize}

\textbf{($\Leftarrow$) Sufficiency:} If conditions (1)-(5) hold, define:
$$\mathcal{A} = \Omega(\widehat{\mathcal{C}})$$

We must show:
\begin{enumerate}
\item $\mathcal{A}$ is a well-defined chiral algebra
\item $\bar{B}(\mathcal{A}) \simeq \widehat{\mathcal{C}}$ (bar-cobar inversion)
\item $\mathcal{A}$ has $\widehat{\mathcal{C}}$ as its Koszul dual
\end{enumerate}

This is established in the following subsections.
\end{proof}

\subsection{Conilpotency and Connectedness}

\begin{lemma}[Conilpotency is Necessary]\label{lem:conilpotency-necessary}
If $\widehat{\mathcal{C}} = \mathcal{A}^!$ for some $\mathcal{A}$, then $\widehat{\mathcal{C}}$ 
is conilpotent.
\end{lemma}

\begin{proof}
Let $I \subseteq \mathcal{A}$ be the augmentation ideal (kernel of the counit). Then:
$$\mathcal{A} = \mathbb{C} \oplus I$$

The Koszul dual is built from $I$:
$$\mathcal{A}^! = \text{Cofree}(sI^*)$$

For any element $c \in \mathcal{A}^!$, write:
$$c = c_0 + c_1 + c_2 + \cdots$$
where $c_n \in (sI^*)^{\otimes n}$.

The iterated comultiplication is:
$$\Delta^n(c) = \sum_{i_0 + \cdots + i_k = n} c_{i_0} \otimes \cdots \otimes c_{i_k}$$

As $n \to \infty$, the image of $\Delta^n$ consists only of elements with arbitrarily 
many tensor factors. Since $I$ is the augmentation ideal, these eventually vanish.

Therefore:
$$\bigcap_n \text{coker}(\Delta^n) = \{0\}$$

This is precisely conilpotency.
\end{proof}

\begin{lemma}[Connectedness Characterizes Augmentation]\label{lem:connectedness-augmentation}
A coalgebra $\widehat{\mathcal{C}}$ is connected (has surjective counit $\epsilon: \widehat{\mathcal{C}} 
\twoheadrightarrow \mathbb{C}$) if and only if it is the dual of an augmented algebra.
\end{lemma}

\begin{proof}
The counit $\epsilon$ of a coalgebra dualizes to the unit $\eta$ of an algebra:
$$\eta: \mathbb{C} \to \mathcal{A} \quad \leftrightarrow \quad \epsilon: \widehat{\mathcal{C}} 
\to \mathbb{C}$$

Surjectivity of $\epsilon$ means:
$$\epsilon(c) \neq 0 \text{ for some } c \in \widehat{\mathcal{C}}$$

This is equivalent to $\eta$ being injective, i.e., $\mathcal{A}$ is augmented.
\end{proof}

\subsection{Geometric Representability}

\begin{definition}[Geometrically Representable Coalgebra]\label{def:geom-representable-coalgebra}
A chiral coalgebra $\widehat{\mathcal{C}}$ is \textbf{geometrically representable} 
if there exists:
\begin{enumerate}
\item A factorization algebra $\mathcal{F}$ on configuration spaces $\{C_n(X)\}$
\item A quasi-isomorphism:
      $$\widehat{\mathcal{C}} \simeq \int_{C_\bullet(X)} \mathcal{F}$$
      (factorization homology)
\end{enumerate}
\end{definition}

\begin{theorem}[Koszul Duals are Geometrically Representable]\label{thm:koszul-geom-rep}
If $\widehat{\mathcal{C}} = \mathcal{A}^!$ for a chiral algebra $\mathcal{A}$, then 
$\widehat{\mathcal{C}}$ is geometrically representable via:
$$\mathcal{A}^! \simeq \bar{B}^{\text{geom}}(\mathcal{A}) = 
\bigoplus_{n \geq 0} \Gamma\left(\overline{C}_n(X), \mathcal{A}^{\boxtimes n} 
\otimes \Omega^\bullet\right)$$
\end{theorem}

\begin{proof}
The geometric bar complex (Definition \ref{def:geometric-bar}) provides the 
geometric realization:

\textbf{Step 1:} The factorization algebra is:
$$\mathcal{F}_{C_n}(U) = \Gamma(U, \mathcal{A}^{\boxtimes n}|_U)$$
for $U \subseteq C_n(X)$.

\textbf{Step 2:} The bar complex computes factorization homology:
$$\bar{B}^{\text{geom}}(\mathcal{A}) = \int_{C_\bullet(X)} \mathcal{F}$$

This was proven in Theorem \ref{thm:bar-factorization-homology}.

\textbf{Step 3:} By bar-cobar duality:
$$\bar{B}^{\text{geom}}(\mathcal{A}) \simeq \mathcal{A}^!$$

Therefore $\mathcal{A}^!$ is geometrically representable.
\end{proof}

\begin{corollary}[Converse: Geometric Representability Implies Koszul]\label{cor:geom-implies-koszul}
If $\widehat{\mathcal{C}}$ is geometrically representable by a factorization algebra 
$\mathcal{F}$ on configuration spaces, and satisfies conilpotency + connectedness, 
then:
$$\widehat{\mathcal{C}} = \mathcal{A}^!$$
for $\mathcal{A} = \Omega(\widehat{\mathcal{C}})$.
\end{corollary}

\subsection{Curvature and Centrality}

\begin{theorem}[Curvature Must Be Central]\label{thm:curvature-central}
Let $\widehat{\mathcal{C}}$ be a curved coalgebra with curvature:
$$\mu_0 \in \widehat{\mathcal{C}}^{\otimes 2}[2]$$

If $\widehat{\mathcal{C}} = \mathcal{A}^!$ for some algebra $\mathcal{A}$, then 
$\mu_0$ must be \textbf{central} in the sense:
$$\mu_0 \text{ commutes with all operations in } \mathcal{A}$$
\end{theorem}

\begin{proof}
The curvature $\mu_0$ in the coalgebra $\widehat{\mathcal{C}} = \mathcal{A}^!$ 
corresponds to a central extension in the algebra $\mathcal{A}$.

\textbf{Step 1: Maurer-Cartan equation.}
The curved structure satisfies:
$$d(\mu_0) + \frac{1}{2}[\mu_0, \mu_0] = 0$$
 
\textbf{Step 2: Duality.}
Under Koszul duality, this equation dualizes to:
$$\partial(\mu_0^*) + \frac{1}{2}\{\mu_0^*, \mu_0^*\} = 0$$

in $\mathcal{A}$.

\textbf{Step 3: Centrality.}
The condition $[\mu_0, \mu_0] = 0$ implies $\mu_0$ generates a central extension:
$$0 \to \mathbb{C} \xrightarrow{\mu_0} \tilde{\mathcal{A}} \to \mathcal{A} \to 0$$

Therefore $\mu_0$ is central in $\mathcal{A}$.
\end{proof}

\begin{example}[Virasoro Central Charge]\label{ex:virasoro-central-charge-curvature}
For the Virasoro algebra:
$$\text{Vir} = \text{span}\{L_n, c\} / ([L_m, L_n] - (m-n)L_{m+n} - \frac{c}{12}(m^3-m)\delta_{m,-n})$$

The central charge $c$ is a curvature term in the dual coalgebra $\text{Vir}^!$.

\textbf{Verification:}
\begin{itemize}
\item $c$ commutes with all $L_n$: $[c, L_n] = 0$
\item $c$ is central: It generates $Z(\text{Vir}) = \mathbb{C} \cdot c$
\item In the dual: $c$ appears as curvature $\mu_0$ in the coalgebra differential
\end{itemize}

This confirms Theorem \ref{thm:curvature-central}.
\end{example}

\subsection{Formal Completeness}

\begin{definition}[Coaugmentation Coideal]\label{def:coaugmentation-coideal}
For a connected coalgebra $\widehat{\mathcal{C}}$ with counit $\epsilon: \widehat{\mathcal{C}} 
\to \mathbb{C}$, the \textbf{coaugmentation coideal} is:
$$\bar{\mathcal{C}} = \ker(\epsilon)$$

This is the ``reduced'' part of the coalgebra (everything that doesn't map to the 
ground field).
\end{definition}

\begin{theorem}[Completion Characterization]\label{thm:completion-characterization}
A coalgebra $\widehat{\mathcal{C}}$ is the Koszul dual of some algebra $\mathcal{A}$ 
if and only if it is \textbf{complete} with respect to its coaugmentation coideal:
$$\widehat{\mathcal{C}} = \varprojlim_n \widehat{\mathcal{C}} / \bar{\mathcal{C}}^n$$
\end{theorem}

\begin{proof}
\textbf{($\Rightarrow$) Necessity:}
If $\widehat{\mathcal{C}} = \mathcal{A}^!$, then the filtration on $\mathcal{A}$ by 
powers of the augmentation ideal induces a cofiltration on $\widehat{\mathcal{C}}$:
$$F^n\widehat{\mathcal{C}} = \{c : \Delta^k(c) \in (\bar{\mathcal{C}})^{\otimes k} 
\text{ for } k \leq n\}$$

The completion is:
$$\widehat{\mathcal{C}} = \varprojlim_n \widehat{\mathcal{C}} / \bar{\mathcal{C}}^n$$

This holds by construction of $\mathcal{A}^!$.

\textbf{($\Leftarrow$) Sufficiency:}
If $\widehat{\mathcal{C}}$ is complete, define:
$$\mathcal{A} = \Omega(\widehat{\mathcal{C}})$$

The completeness ensures that the cobar construction converges, giving a well-defined 
algebra structure on $\mathcal{A}$.

By bar-cobar inversion (Theorem \ref{thm:bar-cobar-inversion-qi}):
$$\bar{B}(\mathcal{A}) \simeq \widehat{\mathcal{C}}$$

Therefore $\widehat{\mathcal{C}} = \mathcal{A}^!$.
\end{proof}

\subsection{Uniqueness of the Algebra}

\begin{theorem}[Uniqueness Up to Quasi-Isomorphism]\label{thm:uniqueness-algebra}
If $\widehat{\mathcal{C}} = \mathcal{A}^! = \mathcal{B}^!$ for two chiral algebras 
$\mathcal{A}$ and $\mathcal{B}$, then:
$$\mathcal{A} \simeq \mathcal{B}$$
(quasi-isomorphic as chiral algebras).
\end{theorem}

\begin{proof}
The cobar construction provides canonical algebra structures:
$$\mathcal{A} \simeq \Omega(\mathcal{A}^!) = \Omega(\widehat{\mathcal{C}}) = 
\Omega(\mathcal{B}^!) \simeq \mathcal{B}$$

All quasi-isomorphisms are via the bar-cobar adjunction (Theorem \ref{thm:bar-cobar-inversion-qi}).
\end{proof}

\begin{remark}[Non-Uniqueness at the Strict Level]\label{rem:non-uniqueness-strict}
The theorem only guarantees quasi-isomorphism, not strict isomorphism. Different 
presentations of the same chiral algebra (e.g., different choices of generators 
and relations) give strictly different algebras that are quasi-isomorphic.

\textbf{Example:} The Heisenberg algebra can be presented as:
\begin{itemize}
\item $\mathcal{H}_1 = \text{Free}(a, a^*) / ([a, a^*] - 1)$
\item $\mathcal{H}_2 = \text{Free}(x, p) / ([x, p] - i\hbar)$
\end{itemize}

These are different presentations (different generators), but $\mathcal{H}_1 \simeq \mathcal{H}_2$ 
as chiral algebras, and both have the same Koszul dual.
\end{remark}

