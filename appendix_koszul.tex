\appendix
\section{Mathematical Foundations of Koszul Duality}

\subsection{Definition and Basic Properties}

\begin{definition}[Koszul Algebra]
An associative algebra $A$ is \emph{Koszul} if:
$$\text{Ext}_A^{i,j}(\mathbb{k}, \mathbb{k}) = 0 \text{ for } i \neq j$$
where the bigrading is by homological degree and internal degree.
\end{definition}

\begin{theorem}[Koszul Duality Theorem]
If $A$ is Koszul, then:
$$A^! := \text{Ext}_A^*(\mathbb{k}, \mathbb{k})$$
is also Koszul, and $(A^!)^! \cong A$.
\end{theorem}

\subsection{Chiral Koszul Duality}

For chiral algebras, we need a modified definition:

\begin{definition}[Chiral Koszul Duality]
Chiral algebras $\mathcal{A}$ and $\mathcal{B}$ are Koszul dual if:
$$\text{RHom}_{\mathcal{A} \otimes \mathcal{B}}(\mathbb{C}, \mathbb{C}) \simeq \mathbb{C}$$
in the derived category of chiral modules.
\end{definition}

\subsection{Curved and Filtered Generalizations}

\begin{definition}[Curved Koszul Duality]
A curved algebra $(A, d, m_0)$ with $d^2 = m_0 \cdot \text{id}$ has curved dual:
$$(A^!, d^!, m_0^!)$$
where $m_0^! = -m_0$ under the pairing.
\end{definition}

\subsection{Computational Tools}

\begin{lemma}[Koszul Complex Resolution]
For Koszul $A$, the minimal resolution of $\mathbb{k}$ is:
$$\cdots \to A \otimes A^!_{(2)} \to A \otimes A^!_{(1)} \to A \to \mathbb{k}$$
where $A^!_{(n)}$ is the degree $n$ part of $A^!$.
\end{lemma}

\subsection{Physical Interpretation}

In physics, Koszul duality appears as:
\begin{itemize}
\item Electric-magnetic duality (abelian case)
\item Open-closed string duality (topological strings)  
\item Holographic duality (AdS/CFT)
\item Mirror symmetry (A-model/B-model)
\end{itemize}

\subsection{Maurer-Cartan Elements and Twisting}

\begin{theorem}[MC Elements Parametrize Deformations]
For a chiral algebra $\mathcal{A}$ and its bar complex $\bar{B}^{\text{ch}}(\mathcal{A})$:

\textbf{1. Maurer-Cartan Equation:}
$$\alpha \in \bar{B}^1_{\text{geom}}(\mathcal{A}), \quad d\alpha + \frac{1}{2}[\alpha, \alpha] = 0$$

\textbf{2. Twisting:}
Each MC element $\alpha$ yields a twisted differential:
$$d_{\alpha} = d + [\alpha, -]$$
with $(d_{\alpha})^2 = 0$.

\textbf{3. Deformation:}
MC elements correspond to first-order deformations of $\mathcal{A}$:
$$\mu_{\alpha}(a \otimes b) = \mu(a \otimes b) + \langle \alpha, a \otimes b \rangle$$

\textbf{4. Geometric Interpretation:}
On configuration spaces, MC elements are:
\begin{itemize}
\item Closed 1-forms on $\overline{C}_2(X)$ with prescribed residues
\item Flat connections on the punctured configuration space
\item Solutions to the classical Yang-Baxter equation
\end{itemize}

\textbf{5. Moduli Space:}
$$\mathcal{M}_{\text{MC}}(\mathcal{A}) = \{\text{MC elements}\}/\text{gauge equivalence}$$
parametrizes deformations of the chiral algebra structure.
\end{theorem}
