\appendix
\chapter{Koszul Duality Across Genera}

\section{Genus-Graded Koszul Duality}

\begin{theorem}[Extended Koszul Duality]
If $(\mathcal{A}, \mathcal{A}^!)$ form a genus-0 Koszul dual pair, then:
$$\left(\bigoplus_{g \geq 0} \mathcal{A}^{(g)}, \bigoplus_{g \geq 0} (\mathcal{A}^!)^{(g)}\right)$$
form a multi-genus Koszul dual pair with pairing:
$$\langle -, - \rangle: \mathcal{A}^{(g)} \otimes (\mathcal{A}^!)^{(g)} \to \mathbb{C}[\![\hbar]\!]$$
where $\hbar$ tracks the genus.
\end{theorem}

\section{Definition and Basic Properties}

\begin{definition}[Genus-Graded Koszul Algebra]
A genus-graded associative algebra $\mathcal{A} = \bigoplus_{g \geq 0} \mathcal{A}^{(g)}$ is \emph{Koszul} if:
$$\text{Ext}_{\mathcal{A}^{(g)}}^{i,j}(\mathbb{k}, \mathbb{k}) = 0 \text{ for } i \neq j$$
where the bigrading is by homological degree and internal degree, and the Koszul property holds at each genus.
\end{definition}

\begin{theorem}[Genus-Graded Koszul Duality Theorem]
If $\mathcal{A}$ is genus-graded Koszul, then:
$$\mathcal{A}^! := \bigoplus_{g \geq 0} \text{Ext}_{\mathcal{A}^{(g)}}^*(\mathbb{k}, \mathbb{k})$$
is also genus-graded Koszul, and $(\mathcal{A}^!)^! \cong \mathcal{A}$.
\end{theorem}

\subsection{Genus-Graded Chiral Koszul Duality}

For chiral algebras across all genera, we need a modified definition:

\begin{definition}[Genus-Graded Chiral Koszul Duality]
Genus-graded chiral algebras $\mathcal{A} = \bigoplus_{g \geq 0} \mathcal{A}^{(g)}$ and $\mathcal{B} = \bigoplus_{g \geq 0} \mathcal{B}^{(g)}$ are Koszul dual if:
$$\text{RHom}_{\mathcal{A}^{(g)} \otimes \mathcal{B}^{(g)}}(\mathbb{C}, \mathbb{C}) \simeq \mathbb{C}$$
in the derived category of chiral modules at each genus $g$, with modular covariance under $\text{Sp}(2g, \mathbb{Z})$ transformations.
\end{definition}

\subsection{Curved and Filtered Generalizations Across Genera}

\begin{definition}[Genus-Graded Curved Koszul Duality]
A genus-graded curved algebra $(\mathcal{A}^{(g)}, d^{(g)}, m_0^{(g)})$ with $(d^{(g)})^2 = m_0^{(g)} \cdot \text{id}$ has curved dual:
$$((\mathcal{A}^{(g)})^!, d^{!(g)}, m_0^{!(g)})$$
where $m_0^{!(g)} = -m_0^{(g)}$ under the genus-graded pairing, with modular corrections from period integrals.
\end{definition}

\subsection{Computational Tools Across Genera}

\begin{lemma}[Genus-Graded Koszul Complex Resolution]
For genus-graded Koszul $\mathcal{A}$, the minimal resolution of $\mathbb{k}$ at genus $g$ is:
$$\cdots \to \mathcal{A}^{(g)} \otimes (\mathcal{A}^!)_{(2)}^{(g)} \to \mathcal{A}^{(g)} \otimes (\mathcal{A}^!)_{(1)}^{(g)} \to \mathcal{A}^{(g)} \to \mathbb{k}$$
where $(\mathcal{A}^!)_{(n)}^{(g)}$ is the degree $n$ part of $\mathcal{A}^!$ at genus $g$, with modular corrections from period integrals.
\end{lemma}

\subsection{Physical Interpretation Across Genera}

In physics, genus-graded Koszul duality appears as:
\begin{itemize}
\item Electric-magnetic duality with genus corrections (abelian case)
\item Open-closed string duality with modular forms (topological strings)  
\item Holographic duality with genus expansion (AdS/CFT)
\item Mirror symmetry with period integrals (A-model/B-model)
\item String amplitudes with genus-graded corrections
\end{itemize}

\subsection{Genus-Graded Maurer-Cartan Elements and Twisting}

\begin{theorem}[Genus-Graded MC Elements Parametrize Deformations]
For a genus-graded chiral algebra $\mathcal{A} = \bigoplus_{g \geq 0} \mathcal{A}^{(g)}$ and its bar complex $\bar{B}(\mathcal{A})$:

\textbf{1. Genus-Graded Maurer-Cartan Equation:}
$$\alpha^{(g)} \in \barBgeom^{(g)}(\mathcal{A}), \quad d^{(g)}\alpha^{(g)} + \frac{1}{2}[\alpha^{(g)}, \alpha^{(g)}] = 0$$
with modular corrections from period integrals.

\textbf{2. Genus-Graded Twisting:}
Each MC element $\alpha^{(g)}$ yields a twisted differential:
$$d_{\alpha^{(g)}}^{(g)} = d^{(g)} + [\alpha^{(g)}, -]$$
with $(d_{\alpha^{(g)}}^{(g)})^2 = 0$ and modular covariance.

\textbf{3. Genus-Graded Deformation:}
MC elements correspond to first-order deformations of $\mathcal{A}^{(g)}$:
$$\mu_{\alpha^{(g)}}^{(g)}(a \otimes b) = \mu^{(g)}(a \otimes b) + \langle \alpha^{(g)}, a \otimes b \rangle$$
with genus corrections.

\textbf{4. Geometric Interpretation Across Genera:}
On configuration spaces, MC elements are:
\begin{itemize}
\item Closed 1-forms on $\overline{C}_2^{(g)}(\Sigma_g)$ with prescribed residues and period integrals
\item Flat connections on the punctured configuration space with modular structure
\item Solutions to the classical Yang-Baxter equation with genus corrections
\end{itemize}

\textbf{5. Genus-Graded Moduli Space:}
$$\mathcal{M}_{\text{MC}}^{(g)}(\mathcal{A}) = \{\text{MC elements at genus } g\}/\text{gauge equivalence}$$
parametrizes deformations of the chiral algebra structure at each genus.
\end{theorem}

\subsection{Koszul Duality at Higher Genus: The Tower Structure}
\label{app:koszul_higher_genus}

The genus 0 Koszul duality:
$$\Omega C_{\bullet}^{(0)}(\mathcal{A}) \simeq \mathcal{A}$$
extends to all genera by the modular operad structure.

\subsubsection{The Genus $g$ Statement}

For each $g \geq 0$, there is a duality:
$$\Omega^{(g)} C_{\bullet}^{(g)}(\mathcal{A}) \simeq \mathcal{A}^{(g)}$$
where:
\begin{itemize}
\item $\Omega^{(g)}$ is the genus $g$ cobar construction
\item $\mathcal{A}^{(g)}$ is the genus $g$ component of $\mathcal{A}$
\end{itemize}

\subsubsection{Compatibility}

The genus stratification satisfies:
$$\partial: C_{\bullet}^{(g)} \to C_{\bullet}^{(g-1)}$$
(boundary/degeneration maps) compatible with:
$$\iota: \mathcal{A}^{(g-1)} \to \mathcal{A}^{(g)}$$
(restriction maps).

This gives a \textbf{tower of Koszul dualities}:
\begin{center}
\begin{tikzcd}[column sep=small]
\cdots \arrow[r] & C_{\bullet}^{(2)}(\mathcal{A}) \arrow[r] \arrow[d, "\Omega^{(2)}"] & 
C_{\bullet}^{(1)}(\mathcal{A}) \arrow[r] \arrow[d, "\Omega^{(1)}"] & 
C_{\bullet}^{(0)}(\mathcal{A}) \arrow[d, "\Omega^{(0)}"] \\
\cdots \arrow[r] & \mathcal{A}^{(2)} \arrow[r] & 
\mathcal{A}^{(1)} \arrow[r] & 
\mathcal{A}^{(0)}
\end{tikzcd}
\end{center}

\subsubsection{The Limit}

Taking the inverse limit:
$$\mathcal{A}_{\text{complete}} = \varprojlim_g \mathcal{A}^{(g)}$$
gives the \textbf{completed chiral algebra}, encoding all genus contributions.

\subsubsection{Modular Invariance}

At each genus $g$, the duality respects the action of the mapping class group $\Gamma_g = \operatorname{MCG}(\Sigma_g)$:
$$\Omega^{(g)}(\sigma^* C_{\bullet}^{(g)}(\mathcal{A})) \simeq \sigma^* \mathcal{A}^{(g)}$$
for $\sigma \in \Gamma_g$.

This ensures that genus $g$ quantum corrections are modular-invariant.
