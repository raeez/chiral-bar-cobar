\chapter{Introduction}

\section{The Consequences of Non-Abelian Poincaré Duality on Quantum Field Theory}
\label{sec:nonabelian-poincare-intro}

\subsection{Beyond Classical Poincaré Duality}

Classical Poincaré duality establishes an isomorphism between homology and cohomology:
\begin{equation}
H_k(M) \cong H^{n-k}(M)^\vee
\end{equation}
for an $n$-dimensional closed oriented manifold $M$. This is fundamentally \emph{abelian}—both sides are vector spaces related by a linear duality.

\begin{principle}[Non-Abelian Poincaré Duality]
\label{prin:nonabelian-poincare}
Non-abelian Poincaré duality, in the sense of Ayala-Francis, extends this to a duality between \emph{algebraic structures}:
\begin{equation}
\int_M \mathcal{A} \simeq \left(\int_M \mathcal{A}^!\right)^\vee
\end{equation}
where:
\begin{itemize}
\item $\mathcal{A}$ is a factorization algebra (encoding local-to-global algebraic data)
\item $\mathcal{A}^!$ is its Koszul dual factorization algebra
\item $\int_M$ denotes factorization homology
\item The duality preserves non-abelian (non-commutative) structure
\end{itemize}
\end{principle}

\subsection{Chiral Algebras as Factorization Algebras}

Following Beilinson-Drinfeld and Francis-Gwilliam, a chiral algebra $\mathcal{A}$ on a curve $X$ is equivalently:

\begin{enumerate}
\item \textbf{BD Perspective}: A $\mathcal{D}_X$-module with chiral operations defined via residues
\item \textbf{Factorization Perspective}: A factorization algebra on $X$ satisfying:
\begin{equation}
\mathcal{A}(U \sqcup V) \xrightarrow{\sim} \mathcal{A}(U) \otimes_{\mathcal{D}_X} \mathcal{A}(V)
\end{equation}
for disjoint open sets $U, V \subset X$
\end{enumerate}

\begin{remark}[Why This Matters]
The factorization property encodes \textbf{locality} of quantum field theory: observations at separated points are independent (factorize). This is the physical content underlying the mathematical structure.
\end{remark}

\section{The Unifying Principle: Non-Abelian Poincaré Duality}
\label{sec:NAP-unifying}

\subsection{Three Perspectives on Duality}

\begin{framework}[The Three-Way Correspondence]\label{framework:three-way}
This manuscript reveals that chiral bar-cobar duality is one manifestation of a profound three-way correspondence:

$$\boxed{
\begin{array}{c}
\text{Non-Abelian Poincaré Duality} \\
\updownarrow \\
\text{Chiral Koszul Duality} \\
\updownarrow \\
\text{Verdier Duality on Configuration Spaces}
\end{array}
}$$

Each perspective illuminates different aspects of the same underlying phenomenon.
\end{framework}

\subsection{Witten's Physical Intuition: Why Non-Abelian?}

\begin{motivation}[From Abelian to Non-Abelian Duality]
Classical Poincaré duality relates homology and cohomology of a manifold:
$$H_k(M, \mathbb{Z}) \cong H^{n-k}(M, \mathbb{Z})^*$$

This is ``abelian'' because we're working with coefficient systems—vector spaces or abelian groups.

In quantum field theory, we need \textbf{non-abelian} generalizations where coefficients are replaced by:
\begin{itemize}
\item Algebras (encoding operator products)
\item Categories (encoding different sectors)
\item Higher structures (encoding quantum corrections)
\end{itemize}

Non-abelian Poincaré duality provides this: instead of pairing cohomology classes, we pair \textbf{factorization algebras}, which encode the full quantum field theory structure.
\end{motivation}

\subsection{Ayala-Francis Framework}

\begin{theorem}[Ayala-Francis Poincaré-Koszul Correspondence]\label{thm:ayala-francis-NAP}
\textup{(Ayala-Francis \cite{AF15}, Theorem 8.11)}

For an oriented d-manifold M and an E_d-algebra A, factorization homology satisfies:
$$\int_M \mathbb{D}(A) \simeq \mathbb{D}\left(\int_{-M} A\right)$$

where:
\begin{itemize}
\item $\mathbb{D}$ denotes duality (Spanier-Whitehead in spectra)
\item $-M$ denotes M with opposite orientation
\item The integral is factorization homology
\end{itemize}

This simultaneously generalizes:
\begin{enumerate}
\item \textbf{Poincaré duality}: Taking A = k (constants)
\item \textbf{Koszul duality}: Taking M = S^d (sphere)
\end{enumerate}
\end{theorem}

\begin{remark}[Why This Unifies Our Story]\label{rem:why-NAP-unifies}
For chiral algebras on curves (d=1), the Ayala-Francis framework specializes to give:

\begin{itemize}
\item \textbf{Manifold structure}: The curve X with its orientation
\item \textbf{Algebra structure}: The chiral algebra $\mathcal{A}$ as factorization algebra
\item \textbf{Duality}: Bar-cobar relating $\mathcal{A}$ and its Koszul dual $\mathcal{A}^!$
\end{itemize}

Our geometric construction via configuration spaces makes this duality completely explicit and computable.
\end{remark}

\subsection{The Three-Way Correspondence}

Our central insight is that chiral Koszul duality sits at the nexus of three perspectives:

\begin{center}
\begin{tikzcd}[row sep=huge, column sep=huge]
\text{Chiral Koszul Duality} \arrow[r, "\text{specializes}"] \arrow[d, "\text{realizes}"'] & \text{Non-Abelian Poincaré Duality} \arrow[d, "\text{via}"] \\
\text{Configuration Space Geometry} \arrow[r, "\text{computes}"'] & \text{Factorization Homology}
\end{tikzcd}
\end{center}

\begin{theorem}[Unification via Configuration Spaces]
\label{thm:three-way-correspondence}
For a chiral Koszul pair $(\mathcal{A}, \mathcal{A}^!)$ on a curve $X$:
\begin{enumerate}
\item \textbf{Algebraic}: $\Omega(\bar{B}(\mathcal{A})) \simeq \mathcal{A}$ (bar-cobar adjunction)
\item \textbf{Geometric}: Both $\bar{B}(\mathcal{A})$ and $\mathcal{A}^!$ realized via integrals on $\overline{C}_n(X)$
\item \textbf{Homological}: 
\begin{equation}
\int_X \mathcal{A} \simeq \left(\int_X \mathcal{A}^!\right)^\vee
\end{equation}
computed by factorization homology
\end{enumerate}
The three perspectives are equivalent and mutually enriching.
\end{theorem}

\subsection{Computational Manifestations}

This framework provides concrete computational tools:

\begin{itemize}
\item \textbf{Witten (Physical)}: OPE residues as Feynman rules
\item \textbf{Kontsevich (Geometric)}: Configuration space integrals compute everything
\item \textbf{Serre (Computational)}: Explicit formulas degree by degree
\item \textbf{Grothendieck (Categorical)}: Functorial uniqueness determines structure
\end{itemize}

We develop each perspective systematically in what follows.

\subsection{Relationship to Topological-Holomorphic Theories}
\label{subsec:topological-holomorphic-distinction}

\begin{remark}[Pure Holomorphic vs Topological-Holomorphic]\label{rem:pure-vs-topological-holomorphic}
Our framework studies \textbf{pure holomorphic} chiral algebras, which should be 
carefully distinguished from the \textbf{topological-holomorphic} theories developed 
by Costello-Li \cite{CL16}, Gaiotto \cite{Gai19}, and Paquette-Williams 
\cite{PW22}:

\begin{center}
\small
\begin{tabular}{|l|l|l|}
\hline
\textbf{Aspect} & \textbf{Pure Holomorphic (Us)} & \textbf{Topological-Holomorphic (CL/PW)} \\
\hline
Dimension & 2D conformal & 4D gauge theory (twisted) \\
Target space & Riemann surface $X$ & Complex 2-fold (Calabi-Yau) \\
Fields & Holomorphic sections & $\bar{\partial}$-closed forms \\
Structure & Chiral algebras (BD) & Factorization algebras (CG) \\
Symmetry & Virasoro (conformal) & Twisted supersymmetry \\
Bar complex & Configuration spaces on $X$ & BV observables \\
Physical origin & 2D CFT, string worldsheet & 4D $\mathcal{N}=2$ gauge theory \\
\hline
\end{tabular}
\end{center}

\textbf{Key distinction:} 
\begin{itemize}
\item \textbf{Our work}: Studies chiral algebras $\mathcal{A}$ on complex curves via 
geometric bar-cobar duality, generalizing Beilinson-Drinfeld to all genera using 
configuration space compactifications.

\item \textbf{Costello-Li}: Studies dimensional reduction of 4D $\mathcal{N}=2$ 
gauge theories to 2D, producing factorization algebras via BV formalism. The ``holomorphic'' 
structure comes from topological twist, not from intrinsic 2D CFT.

\item \textbf{Paquette-Williams}: Studies boundary conditions in holomorphic-topological 
twist, connecting to vertex algebras at interfaces \cite{PW22}.
\end{itemize}
\end{remark}

\section{The Central Mystery}

In two-dimensional conformal field theory, the most fundamental observables are correlation functions of local operators. When two chiral operators $\phi_1(z_1)$ and $\phi_2(z_2)$ approach each other on a Riemann surface, their correlation functions develop singularities controlled by the operator product expansion (OPE):

$$\phi_1(z_1) \phi_2(z_2) \sim \sum_k \frac{C^k_{12}}{(z_1 - z_2)^{h_k}} \phi_k(z_2) + \text{regular terms}$$

The structure constants $C^k_{12}$ encode the complete algebraic structure of the chiral algebra. This local singularity data—purely algebraic in nature—turns out to have a natural geometric interpretation that forms the foundation of our work.

\section{The Key Observation}

The key observation is elementary yet profound: the logarithmic differential form $d\log(z_1 - z_2) = \frac{dz_1 - dz_2}{z_1 - z_2}$ has a simple pole precisely when $z_1 = z_2$. When we compute the residue
$$\text{Res}_{z_1=z_2} d\log(z_1 - z_2) \cdot \phi_1(z_1)\phi_2(z_2) = C^k_{12} \phi_k(z_2)$$
we extract exactly the structure constant from the OPE. This simple fact—that algebraic structure constants become geometric residues—motivates our entire construction.

\section{Why Configuration Spaces?}

But why should we expect such a geometric interpretation to exist? The answer lies in a fundamental principle of quantum field theory: locality. The requirement that operators commute at spacelike separation forces the algebraic structure to be encoded in the singularities as operators approach each other. These singularities naturally live on configuration spaces—the spaces parametrizing positions of operators on the curve. The compactification of these spaces, which adds boundary divisors corresponding to collision patterns, provides the geometric arena where quantum algebra becomes algebraic geometry.

\section{Relationship to Foundational Work}

Beilinson and Drinfeld \cite{BD04} axiomatized 2d quantum field theory as factorization algebras on curves with presentations as $\mathcal{D}$-modules with chiral operations. This paper develops a systematic geometric realization of bar-cobar duality for chiral algebras through configuration space integrals, extending across all genera to incorporate the full spectrum of quantum corrections to all loop orders. The construction naturally produces a theory of chiral koszul dual pairs, vastly extending the classic quadratic koszul duality.

Our perspective draws from three mathematical perspectives: the algebraic approach to chiral algebras via $\mathcal{D}$-modules developed by Beilinson-Drinfeld \cite{BD04}, the geometric configuration space methods pioneered by Kontsevich \cite{Kon94, Kon99}, and the higher categorical framework of factorization homology introduced by Ayala-Francis \cite{AF15}.

\subsection{Relation to Costello-Gwilliam}

Our geometric approach complements the perspective in Costello-Gwilliam's 
\textit{Factorization Algebras in Quantum Field Theory} \cite{CostelloGwilliam}:

\begin{itemize}
\item \textbf{Volume 1}: Foundations of factorization algebras. Our bar complex is 
the derived global sections of a factorization algebra (compare CG Vol. 1, Chapter 5).

\item \textbf{Volume 2}: Renormalization and BV formalism. Our nilpotent completion 
(Appendix on non-quadratic algebras) corresponds to Costello's renormalization 
group flow (CG Vol. 2, Chapters 4-5). The I-adic filtration $I^n$ encodes the 
``effective action at scale $n$''.

\item \textbf{Koszul duality}: CG Vol. 2, \S13 develops Koszul duality for $E_n$ 
operads. Our work extends this to the $E_\infty$ (chiral) setting using configuration 
space integrals.
\end{itemize}

\textbf{Key insight}: Bar-cobar duality for chiral algebras is the curved $L_\infty$ 
version of CG's bar-cobar for factorization algebras. The curvature terms come from 
central extensions (quantum anomalies in physics).

\medskip
\noindent
\textbf{Complete Computational Chapters.} This manuscript includes three comprehensive computational chapters providing excruciating detail:
\begin{enumerate}
\item Chapter \ref{chap:chiral-deformation}: Complete treatment of chiral deformation quantization, extending Kontsevich's formality theorem to curves with explicit formulas for all genera and examples (Heisenberg, affine $\mathfrak{sl}_2$, $W_3$) with all coefficients computed.
\item Chapter \ref{chap:kac-moody}: Kac-Moody Koszul duals with complete OPE structures for $\widehat{\mathfrak{sl}}_2, \widehat{\mathfrak{sl}}_3, \widehat{\mathfrak{sl}}_n, \widehat{E}_8$, bar construction through degree 5, and level shift formulas derived from first principles.
\item Chapter \ref{chap:w-algebras}: W-algebra Koszul duals with complete $W_3$ OPE expanded (not abbreviated), mode commutators with all coefficients, $W_k(\mathfrak{sl}_3)$ from BRST construction step-by-step, and examples at $c=2$ and $c=100$.
\end{enumerate}

\subsection{Connections to Related Mathematical Physics Programs}
\label{subsec:related-programs}

\begin{remark}[Landscape of Holomorphic Field Theories]\label{rem:holomorphic-landscape}
Our chiral bar-cobar duality sits within a broader landscape of holomorphic 
constructions in mathematical physics. We clarify the relationships:

\textbf{1. Beilinson-Drinfeld Chiral Algebras (1995-2004)} \cite{BD04}:
\begin{itemize}
\item Foundation: $\mathcal{D}$-modules on configuration spaces
\item Genus: Primarily genus zero (rational curves)
\item Our contribution: Extension to all genera via moduli spaces
\end{itemize}

\textbf{2. Costello-Gwilliam Factorization Algebras (2017)} \cite{CG17}:
\begin{itemize}
\item Foundation: BV formalism, general manifolds
\item Scope: Arbitrary dimension, topological field theories
\item Connection: Our bar complex $\simeq$ CG factorization homology for chiral algebras
\end{itemize}

\textbf{3. Costello-Li Twisted Supergravity (2016)} \cite{CL16}:
\begin{itemize}
\item Foundation: Topological twist of 4D $\mathcal{N}=2$ theories
\item Method: Dimensional reduction produces 2D factorization algebras
\item Key difference: Their ``holomorphic'' = topologically twisted; 
      ours = genuine 2D CFT holomorphic structure
\end{itemize}

\textbf{4. Gaiotto Holomorphic-Topological Twist (2019)} \cite{Gai19}:
\begin{itemize}
\item Foundation: Boundary conditions in HT twist
\item Focus: Interfaces and defects in gauge theory
\item Connection: W-algebras appear as boundary vertex algebras
\end{itemize}

\textbf{5. Paquette-Williams Boundaries and Interfaces (2022)} \cite{PW22}:
\begin{itemize}
\item Foundation: Vertex algebras at corners in HT theories
\item Method: Quantization of moduli spaces produces vertex algebras
\item Our perspective: These vertex algebras have chiral envelope, 
      amenable to our bar-cobar analysis
\end{itemize}

\textbf{6. Ayala-Francis Factorization Homology (2019)} \cite{AF19}:
\begin{itemize}
\item Foundation: $\infty$-categorical factorization homology
\item Generality: Arbitrary symmetric monoidal $\infty$-categories
\item Connection: Our geometric bar complex computes factorization 
      homology for chiral algebras (Theorem \ref{thm:bar-factorization-homology})
\end{itemize}
\end{remark}

\begin{remark}[Why Both Approaches Are Valuable]\label{rem:complementary-approaches}
The pure holomorphic (BD-style) and topological-holomorphic (CL-style) approaches 
are complementary:

\textbf{Pure holomorphic advantages:}
\begin{itemize}
\item Direct access to conformal symmetry and Virasoro representations
\item Explicit connection to vertex algebra OPE
\item Natural framework for modular forms and elliptic functions
\item Clear geometric meaning of central charges and anomalies
\end{itemize}

\textbf{Topological-holomorphic advantages:}
\begin{itemize}
\item Connection to 4D gauge theory (AGT correspondence)
\item Access to Higgs branch moduli spaces
\item Natural framework for S-duality and mirror symmetry
\item Powerful BV technology for quantization
\end{itemize}

\textbf{Synthesis:} Many W-algebras appear in \emph{both} contexts:
\begin{itemize}
\item In 2D CFT: As extended chiral algebras (our setting)
\item In 4D gauge: As boundary VOAs in HT twist (CL/PW setting)
\end{itemize}

Understanding both perspectives provides the deepest insight. Our bar-cobar duality 
gives the 2D CFT structure, while CL/PW give the 4D gauge theory origin. The 
mathematical structures (factorization algebras, BV complexes) unify both views.
\end{remark}

\section{Main Results and Organization}

Our first result establishes the geometric bar construction for chiral algebras through configuration space integrals. This construction is elementary at its core: we take tensor products of the chiral algebra and integrate logarithmic forms over configuration spaces. The residues at collision divisors extract the algebraic operations:

\begin{theorem}[Geometric Bar Construction, Theorem 3.2]
For a chiral algebra $\mathcal{A}$ on a smooth curve $X$, we construct a geometric bar complex at the chain level:
$$\bar{B}^{\text{geom}}(\mathcal{A})_n = \Gamma\left(\overline{C}_n(X), \mathcal{A}^{\boxtimes n} \otimes \Omega^*_{\text{log}}\right)$$
where $\overline{C}_n(X)$ is the Fulton-MacPherson compactification and $\Omega^*_{\text{log}}$ denotes logarithmic differential forms with poles along boundary divisors. The differential 
$$d = d_{\text{internal}} + d_{\text{residue}} + d_{\text{de Rham}}$$
combines internal operations from $\mathcal{A}$ with residues along collision divisors and the de Rham differential. Concretely, for elements $a_1 \otimes \cdots \otimes a_n \otimes \omega \in \bar{B}^{\text{geom}}(\mathcal{A})_n$:
$$d_{\text{residue}}(a_1 \otimes \cdots \otimes a_n \otimes \omega) = \sum_{i<j} \text{Res}_{D_{ij}}[\omega] \cdot (a_1 \otimes \cdots \otimes \mu(a_i, a_j) \otimes \cdots)$$
The condition $d^2 = 0$ follows from the Arnold-Orlik-Solomon relations among logarithmic forms.
\end{theorem}

\begin{remark}[Conceptual Foundation: NAP Duality]
\label{rem:NAP-foundation}
These constructions are not ad hoc. They arise inevitably from non-abelian Poincaré (NAP) duality, which we develop systematically in Part~II (Chapters on NAP derivation and computations).

The key principle: For a chiral algebra $\mathcal{A}$ viewed as a factorization algebra on $X$, factorization homology satisfies:
$$\int_X \mathcal{A} \simeq \mathbb{D}\left(\int_{-X} \mathcal{A}^!\right)$$

where:
\begin{itemize}
\item $\int_X$ denotes factorization homology (computed by configuration space integrals)
\item $\mathbb{D}$ is Verdier duality (exchanging logarithmic forms and distributions)
\item $-X$ denotes $X$ with opposite orientation
\item $\mathcal{A}^!$ is the \textbf{Koszul dual chiral coalgebra}, defined intrinsically via this duality
\end{itemize}

This framework provides:
\begin{enumerate}
\item An independent construction of $\mathcal{A}^!$ without circularity
\item A geometric proof that $\bar{B}^{\text{ch}}(\mathcal{A}) \simeq \mathcal{A}^!$
\item Systematic computation of Koszul duals for non-quadratic algebras
\item Natural extension to higher genus via modular forms
\end{enumerate}

The bar and cobar complexes are computational manifestations of Verdier duality on configuration spaces---geometry computing algebra.
\end{remark}

We follow with the dual construction—the geometric cobar complex. This construction is equally elementary: we work with distributions (integration kernels) on open configuration spaces:

\begin{theorem}[Geometric Cobar Construction, Theorem 3.5]
For a chiral coalgebra $\mathcal{C}$ on a smooth curve $X$, we construct a geometric cobar complex at the cochain level:
$$\Omega^{\text{geom}}(\mathcal{C})_n = \text{Dist}(C_n(X), \mathcal{C}^{\boxtimes n})$$
consisting of distributional sections (integration kernels) on open configuration spaces with prescribed singularities along diagonals. Concretely, elements are expressions like:
$$K(z_1, \ldots, z_n) = \sum_{\text{poles}} \frac{c_{i_1 \cdots i_k}}{(z_{i_1} - z_{i_2})^{h_1} \cdots (z_{i_{k-1}} - z_{i_k})^{h_{k-1}}}$$
The cobar differential
$$d_{\text{cobar}}(K) = \sum_{i<j} \Delta_{ij}(K) \cdot \delta(z_i - z_j)$$
inserts Dirac distributions that "pull apart" colliding points, implementing the coproduct $\Delta: \mathcal{C} \to \mathcal{C} \otimes \mathcal{C}$.
\end{theorem}

We proced to extend the construction across all genera, incorporating quantum corrections that appear as loop integrals in physics:

\begin{theorem}[Full Genus Bar Complex, Theorem 5.1]
The geometric bar complex extends to all genera $g \geq 0$ as
$$\bar{B}^{\text{full}}(\mathcal{A}) = \bigoplus_{g \geq 0} \lambda^{2g-2} \bar{B}^g(\mathcal{A})$$
where each $\bar{B}^g(\mathcal{A})$ incorporates genus-specific geometry:
\begin{itemize}
\item \textbf{Genus 0}: Logarithmic forms $\eta_{ij} = d\log(z_i - z_j)$ on $\mathbb{P}^1$
\item \textbf{Genus 1}: Elliptic forms on torus $E_\tau = \mathbb{C}/(\mathbb{Z} + \tau\mathbb{Z})$:
  $$\eta_{ij}^{(1)} = d\log\vartheta_1\left(\frac{z_i - z_j}{2\pi i}|\tau\right) + \frac{(z_i - z_j)d\tau}{2\pi i \text{Im}(\tau)}$$
  where $\vartheta_1(z|\tau) = -i\sum_{n \in \mathbb{Z}}(-1)^n q^{(n-1/2)^2}e^{i(2n-1)z}$ with $q = e^{i\pi\tau}$
\item \textbf{Genus $g \geq 2$}: Prime forms and period integrals on hyperbolic surfaces:
  $$\eta_{ij}^{(g)} = d\log E(z_i, z_j) + \sum_{\alpha=1}^g \left(\oint_{A_\alpha} \omega_i\right) \left(\oint_{B_\alpha} \omega_j\right)$$
  where $E(z,w)$ is the prime form and $\{A_\alpha, B_\alpha\}$ are canonical homology cycles
\end{itemize}
The master differential $d^{\text{full}} = \sum_{g} \lambda^{2g-2} d^g$ satisfies $(d^{\text{full}})^2 = 0$, encoding quantum associativity to all loop orders.
\end{theorem}

\section{Main Results - Complete Statements with Proof Locations}
\label{sec:main-results-complete}

\begin{maintheorem}[Geometric Bar-Cobar Duality - Complete Statement]\label{mainthm:bar-cobar-complete}
For a chiral algebra $\mathcal{A}$ on a smooth projective curve $X$, there exists a 
canonical Koszul dual chiral coalgebra $\mathcal{A}^!$ such that:

\begin{enumerate}
\item \textbf{(Functoriality)} The assignment $\mathcal{A} \mapsto \bar{B}_{\text{geom}}(\mathcal{A})$ 
defines a functor:
$$\bar{B}_{\text{geom}}: \text{ChirAlg}(X) \to \text{dgCoalg}(X)$$
\textit{Proven in: Corollary \ref{cor:bar-functorial} (Section 3.2)}

\item \textbf{(Quasi-isomorphism)} The natural maps:
$$\bar{B}_{\text{geom}}(\mathcal{A}) \xrightarrow{\simeq} \mathcal{A}^!$$
$$\Omega_{\text{geom}}(\mathcal{A}^!) \xrightarrow{\simeq} \mathcal{A}$$
are quasi-isomorphisms of chain complexes.
\textit{Proven in: Corollary \ref{cor:bar-cobar-inverse} (Section 3.8)}

\item \textbf{(Adjunction)} The bar and cobar constructions form an adjoint pair:
$$\text{Hom}_{\text{dgCoalg}}(\bar{B}(\mathcal{A}), \mathcal{C}) 
\simeq \text{Hom}_{\text{ChirAlg}}(\mathcal{A}, \Omega(\mathcal{C}))$$
\textit{Follows from: Theorem \ref{thm:bar-cobar-verdier} (Section 3.8)}

\item \textbf{(Higher Genus Extension)} For each genus $g \geq 0$:
$$\bar{B}(\mathcal{A}) = \bigoplus_{g=0}^{\infty} \hbar^{2g-2} \bar{B}_g(\mathcal{A})$$
where $\bar{B}_g$ computes cohomology over $\mathcal{M}_g$ with quantum corrections.
\textit{Proven in: Theorem \ref{thm:CC-acyclicity-higher-genus} (Section 4.10)}

\item \textbf{(BD Compatibility)} For genus 0, this reduces to Beilinson-Drinfeld.
\textit{Verified in: Remark \ref{rem:BD-reference-guide} (Section 3.1)}
\end{enumerate}
\end{maintheorem}

\begin{proof}[Proof Outline and Cross-References]
The complete proof is distributed across the manuscript:

\textbf{Part (1) - Functoriality:}
Corollary \ref{cor:bar-functorial} (Section 3.2)

\textbf{Part (2) - Quasi-isomorphism:}
Theorem \ref{thm:CC-acyclicity-higher-genus} (Section 4.10) + Corollary \ref{cor:bar-cobar-inverse} (Section 3.8)

\textbf{Part (3) - Adjunction:}
Theorem \ref{thm:bar-cobar-verdier} (Section 3.8) + Theorem \ref{thm:verdier-AF-compat} (Section 4.11)

\textbf{Part (4) - Higher Genus:}
Theorem \ref{thm:deformation-obstruction} (Section 3.10) + Lemma \ref{lem:quantum-preserves-acyclicity} (Section 4.10)

\textbf{Part (5) - BD Compatibility:}
Remark \ref{rem:BD-reference-guide} (Section 3.1)
\end{proof}

\begin{maintheorem}[Curved Koszul Duality - Complete Statement]\label{mainthm:curved-complete}
For chiral algebras with central extensions (curved $A_\infty$ structures):

\begin{enumerate}
\item \textbf{(Obstruction Theory)} The failure of $d^2 = 0$ is measured by:
$$Q_g(\mathcal{A}) \subset H^2(\bar{B}_g(\mathcal{A}), Z(\mathcal{A}))$$
where $Z(\mathcal{A})$ is the center.
\textit{Proven in: Lemma \ref{lem:obstruction-class} (Section 3.10)}

\item \textbf{(Deformation-Obstruction Duality)} Perfect pairing:
$$Q_g(\mathcal{A}) \oplus Q_g(\mathcal{A}^!) \simeq H^*(\mathcal{M}_g, Z(\mathcal{A}))$$
\textit{Proven in: Theorem \ref{thm:deformation-obstruction} (Section 3.10)}

\item \textbf{(Completion)} For non-quadratic algebras:
$$\widehat{\mathcal{A}}^! = \varprojlim_n \mathcal{A}^! / I^n$$
\textit{Proven in: Theorem \ref{thm:nilpotent-main} (Appendix B)}

\item \textbf{(Examples)} Heisenberg, Kac-Moody, W-algebras, Virasoro.
\end{enumerate}
\end{maintheorem}

\begin{proof}[Proof Outline]
See Theorem \ref{thm:deformation-obstruction} (Section 3.10) for complete proof. 
Key steps:

1. Curved $A_\infty$ relations ensure $\mu_0 \in Z(\mathcal{A})$
2. Obstructions are classes in $H^2(\bar{B}, Z(\mathcal{A}))$
3. Deformations parametrized by $\text{Ext}^1(\mathcal{A}^!, \mathcal{A}^!)$
4. Serre duality on $\mathcal{M}_g$ gives perfect pairing
5. Completion ensures convergence for non-quadratic cases
\end{proof}

% ================================================================
% PATCH 013 ADDITION TO INTRO: ON-NOSE NILPOTENCE
% ================================================================

\subsection{On-Nose Nilpotence: $d^2 = 0$ Strictly}

A key technical achievement is proving that the bar differential satisfies $d^2 = 0$ 
\textbf{on the nose}, not just up to homotopy. This requires:
\begin{itemize}
\item Central curvature: $\mu_0 \in Z(\mathcal{A})$
\item Arnold relations for residue terms
\item Leibniz compatibility
\item Closedness of quantum correction forms $\omega_g$
\end{itemize}

This on-nose nilpotence allows direct computation of Koszul duals without $\infty$-categorical 
machinery. See \S\ref{sec:on-nose-vs-homotopy} for complete details and verification through genus 5.

\textbf{Main Result (Theorem \ref{thm:central-implies-strict}):} For all chiral algebras with 
central curvature:
$$d_{\text{bar}}^2 = 0 \quad \text{STRICTLY, NOT JUST UP TO HOMOTOPY}$$

This applies to all vertex algebras from conformal field theory, including:
\begin{itemize}
\item Heisenberg $\mathcal{H}_k$ at level $k$
\item Affine Kac-Moody $\widehat{\mathfrak{g}}_k$ at level $k$  
\item Virasoro $\text{Vir}_c$ with central charge $c$
\item $W$-algebras $W_N$ for all $N \geq 3$
\end{itemize}

\begin{maintheorem}[Non-Abelian Poincaré Duality - Complete Statement]\label{mainthm:NAP-complete}
Bar-cobar duality is mediated by Verdier duality on configuration spaces:

\begin{enumerate}
\item \textbf{(Factorization Homology)} Bar computes:
$$\bar{B}(\mathcal{A}) \simeq \int_X \mathcal{A}$$
(Ayala-Francis factorization homology).
\textit{Proven in: Lemma \ref{lem:bar-as-fact-hom-AF} (Section 4.11)}

\item \textbf{(Verdier Dual)} Cobar is:
$$\Omega(\mathcal{C}) \simeq \mathbb{D}\left(\int_{-X} \mathcal{C}\right)$$
where $\mathbb{D}$ is Verdier duality.
\textit{Proven in: Theorem \ref{thm:bar-cobar-verdier} (Section 3.8)}

\item \textbf{(Compatibility)} Geometric duality (Verdier) = topological duality (AF).
\textit{Proven in: Theorem \ref{thm:verdier-AF-compat} (Section 4.11)}
\end{enumerate}
\end{maintheorem}

\begin{proof}[Proof Outline]
See Theorem \ref{thm:verdier-AF-compat} (Section 4.11) for complete proof.
\end{proof}

\subsection{Corollaries and Applications}

\begin{corollary}[Explicit Koszul Pairs]\label{cor:explicit-pairs-intro}
The following are Koszul dual pairs:
\begin{enumerate}
\item Free fermion $\leftrightarrow$ $\beta\gamma$ system
\item Heisenberg $\mathcal{H}_k$ $\leftrightarrow$ Symmetric algebra (curved)
\item Affine Lie $\widehat{\mathfrak{g}}_k$ $\leftrightarrow$ $\widehat{\mathfrak{g}}_{-k-h^{\vee}}$ 
(level-rank duality)
\item W-algebra $\mathcal{W}_N^{-N}$ $\leftrightarrow$ Wakimoto realization
\end{enumerate}
\end{corollary}

\begin{corollary}[Hochschild Cohomology Computation]\label{cor:hochschild-computation-intro}
For a Koszul pair $(\mathcal{A}, \mathcal{A}^!)$:
$$HH^*(\mathcal{A}) \simeq H^*(\bar{B}(\mathcal{A}), \mathcal{A}) 
\simeq H^*(\mathcal{A}^! \otimes \mathcal{A})$$

This gives explicit formulas via configuration space integrals.
\end{corollary}

\section{The Arnold Relations: Foundation of Consistency}

\subsection{Discovery and Significance}

This principle, discovered by V.I. Arnold in studying braid groups, is the cornerstone ensuring $d^2 = 0$ for the bar differential. We provide complete proofs in multiple ways—combinatorial, topological, and operadic—establishing this fundamental identity rigorously. Each approach illuminates different aspects of the underlying geometry.

The Arnold relations state that certain combinations of logarithmic forms vanish identically:

\begin{theorem}[Arnold-Orlik-Solomon Relations - Fundamental]
For logarithmic forms $\eta_{ij} = d\log(z_i - z_j)$ on configuration space, and any subset $S \subset \{1, \ldots, n\}$ with distinct $i, j \notin S$:
$$\sum_{k \in S} (-1)^{|k|} \eta_{ik} \wedge \eta_{kj} \wedge \bigwedge_{l \in S\setminus\{k\}} \eta_{kl} = 0$$
where $|k|$ denotes the position of $k$ in the ordering of $S$.
\end{theorem}

\subsection{Why These Relations Matter}

The Arnold relations are not merely a technical tool—they encode the fundamental consistency of local operator algebras in quantum field theory:

\begin{enumerate}
\item \textbf{Algebraic Consistency}: They ensure the Jacobi identity for the chiral algebra
\item \textbf{Geometric Consistency}: They guarantee that residue extraction is well-defined independent of the order of operations
\item \textbf{Homological Consistency}: They are precisely the condition for $d^2 = 0$ in the bar complex
\item \textbf{Physical Consistency}: They encode the associativity of the operator product expansion
\end{enumerate}

\subsection{Three Perspectives on the Proof}

We establish these relations through three independent proofs, each revealing different aspects:

\textbf{1. Combinatorial Proof (Following Arnold)}:
The relations follow from the elementary identity
$$z_i - z_j = (z_i - z_k) + (z_k - z_j)$$
by taking logarithmic derivatives and carefully tracking the resulting terms. This proof is constructive and yields explicit formulas.

\textbf{2. Topological Proof (Via Stokes' Theorem)}:
Consider the map $S^1 \times C_{|S|}(X) \to C_{|S|+2}(X)$ given by placing points $i$ and $j$ on a small circle. Applying Stokes' theorem to appropriate forms on this space yields the Arnold relations as boundary contributions.

\textbf{3. Operadic Proof (Higher Structure)}:
The configuration space naturally forms an operad with composition given by inserting configurations. The condition that this operad is a complex (has differential squaring to zero) is precisely the Arnold relations.

Complete detailed proofs are provided in Appendix A, with computational examples for small values of $|S|$.

\section{Chiral Hochschild Cohomology and Deformation Theory}

\subsection{From Classical to Chiral}

In classical algebra, Hochschild cohomology controls deformations. For chiral algebras, we have an enriched theory:

\begin{definition}[Chiral Hochschild Complex]
For a chiral algebra $\mathcal{A}$ on a smooth curve $X$, the chiral Hochschild complex is:
$$CH^*(\mathcal{A}) = \text{RHom}_{\mathcal{D}_X}(\bar{B}^{\text{geom}}(\mathcal{A}), \mathcal{A})$$
with differential combining chiral operations and the de Rham differential.
\end{definition}

The geometric realization through our bar construction gives:
$$CH^n(\mathcal{A}) \cong H^n\left(\bar{B}^{\text{geom}}(\mathcal{A}) \otimes_{\mathcal{A}} \mathcal{A}\right)$$

\begin{theorem}[Deformation-Obstruction Theory]
The chiral Hochschild cohomology controls:
\begin{enumerate}
\item $CH^0(\mathcal{A})$ = center of $\mathcal{A}$ (conserved charges in physics)
\item $CH^1(\mathcal{A})$ = infinitesimal deformations (symmetry generators)
\item $CH^2(\mathcal{A})$ = obstructions to extending deformations (marginal operators)
\item $CH^3(\mathcal{A})$ = obstructions to associativity of deformed product
\end{enumerate}
\end{theorem}

\subsection{Periodicity Phenomena}

A remarkable feature of chiral algebras is the appearance of periodicity:

\begin{theorem}[Periodicity in Cohomology]
For certain chiral algebras, the Hochschild cohomology exhibits periodicity:
\begin{enumerate}
\item \textbf{Virasoro}: $CH^{n+2}(\text{Vir}_c) \cong CH^n(\text{Vir}_c) \otimes H^2(\mathcal{M}_{g,n})$ 
\item \textbf{Affine Kac-Moody}: $CH^{n+2h^\vee}(\widehat{\mathfrak{g}}_k) \cong CH^n(\widehat{\mathfrak{g}}_k)$ at critical level
\item \textbf{W-algebras}: Period determined by the principal grading
\end{enumerate}
\end{theorem}

This periodicity reflects deep structure—the cohomology classes correspond to modular forms of specific weights, with periodicity arising from representation theory of $\text{SL}_2(\mathbb{Z})$.

\begin{remark}[On Bar, Cobar, and Koszul Duality]\label{rem:bar-cobar-koszul-distinction}
\textbf{A Critical Distinction:} Throughout this manuscript, we work with three related but distinct concepts that must not be conflated:

\begin{center}
\begin{tabular}{|l|p{10cm}|}
\hline
\textbf{Concept} & \textbf{Precise Meaning} \\
\hline
\textbf{Bar Construction} & \textbf{Functor} $\bar{B}: \text{ChirAlg} \to \text{ChirCoalg}$ \\
 & Maps: Algebra $\mathcal{A} \mapsto$ Coalgebra $\bar{B}(\mathcal{A})$ \\
 & One-way arrow: $\mathcal{A} \to \bar{B}(\mathcal{A})$ \\
\hline
\textbf{Cobar Construction} & \textbf{Functor} $\Omega: \text{ChirCoalg} \to \text{ChirAlg}$ \\
 & Maps: Coalgebra $\mathcal{C} \mapsto$ Algebra $\Omega(\mathcal{C})$ \\
 & One-way arrow: $\mathcal{C} \to \Omega(\mathcal{C})$ \\
\hline
\textbf{Koszul Duality} & \textbf{Equivalence of derived categories} \\
 & Relates: Two algebras $(\mathcal{A}_1, \mathcal{A}_2)$ \\
 & Two-way correspondence: $\mathcal{A}_1 \xleftrightarrow{\text{Koszul}} \mathcal{A}_2$ \\
\hline
\end{tabular}
\end{center}

\textbf{The Relationship:}

For a Koszul dual pair $(\mathcal{A}_1, \mathcal{A}_2)$, the bar and cobar constructions \textit{witness} the duality:
\begin{align*}
\bar{B}(\mathcal{A}_1) &\simeq \mathcal{A}_2^! \quad \text{(bar of first ≃ dual coalgebra of second)} \\
\Omega(\mathcal{A}_2^!) &\simeq \mathcal{A}_1 \quad \text{(cobar reconstructs first from dual of second)}
\end{align*}

But crucially:
\begin{itemize}
\item $\bar{B}$ and $\Omega$ are \textbf{constructions} (functors)
\item Koszul duality is a \textbf{property} (an equivalence relation)
\item NOT every algebra admits a Koszul dual!
\item When it exists, Koszul duality is witnessed by, but not identical to, bar-cobar
\end{itemize}
\end{remark}

\begin{remark}[Common Sources of Confusion]\label{rem:bar-cobar-confusion}
\textbf{Pitfall 1: ``Self-Koszul Duality''}

Saying ``$\mathcal{A}$ is self-Koszul dual'' can mean:
\begin{enumerate}
\item[a)] $\Omega(\bar{B}(\mathcal{A})) \simeq \mathcal{A}$ (bar-cobar inversion - TRUE for any algebra)
\item[b)] $\mathcal{A}^! \simeq \mathcal{A}$ (true Koszul self-duality - RARE!)
\end{enumerate}

Statement (a) is automatic from general theory - it does NOT imply (b)!

\textbf{Example:} Heisenberg $\mathcal{H}_k$ satisfies (a) but NOT (b). Instead: $\mathcal{H}_k^! \simeq \mathcal{H}_{-k}$ (level-shifted).

\textbf{Pitfall 2: ``Bar-Cobar Duality''}

This phrase is ambiguous:
\begin{itemize}
\item As functors: $\bar{B}$ and $\Omega$ are adjoint (standard fact)
\item As duality: Only for Koszul pairs does $\Omega \circ \bar{B} \simeq \text{id}$ (special property)
\end{itemize}

\textbf{Pitfall 3: Direction Matters}

For Koszul pairs:
\begin{itemize}
\item $\bar{B}(\mathcal{A}_1) \to \mathcal{A}_2^!$: algebra$_1$ $\to$ coalgebra (bar direction)
\item $\mathcal{A}_1 \leftarrow \Omega(\mathcal{A}_2^!)$: algebra$_1$ $\leftarrow$ coalgebra (cobar direction)
\item $\mathcal{A}_1 \xleftrightarrow{\text{Koszul}} \mathcal{A}_2$: algebra$_1$ $\leftrightarrow$ algebra$_2$ (duality)
\end{itemize}

These are THREE DIFFERENT ARROWS with different meanings!
\end{remark}

\subsection{The Non-Abelian Poincaré Perspective}

\begin{remark}[NAP View of Bar-Cobar]\label{rem:NAP-bar-cobar}
From the non-abelian Poincaré duality perspective, bar and cobar constructions are manifestations of orientation reversal on curves:

\textbf{Bar Construction:}
$$\bar{B}^{\text{ch}}(\mathcal{A}): X \mapsto \int_X \mathcal{A}$$
computes factorization homology in the standard orientation.

\textbf{Cobar Construction:}
$$\Omega^{\text{ch}}(\mathcal{C}): X \mapsto \int_{-X} \mathcal{C}$$
computes factorization homology in the opposite orientation.

\textbf{Koszul Duality:}
The relationship $\mathcal{A}_1 \xleftrightarrow{\text{Koszul}} \mathcal{A}_2$ means:
$$\int_X \mathcal{A}_1 \simeq \mathbb{D}\left(\int_{-X} \mathcal{A}_2\right)$$

Orientation reversal is the geometric manifestation of Koszul duality!
\end{remark}

\begin{example}[Heisenberg Through NAP Lens]\label{ex:heisenberg-NAP}
For the Heisenberg algebra $\mathcal{H}_k$:

\textbf{Standard Orientation ($X$):}
$$\int_X \mathcal{H}_k = \text{Partition function at level } k$$

\textbf{Opposite Orientation ($-X$):}
$$\int_{-X} \mathcal{H}_k = \text{Partition function at level } -k$$

The level shift k → -k is precisely the effect of orientation reversal!

At genus 1 (torus), this manifests as the modular transformation τ → -1/τ exchanging A-cycles and B-cycles. This is quantum complementarity, viewed through the NAP lens.
\end{example}

\begin{insight}[Grothendieck's Functorial View]\label{insight:grothendieck-NAP}
From Grothendieck's perspective, non-abelian Poincaré duality is the ultimate expression of functoriality:

\begin{center}
\begin{tikzcd}
\text{Oriented manifolds} \arrow[r, "\int"] \arrow[d, "\text{reverse}"'] & \text{Spectra} \arrow[d, "\mathbb{D}"] \\
\text{Opposite orientation} \arrow[r, "\int"] & \text{Dual spectra}
\end{tikzcd}
\end{center}

The entire structure is determined by functoriality and the duality functor $\mathbb{D}$.
\end{insight}

\begin{theorem}[Geometric Bar-Cobar Duality]\label{thm:geometric-bar-cobar}
For a chiral Koszul pair $(\mathcal{A}_1, \mathcal{A}_2)$ on a smooth curve $X$, our geometric constructions establish the duality:
\begin{enumerate}
\item \textbf{Bar construction witness:}
$$\bar{B}^{\text{geom}}(\mathcal{A}_1) \simeq \mathcal{A}_2^! \quad \text{as chiral coalgebras}$$

\item \textbf{Cobar reconstruction witness:}
$$\Omega^{\text{geom}}(\mathcal{A}_2^!) \simeq \mathcal{A}_1 \quad \text{as chiral algebras}$$

\item \textbf{Geometric realization:} The equivalence is realized by Verdier duality:
$$\mathbb{D}_{\overline{C}_*(X)}: \Omega^*_{\log}(\overline{C}_*(X)) \xrightarrow{\sim} \Omega^{d-*}_{\text{dist}}(C_*(X))$$
exchanging logarithmic forms (bar) with distributions (cobar).
\end{enumerate}

\textbf{Non-Abelian Poincaré Interpretation:}
This theorem realizes non-abelian Poincaré duality for the curve X with coefficients in the factorization algebra $\mathcal{A}_1$. The bar construction computes factorization homology; Verdier duality implements the NAP isomorphism; cobar reconstruction completes the circle.
\end{theorem}

\section{Criteria for Existence of Koszul Duals}

Not every chiral algebra admits a Koszul dual. We establish precise criteria:

\begin{theorem}[Existence Criterion for Koszul Duality]
A chiral algebra $\mathcal{A}$ admits a Koszul dual if and only if:
\begin{enumerate}
\item \textbf{Finite generation}: $\mathcal{A}$ is finitely generated as a $\mathcal{D}_X$-module
\item \textbf{Formal smoothness}: $\dim CH^n(\mathcal{A}) < \infty$ for each $n$
\item \textbf{Poincaré duality}: There exists a non-degenerate pairing
   $$CH^i(\mathcal{A}) \times CH^{d-i}(\mathcal{A}) \to \omega_X$$
   for some dimension $d$
\item \textbf{Convergence}: The bar spectral sequence 
   $$E_1^{p,q} = H^q(C_{p+1}(X), \mathcal{A}^{\boxtimes(p+1)}) \Rightarrow H^{p+q}(\bar{B}(\mathcal{A}))$$
   converges
\end{enumerate}
\end{theorem}

For W-algebras, additional structure emerges from quantum Drinfeld-Sokolov reduction:

\begin{theorem}[W-algebra Koszul Duality]
At critical level $k = -h^\vee$:
$$\mathcal{W}^{-h^\vee}(\mathfrak{g}, f) \text{ is Koszul dual to } \mathcal{W}^{-h^\vee}(\mathfrak{g}^\vee, f^\vee)$$
where $\mathfrak{g}^\vee$ is the Langlands dual Lie algebra and $f^\vee$ is the dual nilpotent element.
\end{theorem}

\subsection{The Fundamental Bar-Cobar Relationship}

The central result of this monograph is making precise the relationship between chiral algebras in a Koszul pair. We establish not merely that they are "dual" in some abstract sense, but rather that their bar and cobar constructions provide explicit, mutually inverse transformations.

\begin{theorem}[Extended Koszul Duality, Theorem 4.3]
For a chiral Koszul pair $(\mathcal{A}_1, \mathcal{A}_2)$ of chiral algebras, we establish:

\medskip
\noindent\textbf{I. The Bar-Cobar Isomorphism:}
\begin{enumerate}
\item \textbf{Bar transforms algebra to dual coalgebra:}
   $$\bar{B}^{\text{ch}}(\mathcal{A}_1) \simeq \mathcal{A}_2^! \quad \text{and} \quad \bar{B}^{\text{ch}}(\mathcal{A}_2) \simeq \mathcal{A}_1^!$$
   as quasi-isomorphisms of chiral coalgebras.

\item \textbf{Cobar reconstructs the dual algebra:}
   $$\Omega^{\text{ch}}(\mathcal{A}_2^!) \simeq \mathcal{A}_1 \quad \text{and} \quad \Omega^{\text{ch}}(\mathcal{A}_1^!) \simeq \mathcal{A}_2$$
   as quasi-isomorphisms of chiral algebras.

\item \textbf{Composition gives quasi-isomorphisms to identity:}
   $$\Omega^{\text{ch}}(\bar{B}^{\text{ch}}(\mathcal{A}_i)) \xrightarrow{\sim} \mathcal{A}_i, \quad \bar{B}^{\text{ch}}(\Omega^{\text{ch}}(\mathcal{A}_i^!)) \xrightarrow{\sim} \mathcal{A}_i^!$$
   for $i = 1, 2$, establishing that bar and cobar are quasi-inverse equivalences.
\end{enumerate}

\medskip
\noindent\textbf{II. How Structures Correspond:}
\begin{enumerate}
\item \textbf{Generators and relations interchange:}
   \begin{itemize}
   \item Generating fields of $\mathcal{A}_1$ correspond to relations of $\mathcal{A}_2$
   \item Relations of $\mathcal{A}_1$ correspond to generating fields of $\mathcal{A}_2$
   \item This explains the slogan: "strong coupling $\leftrightarrow$ weak coupling"
   \end{itemize}

\item \textbf{Algebraic operations correspond to coalgebraic operations:}
   \begin{itemize}
   \item Chiral product $\mu: \mathcal{A}_1 \otimes \mathcal{A}_1 \to \mathcal{A}_1$ corresponds to coproduct $\Delta: \mathcal{A}_2^! \to \mathcal{A}_2^! \otimes \mathcal{A}_2^!$
   \item Higher multiplications $m_n$ correspond to higher comultiplications $\Delta_n$
   \item Associativity of products becomes coassociativity of coproducts
   \end{itemize}

\item \textbf{OPE pole orders encode coproduct terms:}
   \begin{itemize}
   \item An OPE singularity $\phi_1(z)\phi_2(w) \sim \frac{a}{(z-w)^k}$ in $\mathcal{A}_1$ becomes a coproduct term in $\mathcal{A}_2^!$
   \item The residue map $\text{Res}_{z=w}$ extracts coproduct coefficients from OPE data
   \item Distribution-valued correlators in $\mathcal{A}_2$ reconstruct OPE structure of $\mathcal{A}_1$
   \end{itemize}
\end{enumerate}

\medskip
\noindent\textbf{III. Geometric Realization:}

The abstract isomorphisms are realized geometrically through configuration space integration:

\begin{enumerate}
\item \textbf{Perfect pairing via integration:}
   $$\langle \omega_{\text{bar}}, K_{\text{cobar}} \rangle = \int_{\overline{C}_n(X)} \omega_{\text{bar}} \wedge \iota^* K_{\text{cobar}}$$
   where $\omega_{\text{bar}} \in \bar{B}^{\text{ch}}(\mathcal{A}_1)$ is a logarithmic form, $K_{\text{cobar}} \in \Omega^{\text{ch}}(\mathcal{A}_2^!)$ is a distribution-valued kernel, and $\iota: C_n(X) \hookrightarrow \overline{C}_n(X)$ is the inclusion of open into compactified configuration space.

\item \textbf{Residues extract coalgebra structure:}
   The differential on the bar side:
   $$d_{\text{bar}} = \sum_{D \in \text{Bdry}} (-1)^{|D|} \text{Res}_D$$
   computes coproduct operations by extracting residues at collision divisors.

\item \textbf{Distributions reconstruct algebra structure:}
   The differential on the cobar side:
   $$d_{\text{cobar}} = \sum_{i<j} \Delta_{ij} \cdot \delta(z_i - z_j)$$
   reconstructs products by inserting distributional singularities.
\end{enumerate}

\medskip
\noindent\textbf{IV. Extensions:}
\begin{enumerate}
\item \textbf{Curved algebras:} The duality extends to curved $A_\infty$ structures with curvature $\kappa \in \mathcal{A}^{\otimes 2}[2]$ satisfying the Maurer-Cartan equation

\item \textbf{Filtered structures:} Koszul pairs of filtered chiral algebras satisfy graded duality at each filtration level

\item \textbf{Higher genus corrections:} At genus $g \geq 1$, quantum corrections enter through period integrals, with complementary deformation-obstruction spaces
\end{enumerate}
\end{theorem}

\begin{remark}[Slogan]\label{rem:koszul-slogan-corrected}
In maximal generality, the content of chiral Koszul duality is:
\begin{center}
\textit{The bar construction transforms $\mathcal{A}_1$ into its Koszul dual coalgebra $\mathcal{A}_2^! \simeq \bar{B}^{\text{ch}}(\mathcal{A}_1)$, from which the cobar construction reconstructs $\mathcal{A}_1$. Symmetrically, $\bar{B}^{\text{ch}}(\mathcal{A}_2) \simeq \mathcal{A}_1^!$ and $\Omega^{\text{ch}}(\mathcal{A}_1^!) \simeq \mathcal{A}_2$. Each algebra encodes complete information about its dual partner via coalgebra structures.}
\end{center}

More precisely, for a Koszul pair $(\mathcal{A}_1, \mathcal{A}_2)$:
\begin{itemize}
\item Bar transforms algebra structures to coalgebra structures: $\bar{B}^{\text{ch}}(\mathcal{A}_i)$ has coproduct operations encoding products of $\mathcal{A}_i$
\item Cobar inverts this: $\Omega^{\text{ch}}(\bar{B}^{\text{ch}}(\mathcal{A}_i)) \simeq \mathcal{A}_i$ quasi-isomorphically
\item The duality interchanges the two algebras: $\bar{B}^{\text{ch}}(\mathcal{A}_1) \simeq \mathcal{A}_2^!$ and cobar rebuilds the \emph{partner} $\Omega^{\text{ch}}(\mathcal{A}_2^!) \simeq \mathcal{A}_2$
\end{itemize}

This provides a computational bridge where bar and cobar establish mutually quasi-inverse equivalences, with algebraic operations on one side corresponding to coalgebraic operations defining its Koszul dual partner.
\end{remark}

\begin{remark}[Two Distinct Phenomena]
It is critical to distinguish:

\textbf{Phenomenon 1: Bar-Cobar Inversion} (same algebra)
$$\text{For any algebra } \mathcal{A}: \Omega^{\text{ch}}(\bar{B}^{\text{ch}}(\mathcal{A})) \xrightarrow{\sim} \mathcal{A}$$

\textbf{Phenomenon 2: Koszul Duality} (different algebras)
$$\text{For Koszul pair } (\mathcal{A}_1, \mathcal{A}_2): \bar{B}^{\text{ch}}(\mathcal{A}_1) \simeq \mathcal{A}_2^! \text{ and } \Omega^{\text{ch}}(\mathcal{A}_2^!) \simeq \mathcal{A}_2$$

The bar-cobar composition $\Omega \circ \bar{B}$ always reconstructs the SAME algebra (up to quasi-isomorphism), while Koszul duality relates DIFFERENT algebras through their coalgebra incarnations.
\end{remark}

\section{Heisenberg Koszul Duality}
\label{sec:heisenberg-correction-intro}


\subsection{The Koszul Dual of Heisenberg}

\begin{theorem}[Heisenberg Koszul Dual - Corrected]
\label{thm:heisenberg-koszul-correct-intro}
Let $\mathcal{H}_k$ be the Heisenberg chiral algebra at level $k$. Then:
\begin{equation}
\bar{B}(\mathcal{H}_k) \simeq \text{Sym}(V)^!
\end{equation}
where $\text{Sym}(V)$ is the symmetric algebra (commutative) on the underlying vector space $V$ of currents.

Equivalently, via cobar reconstruction:
\begin{equation}
\Omega(\bar{B}(\mathcal{H}_k)) \simeq \mathcal{H}_k
\end{equation}
demonstrating that bar-cobar provides a resolution, but:
\begin{equation}
\mathcal{H}_k^! \simeq \text{Sym}(V) \quad \text{(commutative algebra)}
\end{equation}
\end{theorem}

\subsection{Why Confusion May Arise}

There are \emph{three different} mathematical phenomena involving Heisenberg algebras that can be conflated:

\begin{enumerate}
\item \textbf{Bar-Cobar Koszul Duality}: $\mathcal{H}_k^! \simeq \text{Sym}(V)$ (this paper's focus)
\item \textbf{Representation-Theoretic Duality}: Langlands/level-rank duality $k \leftrightarrow -k-2h^\vee$ (for Heisenberg, $h^\vee = 0$, so $k \leftrightarrow -k$)
\item \textbf{Boson-Fermion Correspondence}: $\mathcal{H}_k \simeq \text{Fermion}^{\otimes 2}$ (equivalence of categories of representations)
\end{enumerate}

These are \textbf{different structures} on the same algebra:
\begin{itemize}
\item (1) relates the algebra to its coalgebra Koszul dual
\item (2) relates representation categories
\item (3) relates different algebraic structures with equivalent module categories
\end{itemize}


\begin{remark}[Critical Distinction: Bar-Cobar Inversion vs Koszul Duality]\label{rem:inversion-vs-duality-distinguished}
The manuscript establishes two logically distinct phenomena that must not be confused. Understanding this distinction is essential for correctly interpreting all subsequent results.

\medskip
\noindent\textbf{PHENOMENON 1: Bar-Cobar Inversion (Universal Property)}

For \emph{any} chiral algebra $\mathcal{A}$, the bar and cobar constructions are quasi-inverse:
$$\Omega^{\text{ch}}(\bar{B}^{\text{ch}}(\mathcal{A})) \xrightarrow{\sim} \mathcal{A}$$

\textbf{Status:} This is a \textsc{tautology}. It holds by the definition of bar-cobar adjunction. The cobar functor is constructed precisely to be the left adjoint to the bar functor, ensuring this quasi-isomorphism.

\textbf{Interpretation:} ``If you apply the bar construction (algebra $\to$ coalgebra), then apply cobar (coalgebra $\to$ algebra), you return to where you started (up to quasi-isomorphism).''

\textbf{Analogy:} This is like saying: ``If you translate English to French, then translate back to English, you recover the original meaning.'' It's a tautology about translation having an inverse.

\medskip
\noindent\textbf{PHENOMENON 2: Koszul Duality (Special Property)}

For a \textsc{Koszul pair} $(\mathcal{A}_1, \mathcal{A}_2)$ of chiral algebras, we have the much stronger statement:
\begin{align*}
\bar{B}^{\text{ch}}(\mathcal{A}_1) &\simeq (\mathcal{A}_2)^! \quad \text{(bar of algebra 1 = dual cooperad of algebra 2)} \\
\bar{B}^{\text{ch}}(\mathcal{A}_2) &\simeq (\mathcal{A}_1)^! \quad \text{(bar of algebra 2 = dual cooperad of algebra 1)} \\
\Omega^{\text{ch}}((\mathcal{A}_1)^!) &\simeq \mathcal{A}_2 \quad \text{(cobar of dual 1 rebuilds algebra 2)} \\
\Omega^{\text{ch}}((\mathcal{A}_2)^!) &\simeq \mathcal{A}_1 \quad \text{(cobar of dual 2 rebuilds algebra 1)}
\end{align*}

\textbf{Status:} This is a \textsc{non-trivial theorem}. It requires special algebraic structure and must be verified case-by-case.

\textbf{Interpretation:} ``The bar construction of algebra 1 doesn't just encode algebra 1 - it encodes a \emph{different} algebra (algebra 2)!''

\textbf{Analogy:} This is like discovering: ``If you translate Phenomenon 1 (English) to mathematics (French), you don't get back English - you get German! And translating Phenomenon 2 (German) to mathematics gives you English!'' This is highly non-trivial.

\medskip
\noindent\textbf{THE KEY DISTINCTION}

\begin{center}
\fbox{\begin{minipage}{0.9\textwidth}
\textbf{Bar-cobar inversion} is about \emph{returning home}: \\
$\mathcal{A} \xrightarrow{\bar{B}} \bar{B}(\mathcal{A}) \xrightarrow{\Omega} \mathcal{A}$

\vspace{0.3cm}

\textbf{Koszul duality} is about \emph{finding your partner}: \\
$\mathcal{A}_1 \xrightarrow{\bar{B}} (\mathcal{A}_2)^! \xrightarrow{\Omega} \mathcal{A}_2 \neq \mathcal{A}_1$
\end{minipage}}
\end{center}

\medskip
\noindent\textbf{THE DIAGNOSTIC TEST}

To determine whether $(\mathcal{A}_1, \mathcal{A}_2)$ form a Koszul dual pair, compute:
$$\Omega^{\text{ch}}(\bar{B}^{\text{ch}}(\mathcal{A}_1)) \stackrel{?}{\simeq} \mathcal{A}_2$$

\textbf{Outcome 1:} If $\Omega^{\text{ch}}(\bar{B}^{\text{ch}}(\mathcal{A}_1)) \simeq \mathcal{A}_1$ (inversion), then $\mathcal{A}_1$ is \textsc{not} part of a non-trivial Koszul pair.

\textbf{Outcome 2:} If $\Omega^{\text{ch}}(\bar{B}^{\text{ch}}(\mathcal{A}_1)) \simeq \mathcal{A}_2$ with $\mathcal{A}_2 \neq \mathcal{A}_1$, then $(\mathcal{A}_1, \mathcal{A}_2)$ \textsc{are} Koszul dual.

\textbf{Outcome 3:} If $\Omega^{\text{ch}}(\bar{B}^{\text{ch}}(\mathcal{A}_1)) \simeq \mathcal{A}_1$ but through non-trivial quasi-isomorphism (e.g., Yangian), then $\mathcal{A}_1$ is \textsc{Koszul self-dual}.

\medskip
\noindent\textbf{CONCRETE EXAMPLES ILLUSTRATING THE DISTINCTION}

\textbf{Example 1: Free Fermions $\psi$ (Genuine Koszul Pair)}
\begin{itemize}
\item Inversion: $\Omega^{\text{ch}}(\bar{B}^{\text{ch}}(\psi)) \simeq \psi$ \checkmark \quad (always true)
\item Koszul duality: $\Omega^{\text{ch}}(\bar{B}^{\text{ch}}(\psi)) \simeq \beta\gamma \neq \psi$ \checkmark \quad (non-trivial!)
\end{itemize}
Resolution: Both are true! The bar coalgebra $\bar{B}^{\text{ch}}(\psi)$ has two faces:
\begin{enumerate}
\item As the bar of $\psi$, cobar returns $\psi$ (inversion)
\item As the dual cooperad $(\beta\gamma)^!$, cobar builds $\beta\gamma$ (Koszul duality)
\end{enumerate}
The key: $\bar{B}^{\text{ch}}(\psi) = (\beta\gamma)^!$ as cooperads!

\textbf{Example 2: Heisenberg $\mathcal{H}$ (NOT Self-Dual)}
\begin{itemize}
\item Inversion: $\Omega^{\text{ch}}(\bar{B}^{\text{ch}}(\mathcal{H})) \simeq \mathcal{H}$ \checkmark \quad (always true)
\item Koszul duality: $\Omega^{\text{ch}}(\bar{B}^{\text{ch}}(\mathcal{H})) \simeq \text{Sym}(V) \neq \mathcal{H}$ \checkmark \quad (non-trivial!)
\end{itemize}
Resolution: The Heisenberg algebra is NOT Koszul self-dual. Its bar coalgebra is:
$$\bar{B}^{\text{ch}}(\mathcal{H}) = (\text{Sym}(V))^!$$
the dual cooperad of the symmetric algebra.

\textbf{Example 3: Yangian $Y(\mathfrak{g})$ (Self-Dual)}
\begin{itemize}
\item Inversion: $\Omega^{\text{ch}}(\bar{B}^{\text{ch}}(Y)) \simeq Y$ \checkmark \quad (always true)
\item Koszul duality: $\Omega^{\text{ch}}(\bar{B}^{\text{ch}}(Y)) \simeq Y$ \checkmark \quad (happens to coincide!)
\end{itemize}
Resolution: The Yangian IS Koszul self-dual. We have:
$$\bar{B}^{\text{ch}}(Y(\mathfrak{g})) = (Y(\mathfrak{g}))^!$$
The bar coalgebra equals the dual of the same Yangian, not of a different algebra.

\medskip
\noindent\textbf{MATHEMATICAL EXPLANATION OF THE RESOLUTION}

How can both statements be true simultaneously? The resolution involves careful tracking of categorical structures:

\textbf{Category-theoretic view:} The bar construction is a functor:
$$\bar{B}^{\text{ch}}: \mathsf{ChirAlg}_X \to \mathsf{ChirCoalg}_X^{\text{op}}$$

For an algebra $\mathcal{A}_1$, the coalgebra $\bar{B}^{\text{ch}}(\mathcal{A}_1)$ encodes:
\begin{enumerate}
\item Its own inverse image under cobar: $\Omega^{\text{ch}}(\bar{B}^{\text{ch}}(\mathcal{A}_1)) \simeq \mathcal{A}_1$
\item Potentially the dual of a different algebra: $\bar{B}^{\text{ch}}(\mathcal{A}_1) \stackrel{?}{=} (\mathcal{A}_2)^!$
\end{enumerate}

When item (2) holds with $\mathcal{A}_2 \neq \mathcal{A}_1$, we have genuine Koszul duality. The coalgebra structure simultaneously encodes two different algebras through different functorial operations!

\textbf{Slogan:} \emph{The bar coalgebra is an ambiguous figure - it can be read as encoding its source algebra (via cobar inversion) or as encoding a different dual algebra (via Koszul duality). For Koszul pairs, both readings are valid and yield different algebras.}
\end{remark}


We further establish a fundamental relationship between Koszul duality and quantum corrections:

\begin{theorem}[Koszul Complementarity, Theorem 6.5.1]
For a Koszul dual pair $(\mathcal{A}, \mathcal{A}^!)$ of chiral algebras on a genus $g$ surface, the spaces of quantum corrections to the Arnold relations satisfy:
$$\mathcal{Q}_g(\mathcal{A}) \oplus \mathcal{Q}_g(\mathcal{A}^!) \cong H^*(\overline{\mathcal{M}}_{g,n}, \mathbb{C})$$
\end{theorem}

This reveals that Koszul dual chiral algebras have complementary quantum corrections—what one algebra sees as a deformation, its dual sees as an obstruction, and vice versa. This provides a complete classification of quantum corrections through Koszul duality and explains the deep relationship between bosonic and fermionic theories in physics.



\section{Concrete Computational Power}

Throughout the paper we utilize the principle that chiral algebraic structures naturally live on configuration spaces, with the bar-cobar construction providing the dictionary between algebraic and geometric perspectives. This geometric realization transforms abstract algebraic computations into concrete integrations that can be explicitly performed. 

We compute concrete examples that demonstrate the full power of our approach:
\begin{itemize}
\item \textbf{The Heisenberg vertex algebra}: We show how the central extension appears geometrically from the failure of logarithmic forms to satisfy exact Arnold relations at genus one
\item \textbf{Free fermions and boson-fermion correspondence}: The bar complex of free fermions is quasi-isomorphic to the Koszul dual coalgebra of free bosons, $\bar{B}^{\text{ch}}(\text{fermions}) \simeq (\text{bosons})^!$, while the cobar construction establishes the inverse relationship $\Omega^{\text{ch}}((\text{fermions})^!) \simeq \text{bosons}$, realizing bosonization geometrically through the bar-cobar duality dictionary
\item \textbf{$\beta\gamma$ systems}: Complete computation through degree 5, with explicit Koszul dual identification
\item \textbf{W-algebras at critical level}: The bar complex simplifies dramatically, with differential given entirely by screening charges
\item \textbf{Affine Kac-Moody algebras}: We compute their bar complexes and show how quantum deformations arise from higher genus contributions
\end{itemize}

Each example is worked out completely, with all differentials computed explicitly and cohomology determined.

\section{From Local Physics to Global Geometry}


\subsection{Why Configuration Spaces: The Factorization Perspective}

A deeper reason for the appearance of configuration spaces comes from understanding chiral algebras as factorization algebras—a perspective developed by Ayala-Francis \cite{AF15} building on ideas of Lurie \cite{HA} and Costello-Gwilliam \cite{CG17}. 

In this view, a chiral algebra assigns:
\begin{enumerate}
\item To each open set $U \subset X$, a vector space $\mathcal{F}(U)$
\item To disjoint unions, a factorization isomorphism: $\mathcal{F}(U \sqcup V) \cong \mathcal{F}(U) \otimes \mathcal{F}(V)$
\item To inclusions, structure maps satisfying coherence conditions
\end{enumerate}

The configuration spaces encode all possible ways points can be distributed in open sets, making them the natural domain for understanding factorization structures.





\section{Structure of This Paper}

\textbf{Part I: Foundations and Mathematical Framework}
\begin{itemize}
\item Chapter 1: This overview
\item Chapter 2: Chiral algebras following Beilinson-Drinfeld, with explicit connection to our geometric approach
\item Chapter 3: Chiral Hochschild cohomology and deformation theory
\end{itemize}

\textbf{Part II: Configuration Spaces and Geometry}
\begin{itemize}
\item Chapter 4: Fulton-MacPherson compactification with explicit local coordinates
\item Chapter 5: Logarithmic differential forms and proof of Arnold relations
\item Chapter 6: Higher genus phenomena, prime forms, and modular forms
\end{itemize}

\textbf{Part III: Bar and Cobar Constructions}
\begin{itemize}
\item Chapter 7: The geometric bar complex, proof of $d^2 = 0$
\item Chapter 8: The geometric cobar complex, distribution theory, well-definedness
\item Chapter 9: $A_\infty$ structures and higher operations
\end{itemize}

\textbf{Part IV: Koszul Duality and Complete Examples}
\begin{itemize}
\item Chapter 10: Extended Koszul duality theory, criteria for existence
\item Chapter 11: Complete computation for $\beta\gamma$ system
\item Chapter 12: W-algebras at critical level, screening charges
\item Chapter 13: Physical applications, holographic duality
\end{itemize}

\textbf{Appendices}
\begin{itemize}
\item Appendix A: Complete proofs of Arnold relations
\item Appendix B: Theta functions and modular forms
\item Appendix C: Spectral sequences and computational tools
\item Appendix D: Consistency checks and cross-validation
\end{itemize}

\chapter{Alternate Introduction}
The unifying principle throughout: \emph{chiral algebraic structures naturally live on configuration spaces, with the bar-cobar construction providing the precise dictionary between abstract algebra and concrete geometry}. This perspective transforms seemingly intractable algebraic computations into explicit geometric calculations that can be carried out systematically.

To extract the algebraic structure from these singularities, we need to compactify the configuration space in a controlled manner. The Fulton-MacPherson compactification $\overline{C}_n(X)$ adds boundary divisors $D_{ij}$ corresponding to all possible collision patterns, with normal crossing singularities that enable systematic residue calculus. When operators $i$ and $j$ collide, we blow up the diagonal, introducing a new coordinate $\epsilon_{ij} = z_i - z_j$ and angular coordinate $\theta_{ij}$. The divisor $D_{ij} = \{\epsilon_{ij} = 0\}$ is where the collision occurs.

This is where geometry enters: the abstract algebraic operations of the chiral algebra become residue operations along geometric divisors. The residue
$$\text{Res}_{D_{ij}}[\eta_{ij} \cdot \phi_i \otimes \phi_j] = C^k_{ij} \phi_k$$
extracts precisely the OPE coefficient, transforming algebra into geometry through the residue theorem.

\subsection{The Factorization Perspective}

A deep reason for the appearance of configuration spaces comes from understanding chiral algebras as factorization algebras—a perspective developed by Costello-Gwilliam \cite{CG17}. This viewpoint explains not just how but why configuration spaces appear.

In the 1960s, mathematicians studying algebraic topology wanted to understand how local algebraic structures (like multiplication) extend to global ones. The key insight was that "locality" means assigning algebraic data to open sets with compatibility conditions. For an algebraic structure to be "local" on a curve $X$, we need:

1. **Assignment**: To each open $U \subset X$, assign an algebra $\mathcal{F}(U)$
2. **Restriction**: If $V \subset U$, have restriction maps $\mathcal{F}(U) \to \mathcal{F}(V)$
3. **Factorization**: If $U_1, U_2 \subset U$ are disjoint, the algebras multiply:
   $$\mathcal{F}(U_1) \otimes \mathcal{F}(U_2) \to \mathcal{F}(U)$$

This factorization property—that disjoint regions contribute independently—forces us to consider all possible configurations of points. The factorization homology
$$\int_X \mathcal{A} = \text{colim}_n \left[ \mathcal{A}^{\otimes n} \otimes_{(\mathcal{D}_X)^{\otimes n}} \mathcal{D}_{C_n(X)} \right]$$
computes global sections by integrating over configuration spaces.

The bar construction emerges as the dual perspective: instead of building up from local to global via factorization, we resolve the global structure into its local constituents via the bar resolution.

\subsection{The Prism Principle: Decomposing Structure Through Geometry}

We introduce a guiding principle that illuminates our construction and recurs throughout the paper:

\textbf{The Prism Principle}: The geometric bar complex acts as a mathematical prism that decomposes chiral algebras into their operadic spectrum. Just as a physical prism separates white light into constituent colors by frequency, the logarithmic forms $d\log(z_i - z_j)$ separate the global chiral structure into constituent operator product coefficients by conformal weight.

To make this precise: each boundary divisor $D_I$ in $\overline{C}_n(X)$ corresponding to a collision pattern $I$ represents a "spectral line"—a specific channel in the operator product expansion. The residue operation
$$\text{Res}_{D_I}: \Omega^*_{\text{log}}(\overline{C}_n(X)) \to \Omega^*(D_I)$$
extracts the structure constant for that channel. Just as different wavelengths of light refract at different angles through a prism, different conformal weights appear at different codimension strata in the configuration space.

The complete set of residues along all boundary divisors recovers the full algebraic structure:
$$\mathcal{A} = \bigoplus_{\text{strata}} \text{Res}_{\text{stratum}}[\bar{B}^{\text{geom}}(\mathcal{A})]$$

This geometric spectroscopy transforms abstract algebraic structures into explicit geometric data, providing both conceptual clarity and computational power. Every algebraic relation in the chiral algebra corresponds to a geometric relation among residues (the Arnold-Orlik-Solomon relations), and every deformation of the algebraic structure corresponds to a deformation of the differential forms on configuration spaces.

\section{Historical Development and Mathematical Framework}

\subsection{The Evolution of Operadic Theory: Classical Operads, Loop Spaces and Algebraic Structures}

To understand how our geometric construction fits into the broader mathematical landscape, we trace the historical development of the key ideas, showing how each construction arose from concrete problems.

In 1972, J. Peter May \cite{May72} was studying iterated loop spaces $\Omega^n\Sigma^n X$—spaces of maps from $n$-spheres to themselves that fix a basepoint. These spaces have a multiplication coming from concatenation of loops, but the multiplication is only associative up to homotopy. May needed a way to encode these "up to homotopy" algebraic structures systematically.

This led him to introduce operads: collections $\mathcal{P}(n)$ of $n$-ary operations with composition rules. An operad $\mathcal{P}$ consists of:
- Objects $\mathcal{P}(n)$ representing $n$-ary operations  
- Composition maps $\gamma: \mathcal{P}(k) \otimes \mathcal{P}(n_1) \otimes \cdots \otimes \mathcal{P}(n_k) \to \mathcal{P}(n_1 + \cdots + n_k)$
- Symmetric group actions $\Sigma_n \times \mathcal{P}(n) \to \mathcal{P}(n)$ permuting inputs

The fundamental examples encode familiar algebraic structures:
- **Associative operad** $\text{Ass}$: One operation per arity, $\text{Ass}(n) = \mathbb{k}[\Sigma_n]$
- **Commutative operad** $\text{Com}$: All operations identical, $\text{Com}(n) = \mathbb{k}$  
- **Lie operad** $\text{Lie}$: Bracket operations with Jacobi identity

Boardman and Vogt \cite{BV73} simultaneously developed a similar theory, showing these structures control homotopy-coherent algebras. The bar construction for operads, $B_{\mathcal{P}}(A)$, computes derived functors and provides resolutions.

\subsection{Koszul Duality: The Hidden Symmetry}

In 1994, Victor Ginzburg and Mikhail Kapranov \cite{GK94} made a remarkable discovery while studying quadratic algebras. They found that certain pairs of operads are "dual" in a precise homological sense. For a quadratic operad $\mathcal{P} = \text{Free}(E)/(R)$ with generators $E$ and relations $R$, they defined the dual operad
$$\mathcal{P}^! = \text{Free}(s^{-1}E^*)/(R^\perp)$$
with dualized generators and orthogonal relations.

The fundamental theorem: if $\mathcal{P}$ is Koszul (acyclic bar complex), then
$$H_*(\text{Bar}(\mathcal{P})) \cong \mathcal{P}^!$$

The paradigmatic example is Com-Lie duality:
- The commutative operad has trivial relations (everything commutes)
- Its dual, the Lie operad, has maximal relations (antisymmetry and Jacobi)
- The bar complex of Com computes the homology of Lie

This duality would later connect to physics through the state-operator correspondence in CFT.

\subsection{Configuration Spaces: Where Algebra Meets Topology}

The connection to geometry emerged through May's little disks operads $\mathcal{D}_n$. The space $\mathcal{D}_n(k)$ consists of $k$ disjoint embedded $n$-dimensional disks in the unit $n$-disk. These spaces naturally parametrize ways to combine operations geometrically.

In 1976, Fred Cohen \cite{Coh76} proved the fundamental result:
$$H_*(\mathcal{D}_n(k)) \cong H_*(C_k(\mathbb{R}^n))$$

The homology of little disks equals the homology of configuration spaces! This revealed that:
- Operadic structures naturally live on configuration spaces
- Algebraic operations correspond to geometric strata
- The combinatorics of operations matches the topology of point configurations

The Fulton-MacPherson compactification $\overline{C}_n(X)$, originally developed for intersection theory, provided the right framework. It adds boundary divisors for all collision patterns with normal crossings, enabling systematic residue calculus.

\subsection{Chiral Algebras: The Geometric Revolution}

\subsection{Beilinson-Drinfeld: From Vertex Algebras to Geometry}

In the 1980s, physicists had developed vertex algebras to axiomatize 2D conformal field theory. These were algebraic structures with a formal variable $z$ and complicated identities. The theory was powerful but coordinate-dependent and hard to globalize.

In 2004, Alexander Beilinson and Vladimir Drinfeld \cite{BD04} revolutionized the subject by introducing chiral algebras—a coordinate-free geometric reformulation. The key innovation: replace the formal variable with actual points on a curve.

A chiral algebra on a curve $X$ consists of:
- A $\mathcal{D}_X$-module $\mathcal{A}$ (sheaf with differential operator action)
- A chiral operation $\mu: j_*j^*(\mathcal{A} \boxtimes \mathcal{A}) \to \Delta_*\mathcal{A}$

Here $j: X \times X \setminus \Delta \to X \times X$ excludes the diagonal, and $\Delta: X \to X \times X$ is the diagonal embedding. The operation $\mu$ encodes how fields multiply when they approach each other.

The fundamental theorem: chiral algebras on $\mathbb{P}^1$ are equivalent to vertex algebras. But chiral algebras make sense on any curve, opening new vistas:
- Study vertex algebras on higher genus curves
- Use algebraic geometry tools (D-modules, perverse sheaves)
- Connect to geometric Langlands program

The chiral operad has operations
$$\mathcal{P}^{\text{ch}}_X(n) = H^0(\overline{C}_n(X), \omega_{\overline{C}_n(X)}^{\text{log}})$$
—logarithmic forms on compactified configuration spaces!

\subsection{Factorization Algebras: The Higher Categorical View}

The modern perspective emerged from Jacob Lurie's higher algebra \cite{HA}, developed around 2009. Lurie showed that factorization algebras encode local-to-global principles in a precise $\infty$-categorical framework.

David Ayala and John Francis \cite{AF15} formulated a theory of factorization algebras that views chiral algebras as $E_2$-algebras (disk algebras) on curves with additional holomorphic structure. This explains why configuration spaces appear:
- Factorization encodes locality geometrically
- Configuration spaces parametrize ways regions can be disjoint
- The Ran space $\text{Ran}(X)$ is the universal recipient of factorization

Kevin Costello and Owen Gwilliam \cite{CG17} developed perturbative quantum field theory using factorization algebras, showing this isn't just abstract mathematics but the natural language for quantum fields.

\subsection{The Bar-Cobar Construction: From Abstract to Geometric}

\subsection{Abstract Bar-Cobar Duality}

The bar construction transforms algebras into coalgebras and vice versa for the cobar construction. For an augmented operad $\mathcal{P}$:
$$\text{Bar}(\mathcal{P}) = T^c(s\bar{\mathcal{P}})$$
the cofree cooperad on the suspended augmentation ideal.

Dually, the cobar construction:
$$\Omega(\mathcal{C}) = T(s^{-1}\bar{\mathcal{C}})$$
transforms cooperads into operads.

These form an adjunction:
$$\text{Bar}: \text{Operads} \rightleftarrows \text{Cooperads}^{\text{op}} : \Omega$$

When $\mathcal{P}$ is Koszul, this becomes an equivalence of derived categories—bar and cobar are quasi-inverse.

\subsection{Geometric Realization for Chiral Algebras}

Our key contribution is showing this abstract duality has a natural geometric realization through configuration spaces.

The geometric bar complex realizes the abstract bar construction concretely:
$$\bar{B}^{\text{geom}}(\mathcal{A})_n = \Gamma\left(\overline{C}_n(X), \mathcal{A}^{\boxtimes n} \otimes \Omega^*_{\text{log}}\right)$$

Elements are explicit differential forms with logarithmic singularities:
$$\omega = (a_1 \otimes \cdots \otimes a_n) \cdot \eta_{i_1j_1} \wedge \cdots \wedge \eta_{i_kj_k}$$

The differential uses residues:
$$d_{\text{residue}}(\omega) = \sum_{\text{divisors}} \text{Res}_{D}[\omega]$$

This makes the abstract construction completely computable!

Similarly, the geometric cobar complex:
$$\Omega^{\text{geom}}(\mathcal{C})_n = \text{Dist}(C_n(X), \mathcal{C}^{\boxtimes n})$$

Elements are integration kernels:
$$K(z_1, \ldots, z_n) = \frac{c(z_1, \ldots, z_n)}{(z_1-z_2)^{h_1} \cdots (z_{n-1}-z_n)^{h_{n-1}}}$$

The cobar differential inserts delta functions:
$$d_{\text{cobar}}(K) = \sum_{i<j} \Delta_{ij}(K) \cdot \delta(z_i - z_j)$$

\subsection{Chain/Cochain Level Precision}

Our constructions work at the chain/cochain level, not just homology:
- Bar complex: actual chains on configuration spaces  
- Cobar complex: actual cochains (distributions)
- Computations: explicit integrals and residues

This precision enables concrete calculations impossible at the homology level.

\section{Quantum Corrections and Higher Genus}

\subsection{Why Higher Genus Matters: From Trees to Loops}

In quantum field theory, Feynman diagrams organize perturbation theory. Tree diagrams give classical physics; loops give quantum corrections. In our geometric framework:
- **Genus 0** (sphere): Tree-level, classical, rational functions
- **Genus 1** (torus): One-loop, elliptic functions, modular forms
- **Genus $g \geq 2$**: Multi-loop, automorphic forms, period integrals

Each genus contributes fundamentally new structures that don't exist at lower genus.

\subsection{Genus Zero: The Classical World}

On the sphere $\mathbb{P}^1$, everything is rational. The logarithmic forms
$$\eta_{ij} = d\log(z_i - z_j) = \frac{dz_i - dz_j}{z_i - z_j}$$
have simple poles along collision divisors.

These satisfy the Arnold relations (discovered by V.I. Arnold studying braid groups):
$$\eta_{12} \wedge \eta_{23} + \eta_{23} \wedge \eta_{31} + \eta_{31} \wedge \eta_{12} = 0$$

This relation is exact at genus zero—no quantum corrections yet.

\subsection{Genus One: Enter the Quantum}

On a torus $E_\tau = \mathbb{C}/(\mathbb{Z} + \tau\mathbb{Z})$ with modular parameter $\tau \in \mathbb{H}$ (upper half-plane), rational functions become elliptic functions.

The logarithmic form becomes:
$$\eta_{ij}^{(1)} = d\log\vartheta_1\left(\frac{z_i - z_j}{2\pi i}\Big|\tau\right)$$

where $\vartheta_1$ is the odd Jacobi theta function:
$$\vartheta_1(z|\tau) = -i\sum_{n \in \mathbb{Z}}(-1)^n q^{(n-1/2)^2}e^{i(2n-1)z}, \quad q = e^{i\pi\tau}$$

Crucially, the Arnold relation acquires a quantum correction:
$$\eta_{12}^{(1)} \wedge \eta_{23}^{(1)} + \eta_{23}^{(1)} \wedge \eta_{31}^{(1)} + \eta_{31}^{(1)} \wedge \eta_{12}^{(1)} = 2\pi i \cdot \frac{dz \wedge d\bar{z}}{2i\text{Im}(\tau)}$$

The right side is the volume form on the torus! This non-zero correction encodes:
- Central extensions in the chiral algebra
- Anomalies in the quantum field theory
- Modular transformations under $\text{SL}_2(\mathbb{Z})$

\textbf{Concrete Example}: For the Heisenberg algebra with generators $a_n$ and OPE $[a_m, a_n] = m\delta_{m+n,0} \cdot c$, the central charge $c$ appears precisely from this genus-one correction.

\subsection{Higher Genus: The Full Quantum Theory}

For genus $g \geq 2$, surfaces have hyperbolic metrics. New structures appear:

**Period Matrices**: Choose canonical cycles $\{A_\alpha, B_\beta\}_{\alpha,\beta=1}^g$ with intersection
$$A_\alpha \cap B_\beta = \delta_{\alpha\beta}, \quad A_\alpha \cap A_\beta = B_\alpha \cap B_\beta = 0$$

The period matrix
$$\Omega_{\alpha\beta} = \oint_{B_\beta} \omega_\alpha$$
where $\{\omega_\alpha\}$ are holomorphic differentials, lives in the Siegel upper half-space $\mathcal{H}_g$.

**Prime Forms**: The fundamental building block $E(z,w)$ is a $(-1/2, -1/2)$ differential with a simple zero at $z = w$ and no other zeros. It generalizes $(z-w)$ from genus zero.

**Logarithmic Forms at Genus $g$**:
$$\eta_{ij}^{(g)} = d\log E(z_i, z_j) + \sum_{\alpha,\beta=1}^g \left(\oint_{A_\alpha} \omega^{(i)}\right) \Omega_{\alpha\beta}^{-1} \left(\oint_{B_\beta} \omega^{(j)}\right)$$

The second term involves period integrals around cycles—a genuinely new quantum phenomenon!

\subsection{The Master Differential and Quantum Associativity}

The full genus bar complex assembles all contributions:
$$\bar{B}^{\text{full}}(\mathcal{A}) = \bigoplus_{g \geq 0} \lambda^{2g-2} \bar{B}^g(\mathcal{A})$$

Here $\lambda$ is the string coupling (genus expansion parameter). The master differential
$$d^{\text{full}} = \sum_{g \geq 0} \lambda^{2g-2} d^g$$

Each $d^g$ incorporates:
- Residues at collision divisors in $\overline{C}_n(\Sigma_g)$
- Period integrals $\oint_{A_\alpha} \omega$
- Modular forms encoding $\text{Sp}(2g, \mathbb{Z})$ transformations

The miracle: $(d^{\text{full}})^2 = 0$ encodes quantum associativity to all orders!

Expanding in $\lambda$:
- Order $\lambda^{-2}$: Classical associativity (tree level)
- Order $\lambda^0$: One-loop anomaly cancellation  
- Order $\lambda^{2g-2}$: $g$-loop quantum consistency

The geometry of moduli spaces ensures these relations automatically.

\section{Extended Koszul Duality and the theory of Chiral Koszul Pairs}

\subsection{Classical Koszul Duality: The Algebraic Foundation}

In 1970, Stewart Priddy was studying the homology of symmetric groups. He discovered that certain pairs of algebras are "dual" in a remarkable way. For a quadratic algebra
$$A = T(V)/(R), \quad R \subset V^{\otimes 2}$$
the Koszul dual is
$$A^! = T(V^*)/(R^\perp)$$
where $R^\perp = \{f \in (V^*)^{\otimes 2} : f(R) = 0\}$.

The fundamental property: if $A$ is Koszul (bar complex acyclic except in top degree), then
$$\text{Ext}_A^*(k, k) \cong A^!$$

This duality interchanges fundamental structures:
- **Generators** $\leftrightarrow$ **Relations**
- **Multiplication** $\leftrightarrow$ **Comultiplication**
- **Augmentation** $\leftrightarrow$ **Coaugmentation**

\textbf{Classical Examples}:
1. **Symmetric-Exterior Duality**: $S(V) \leftrightarrow \Lambda(V^*)$
   - Symmetric: commutative, no relations beyond commutativity
   - Exterior: anticommutative, maximal relations ($v \wedge v = 0$)

2. **Universal Enveloping-Chevalley-Eilenberg**: $U(\mathfrak{g}) \leftrightarrow CE^*(\mathfrak{g})$
   - Universal enveloping: encodes Lie bracket
   - Chevalley-Eilenberg: computes Lie algebra cohomology

\subsection{Com-Lie Duality: The Geometric Bridge}

The most important example connects commutative and Lie structures.

\subsection{The Commutative Side}

The bar complex of the commutative operad:
$$\text{Bar}(\text{Com})(n) = \bigoplus_{\text{trees } T} \mathbb{k}[T]$$
sums over trees with $n$ leaves. The differential contracts edges.

Geometrically, this equals chains on the partition lattice:
$$\text{Bar}(\text{Com})(n) \cong \tilde{C}_*(\bar{\Pi}_n)$$
where $\Pi_n$ = partitions of $\{1, \ldots, n\}$ ordered by refinement.

The crucial fact: boundary strata of $\overline{C}_n(\mathbb{P}^1)$ correspond to partitions! A partition $\pi$ corresponds to the stratum where points collide according to blocks of $\pi$.

\subsection{The Lie Side}

The homology computes:
$$H_{n-2}(\bar{\Pi}_n) \cong \text{Lie}(n) \otimes \text{sgn}_n$$

Bracket operations emerge from cycles in the partition complex!

\subsection{Our Geometric Enhancement}

In the chiral setting, Com-Lie duality becomes:
- **Commutative chiral**: Free commutative chiral algebra
- **Lie chiral**: Affine Lie algebra (current algebra)

The geometric bar complex enriches the partition complex:
$$\bar{B}^{\text{ch}}(\text{Com}_{\text{ch}}) = \tilde{C}_*(\bar{\Pi}_n) \otimes \Omega^*_{\text{log}}(\overline{C}_n(X))$$

Now we have:
- Combinatorics from partitions (discrete)
- Geometry from configuration spaces (continuous)
- Logarithmic forms encoding conformal weights

This enrichment captures:
- Central extensions from genus-one  
- Quantum groups from higher genus
- Modular transformations from $\text{SL}_2(\mathbb{Z})$ action

\subsection{Chiral Quadratic Algebras}

For chiral algebras, "quadratic" requires locality. Following Beilinson-Drinfeld and recent work by Gui-Li-Zeng \cite{GL22}:

A chiral quadratic datum consists of:
- Locally free sheaf $N$ on $X$ (generators)
- Subsheaf $P \subset j_*j^*(N \boxtimes N)$ with $P|_U = N \boxtimes N|_U$ (relations)

The locality condition means relations only appear at collisions—away from the diagonal, fields commute freely.

The dual datum:
$$(N, P) \mapsto (s^{-1}N^{\vee}_{\omega^{-1}}, P^\perp)$$

The pairing is computed by residues:
$$\langle n_1 \otimes n_2, m_1 \otimes m_2 \rangle = \text{Res}_{z_1 = z_2} \langle n_1, m_1 \rangle(z_1) \langle n_2, m_2 \rangle(z_2) dz_1 dz_2$$

This residue pairing geometrically realizes the algebraic duality.

\subsection{Beyond Quadratic: Curved and Filtered Extensions}

Many important examples aren't quadratic. We extend Koszul duality to:

\subsection{Curved Algebras}

A curved chiral algebra has curvature $\kappa \in \mathcal{A}^{\otimes 2}[2]$ with
$$d\kappa + \frac{1}{2}[\kappa, \kappa] = 0$$
(Maurer-Cartan equation).

The bar differential becomes:
$$d_{\text{curved}} = d + m_0(\kappa)$$

\textbf{Example}: The $\beta\gamma$ system has fields $\beta, \gamma$ with OPE $\beta(z)\gamma(w) \sim (z-w)^{-1}$. The curvature
$$\kappa = \int \beta\gamma$$
encodes the non-zero vacuum expectation value.

\subsection{Filtered Algebras}  

W-algebras have natural filtrations by conformal weight:
$$F_0 \mathcal{W} \subset F_1 \mathcal{W} \subset F_2 \mathcal{W} \subset \cdots$$

The associated graded recovers simpler structures. The bar complex respects filtrations:
$$F_p\bar{B}(\mathcal{W}) = \bigoplus_{i_1 + \cdots + i_n \leq p} \bar{B}(F_{i_1} \otimes \cdots \otimes F_{i_n})$$

A spectral sequence computes corrections order by order.

\subsection{Poincaré-Verdier Duality: The Geometric Heart}

The bar-cobar duality realizes as Poincaré-Verdier duality:
$$\bar{B}^{\text{ch}}(\mathcal{A}) \cong \mathbb{D}(\Omega^{\text{ch}}(\mathcal{A}^!))$$

The pairing:
$$\langle \omega_{\text{bar}}, K_{\text{cobar}} \rangle = \int_{\overline{C}_n(X)} \omega_{\text{bar}} \wedge \iota^* K_{\text{cobar}}$$

This exchanges:
- **Compactification** $\leftrightarrow$ **Localization**  
- **Logarithmic forms** $\leftrightarrow$ **Distributions**
- **Residues** $\leftrightarrow$ **Principal values**
- **Boundary divisors** $\leftrightarrow$ **Propagators**

The duality is computed by explicit integration—completely constructive!

\section{Concrete Examples and Applications}

\subsection{The Heisenberg Vertex Algebra}

The Heisenberg algebra is generated by $a_n$ ($n \in \mathbb{Z}$) with
$$[a_m, a_n] = m\delta_{m+n,0} \cdot c$$

The central charge $c$ appears from genus-one geometry:

At genus 0: $\eta_{12} \wedge \eta_{21} = 0$ (exact relation)
At genus 1: $\eta_{12}^{(1)} \wedge \eta_{21}^{(1)} = 2\pi i \omega_\tau$ (quantum correction)

The bar complex:
$$\bar{B}^0(\text{Heis}) = \text{Polynomial differential forms on } \overline{C}_n(\mathbb{P}^1)$$
$$\bar{B}^1(\text{Heis}) = \text{Elliptic forms with modular weight}$$

\subsection{Free Fermions and Boson-Fermion Correspondence}

Free fermions: $\psi(z)\psi(w) \sim (z-w)^{-1}$

Bar complex:
$$\bar{B}(\text{Fermion}) = \Lambda^*(\mathbb{C}^n) \otimes \Omega^*_{\text{log}}(\overline{C}_n)$$

The cobar of the bar recovers free bosons:
$$\Omega(\bar{B}(\text{Fermion})) \simeq \text{Heisenberg}$$

This geometrically realizes bosonization!

\subsection{W-Algebras at Critical Level}

For $\mathcal{W}^k(\mathfrak{g}, f)$ at critical level $k = -h^\vee$:

$$\bar{B}(\mathcal{W}^{-h^\vee}) = \text{Screening charges} \otimes \Omega^*_{\text{log}}$$

The differential is entirely screening operators—dramatic simplification!

\section{Chiral Hochschild Cohomology}

The bar complex computes a chiral version of Hochschild cohomology:

\begin{definition}[Chiral Hochschild Complex]
For a chiral algebra $\mathcal{A}$, define:
$$\ChirHoch^*(\mathcal{A}) = H^*(\text{RHom}_{\mathcal{A}^e}(\mathcal{A}, \mathcal{A}))$$
where $\mathcal{A}^e$ is the chiral enveloping algebra.
\end{definition}

\begin{theorem}[Geometric Realization]
$$\ChirHoch^n(\mathcal{A}) \cong H^n(\barBgeom(\mathcal{A}) \otimes_{\mathcal{A}} \mathcal{A})$$
\end{theorem}

Physical interpretation:
\begin{itemize}
\item $\ChirHoch^0$: Center (conserved charges)
\item $\ChirHoch^1$: Derivations (symmetries)
\item $\ChirHoch^2$: Deformations (marginal operators)
\item $\ChirHoch^3$: Obstructions
\end{itemize}

\section{Criteria for Koszul Pairs}

Not every chiral algebra admits a Koszul dual. We establish:

\begin{theorem}[Koszul Criterion]
A chiral algebra $\mathcal{A}$ admits a Koszul dual iff:
\begin{enumerate}
\item Finite generation over $\mathcal{D}_X$
\item Formal smoothness: $\dim \ChirHoch^n(\mathcal{A}) < \infty$
\item Poincaré duality: $\ChirHoch^i \times \ChirHoch^{d-i} \to \omega_X$
\item Bar spectral sequence converges
\end{enumerate}
\end{theorem}

For W-algebras, additional structure emerges:

\begin{theorem}[W-algebra Koszul Duality]
At critical level $k = \critLevel$:
$$\Walg^{\critLevel}(\mathfrak{g}, f) \text{ is Koszul dual to } \Walg^{\critLevel}(\mathfrak{g}^\vee, f^\vee)$$
where $\mathfrak{g}^\vee$ is the Langlands dual.
\end{theorem}

Complete proofs with explicit examples follow in the main text.

\subsection{Clarifications on Scope and Terminology}
\label{subsec:scope-terminology}

\begin{remark}[What This Manuscript Does and Does Not Cover]\label{rem:scope-explicit}
To avoid confusion with related programs, we explicitly state:

\textbf{This manuscript DOES cover:}
\begin{enumerate}
\item Chiral algebras on Riemann surfaces (all genera)
\item Geometric bar-cobar duality via configuration spaces
\item Extension of Beilinson-Drinfeld to higher genus
\item Explicit computations for Heisenberg, Kac-Moody, W-algebras
\item Connection to vertex algebras and CFT
\item Factorization homology for chiral algebras
\end{enumerate}

\textbf{This manuscript does NOT directly cover (but makes connections to):}
\begin{enumerate}
\item Topological twists of 4D gauge theories
\item Dimensional reduction from higher dimensions
\item Full BV-BRST quantization (only BV complex structure)
\item Holomorphic-topological boundary conditions (only vertex algebra aspects)
\item AGT correspondence (only W-algebra perspective)
\end{enumerate}
\end{remark}


\section{Structure of This Paper}

\textbf{Part II: Configuration Spaces and Geometry} (Chapters 2-3)
- Chapter 2: Fulton-MacPherson compactification, explicit coordinates
- Chapter 3: Logarithmic forms, Arnold relations across genera

\textbf{Part III: Bar and Cobar Constructions} (Chapters 4-5)
- Chapter 4: Geometric bar complex, proof of $d^2 = 0$
- Chapter 5: Geometric cobar, distributions, $A_\infty$ structures

\textbf{Part IV: Koszul Duality and Applications} (Chapters 6-8)
- Chapter 6: Extended Koszul duality, curved and filtered cases
- Chapter 7: W-algebras, screening charges, representation theory
- Chapter 8: Holographic duality, AdS/CFT as Koszul duality

The unifying principle: chiral algebraic structures naturally live on configuration spaces, with bar-cobar constructions providing the dictionary between algebra and geometry. Our chain-level geometric realization makes everything computable through explicit integration.

