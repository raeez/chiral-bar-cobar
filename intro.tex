\chapter{Introduction}

\section{Poincaré Duality and Quantum Field Theory}
\label{sec:nonabelian-poincare-intro}

\subsection{Beyond Classical Poincaré Duality}

Classical Poincaré duality establishes an isomorphism between homology and cohomology:
\begin{equation}
H_k(M) \cong H^{n-k}(M)^\vee
\end{equation}
for an $n$-dimensional closed oriented manifold $M$. This is fundamentally \emph{abelian}—both sides are vector spaces related by a linear duality.

\begin{principle}[Non-Abelian Poincaré Duality]
\label{prin:nonabelian-poincare}
Non-abelian Poincaré duality, in the sense of Ayala-Francis, extends this to a duality between \emph{algebraic structures}:
\begin{equation}
\int_M \mathcal{A} \simeq \left(\int_{-M} \mathcal{A}^!\right)^\vee
\end{equation}
where:
\begin{itemize}
\item $\mathcal{A}$ is a factorization algebra (encoding local-to-global algebraic data)
\item $\mathcal{A}^!$ is its Koszul dual factorization algebra
\item $-M$ is $M$ with the reversed orientation
\item $\int_M$ denotes factorization homology
\item The duality preserves non-abelian (non-commutative) structure
\end{itemize}
\end{principle}

\subsection{Chiral Algebras as Factorization Algebras}

Following Beilinson-Drinfeld and Francis-Gwilliam, a chiral algebra $\mathcal{A}$ on a curve $X$ is equivalently:

\begin{enumerate}
\item \textbf{BD Perspective}: A $\mathcal{D}_X$-module with chiral operations defined via residues
\item \textbf{Factorization Perspective}: A factorization algebra on $X$ satisfying:
\begin{equation}
\mathcal{A}(U \sqcup V) \xrightarrow{\sim} \mathcal{A}(U) \otimes_{\mathcal{D}_X} \mathcal{A}(V)
\end{equation}
for disjoint open sets $U, V \subset X$
\end{enumerate}

\begin{remark}[Why This Matters]
The factorization property encodes \textbf{locality} of quantum field theory: observations at separated points are independent (factorize). This is the physical content underlying the mathematical structure.
\end{remark}

\section{Three Facets of the Same Phenomenon}
\label{sec:NAP-unifying}

\subsection{The Three-Way Correspondence}

Our central insight is that chiral Koszul duality sits at the nexus of three perspectives:

\begin{center}
\begin{tikzcd}[row sep=huge, column sep=huge]
\text{Chiral Koszul Duality} \arrow[r, "\text{specializes}"] \arrow[d, "\text{realizes}"'] & \text{Non-Abelian Poincaré Duality} \arrow[d, "\text{via}"] \\
\text{Configuration Space Geometry} \arrow[r, "\text{computes}"'] & \text{Factorization Homology}
\end{tikzcd}
\end{center}

\begin{theorem}[Unification via Configuration Spaces]
\label{thm:three-way-correspondence}
For a chiral Koszul pair $(\mathcal{A}, \mathcal{A}^!)$ on a curve $X$:
\begin{enumerate}
\item \textbf{Algebraic}: $\Omega(\bar{B}(\mathcal{A})) \simeq \mathcal{A}$ (bar-cobar adjunction)
\item \textbf{Geometric}: Both $\bar{B}(\mathcal{A})$ and $\mathcal{A}^!$ realized via integrals on $\overline{C}_n(X)$
\item \textbf{Homological}: 
\begin{equation}
\int_X \mathcal{A} \simeq \left(\int_X \mathcal{A}^!\right)^\vee
\end{equation}
computed by factorization homology
\end{enumerate}
The three perspectives are equivalent and mutually enriching.
\end{theorem}


\section{The Central Mystery}

In two-dimensional conformal field theory, the most fundamental observables are correlation functions of local operators. When two chiral operators $\phi_1(z_1)$ and $\phi_2(z_2)$ approach each other on a Riemann surface, their correlation functions develop singularities controlled by the operator product expansion (OPE):

$$\phi_1(z_1) \phi_2(z_2) \sim \sum_k \frac{C^k_{12}}{(z_1 - z_2)^{h_k}} \phi_k(z_2) + \text{regular terms}$$

The structure constants $C^k_{12}$ encode the complete algebraic structure of the chiral algebra. This local singularity data—purely algebraic in nature—turns out to have a natural geometric interpretation that forms the foundation of our work.

\section{The Key Observation}

The key observation is elementary yet profound: the logarithmic differential form $d\log(z_1 - z_2) = \frac{dz_1 - dz_2}{z_1 - z_2}$ has a simple pole precisely when $z_1 = z_2$. When we compute the residue
$$\text{Res}_{z_1=z_2} d\log(z_1 - z_2) \cdot \phi_1(z_1)\phi_2(z_2) = C^k_{12} \phi_k(z_2)$$
we extract exactly the structure constant from the OPE. This simple fact—that algebraic structure constants become geometric residues—motivates our entire construction.

\section{Why Configuration Spaces?}

But why should we expect such a geometric interpretation to exist? The answer lies in a fundamental principle of quantum field theory: locality. The requirement that operators commute at spacelike separation forces the algebraic structure to be encoded in the singularities as operators approach each other. These singularities naturally live on configuration spaces—the spaces parametrizing positions of operators on the curve. The compactification of these spaces, which adds boundary divisors corresponding to collision patterns, provides the geometric arena where quantum algebra becomes algebraic geometry.

\section{Relationship to Foundational Work}

Beilinson and Drinfeld \cite{BD04} axiomatized 2d quantum field theory as factorization algebras on curves with presentations as $\mathcal{D}$-modules with chiral operations. This paper develops a systematic geometric realization of bar-cobar duality for chiral algebras through configuration space integrals, extending across all genera to incorporate the full spectrum of quantum corrections to all loop orders. The construction naturally produces a theory of chiral koszul dual pairs, vastly extending the classic quadratic koszul duality.

Our perspective draws from three mathematical perspectives: the algebraic approach to chiral algebras via $\mathcal{D}$-modules developed by Beilinson-Drinfeld \cite{BD04}, the geometric configuration space methods pioneered by Kontsevich \cite{Kon94, Kon99}, and the higher categorical framework of factorization homology introduced by Ayala-Francis \cite{AF15}.

\subsection{Relation to Costello-Gwilliam}

Our geometric approach complements the perspective in Costello-Gwilliam's 
\textit{Factorization Algebras in Quantum Field Theory} \cite{CostelloGwilliam}:

\begin{itemize}
\item \textbf{Volume 1}: Foundations of factorization algebras. Our bar complex is 
the derived global sections of a factorization algebra (compare CG Vol. 1, Chapter 5).

\item \textbf{Volume 2}: Renormalization and BV formalism. Our nilpotent completion 
(Appendix on non-quadratic algebras) corresponds to Costello's renormalization 
group flow (CG Vol. 2, Chapters 4-5). The I-adic filtration $I^n$ encodes the 
``effective action at scale $n$''.

\item \textbf{Koszul duality}: CG Vol. 2, \S13 develops Koszul duality for $E_n$ 
operads. Our work extends this to the $E_\infty$ (chiral) setting using configuration 
space integrals.
\end{itemize}

\textbf{Key insight}: Bar-cobar duality for our chiral algebras forms the curved $L_\infty$ 
version of CG's bar-cobar for factorization algebras. The curvature terms come from 
central extensions (quantum anomalies in physics).

\begin{enumerate}
\item Chapter \ref{chap:chiral-deformation}: Complete treatment of chiral deformation quantization, extending Kontsevich's formality theorem to curves with explicit formulas for all genera and examples (Heisenberg, affine $\mathfrak{sl}_2$, $W_3$) with all coefficients computed.
\item Chapter \ref{chap:kac-moody}: Kac-Moody Koszul duals with complete OPE structures for $\widehat{\mathfrak{sl}}_2, \widehat{\mathfrak{sl}}_3, \widehat{\mathfrak{sl}}_n, \widehat{E}_8$, bar construction through degree 5, and level shift formulas derived from first principles.
\item Chapter \ref{chap:w-algebras}: W-algebra Koszul duals with concrete $W_3$ OPE expanded mode commutators, $W_k(\mathfrak{sl}_3)$ from BRST construction step-by-step, and examples at $c=2$ and $c=100$.
\end{enumerate}

\subsection{Connections to Related Mathematical Physics Programs}
\label{subsec:related-programs}

\begin{remark}[Landscape of Holomorphic Field Theories]\label{rem:holomorphic-landscape}
Our chiral bar-cobar duality sits within a broader landscape of holomorphic 
constructions in mathematical physics. We clarify the relationships:

\textbf{1. Beilinson-Drinfeld Chiral Algebras (1995-2004)} \cite{BD04}:
\begin{itemize}
\item Foundation: $\mathcal{D}$-modules on configuration spaces
\item Genus: Primarily genus zero (rational curves)
\item This manuscript provideos a homotopy-geometric construction of the Bar complex that is left implicit in their work, and further extends the construction to all genera
\end{itemize}

\textbf{2. Costello-Gwilliam Factorization Algebras (2017)} \cite{CG17}:
\begin{itemize}
\item Foundation: BV formalism, general manifolds
\item Scope: Arbitrary dimension, topological field theories
\item Connection: Our bar complex $\simeq$ CG factorization homology for chiral algebras
\end{itemize}

\textbf{3. Costello-Li Twisted Supergravity (2016)} \cite{CL16}:
\begin{itemize}
\item Foundation: Topological twist of 4D $\mathcal{N}=2$ theories
\item Method: Dimensional reduction produces 2D factorization algebras
\end{itemize}

\textbf{4. Gaiotto Holomorphic-Topological Twist (2019)} \cite{Gai19}:
\begin{itemize}
\item Foundation: Boundary conditions in HT twist
\item Focus: Interfaces and defects in gauge theory
\item Connection: W-algebras appear as boundary vertex algebras
\end{itemize}

\textbf{5. Paquette-Williams Boundaries and Interfaces (2022)} \cite{PW22}:
\begin{itemize}
\item Foundation: Vertex algebras at corners in HT theories
\item Method: Quantization of moduli spaces produces vertex algebras
\item Our perspective: These vertex algebras have chiral envelope, 
      amenable to our bar-cobar analysis
\end{itemize}

\textbf{6. Ayala-Francis Factorization Homology (2019)} \cite{AF19}:
\begin{itemize}
\item Foundation: $\infty$-categorical factorization homology
\item Generality: Arbitrary symmetric monoidal $\infty$-categories
\item Connection: Our geometric bar complex computes factorization 
      homology for chiral algebras (Theorem \ref{thm:bar-factorization-homology})
\end{itemize}
\end{remark}


\section{Main Results and Organization}

Our first result establishes the geometric bar construction for chiral algebras through configuration space integrals. This construction is elementary at its core: we take tensor products of the chiral algebra and integrate logarithmic forms over configuration spaces. The residues at collision divisors extract the algebraic operations:

\begin{theorem}[Geometric Bar Construction, Theorem 3.2]
For a chiral algebra $\mathcal{A}$ on a smooth curve $X$, we construct a geometric bar complex at the chain level:
$$\bar{B}^{\text{geom}}(\mathcal{A})_n = \Gamma\left(\overline{C}_n(X), \mathcal{A}^{\boxtimes n} \otimes \Omega^*_{\text{log}}\right)$$
where $\overline{C}_n(X)$ is the Fulton-MacPherson compactification and $\Omega^*_{\text{log}}$ denotes logarithmic differential forms with poles along boundary divisors. The differential 
$$d = d_{\text{internal}} + d_{\text{residue}} + d_{\text{de Rham}}$$
combines internal operations from $\mathcal{A}$ with residues along collision divisors and the de Rham differential. Concretely, for elements $a_1 \otimes \cdots \otimes a_n \otimes \omega \in \bar{B}^{\text{geom}}(\mathcal{A})_n$:
$$d_{\text{residue}}(a_1 \otimes \cdots \otimes a_n \otimes \omega) = \sum_{i<j} \text{Res}_{D_{ij}}[\omega] \cdot (a_1 \otimes \cdots \otimes \mu(a_i, a_j) \otimes \cdots)$$
The condition $d^2 = 0$ follows from the Arnold-Orlik-Solomon relations among logarithmic forms.
\end{theorem}

\begin{remark}[Conceptual Foundation for the Duality]
\label{rem:NAP-foundation}
These constructions are not ad hoc. They arise inevitably from non-abelian Poincaré (NAP) duality, which we develop systematically in Part~II (Chapters on NAP derivation and computations).

The key principle: For a chiral algebra $\mathcal{A}$ viewed as a factorization algebra on $X$, factorization homology satisfies:
$$\int_X \mathcal{A} \simeq \mathbb{D}\left(\int_{-X} \mathcal{A}^!\right)$$

where:
\begin{itemize}
\item $\int_X$ denotes factorization homology (computed by configuration space integrals)
\item $\mathbb{D}$ is Verdier duality (exchanging logarithmic forms and distributions)
\item $-X$ denotes $X$ with opposite orientation
\item $\mathcal{A}^!$ is the \textbf{Koszul dual chiral coalgebra}, defined intrinsically via this duality
\end{itemize}

This framework provides:
\begin{enumerate}
\item An independent construction of $\mathcal{A}^!$ without circularity
\item A geometric proof that $\bar{B}^{\text{ch}}(\mathcal{A}) \simeq \mathcal{A}^!$
\item Systematic computation of Koszul duals for non-quadratic algebras
\item Natural extension to higher genus via modular forms
\end{enumerate}

The bar and cobar complexes are related by Verdier duality on the configuration spaces.
\end{remark}

We follow with the dual construction—the geometric cobar complex. This construction is equally elementary: we work with distributions (integration kernels) on open configuration spaces:

\begin{theorem}[Geometric Cobar Construction, Theorem 3.5]
For a chiral coalgebra $\mathcal{C}$ on a smooth curve $X$, we construct a geometric cobar complex at the cochain level:
$$\Omega^{\text{geom}}(\mathcal{C})_n = \text{Dist}(C_n(X), \mathcal{C}^{\boxtimes n})$$
consisting of distributional sections (integration kernels) on open configuration spaces with prescribed singularities along diagonals. Concretely, elements are expressions like:
$$K(z_1, \ldots, z_n) = \sum_{\text{poles}} \frac{c_{i_1 \cdots i_k}}{(z_{i_1} - z_{i_2})^{h_1} \cdots (z_{i_{k-1}} - z_{i_k})^{h_{k-1}}}$$
The cobar differential
$$d_{\text{cobar}}(K) = \sum_{i<j} \Delta_{ij}(K) \cdot \delta(z_i - z_j)$$
inserts Dirac distributions that "pull apart" colliding points, implementing the coproduct $\Delta: \mathcal{C} \to \mathcal{C} \otimes \mathcal{C}$.
\end{theorem}

We proceed to extend the construction across all genera, incorporating quantum corrections that appear as loop integrals in physics:

\begin{theorem}[Full Genus Bar Complex, Theorem 5.1]
The geometric bar complex extends to all genera $g \geq 0$ as
$$\bar{B}^{\text{full}}(\mathcal{A}) = \bigoplus_{g \geq 0} \lambda^{2g-2} \bar{B}^g(\mathcal{A})$$
where each $\bar{B}^g(\mathcal{A})$ incorporates genus-specific geometry:
\begin{itemize}
\item \textbf{Genus 0}: Logarithmic forms $\eta_{ij} = d\log(z_i - z_j)$ on $\mathbb{P}^1$
\item \textbf{Genus 1}: Elliptic forms on torus $E_\tau = \mathbb{C}/(\mathbb{Z} + \tau\mathbb{Z})$:
  $$\eta_{ij}^{(1)} = d\log\vartheta_1\left(\frac{z_i - z_j}{2\pi i}|\tau\right) + \frac{(z_i - z_j)d\tau}{2\pi i \text{Im}(\tau)}$$
  where $\vartheta_1(z|\tau) = -i\sum_{n \in \mathbb{Z}}(-1)^n q^{(n-1/2)^2}e^{i(2n-1)z}$ with $q = e^{i\pi\tau}$
\item \textbf{Genus $g \geq 2$}: Prime forms and period integrals on hyperbolic surfaces:
  $$\eta_{ij}^{(g)} = d\log E(z_i, z_j) + \sum_{\alpha=1}^g \left(\oint_{A_\alpha} \omega_i\right) \left(\oint_{B_\alpha} \omega_j\right)$$
  where $E(z,w)$ is the prime form and $\{A_\alpha, B_\alpha\}$ are canonical homology cycles
\end{itemize}
The master differential $d^{\text{full}} = \sum_{g} \lambda^{2g-2} d^g$ satisfies $(d^{\text{full}})^2 = 0$, encoding quantum associativity to all loop orders.
\end{theorem}

\section{Main Results - Complete Statements with Proof Locations}
\label{sec:main-results-complete}

\begin{maintheorem}[Geometric Bar-Cobar Duality - Complete Statement]\label{mainthm:bar-cobar-complete}
For a koszul chiral algebra $\mathcal{A}$ on a smooth projective curve $X$, there exists a 
canonical Koszul dual chiral coalgebra $\mathcal{A}^!$ such that:

\begin{enumerate}
\item \textbf{(Functoriality)} The assignment $\mathcal{A} \mapsto \bar{B}_{\text{geom}}(\mathcal{A})$ 
defines a functor:
$$\bar{B}_{\text{geom}}: \text{ChirAlg}(X) \to \text{dgCoalg}(X)$$
\textit{Proven in: Corollary \ref{cor:bar-functorial} (Section 3.2)}

\item \textbf{(Quasi-isomorphism)} The natural maps:
$$\bar{B}_{\text{geom}}(\mathcal{A}) \xrightarrow{\simeq} \mathcal{A}^!$$
$$\Omega_{\text{geom}}(\mathcal{A}^!) \xrightarrow{\simeq} \mathcal{A}$$
are quasi-isomorphisms of chain complexes.
\textit{Proven in: Corollary \ref{cor:bar-cobar-inverse} (Section 3.8)}

\item \textbf{(Adjunction)} The bar and cobar constructions form an adjoint pair:
$$\text{Hom}_{\text{dgCoalg}}(\bar{B}(\mathcal{A}), \mathcal{C}) 
\simeq \text{Hom}_{\text{ChirAlg}}(\mathcal{A}, \Omega(\mathcal{C}))$$
\textit{Follows from: Theorem \ref{thm:bar-cobar-verdier} (Section 3.8)}

\item \textbf{(Higher Genus Extension)} For each genus $g \geq 0$:
$$\bar{B}(\mathcal{A}) = \bigoplus_{g=0}^{\infty} \hbar^{2g-2} \bar{B}_g(\mathcal{A})$$
where $\bar{B}_g$ computes cohomology over $\mathcal{M}_g$ with quantum corrections.
\textit{Proven in: Theorem \ref{thm:CC-acyclicity-higher-genus} (Section 4.10)}

\item \textbf{(BD Compatibility)} For genus 0, this reduces to Beilinson-Drinfeld.
\textit{Verified in: Remark \ref{rem:BD-reference-guide} (Section 3.1)}
\end{enumerate}
\end{maintheorem}

\begin{proof}[Proof Outline and Cross-References]
The complete proof is distributed across the manuscript:

\textbf{Part (1) - Functoriality:}
Corollary \ref{cor:bar-functorial} (Section 3.2)

\textbf{Part (2) - Quasi-isomorphism:}
Theorem \ref{thm:CC-acyclicity-higher-genus} (Section 4.10) + Corollary \ref{cor:bar-cobar-inverse} (Section 3.8)

\textbf{Part (3) - Adjunction:}
Theorem \ref{thm:bar-cobar-verdier} (Section 3.8) + Theorem \ref{thm:verdier-AF-compat} (Section 4.11)

\textbf{Part (4) - Higher Genus:}
Theorem \ref{thm:deformation-obstruction} (Section 3.10) + Lemma \ref{lem:quantum-preserves-acyclicity} (Section 4.10)

\textbf{Part (5) - BD Compatibility:}
Remark \ref{rem:BD-reference-guide} (Section 3.1)
\end{proof}

\begin{maintheorem}[Curved Koszul Duality - Complete Statement]\label{mainthm:curved-complete}
For chiral algebras with central extensions (curved $A_\infty$ structures):

\begin{enumerate}
\item \textbf{(Obstruction Theory)} The failure of $d^2 = 0$ is measured by:
$$Q_g(\mathcal{A}) \subset H^2(\bar{B}_g(\mathcal{A}), Z(\mathcal{A}))$$
where $Z(\mathcal{A})$ is the center.
\textit{Proven in: Lemma \ref{lem:obstruction-class} (Section 3.10)}

\item \textbf{(Deformation-Obstruction Duality)} Perfect pairing:
$$Q_g(\mathcal{A}) \oplus Q_g(\mathcal{A}^!) \simeq H^*(\mathcal{M}_g, Z(\mathcal{A}))$$
\textit{Proven in: Theorem \ref{thm:deformation-obstruction} (Section 3.10)}

\item \textbf{(Completion)} For non-quadratic algebras:
$$\widehat{\mathcal{A}}^! = \varprojlim_n \mathcal{A}^! / I^n$$
\textit{Proven in: Theorem \ref{thm:nilpotent-main} (Appendix B)}

\end{enumerate}
\end{maintheorem}

\begin{proof}[Proof Outline]
See Theorem \ref{thm:deformation-obstruction} (Section 3.10) for complete proof. 
Key steps:

1. Curved $A_\infty$ relations ensure $\mu_0 \in Z(\mathcal{A})$
2. Obstructions are classes in $H^2(\bar{B}, Z(\mathcal{A}))$
3. Deformations parametrized by $\text{Ext}^1(\mathcal{A}^!, \mathcal{A}^!)$
4. Serre duality on $\mathcal{M}_g$ gives perfect pairing
5. Completion ensures convergence for non-quadratic cases
\end{proof}

Symplectic bosons and chiral fermions, Kac-Moody, and W-algebras form concrete examples throughout the manuscript.


% ================================================================
% PATCH 013 ADDITION TO INTRO: ON-NOSE NILPOTENCE
% ================================================================

\subsection{Strict Nilpotence: $d^2 = 0$}

A key technical fact is the proof that the bar differential satisfies $d^2 = 0$ 
\textbf{on the nose}, not just up to homotopy. This requires:
\begin{itemize}
\item Central curvature: $\mu_0 \in Z(\mathcal{A})$
\item Arnold relations for residue terms
\item Leibniz compatibility
\item Closedness of quantum correction forms $\omega_g$
\end{itemize}

This on-nose nilpotence allows direct computation of Koszul duals without $\infty$-categorical 
machinery. See \S\ref{sec:on-nose-vs-homotopy} for complete details and verification through genus 5.

\textbf{Main Result (Theorem \ref{thm:central-implies-strict}):} For all chiral algebras with 
central curvature:
$$d_{\text{bar}}^2 = 0 \quad \text{strict equality, \textit{not just up to homotopy}}$$

This applies to all vertex algebras from conformal field theory, including:
\begin{itemize}
\item Heisenberg $\mathcal{H}_k$ at level $k$
\item Affine Kac-Moody $\widehat{\mathfrak{g}}_k$ at level $k$  
\item Virasoro $\text{Vir}_c$ with central charge $c$
\item $W$-algebras $W_N$ for all $N \geq 3$
\end{itemize}

\begin{maintheorem}[Non-Abelian Poincaré Duality - Complete Statement]\label{mainthm:NAP-complete}
Bar-cobar duality is mediated by Verdier duality on configuration spaces:

\begin{enumerate}
\item \textbf{(Factorization Homology)} Bar computes:
$$\bar{B}(\mathcal{A}) \simeq \int_X \mathcal{A}$$
(Ayala-Francis factorization homology).
\textit{Proven in: Lemma \ref{lem:bar-as-fact-hom-AF} (Section 4.11)}

\item \textbf{(Verdier Dual)} Cobar is:
$$\Omega(\mathcal{C}) \simeq \mathbb{D}\left(\int_{-X} \mathcal{C}\right)$$
where $\mathbb{D}$ is Verdier duality.
\textit{Proven in: Theorem \ref{thm:bar-cobar-verdier} (Section 3.8)}

\item \textbf{(Compatibility)} Geometric duality (Verdier) specializes to topological duality as $D$-modules specialize to abelian groups, and the dualities intertwine across the specialization map.
\textit{Proven in: Theorem \ref{thm:verdier-AF-compat} (Section 4.11)}
\end{enumerate}
\end{maintheorem}

\begin{proof}[Proof Outline]
See Theorem \ref{thm:verdier-AF-compat} (Section 4.11) for complete proof.
\end{proof}

\subsection{Corollaries and Applications}

\begin{corollary}[Explicit Koszul Pairs]\label{cor:explicit-pairs-intro}
The following are Koszul dual pairs:
\begin{enumerate}
\item Free fermion $\leftrightarrow$ $\beta\gamma$ system
\item Heisenberg $\mathcal{H}_k$ $\leftrightarrow$ DG Symmetric Chiral Algebra (curved)
\item Affine Lie $\widehat{\mathfrak{g}}_k$ $\leftrightarrow$ $CE_{ch}^*(\mathfrak{g})$ 

\item W-algebra $\mathcal{W}_N^{-N}$ $\leftrightarrow$ Wakimoto realization
\end{enumerate}
\end{corollary}

\begin{corollary}[Hochschild Cohomology Computation]\label{cor:hochschild-computation-intro}
For a Koszul pair $(\mathcal{A}, \mathcal{A}^!)$:
$$HH^*(\mathcal{A}) \simeq H^*(\bar{B}(\mathcal{A}), \mathcal{A}) 
\simeq H^*(\mathcal{A}^! \otimes \mathcal{A})$$

Expressed via explicit integration formulas over configuration space integrals.
\end{corollary}

\section{The Arnold Relations: Foundation of Consistency}

\subsection{Discovery and Significance}

This principle, discovered by V.I. Arnold in studying braid groups, is the cornerstone ensuring $d^2 = 0$ for the bar differential. We provide complete proofs in multiple ways—combinatorial, topological, and operadic—establishing this fundamental identity from different points of view. Each approach illuminates different aspects of the underlying geometry.

The Arnold relations state that certain combinations of logarithmic forms vanish identically:

\begin{theorem}[Arnold-Orlik-Solomon Relations - Fundamental]
For logarithmic forms $\eta_{ij} = d\log(z_i - z_j)$ on configuration space, and any subset $S \subset \{1, \ldots, n\}$ with distinct $i, j \notin S$:
$$\sum_{k \in S} (-1)^{|k|} \eta_{ik} \wedge \eta_{kj} \wedge \bigwedge_{l \in S\setminus\{k\}} \eta_{kl} = 0$$
where $|k|$ denotes the position of $k$ in the ordering of $S$.
\end{theorem}

\subsection{Why These Relations Matter}

The Arnold relations are not merely a technical tool—they encode the fundamental consistency of local operator algebras in quantum field theory:

\begin{enumerate}
\item \textbf{Algebraic Consistency}: They ensure the Jacobi identity for the chiral algebra
\item \textbf{Geometric Consistency}: They guarantee that residue extraction is well-defined independent of the order of operations
\item \textbf{Homological Consistency}: They are precisely the condition for $d^2 = 0$ in the bar complex
\item \textbf{Physical Consistency}: They encode the associativity of the operator product expansion
\end{enumerate}

\subsection{Three Perspectives on the Proof}

We establish these relations through three independent proofs, each revealing different aspects:

\textbf{1. Combinatorial Proof (Following Arnold)}:
The relations follow from the elementary identity
$$z_i - z_j = (z_i - z_k) + (z_k - z_j)$$
by taking logarithmic derivatives and carefully tracking the resulting terms. This proof is constructive and yields explicit formulas.

\textbf{2. Topological Proof (Via Stokes' Theorem)}:
Consider the map $S^1 \times C_{|S|}(X) \to C_{|S|+2}(X)$ given by placing points $i$ and $j$ on a small circle. Applying Stokes' theorem to appropriate forms on this space yields the Arnold relations as boundary contributions.

\textbf{3. Operadic Proof (Higher Structure)}:
The configuration space naturally forms an operad with composition given by inserting configurations. The condition that this operad is a complex (has differential squaring to zero) is precisely the Arnold relations.

Complete detailed proofs are provided in Appendix A, with computational examples for small values of $|S|$.

\section{Chiral Hochschild Cohomology and Deformation Theory}

\subsection{From Classical to Chiral}

In classical algebra, Hochschild cohomology controls deformations. For chiral algebras, we have an enriched theory:

\begin{definition}[Chiral Hochschild Complex]
For a chiral algebra $\mathcal{A}$ on a smooth curve $X$, the chiral Hochschild complex is:
$$CH^*(\mathcal{A}) = \text{RHom}_{\mathcal{D}_X}(\bar{B}^{\text{geom}}(\mathcal{A}), \mathcal{A})$$
with differential combining chiral operations and the de Rham differential.
\end{definition}

The geometric realization through our bar construction gives:
$$CH^n(\mathcal{A}) \cong H^n\left(\bar{B}^{\text{geom}}(\mathcal{A}) \otimes_{\mathcal{A}} \mathcal{A}\right)$$

\begin{theorem}[Deformation-Obstruction Theory]
The chiral Hochschild cohomology controls:
\begin{enumerate}
\item $CH^0(\mathcal{A})$ = center of $\mathcal{A}$ (conserved charges in physics)
\item $CH^1(\mathcal{A})$ = infinitesimal deformations (symmetry generators)
\item $CH^2(\mathcal{A})$ = obstructions to extending deformations (marginal operators)
\item $CH^3(\mathcal{A})$ = obstructions to associativity of deformed product
\end{enumerate}
\end{theorem}

\subsection{Periodicity Phenomena}

A remarkable feature of chiral algebras is the appearance of periodicity:

\begin{theorem}[Periodicity in Cohomology]
For certain chiral algebras, the Hochschild cohomology exhibits periodicity:
\begin{enumerate}
\item \textbf{Virasoro}: $CH^{n+2}(\text{Vir}_c) \cong CH^n(\text{Vir}_c) \otimes H^2(\mathcal{M}_{g,n})$ 
\item \textbf{Affine Kac-Moody}: $CH^{n+2h^\vee}(\widehat{\mathfrak{g}}_k) \cong CH^n(\widehat{\mathfrak{g}}_k)$ at critical level
\item \textbf{W-algebras}: Period determined by the principal grading
\end{enumerate}
\end{theorem}

This periodicity reflects hidden structure from the point of view of the genus $0$ theory—the cohomology classes correspond to modular forms of specific weights, with periodicity arising from representation theory of $\text{SL}_2(\mathbb{Z})$.



\subsection{The Non-Abelian Poincaré Perspective}

\begin{remark}[NAP View of Bar-Cobar]\label{rem:NAP-bar-cobar}
From the non-abelian Poincaré duality perspective, bar and cobar constructions are manifestations of orientation reversal on curves:

\textbf{Bar Construction:}
$$\bar{B}^{\text{ch}}(\mathcal{A}): X \mapsto \int_X \mathcal{A}$$
computes factorization homology in the standard orientation.

\textbf{Cobar Construction:}
$$\Omega^{\text{ch}}(\mathcal{C}): X \mapsto \int_{-X} \mathcal{C}$$
computes factorization homology in the opposite orientation.

\textbf{Koszul Duality:}
The relationship $\mathcal{A}_1 \xleftrightarrow{\text{Koszul}} \mathcal{A}_2$ means:
$$\int_X \mathcal{A}_1 \simeq \mathbb{D}\left(\int_{-X} \mathcal{A}_2\right)$$

Orientation reversal is the geometric manifestation of Koszul duality!
\end{remark}


\begin{insight}[Grothendieck's Functorial View]\label{insight:grothendieck-NAP}
From an abstract perspective, non-abelian Poincaré duality is an expression of functoriality:

\begin{center}
\begin{tikzcd}
\text{Oriented manifolds} \arrow[r, "\int"] \arrow[d, "\text{reverse}"'] & \text{Spectra} \arrow[d, "\mathbb{D}"] \\
\text{Opposite orientation} \arrow[r, "\int"] & \text{Dual spectra}
\end{tikzcd}
\end{center}

The entire structure is determined by functoriality and the duality functor $\mathbb{D}$.
\end{insight}

\begin{theorem}[Geometric Bar-Cobar Duality]\label{thm:geometric-bar-cobar}
For a chiral Koszul pair $(\mathcal{A}_1, \mathcal{A}_2)$ on a smooth curve $X$, our geometric constructions establish the duality:
\begin{enumerate}
\item \textbf{Bar construction witness:}
$$\bar{B}^{\text{geom}}(\mathcal{A}_1) \simeq \mathcal{A}_2^! \quad \text{as chiral coalgebras}$$

\item \textbf{Cobar reconstruction witness:}
$$\Omega^{\text{geom}}(\mathcal{A}_2^!) \simeq \mathcal{A}_1 \quad \text{as chiral algebras}$$

\item \textbf{Geometric realization:} The equivalence is realized by Verdier duality:
$$\mathbb{D}_{\overline{C}_*(X)}: \Omega^*_{\log}(\overline{C}_*(X)) \xrightarrow{\sim} \Omega^{d-*}_{\text{dist}}(C_*(X))$$
exchanging logarithmic forms (bar) with distributions (cobar).
\end{enumerate}

\textbf{Non-Abelian Poincaré Interpretation:}
This theorem realizes non-abelian Poincaré duality for the curve X with coefficients in the factorization algebra $\mathcal{A}_1$. The bar construction computes factorization homology; Verdier duality implements the NAP isomorphism.
\end{theorem}

\section{Criteria for Existence of Koszul Duals}

Not every chiral algebra admits a Koszul dual. We establish precise criteria:

\begin{theorem}[Existence Criterion for Koszul Duality]
A chiral algebra $\mathcal{A}$ admits a Koszul dual if and only if:
\begin{enumerate}
\item \textbf{Finite generation}: $\mathcal{A}$ is finitely generated as a $\mathcal{D}_X$-module
\item \textbf{Formal smoothness}: $\dim CH^n(\mathcal{A}) < \infty$ for each $n$
\item \textbf{Poincaré duality}: There exists a non-degenerate pairing
   $$CH^i(\mathcal{A}) \times CH^{d-i}(\mathcal{A}) \to \omega_X$$
   for some dimension $d$
\item \textbf{Convergence}: The bar spectral sequence 
   $$E_1^{p,q} = H^q(C_{p+1}(X), \mathcal{A}^{\boxtimes(p+1)}) \Rightarrow H^{p+q}(\bar{B}(\mathcal{A}))$$
   converges
\end{enumerate}
\end{theorem}


\subsection{The Fundamental Bar-Cobar Relationship}

The central result of this monograph is making precise the relationship between chiral algebras in a Koszul pair. We establish not merely that they are "dual" in some abstract sense, but rather that their bar and cobar constructions provide explicit, mutually inverse transformations.

\begin{theorem}[Extended Koszul Duality, Theorem 4.3]
For a chiral Koszul pair $(\mathcal{A}_1, \mathcal{A}_2)$ of chiral algebras, we establish:

\medskip
\noindent\textbf{I. The Bar-Cobar Isomorphism:}
\begin{enumerate}
\item \textbf{Bar transforms algebra to dual coalgebra:}
   $$\bar{B}^{\text{ch}}(\mathcal{A}_1) \simeq \mathcal{A}_2^! \quad \text{and} \quad \bar{B}^{\text{ch}}(\mathcal{A}_2) \simeq \mathcal{A}_1^!$$
   as quasi-isomorphisms of chiral coalgebras.

\item \textbf{Cobar reconstructs the dual algebra:}
   $$\Omega^{\text{ch}}(\mathcal{A}_2^!) \simeq \mathcal{A}_1 \quad \text{and} \quad \Omega^{\text{ch}}(\mathcal{A}_1^!) \simeq \mathcal{A}_2$$
   as quasi-isomorphisms of chiral algebras.

\item \textbf{Composition gives quasi-isomorphisms to identity:}
   $$\Omega^{\text{ch}}(\bar{B}^{\text{ch}}(\mathcal{A}_i)) \xrightarrow{\sim} \mathcal{A}_i, \quad \bar{B}^{\text{ch}}(\Omega^{\text{ch}}(\mathcal{A}_i^!)) \xrightarrow{\sim} \mathcal{A}_i^!$$
   for $i = 1, 2$, establishing that bar and cobar are quasi-inverse equivalences.
\end{enumerate}

\medskip
\noindent\textbf{II. How Structures Correspond:}
\begin{enumerate}
\item \textbf{Generators and relations interchange:}
   \begin{itemize}
   \item Generating fields of $\mathcal{A}_1$ correspond to relations of $\mathcal{A}_2$
   \item Relations of $\mathcal{A}_1$ correspond to generating fields of $\mathcal{A}_2$
   \item This explains the slogan: "strong coupling $\leftrightarrow$ weak coupling"
   \end{itemize}

\item \textbf{Algebraic operations correspond to coalgebraic operations:}
   \begin{itemize}
   \item Chiral product $\mu: \mathcal{A}_1 \otimes \mathcal{A}_1 \to \mathcal{A}_1$ corresponds to coproduct $\Delta: \mathcal{A}_2^! \to \mathcal{A}_2^! \otimes \mathcal{A}_2^!$
   \item Higher multiplications $m_n$ correspond to higher comultiplications $\Delta_n$
   \item Associativity of products becomes coassociativity of coproducts
   \end{itemize}

\item \textbf{OPE pole orders encode coproduct terms:}
   \begin{itemize}
   \item An OPE singularity $\phi_1(z)\phi_2(w) \sim \frac{a}{(z-w)^k}$ in $\mathcal{A}_1$ becomes a coproduct term in $\mathcal{A}_2^!$
   \item The residue map $\text{Res}_{z=w}$ extracts coproduct coefficients from OPE data
   \item Distribution-valued correlators in $\mathcal{A}_2$ reconstruct OPE structure of $\mathcal{A}_1$
   \end{itemize}
\end{enumerate}

\medskip
\noindent\textbf{III. Geometric Realization:}

The abstract isomorphisms are realized geometrically through configuration space integration:

\begin{enumerate}
\item \textbf{Perfect pairing via integration:}
   $$\langle \omega_{\text{bar}}, K_{\text{cobar}} \rangle = \int_{\overline{C}_n(X)} \omega_{\text{bar}} \wedge \iota^* K_{\text{cobar}}$$
   where $\omega_{\text{bar}} \in \bar{B}^{\text{ch}}(\mathcal{A}_1)$ is a logarithmic form, $K_{\text{cobar}} \in \Omega^{\text{ch}}(\mathcal{A}_2^!)$ is a distribution-valued kernel, and $\iota: C_n(X) \hookrightarrow \overline{C}_n(X)$ is the inclusion of open into compactified configuration space.

\item \textbf{Residues extract coalgebra structure:}
   The differential on the bar side:
   $$d_{\text{bar}} = \sum_{D \in \text{Bdry}} (-1)^{|D|} \text{Res}_D$$
   computes coproduct operations by extracting residues at collision divisors.

\item \textbf{Distributions reconstruct algebra structure:}
   The differential on the cobar side:
   $$d_{\text{cobar}} = \sum_{i<j} \Delta_{ij} \cdot \delta(z_i - z_j)$$
   reconstructs products by inserting distributional singularities.
\end{enumerate}

\medskip
\noindent\textbf{IV. Extensions:}
\begin{enumerate}
\item \textbf{Curved algebras:} The duality extends to curved $A_\infty$ structures with curvature $\kappa \in \mathcal{A}^{\otimes 2}[2]$ satisfying the Maurer-Cartan equation

\item \textbf{Filtered structures:} Koszul pairs of filtered chiral algebras satisfy graded duality at each filtration level

\item \textbf{Higher genus corrections:} At genus $g \geq 1$, quantum corrections enter through period integrals, with complementary deformation-obstruction spaces
\end{enumerate}
\end{theorem}





We further establish a fundamental relationship between Koszul duality and quantum corrections:

\begin{theorem}[Koszul Complementarity, Theorem 6.5.1]
For a Koszul dual pair $(\mathcal{A}, \mathcal{A}^!)$ of chiral algebras on a genus $g$ surface, the spaces of quantum corrections to the Arnold relations satisfy:
$$\mathcal{Q}_g(\mathcal{A}) \oplus \mathcal{Q}_g(\mathcal{A}^!) \cong H^*(\overline{\mathcal{M}}_{g,n}, \mathbb{C})$$
\end{theorem}

This reveals that Koszul dual chiral algebras have complementary quantum corrections—what one algebra sees as a deformation, its dual sees as an obstruction, and vice versa. This provides a complete classification of quantum corrections through Koszul duality and explains the deep relationship between bosonic and fermionic theories in physics.



\section{Concrete Computational Power}

Throughout the paper we utilize the principle that chiral algebraic structures naturally live on configuration spaces, with the bar-cobar construction providing the dictionary between algebraic and geometric perspectives. This geometric realization transforms abstract algebraic computations into concrete integrations that can be explicitly performed. 

We compute concrete examples that demonstrate the full power of our approach:
\begin{itemize}
\item \textbf{The Heisenberg vertex algebra}: We show how the central extension appears geometrically from the failure of logarithmic forms to satisfy exact Arnold relations at genus one
\item \textbf{Free fermions and boson-fermion correspondence}: The bar complex of free fermions is quasi-isomorphic to the Koszul dual coalgebra of symplectic bosons, $\bar{B}^{\text{ch}}(\text{fermions}) \simeq (\text{bosons})^!$, while the cobar construction establishes the inverse relationship $\Omega^{\text{ch}}((\text{fermions})^!) \simeq \text{bosons}$, realizing boson-fermion duality geometrically through the bar-cobar duality adjunction
\item \textbf{$\beta\gamma$ systems}: Complete computation through degree 5, with explicit Koszul dual identification
\item \textbf{W-algebras at critical level}: The bar complex simplifies dramatically, with differential given entirely by screening charges
\item \textbf{Affine Kac-Moody algebras}: We compute their bar complexes and show how quantum deformations arise from higher genus contributions
\end{itemize}

Each example is worked out completely, with all differentials computed explicitly and cohomology determined.