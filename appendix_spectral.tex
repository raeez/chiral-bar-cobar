\section{Spectral Sequences for Higher Genus}

\subsection{The Hodge-to-de Rham Spectral Sequence}

For the universal curve $\pi: \mathcal{C}_g \to \mathcal{M}_g$:

$E_1^{p,q} = H^q(\mathcal{M}_g, R^p\pi_*\Omega_{\mathcal{C}_g/\mathcal{M}_g}) \Rightarrow H^{p+q}_{\text{dR}}(\mathcal{C}_g)$

The differentials encode:
\begin{itemize}
\item $d_1$: Gauss-Manin connection
\item $d_2$: Kodaira-Spencer map  
\item $d_r$ ($r \geq 3$): Higher deformations
\end{itemize}

\subsection{The Bar Complex Spectral Sequence}

$E_2^{p,q} = H^p(\overline{\mathcal{M}}_{g,n}, \underline{H}^q(\bar{B}^{(g)}(\mathcal{A}))) \Rightarrow H^{p+q}(\bar{B}^{\text{total}}(\mathcal{A}))$

where $\underline{H}^q$ denotes the local system of bar cohomology groups.

\subsection{Convergence and Degeneration}

\begin{theorem}[Convergence Criterion]
The spectral sequence converges if:
\begin{enumerate}
\item The chiral algebra $\mathcal{A}$ is rational (finitely many irreps)
\item The genus expansion parameter satisfies $|g_s| < \epsilon(\mathcal{A})$
\item The moduli space $\overline{\mathcal{M}}_{g,n}$ is replaced by its Deligne-Mumford compactification
\end{enumerate}
\end{theorem}

\begin{theorem}[Degeneration at $E_2$]
For special values of central charge:
\begin{itemize}
\item $c = 0$: Topological theory, degenerates at $E_1$
\item $c = 26$: Critical bosonic string, degenerates at $E_2$  
\item $c = 15$: Critical superstring, degenerates at $E_2$
\end{itemize}
\end{theorem}

\subsection{Connection to Genus Expansion and Feynman Diagrams}

\begin{theorem}[Spectral Sequence = Genus Expansion]\label{thm:ss-genus}
The spectral sequence computing $H^*(\bar{B}(\mathcal{A}))$ has pages corresponding to genus contributions:

\textbf{Page $E_r$}: Contributions from graphs/diagrams with up to $(r-1)$ loops.

More precisely:
\begin{itemize}
\item $E_1^{p,q}$: Tree-level (genus 0), no loops
\item $E_2^{p,q}$: One-loop corrections (genus 1)  
\item $E_3^{p,q}$: Two-loop corrections (genus 2)
\item $E_\infty^{p,q}$: Sum over all genera
\end{itemize}

\textbf{Physical interpretation}:
$$E_r \approx \text{Feynman graphs with } \leq (r-1) \text{ loops}$$

The differentials $d_r$ implement quantum corrections by integrating over moduli spaces of curves:
$$d_r: E_r^{p,q} \to E_r^{p+r,q-r+1}$$
corresponds to
$$\int_{\overline{\mathcal{M}}_{g,n}} \omega_{\text{correlator}}$$
where $g = r-1$ is the genus.
\end{theorem}

\begin{proof}[Sketch]
The filtration on $\bar{B}(\mathcal{A})$ by pole order along collision divisors corresponds to the loop expansion:

- Order-0 pole: tree diagram (no internal lines)
- Order-1 pole: one internal loop  
- Order-$n$ pole: $n$ internal loops

The $E_r$ page consists of equivalence classes of diagrams modulo those with $<r$ loops.
\end{proof}

\subsection{Computational Tools}

The differentials can be computed via:
\begin{enumerate}
\item \textbf{Čech cohomology:} Cover $\overline{\mathcal{M}}_{g,n}$ by affine opens
\item \textbf{Dolbeault cohomology:} Use $\bar{\partial}$-operator techniques
\item \textbf{Combinatorial models:} Jenkins-Strebel differentials
\item \textbf{Topological recursion:} Eynard-Orantin formalism
\end{enumerate}

\subsection{Spectral Sequence for Bar Complex}

\begin{theorem}[Bar Spectral Sequence]
The filtration by configuration degree yields a spectral sequence:
$$E_1^{p,q} = H^q(\overline{C}_{p+1}(X), j_*j^*\mathcal{A}^{\boxtimes(p+1)}) \Rightarrow H^{p+q}(\bar{B}^{\text{ch}}(\mathcal{A}))$$

\textbf{Key Properties:}
\begin{enumerate}
\item $E_2$ page: Computed by residues at boundary divisors
\item Convergence: Always for finite-type chiral algebras
\item Degeneration: At $E_2$ for Koszul algebras (quadratic with no higher relations)
\item Differential $d_r$: Encodes $(r+1)$-fold collisions
\end{enumerate}

\textbf{Application to Free Fermions:}
\begin{itemize}
\item $E_1^{p,0} = \wedge^p(\mathcal{F} \otimes H^0(X, \omega_X))$ 
\item $d_1 = 0$ (no relations beyond anticommutativity)
\item Collapses at $E_1 = E_{\infty}$
\item Recovers $\bar{B}^{\text{ch}}(\mathcal{F}) = \wedge^{\bullet}(\mathcal{F}[1])$
\end{itemize}

\textbf{Application to W-algebras:}
For $\mathcal{W}_k(\mathfrak{g}, f)$ at admissible level:
\begin{itemize}
\item $E_1$: Free generators from W-currents
\item $E_2$: Normal ordered products and null fields
\item $E_3$: Quantum corrections from BRST cohomology
\item Convergence requires careful analysis of Virasoro representations
\end{itemize}
\end{theorem}

\begin{example}[Computing $E_2$ Page]
For a chiral algebra with generators $\phi_i$ of conformal weight $h_i$:

$$E_2^{p,q} = \frac{\text{Ker}(d_1: E_1^{p,q} \to E_1^{p+1,q})}{\text{Im}(d_1: E_1^{p-1,q} \to E_1^{p,q})}$$

where $d_1$ is computed from OPE residues:
$$d_1(\phi_{i_1} \otimes \cdots \otimes \phi_{i_p}) = \sum_{j<k} \sum_\ell C_{i_j i_k}^\ell \phi_{i_1} \otimes \cdots \widehat{i_j} \cdots \widehat{i_k} \cdots \otimes \phi_\ell$$
\end{example}

\begin{remark}[Physical Interpretation]
In string theory:
\begin{itemize}
\item $E_1$: Off-shell string states
\item $d_1$: BRST operator
\item $E_2$: Physical (on-shell) states
\item Higher pages: Quantum corrections and anomalies
\end{itemize}
\end{remark}
