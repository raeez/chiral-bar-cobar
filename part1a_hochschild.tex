\chapter{Chiral Hochschild Cohomology and Deformation Theory}

\section{Classical to Chiral}

\subsection{Review of Classical Hochschild}

For an associative algebra $A$ over $\mathbb{C}$, the Hochschild cohomology $HH^*(A, M)$ with coefficients in an $A$-bimodule $M$ is computed by:

$$HH^n(A, M) = \text{Ext}^n_{A \otimes A^{op}}(A, M)$$

The bar resolution provides the computational tool:
$$\cdots \to A \otimes A \otimes A \xrightarrow{b} A \otimes A \xrightarrow{b} A$$
where $b(a_0 \otimes \cdots \otimes a_{n+1}) = \sum_{i=0}^n (-1)^i a_0 \otimes \cdots \otimes a_ia_{i+1} \otimes \cdots \otimes a_{n+1}$.

\subsection{Chiral Enhancement}

For chiral algebras, the situation is richer due to:
\begin{itemize}
\item Locality constraints from OPE
\item Geometric structure from the curve $X$
\item Higher operations from $A_\infty$ structure
\end{itemize}

\begin{definition}[Chiral Hochschild Complex]
For a chiral algebra $\mathcal{A}$ on $X$, the chiral Hochschild complex is:
$$\ChirHoch^*(\mathcal{A}, \mathcal{M}) = \text{RHom}_{\mathcal{D}_X}(\barBgeom(\mathcal{A}), \mathcal{M})$$
where $\mathcal{M}$ is a chiral $\mathcal{A}$-module.
\end{definition}

\begin{theorem}[Comparison with Classical]
There is a spectral sequence:
$$E_2^{p,q} = HH^p(\mathcal{A}_0, H^q(\Omega^*_X)) \Rightarrow \ChirHoch^{p+q}(\mathcal{A})$$
where $\mathcal{A}_0$ is the fiber at a point.
\end{theorem}

\section{Periodicity Phenomena}

\subsection{Virasoro Periodicity}

\begin{theorem}[Virasoro Hochschild Cohomology]
For the Virasoro algebra at central charge $c$:
$$\ChirHoch^{n+2}(\text{Vir}_c) \cong \ChirHoch^n(\text{Vir}_c) \otimes H^2(\mathcal{M}_{g,n})$$
The period is 2, reflecting the conformal weight of the stress tensor.
\end{theorem}

\begin{proof}
The stress tensor $T$ has weight 2. The multiplication by $T$ induces:
$$\cup T: \ChirHoch^n \to \ChirHoch^{n+2}$$
At generic $c$, this is an isomorphism for $n \geq 2$.
\end{proof}

\subsection{Affine Kac-Moody Periodicity}

\begin{theorem}[Critical Level Periodicity]
For $\widehat{\mathfrak{g}}_k$ at critical level $k = \critLevel$:
$$\ChirHoch^{n+2h^\vee}(\widehat{\mathfrak{g}}_{\critLevel}) \cong \ChirHoch^n(\widehat{\mathfrak{g}}_{\critLevel})$$
where $h^\vee$ is the dual Coxeter number.
\end{theorem}

This periodicity arises from the center at critical level being large (Feigin-Frenkel center).

\subsection{W-algebra Periodicity}

For W-algebras, the periodicity depends on the principal grading:

\begin{theorem}[W-algebra Cohomology]
$$\ChirHoch^*(\Walg^k(\mathfrak{g}, f)) = \bigoplus_{j \in \mathbb{Z}/d\mathbb{Z}} \ChirHoch^*_j$$
where $d$ is determined by the nilpotent orbit of $f$.
\end{theorem}

\section{Deformation Theory}

\subsection{Infinitesimal Deformations}

\begin{theorem}[Deformation Classification]
\begin{enumerate}
\item $\ChirHoch^1(\mathcal{A})$ parametrizes infinitesimal deformations
\item $\ChirHoch^2(\mathcal{A})$ contains obstructions
\item Unobstructed deformations correspond to marginal operators in CFT
\end{enumerate}
\end{theorem}

\begin{example}[Marginal Deformations of $\beta\gamma$]
For the $\beta\gamma$ system:
$$\ChirHoch^2(\beta\gamma) = \mathbb{C} \cdot [\beta\gamma]$$
The class $[\beta\gamma]$ corresponds to the exactly marginal operator changing the conformal weights.
\end{example}

\subsection{Formal Deformation Theory}

The formal deformation space is controlled by the differential graded Lie algebra:
$$\mathfrak{g}_{\mathcal{A}} = \ChirHoch^*(\mathcal{A}, \mathcal{A})[1]$$
with bracket induced by the cup product.

\begin{theorem}[Maurer-Cartan Equation]
Formal deformations correspond to solutions of:
$$d\alpha + \frac{1}{2}[\alpha, \alpha] = 0, \quad \alpha \in \mathfrak{g}_{\mathcal{A}}^1$$
\end{theorem}

\section{Physical Applications}

\subsection{Marginal Operators and RG Flow}

In CFT, marginal operators have dimension $(1,1)$. They correspond to:
$$\ChirHoch^2_{\text{marginal}}(\mathcal{A}) = \{\omega \in \ChirHoch^2 : h(\omega) = 1\}$$

The beta function vanishes iff the obstruction in $\ChirHoch^3$ vanishes.

\subsection{String Field Theory}

The bar complex computes the BRST cohomology:
$$H^*_{\text{BRST}}(\text{String}[\mathcal{A}]) \cong \ChirHoch^*(\mathcal{A})$$

String vertices are encoded in the $A_\infty$ structure:
\begin{itemize}
\item $m_2$: Three-string vertex
\item $m_3$: Four-string contact term
\item Higher $m_k$: Multi-string interactions
\end{itemize}

\section{Computational Tools}

\subsection{Spectral Sequences}

The bar complex induces several spectral sequences:

\begin{theorem}[Bar Spectral Sequence]
$$E_1^{p,q} = H^q(\ConfigSpace{p}(X), \mathcal{A}^{\boxtimes p}) \Rightarrow \ChirHoch^{p+q}(\mathcal{A})$$
\end{theorem}

\subsection{Explicit Computations}

For the Heisenberg algebra:
$$\ChirHoch^n(\text{Heis}) = \begin{cases}
\mathbb{C}[c] & n = 0 \text{ (center)} \\
0 & n = 1 \\
\mathbb{C} & n = 2 \text{ (central extension)} \\
0 & n \geq 3
\end{cases}$$

For free fermions:
$$\ChirHoch^*(\text{Fermions}) = \Lambda^*[\xi_1, \xi_2]$$
reflecting the fermionic nature.