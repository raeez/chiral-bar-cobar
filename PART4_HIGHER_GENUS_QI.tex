
%================================================================
% HIGHER GENUS QUASI-ISOMORPHISM
%================================================================

\section{Bar-Cobar Quasi-Isomorphism at Higher Genus}
\label{sec:bar-cobar-qi-higher-genus}

\begin{theorem}[Higher Genus Inversion]\label{thm:higher-genus-inversion}
The bar-cobar inversion quasi-isomorphism from Theorem \ref{thm:bar-cobar-inversion-qi} 
holds at each genus $g$:
$$\psi_g: \Omega_g(\bar{B}_g(\mathcal{A})) \xrightarrow{\sim} \mathcal{A}_g$$

where $\mathcal{A}_g$ denotes the genus-$g$ component of $\mathcal{A}$ (contributions 
from curves of genus $g$).
\end{theorem}

\begin{proof}
The proof extends the genus-zero result of Beilinson-Drinfeld to all genera.

\textbf{Step 1: Moduli space stratification.}

The moduli space $\overline{\mathcal{M}}_g$ has a natural stratification by stable 
graphs:
$$\overline{\mathcal{M}}_g = \bigcup_{\Gamma} \mathcal{M}_\Gamma$$

Each stratum $\mathcal{M}_\Gamma$ corresponds to curves with a specific degeneracy 
pattern encoded by graph $\Gamma$.

\textbf{Step 2: Induction on strata.}

We prove $\psi_g$ is a quasi-isomorphism by induction on strata (increasing 
complexity of degeneracy):

\textbf{Base case:} The open stratum $\mathcal{M}_g^{\text{smooth}} \subset \overline{\mathcal{M}}_g$ 
(smooth curves). Here:
$$\bar{B}_g^n(\mathcal{A})|_{\mathcal{M}_g^{\text{smooth}}} = 
\int_{\mathcal{M}_g^{\text{smooth}}} \omega_g \wedge \text{(correlation functions)}$$

where $\omega_g$ are the holomorphic $g$-forms from Section \ref{sec:holomorphic-forms-genus-g}.

On smooth curves, the bar-cobar inversion reduces to:
\begin{itemize}
\item Bar = residues at collision divisors
\item Cobar = distributions on diagonals
\item Pairing = residue-distribution duality
\end{itemize}

This is a quasi-isomorphism by Verdier duality (Theorem \ref{thm:verdier-bar-cobar}).

\textbf{Inductive step:} Consider a boundary stratum $\mathcal{M}_\Gamma$ of 
codimension $k$. By inductive hypothesis, $\psi_g$ is a quasi-isomorphism on all 
strata of codimension $< k$.

The restriction to $\mathcal{M}_\Gamma$ factors as:
$$\psi_g|_{\mathcal{M}_\Gamma}: \Omega_g(\bar{B}_g(\mathcal{A}))|_{\mathcal{M}_\Gamma} 
\to \mathcal{A}_g|_{\mathcal{M}_\Gamma}$$

\textbf{Key lemma:} Gluing formulas (Theorem \ref{thm:gluing-formulas-higher-genus}) 
ensure that $\psi_g$ extends across $\mathcal{M}_\Gamma$ as a quasi-isomorphism.

\begin{lemma}[Extension Across Boundary]\label{lem:extension-across-boundary-qi}
If $\psi$ is a quasi-isomorphism on $U$ open, and extends continuously to $\bar{U}$, 
and the gluing formula holds at $\partial U$, then $\psi$ is a quasi-isomorphism 
on $\bar{U}$.
\end{lemma}

\begin{proof}[Proof of Lemma]
Use the long exact sequence in cohomology:
$$\cdots \to H^i(\bar{U}, \mathcal{F}) \to H^i(U, \mathcal{F}) \to 
H^{i+1}_{\partial U}(\bar{U}, \mathcal{F}) \to \cdots$$

If $\psi$ induces isomorphism on $H^i(U)$ and $H^{i+1}_{\partial U}$ (by gluing), 
then by five-lemma, it induces isomorphism on $H^i(\bar{U})$.
\end{proof}

Applying this lemma at each boundary stratum completes the induction.

\textbf{Step 3: Completeness.}

Since $\overline{\mathcal{M}}_g$ is a finite union of strata, and $\psi_g$ is a 
quasi-isomorphism on each stratum, it is a quasi-isomorphism on all of $\overline{\mathcal{M}}_g$.
\end{proof}

