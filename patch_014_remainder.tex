% This file contains the remaining PATCH 014 content to be inserted

\subsection{When Does Filtering Degenerate to Curved?}

\begin{proposition}[Filtered $\Rightarrow$ Curved]\label{prop:filtered-to-curved}
A filtered chiral algebra $\mathcal{A}$ has an associated \textbf{curved structure} if:
\begin{enumerate}
\item The filtration is \textbf{finite-dimensional at each level}: 
$\dim(F_k\mathcal{A}/F_{k-1}\mathcal{A}) < \infty$ for all $k$
\item The associated graded $\text{gr}(\mathcal{A})$ is generated by $\text{gr}^1(\mathcal{A})$
\item All higher relations are \textbf{consequences} of lower ones plus curvature
\end{enumerate}

In this case, the filtered structure \textbf{degenerates} to a curved structure with:
$$\mu_0 \in F_2\mathcal{A}$$
encoding the deviation from quadratic.
\end{proposition}

\begin{example}[Virasoro: Filtered Degenerates to Curved]\label{ex:vir-filtered-to-curved}
The Virasoro algebra can be viewed as:

\textbf{Option 1 - Filtered:}
\begin{align}
F_0\text{Vir} &= \mathbb{C} \cdot \mathbf{1} \\
F_1\text{Vir} &= F_0 \oplus \mathbb{C} \cdot \partial T \\
F_2\text{Vir} &= F_1 \oplus \mathbb{C} \cdot T \\
F_3\text{Vir} &= F_2 \oplus \mathbb{C} \cdot \partial^2 T \\
&\vdots
\end{align}

\textbf{Option 2 - Curved:}
\begin{itemize}
\item Generators: $V = \mathbb{C} \cdot T$
\item Curvature: $\mu_0 = c \cdot \mathbf{1}$
\item Higher ops: $m_3(T \otimes T \otimes T)$ (Schwarzian)
\end{itemize}

\textbf{Why they're equivalent:}
The filtration $F_k$ is generated by $T$ and its derivatives up to order $k-2$. All composite 
operators like $\partial^n T$ are derivatives of the single generator $T$, so the algebra is 
``effectively'' curved rather than truly filtered.

The curvature $\mu_0 = c$ captures the failure of $T$ to be a quadratic generator.
\end{example}

\begin{example}[$W_3$: Truly Filtered, NOT Curved]\label{ex:w3-not-curved}
The $W_3$ algebra is \textbf{genuinely filtered} because:
\begin{itemize}
\item Generators: $L$ (dimension 2) AND $W$ (dimension 3)
\item Composite operators: $(L \cdot L)$, $(L \cdot W)$, etc. appear in OPE
\item These composites are \textbf{not} derivatives of $L$ or $W$
\end{itemize}

Therefore $W_3$ cannot be reduced to a curved algebra with finite-dimensional generators. 
It requires the full filtered framework.

\textbf{Key distinction:}
\begin{itemize}
\item Virasoro: $T$ and all $\partial^n T$ are ``the same'' generator (derivatives)
\item $W_3$: $L$, $W$, and $(L \cdot L)$ are \textbf{independent} generators
\end{itemize}

This is why $W_3$ requires completion while Heisenberg and Kac-Moody do not!
\end{example}

\subsection{Explicit Calculations: Three Examples}

\subsubsection{Heisenberg (Quadratic): No Completion}

\begin{example}[Heisenberg - Explicit Bar Complex]\label{ex:heisenberg-bar-explicit}
For $\mathcal{H}_k$ with generator $J$:

\textbf{Bar complex:}
\begin{align}
\bar{B}^0(\mathcal{H}_k) &= \mathbb{C} \cdot \mathbf{1} \\
\bar{B}^1(\mathcal{H}_k) &= \mathbb{C} \cdot J \\
\bar{B}^2(\mathcal{H}_k) &= \mathbb{C} \cdot (J \otimes J) \\
\bar{B}^3(\mathcal{H}_k) &= \mathbb{C} \cdot (J \otimes J \otimes J) \\
&\vdots
\end{align}

\textbf{Bar differential:}
\begin{align}
d(J) &= 0 \\
d(J \otimes J) &= \text{Res}_{z=w}(J(z)J(w)) = k \cdot \mathbf{1} \\
d(J \otimes J \otimes J) &= \text{Res}(J \otimes J(z)J(w)) + \text{Res}(J(z)J(w) \otimes J) \\
&= k(J \otimes \mathbf{1}) + k(\mathbf{1} \otimes J) = k \cdot J \otimes (2 \text{ ways})
\end{align}

\textbf{Cohomology:}
\begin{align}
H^0(\bar{B}(\mathcal{H}_k)) &= \mathbb{C} \cdot \mathbf{1} \\
H^1(\bar{B}(\mathcal{H}_k)) &= 0 \quad \text{(since $d(J \otimes J) = k \neq 0$)} \\
H^n(\bar{B}(\mathcal{H}_k)) &= 0 \quad \text{for } n \geq 2
\end{align}

\textbf{Koszul dual:}
$$\mathcal{H}_k^! = H^*(\bar{B}(\mathcal{H}_k)) = \mathbb{C} \cdot \mathbf{1} = \text{Sym}^0(V^*)$$

(The dual is just the trivial algebra, since all $J$ products give $k \cdot \mathbf{1}$.)

\textbf{No completion needed!} The bar complex is finite-dimensional at each degree and 
converges immediately.
\end{example}

\subsubsection{Virasoro (Curved): Sometimes Completion}

\begin{example}[Virasoro - Bar Complex Requires Completion]\label{ex:vir-bar-completion}
For $\text{Vir}_c$ with generator $T$:

\textbf{Bar complex (before completion):}
\begin{align}
\bar{B}^0(\text{Vir}) &= \mathbb{C} \cdot \mathbf{1} \\
\bar{B}^1(\text{Vir}) &= \mathbb{C} \cdot T \oplus \mathbb{C} \cdot \partial T \oplus \cdots \\
\bar{B}^2(\text{Vir}) &= (\mathbb{C} \cdot T \oplus \cdots)^{\otimes 2} \\
&\vdots
\end{align}

\textbf{Issue:} The space $\bar{B}^1$ is \textbf{infinite-dimensional} because it includes 
all derivatives $\partial^n T$ for $n \geq 0$.

\textbf{Completion:} Define the augmentation ideal:
$$I = \langle T, \partial T, \partial^2 T, \ldots \rangle$$

Complete with respect to $I$:
$$\widehat{\bar{B}}(\text{Vir}) = \varprojlim_n \bar{B}(\text{Vir})/I^n$$

\textbf{Completed differential:}
$$\widehat{d}(T \otimes T) = \text{Res}(T(z)T(w)) + c \cdot \mathbf{1}$$

The curvature $c$ ensures $\widehat{d}^2 = 0$ on the completed complex.

\textbf{Cohomology:}
$$H^*(\widehat{\bar{B}}(\text{Vir})) = \widehat{U(\text{Vir})}^*_{-c}$$
is the completed dual universal enveloping algebra.

\textbf{Completion essential!} Without completion, the bar complex doesn't converge and the 
Koszul dual is not well-defined.
\end{example}

\subsubsection{$W_3$ (Filtered): Always Completion}

\begin{example}[$W_3$ - Bar Complex Must Be Completed]\label{ex:w3-bar-completion}
For $W_3$ with generators $L$ (dimension 2) and $W$ (dimension 3):

\textbf{Bar complex (before completion):}
\begin{align}
\bar{B}^0(W_3) &= \mathbb{C} \cdot \mathbf{1} \\
\bar{B}^1(W_3) &= \mathbb{C} \cdot L \oplus \mathbb{C} \cdot W \oplus \text{(derivatives and composites)} \\
\bar{B}^2(W_3) &= \text{(all pairs)} \\
&\vdots
\end{align}

\textbf{Problem:} Already at degree 1, we have:
\begin{itemize}
\item Generators: $L$, $W$
\item First derivatives: $\partial L$, $\partial W$
\item Second derivatives: $\partial^2 L$, $\partial^2 W$
\item Composites: $(L \cdot L)$, $(L \cdot W)$, $(W \cdot W)$
\item Higher composites: $(\partial L \cdot L)$, etc.
\end{itemize}

This is \textbf{infinite-dimensional} even before taking products!

\textbf{Filtration:} Filter by total operator dimension:
\begin{align}
F_0 &= \mathbb{C} \cdot \mathbf{1} \\
F_2 &= F_0 \oplus \mathbb{C} \cdot L \\
F_3 &= F_2 \oplus \mathbb{C} \cdot W \oplus \mathbb{C} \cdot \partial L \\
F_4 &= F_3 \oplus \mathbb{C} \cdot \partial W \oplus \mathbb{C} \cdot \partial^2 L 
\oplus \mathbb{C} \cdot (L \cdot L) \\
&\vdots
\end{align}

\textbf{Completed bar complex:}
$$\widehat{\bar{B}}(W_3) = \varprojlim_n \bar{B}(W_3)/F_n$$

\textbf{Completed differential:}
$$\widehat{d}(W \otimes W) = \text{Res}(W(z)W(w)) + \text{(composite terms)} + c \cdot \mathbf{1}$$

The composite terms involve $(L \cdot L)$ and higher, which are not in the span of $\{L, W\}$.

\textbf{Cohomology:}
$$H^*(\widehat{\bar{B}}(W_3)) = \widehat{\text{CoW}_3}$$
is the completed cooperad structure dual to $W_3$.

\textbf{Completion absolutely essential!} Without it, the bar construction doesn't even make sense.
\end{example}

\subsection{Convergence Criteria}

\begin{theorem}[Convergence of Bar Construction]\label{thm:bar-convergence}
For a chiral algebra $\mathcal{A}$, the bar construction $\bar{B}(\mathcal{A})$ converges 
(without completion) if and only if:
\begin{enumerate}
\item $\dim(\bar{B}^n(\mathcal{A})) < \infty$ for all $n$
\item $\lim_{n \to \infty} \dim(\bar{B}^n(\mathcal{A}))^{1/n} < \infty$ (growth condition)
\item The differential $d$ preserves the grading
\end{enumerate}

\textbf{Sufficient condition:} $\mathcal{A}$ is quadratic.

\textbf{Necessary completion:} If any condition fails, must complete $\widehat{\bar{B}}(\mathcal{A})$.
\end{theorem}

\subsection{Physical Interpretation}

\subsubsection{From Witten's Perspective}

\textbf{Quadratic algebras} correspond to \textbf{free field theories}:
\begin{itemize}
\item Heisenberg $\leftrightarrow$ Free boson
\item Kac-Moody $\leftrightarrow$ WZW model (free fermions in Lie algebra)
\end{itemize}

\textbf{Curved algebras} correspond to \textbf{interacting theories with anomalies}:
\begin{itemize}
\item Virasoro $\leftrightarrow$ Conformal anomaly in 2D gravity
\item Central charge $c$ measures quantum breaking of scale invariance
\end{itemize}

\textbf{Filtered algebras} correspond to \textbf{theories with composite operators}:
\begin{itemize}
\item $W_3$ $\leftrightarrow$ Toda field theory (non-linear interactions)
\item Composite operators $(L \cdot L)$ arise from operator products
\end{itemize}

\textbf{General algebras} correspond to \textbf{non-local theories}:
\begin{itemize}
\item $W_\infty$ $\leftrightarrow$ 2D gravity with infinitely many fields
\item No local Lagrangian description
\end{itemize}

\subsubsection{From Kontsevich's Geometric Viewpoint}

The filtration level corresponds to \textbf{codimension of collision loci}:

\begin{itemize}
\item \textbf{Quadratic}: Only pairwise collisions $(z_i = z_j)$ contribute
\item \textbf{Curved}: Central terms from $n$-point collisions on $S^1$
\item \textbf{Filtered}: Higher codimension strata in configuration space
\item \textbf{General}: Configuration space is not well-behaved
\end{itemize}

The completion $\widehat{\bar{B}}(\mathcal{A})$ is the \textbf{formal neighborhood} of the 
diagonal in configuration space!

\subsection{Summary and Decision Tree}

\begin{remark}[Takeaway for Practitioners]\label{rem:practitioner-takeaway}
\textbf{Before computing Koszul dual, always ask:}
\begin{enumerate}
\item Is my algebra quadratic? $\Rightarrow$ Proceed directly
\item Is it curved with central curvature? $\Rightarrow$ Check if $\dim(\bar{B}^1) < \infty$
  \begin{itemize}
  \item If yes: No completion
  \item If no: Complete!
  \end{itemize}
\item Does it have composite operators? $\Rightarrow$ Must complete
\item Is the generating space infinite-dimensional? $\Rightarrow$ May not have Koszul dual
\end{enumerate}

\textbf{Most vertex algebras from CFT are either quadratic or curved with finite-dimensional 
$\bar{B}^1$, so Koszul duality works!}
\end{remark}

