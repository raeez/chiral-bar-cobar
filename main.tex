% Choose compilation method:
% Option 1: For XeLaTeX/LuaLaTeX with true Adobe Garamond Pro (commercial font)
% \documentclass[11pt]{article}
% Option 2: For pdfLaTeX with EB Garamond (excellent free alternative)
\documentclass[11pt]{article}

% ==========================================
% OPTION 1: XeLaTeX/LuaLaTeX with Adobe Garamond Pro
% Uncomment this section if you have Adobe Garamond Pro installed
% and compile with 


% ==========================================
% \usepackage{fontspec}
% \usepackage{unicode-math}
% 
% % Main text font with optimal settings
% \setmainfont{Adobe Garamond Pro}[
%   Ligatures = {Common, TeX},
%   Numbers = {OldStyle, Proportional},
%   Scale = 1.0,
%   Kerning = On,
%   BoldFont = {Adobe Garamond Pro Bold},
%   ItalicFont = {Adobe Garamond Pro Italic},
%   BoldItalicFont = {Adobe Garamond Pro Bold Italic},
%   SmallCapsFeatures = {Letters = SmallCaps},
%   OpticalSize = Auto
% ]
% 
% % Mathematics font (Garamond-compatible)
% \setmathfont{Garamond-Math.otf}[
%   Scale = MatchLowercase,
%   StylisticSet = {1,2,3,4,5,6,7}
% ]
% 
% % Alternative math font if Garamond-Math not available
% % \setmathfont{TeX Gyre Termes Math}[Scale=MatchLowercase]

% ==========================================
% OPTION 2: pdfLaTeX with EB Garamond (recommended free alternative)
% This provides excellent quality and works with standard pdfLaTeX
% ==========================================
\usepackage[T1]{fontenc}
\usepackage[utf8]{inputenc}

% EB Garamond with full features
\usepackage[
  cmintegrals,
  cmbraces,
  ebgaramond
]{newtxmath}
\usepackage{ebgaramond}

% Fine-tuning for optimal appearance
\usepackage[
  activate={true,nocompatibility},
  final,
  tracking=true,
  kerning=true,
  spacing=true,
  factor=1100,
  stretch=10,
  shrink=10
]{microtype}

% Enhanced kerning for specific pairs
\SetExtraKerning[unit=space]
  {encoding={*}, family={*}, series={*}, size={footnotesize,small,normalsize}}
  {\textemdash={400,400}, % Em dash spacing
   "28={,150}, % Left parenthesis
   "29={150,}, % Right parenthesis  
   \textquotedblleft={,150},
   \textquotedblright={150,}}

% Improved math spacing
\usepackage{mleftright}
\mleftright
% Add necessary packages for rigorous mathematics
\usepackage{amsmath,amssymb,amsthm}
\usepackage{tikz-cd}
\usepackage{hyperref}
\usepackage{amsfonts}
\usepackage{tikz}
\usepackage[margin=1in]{geometry}
\usepackage{algorithm}
\usepackage{algpseudocode}  % or \usepackage{algorithmic} for the older version
\usepackage{tcolorbox}

% ==========================================
% MATHEMATICAL TYPOGRAPHY ENHANCEMENTS
% ==========================================

% Better spacing in math mode
\usepackage{mathtools}
\mathtoolsset{showonlyrefs,showmanualtags}

% Improved theorem environments with Garamond-appropriate spacing
\usepackage{thmtools}
\declaretheoremstyle[
  spaceabove=\topsep,
  spacebelow=\topsep,
  headfont=\normalfont\scshape,
  notefont=\normalfont\itshape,
  bodyfont=\normalfont,
  postheadspace=0.5em,
  headpunct={.}
]{garamondthm}

\declaretheoremstyle[
  spaceabove=\topsep,
  spacebelow=\topsep,
  headfont=\normalfont\itshape,
  notefont=\normalfont\itshape,
  bodyfont=\normalfont,
  postheadspace=0.5em,
  headpunct={.}
]{garamonddef}



% Orientation line used in §6 (keeps your existing symbol but defines it cleanly).
\newcommand{\orline}[1]{\mathrm{or}_{#1}}


% Define theorem environments properly
% Updated theorem styles for Garamond
\declaretheorem[
  style=garamondthm,
  name=Theorem,
  numberwithin=section
]{theorem}
\declaretheorem[
  style=garamondthm,
  name=Lemma,
  sibling=theorem
]{lemma}
\declaretheorem[
  style=garamondthm,
  name=Proposition,
  sibling=theorem
]{proposition}
\declaretheorem[
  style=garamondthm,
  name=Corollary,
  sibling=theorem
]{corollary}

\declaretheorem[style=garamonddef, name=Definition, sibling=theorem]{definition}
\declaretheorem[style=garamonddef, name=Example, sibling=theorem]{example}
\declaretheorem[style=garamonddef, name=Remark, sibling=theorem]{remark}
\declaretheorem[style=garamonddef, name=Conjecture, sibling=theorem]{conjecture}
\declaretheorem[style=garamonddef, name=Notation, sibling=theorem]{notation}
\declaretheorem[style=garamonddef, name=Convention, sibling=theorem]{convention}
\declaretheorem[style=garamonddef, name=Spectral Sequence, sibling=theorem]{spectralsequence}


 
% Essential operators and symbols
\DeclareMathOperator{\Hom}{Hom}
\DeclareMathOperator{\End}{End}
\DeclareMathOperator{\Res}{Res}
\DeclareMathOperator{\Ind}{Ind}
\DeclareMathOperator{\colim}{colim}
\DeclareMathOperator{\Ran}{Ran}
\DeclareMathOperator{\sgn}{sgn}
\DeclareMathOperator{\Free}{Free}
\DeclareMathOperator{\Cofree}{Cofree}
\DeclareMathOperator{\Com}{Com}
\DeclareMathOperator{\Lie}{Lie}
\DeclareMathOperator{\Spec}{Spec}
 
% Custom commands for clarity and consistency
\newcommand{\C}{\mathbb{C}}
\newcommand{\Z}{\mathbb{Z}}
\newcommand{\R}{\mathbb{R}}
\newcommand{\cA}{\mathcal{A}}
\newcommand{\cB}{\mathcal{B}}
\newcommand{\cV}{\mathcal{V}}
\newcommand{\barC}{\overline{C}}
\newcommand{\barB}{\overline{B}}
\newcommand{\barPi}{\overline{\Pi}}

% Unified notation system - Patch 19
\newcommand{\chirbar}{\bar{B}^{\text{ch}}}  % geometric bar complex
\newcommand{\chircobar}{\Omega^{\text{ch}}}  % geometric cobar complex
\newcommand{\Conf}[1]{\overline{C}_{#1}}     % compactified configuration space
\newcommand{\OPE}[2]{\phi_{#1}(z)\phi_{#2}(w)} % operator product
\newcommand{\MC}{\text{MC}}                   % Maurer-Cartan
        % category of chiral algebras

% ==========================================
% TYPOGRAPHY SETTINGS FOR OPTIMAL READABILITY
% ==========================================

% Line spacing optimized for Garamond
\usepackage{setspace}
\setstretch{1.08} % Slightly increased for Garamond's proportions

% Page layout adjusted for Garamond's characteristics
\usepackage{geometry}[
  top=1.2in,
  bottom=1.2in,
  left=1.25in,
  right=1.25in,
  footskip=0.5in
]

% Section formatting with small caps (beautiful in Garamond)
\usepackage{titlesec}
\titleformat{\section}
  {\normalfont\Large\scshape}
  {\thesection}
  {1em}
  {}
\titleformat{\subsection}
  {\normalfont\large\scshape}
  {\thesubsection}
  {1em}
  {}

% ==========================================

\title{\it{Chiral Bar-Cobar Adjunctions:\\
Geometric Realization via Configuration Spaces}}
\author{Raeez Lorgat}
\date{September 21, 2025}
 
\begin{document}

% ==========================================
% NOTATION CONSISTENCY DECLARATIONS
% ==========================================
\newcommand{\barBgeom}{\bar{B}_{\text{geom}}}  % Use consistently
\newcommand{\barBch}{\bar{B}^{\text{ch}}}       % Alternative notation
\newcommand{\Omegach}{\Omega^{\text{ch}}}       % Cobar notation
\newcommand{\ConfigSpace}[1]{\overline{C}_{#1}(X)}  % Configuration space
\newcommand{\LogForm}[2]{\eta_{#1#2}}           % Logarithmic forms
\newcommand{\OPEcoeff}[4]{C_{#1#2}^{#3,#4}}    % OPE coefficients
\newcommand{\ChirAlg}{\mathsf{ChirAlg}}         % Category of chiral algebras
\newcommand{\dgCoalg}{\mathsf{dgCoalg}}         % Category of dg coalgebras

% From this point forward:
% - Always use \barBgeom for the geometric bar complex
% - Always use \ConfigSpace{n} for configuration spaces
% - Always use \LogForm{i}{j} for logarithmic forms



\maketitle
 
\begin{abstract}
{\itshape
\noindent % Better alignment for abstract in Garamond
% Optional: Use slightly smaller size for abstract
% \small
% Add subtle spacing
\setstretch{1.05}
\noindent
Starting from the operadic formalism of two-dimensional conformally invariant quantum field theory and its generalizations, as developed in the notion of a Chiral Algebra by Beilinson-Drinfeld \cite{BD}, we develop a comprehensive geometric framework for bar-cobar duality of chiral algebras via configuration space integrals and logarithmic differential forms. We construct an explicit geometric realization of the bar complex
for chiral algebras, establishing functoriality, uniqueness up to canonical isomorphism,  and essential surjectivity onto the category of conilpotent chiral coalgebras. The dual cobar construction for chiral coalgebras is in parallel realized through Čech-type complexes on configuration spaces. We demonstrate that the corresponding (co)differential given by residues along boundary divisors satisfies $d^2 = 0$, and realize Bar-Cobar duality as a direct manifestation of Poincare-Verdier duality over the global configuration space. 

\medskip
\noindent
The constructions naturally encode canonical $A_\infty$ structures, with higher homotopies determined by Arnold-Orlik-Solomon relations among logarithmic (distributional) forms on boundary strata. These geometric dualities vastly extend the notion of Koszul duality for chiral algebras to a more general theory of \textbf{Koszul dual pairs}, servicing a range of applications from deformation theory, bulk-boundary correspondences in supersymmetric gauge and field theory, to modern approaches to quantum gravity via twisted holography.

\medskip
\noindent
In service of these applications we treat Koszul duality computationally via the \textbf{The Prism Principle:} Just as a physical prism decomposes white light into its spectrum, our geometric bar complex acts as a mathematical prism for chiral algebras. The logarithmic forms $d\log(z_i - z_j)$ on configuration spaces $\overline{C}_n(X)$ separate the global chiral structure into its constituent OPE coefficients through residue calculus. Each collision divisor $D_{ij}$ corresponds to a specific "spectral line" --- an operator product channel --- with residues extracting the corresponding structure constants. This geometric spectroscopy provides both conceptual clarity and computational power, transforming abstract algebraic structures into concrete geometric data.
}
\end{abstract}
\tableofcontents

\medskip
\noindent

\begin{remark}[Notation Convention]
Throughout this manuscript:
\begin{itemize}
\item $\barBgeom(\mathcal{A})$ denotes the geometric bar complex
\item $\barBch(\mathcal{A})$ denotes the abstract chiral bar complex (when distinction needed)
\item $\ConfigSpace{n} = \overline{C}_n(X)$ is the compactified configuration space
\item $\LogForm{i}{j} = d\log(z_i - z_j)$ are the logarithmic 1-forms
\end{itemize}
\end{remark}

% ==========================================
% PART I: FOUNDATIONS AND MOTIVATION
% ==========================================
\part{Foundations and Motivation}

\include{"part1.tex"}

% ==========================================
% PART II: CONFIGURATION SPACES AND GEOMETRY
% ==========================================
\part{Configuration Spaces and Geometry}

\include{"part2.tex"}

% ==========================================
% PART III: BAR AND COBAR CONSTRUCTIONS
% ==========================================
\part{Bar and Cobar Constructions}

\include{"part3.tex"}

\include{"part4.tex"}

% ==========================================
% PART IV: KOSZUL DUALITY AND EXAMPLES
% ==========================================
\part{Koszul Duality and Applications}


\include{"part5.tex"}


% ==========================================
% PART V: PHYSICAL APPLICATIONS AND HOLOGRAPHY
% ==========================================
\part{Physical Applications and Holography}
\include{"part6.tex"}

% ==========================================
% PART VII: COMPUTATIONAL METHODS AND ALGORITHMS
% ==========================================
\part{Computational Methods and Algorithms}

% This part will contain computational details and algorithms
% extracted from various sections

% ==========================================
% APPENDICES
% ==========================================
\appendix

\section{Theta Functions and Modular Forms}

\subsection{Classical Theta Functions}

The four Jacobi theta functions form the basis for all elliptic constructions:

\begin{definition}[Jacobi Theta Functions]
\begin{align}
\vartheta_{00}(z|\tau) &\equiv \vartheta_3(z|\tau) = \sum_{n \in \mathbb{Z}} q^{n^2} e^{2\pi inz} \\
\vartheta_{01}(z|\tau) &\equiv \vartheta_4(z|\tau) = \sum_{n \in \mathbb{Z}}(-1)^n q^{n^2} e^{2\pi inz} \\
\vartheta_{10}(z|\tau) &\equiv \vartheta_2(z|\tau) = \sum_{n \in \mathbb{Z}} q^{(n+1/2)^2} e^{2\pi i(n+1/2)z} \\
\vartheta_{11}(z|\tau) &\equiv \vartheta_1(z|\tau) = -i\sum_{n \in \mathbb{Z}}(-1)^n q^{(n+1/2)^2} e^{2\pi i(n+1/2)z}
\end{align}
where $q = e^{2\pi i\tau}$ is the nome.
\end{definition}

\subsection{Modular Transformation Laws}

Under the generators of $SL_2(\mathbb{Z})$:
$$T: \tau \mapsto \tau + 1, \quad S: \tau \mapsto -1/\tau$$

The theta functions transform as:
\begin{align}
\vartheta_{ab}(z|\tau+1) &= e^{-\pi ia/2} \vartheta_{a,b+a}(z|\tau) \\
\vartheta_{ab}(z/\tau|-1/\tau) &= (-i\tau)^{1/2} e^{\pi iz^2/\tau} \sum_{cd} K_{ab,cd} \vartheta_{cd}(z|\tau)
\end{align}
where $K$ is the kernel matrix encoding the modular transformation.

\subsection{Higher Genus Theta Functions}

For genus $g$, theta functions depend on $g \times g$ period matrices $\Omega$:
$$\Theta[\epsilon](z|\Omega) = \sum_{n \in \mathbb{Z}^g} \exp\left[\pi i(n+\epsilon')^t\Omega(n+\epsilon') + 2\pi i(n+\epsilon')^t(z+\epsilon'')\right]$$
where $\epsilon = (\epsilon', \epsilon'') \in (\mathbb{Z}_2)^{2g}$ is the characteristic.

\subsection{Elliptic and Siegel Modular Forms}

\begin{definition}[Weight $k$ Modular Form]
A holomorphic function $f: \mathfrak{h} \to \mathbb{C}$ is a modular form of weight $k$ if:
$$f\left(\frac{a\tau+b}{c\tau+d}\right) = (c\tau+d)^k f(\tau)$$
for all $\begin{pmatrix} a & b \\ c & d \end{pmatrix} \in SL_2(\mathbb{Z})$.
\end{definition}

Key examples:
\begin{itemize}
\item Eisenstein series: $E_{2k}(\tau) = 1 - \frac{4k}{B_{2k}}\sum_{n=1}^\infty \sigma_{2k-1}(n)q^n$
\item Dedekind eta: $\eta(\tau) = q^{1/24}\prod_{n=1}^\infty(1-q^n)$
\item Discriminant: $\Delta(\tau) = \eta(\tau)^{24} = q\prod_{n=1}^\infty(1-q^n)^{24}$
\end{itemize}

For genus $g$, Siegel modular forms are functions on the Siegel upper half-space $\mathfrak{h}_g$ transforming under $Sp_{2g}(\mathbb{Z})$.

\subsection{Elliptic Polylogarithms}

The elliptic polylogarithms generalize classical polylogarithms:
\[
\text{Li}_n^{(g)}(z;\tau) = \sum_{k=1}^\infty \frac{q^k}{k^n}\frac{1}{1-zq^k}
\]

These appear in the genus $g$ bar differentials as:
\[
d^{(g)}_{\text{ell}} = \sum_{n=2}^{2g} \text{Li}_n^{(g)}(e^{2\pi iz};\tau) \cdot \eta^{\otimes n}
\]

\section{Spectral Sequences for Higher Genus}

\subsection{The Hodge-to-de Rham Spectral Sequence}

For the universal curve $\pi: \mathcal{C}_g \to \mathcal{M}_g$:

$E_1^{p,q} = H^q(\mathcal{M}_g, R^p\pi_*\Omega_{\mathcal{C}_g/\mathcal{M}_g}) \Rightarrow H^{p+q}_{\text{dR}}(\mathcal{C}_g)$

The differentials encode:
\begin{itemize}
\item $d_1$: Gauss-Manin connection
\item $d_2$: Kodaira-Spencer map  
\item $d_r$ ($r \geq 3$): Higher deformations
\end{itemize}

\subsection{The Bar Complex Spectral Sequence}

$E_2^{p,q} = H^p(\overline{\mathcal{M}}_{g,n}, \underline{H}^q(\bar{B}^{(g)}(\mathcal{A}))) \Rightarrow H^{p+q}(\bar{B}^{\text{total}}(\mathcal{A}))$

where $\underline{H}^q$ denotes the local system of bar cohomology groups.

\subsection{Convergence and Degeneration}

\begin{theorem}[Convergence Criterion]
The spectral sequence converges if:
\begin{enumerate}
\item The chiral algebra $\mathcal{A}$ is rational (finitely many irreps)
\item The genus expansion parameter satisfies $|g_s| < \epsilon(\mathcal{A})$
\item The moduli space $\overline{\mathcal{M}}_{g,n}$ is replaced by its Deligne-Mumford compactification
\end{enumerate}
\end{theorem}

\begin{theorem}[Degeneration at $E_2$]
For special values of central charge:
\begin{itemize}
\item $c = 0$: Topological theory, degenerates at $E_1$
\item $c = 26$: Critical bosonic string, degenerates at $E_2$  
\item $c = 15$: Critical superstring, degenerates at $E_2$
\end{itemize}
\end{theorem}

\subsection{Computational Tools}

The differentials can be computed via:
\begin{enumerate}
\item \textbf{Čech cohomology:} Cover $\overline{\mathcal{M}}_{g,n}$ by affine opens
\item \textbf{Dolbeault cohomology:} Use $\bar{\partial}$-operator techniques
\item \textbf{Combinatorial models:} Jenkins-Strebel differentials
\item \textbf{Topological recursion:} Eynard-Orantin formalism
\end{enumerate}

\subsection{Spectral Sequence for Bar Complex}

\begin{theorem}[Bar Spectral Sequence]
The filtration by configuration degree yields a spectral sequence:
$$E_1^{p,q} = H^q(\overline{C}_{p+1}(X), j_*j^*\mathcal{A}^{\boxtimes(p+1)}) \Rightarrow H^{p+q}(\bar{B}^{\text{ch}}(\mathcal{A}))$$

\textbf{Key Properties:}
\begin{enumerate}
\item $E_2$ page: Computed by residues at boundary divisors
\item Convergence: Always for finite-type chiral algebras
\item Degeneration: At $E_2$ for Koszul algebras (quadratic with no higher relations)
\item Differential $d_r$: Encodes $(r+1)$-fold collisions
\end{enumerate}

\textbf{Application to Free Fermions:}
\begin{itemize}
\item $E_1^{p,0} = \wedge^p(\mathcal{F} \otimes H^0(X, \omega_X))$ 
\item $d_1 = 0$ (no relations beyond anticommutativity)
\item Collapses at $E_1 = E_{\infty}$
\item Recovers $\bar{B}^{\text{ch}}(\mathcal{F}) = \wedge^{\bullet}(\mathcal{F}[1])$
\end{itemize}

\textbf{Application to W-algebras:}
For $\mathcal{W}_k(\mathfrak{g}, f)$ at admissible level:
\begin{itemize}
\item $E_1$: Free generators from W-currents
\item $E_2$: Normal ordered products and null fields
\item $E_3$: Quantum corrections from BRST cohomology
\item Convergence requires careful analysis of Virasoro representations
\end{itemize}
\end{theorem}

\begin{example}[Computing $E_2$ Page]
For a chiral algebra with generators $\phi_i$ of conformal weight $h_i$:

$$E_2^{p,q} = \frac{\text{Ker}(d_1: E_1^{p,q} \to E_1^{p+1,q})}{\text{Im}(d_1: E_1^{p-1,q} \to E_1^{p,q})}$$

where $d_1$ is computed from OPE residues:
$$d_1(\phi_{i_1} \otimes \cdots \otimes \phi_{i_p}) = \sum_{j<k} \sum_\ell C_{i_j i_k}^\ell \phi_{i_1} \otimes \cdots \widehat{i_j} \cdots \widehat{i_k} \cdots \otimes \phi_\ell$$
\end{example}

\begin{remark}[Physical Interpretation]
In string theory:
\begin{itemize}
\item $E_1$: Off-shell string states
\item $d_1$: BRST operator
\item $E_2$: Physical (on-shell) states
\item Higher pages: Quantum corrections and anomalies
\end{itemize}
\end{remark}

\appendix
\chapter{Koszul Duality Across Genera}

\section{Genus-Graded Koszul Duality}

\begin{theorem}[Extended Koszul Duality]
If $(\mathcal{A}, \mathcal{A}^!)$ form a genus-0 Koszul dual pair, then:
$$\left(\bigoplus_{g \geq 0} \mathcal{A}^{(g)}, \bigoplus_{g \geq 0} (\mathcal{A}^!)^{(g)}\right)$$
form a multi-genus Koszul dual pair with pairing:
$$\langle -, - \rangle: \mathcal{A}^{(g)} \otimes (\mathcal{A}^!)^{(g)} \to \mathbb{C}[\![\hbar]\!]$$
where $\hbar$ tracks the genus.
\end{theorem}

\section{Definition and Basic Properties}

\begin{definition}[Genus-Graded Koszul Algebra]
A genus-graded associative algebra $\mathcal{A} = \bigoplus_{g \geq 0} \mathcal{A}^{(g)}$ is \emph{Koszul} if:
$$\text{Ext}_{\mathcal{A}^{(g)}}^{i,j}(\mathbb{k}, \mathbb{k}) = 0 \text{ for } i \neq j$$
where the bigrading is by homological degree and internal degree, and the Koszul property holds at each genus.
\end{definition}

\begin{theorem}[Genus-Graded Koszul Duality Theorem]
If $\mathcal{A}$ is genus-graded Koszul, then:
$$\mathcal{A}^! := \bigoplus_{g \geq 0} \text{Ext}_{\mathcal{A}^{(g)}}^*(\mathbb{k}, \mathbb{k})$$
is also genus-graded Koszul, and $(\mathcal{A}^!)^! \cong \mathcal{A}$.
\end{theorem}

\subsection{Genus-Graded Chiral Koszul Duality}

For chiral algebras across all genera, we need a modified definition:

\begin{definition}[Genus-Graded Chiral Koszul Duality]
Genus-graded chiral algebras $\mathcal{A} = \bigoplus_{g \geq 0} \mathcal{A}^{(g)}$ and $\mathcal{B} = \bigoplus_{g \geq 0} \mathcal{B}^{(g)}$ are Koszul dual if:
$$\text{RHom}_{\mathcal{A}^{(g)} \otimes \mathcal{B}^{(g)}}(\mathbb{C}, \mathbb{C}) \simeq \mathbb{C}$$
in the derived category of chiral modules at each genus $g$, with modular covariance under $\text{Sp}(2g, \mathbb{Z})$ transformations.
\end{definition}

\subsection{Curved and Filtered Generalizations Across Genera}

\begin{definition}[Genus-Graded Curved Koszul Duality]
A genus-graded curved algebra $(\mathcal{A}^{(g)}, d^{(g)}, m_0^{(g)})$ with $(d^{(g)})^2 = m_0^{(g)} \cdot \text{id}$ has curved dual:
$$((\mathcal{A}^{(g)})^!, d^{!(g)}, m_0^{!(g)})$$
where $m_0^{!(g)} = -m_0^{(g)}$ under the genus-graded pairing, with modular corrections from period integrals.
\end{definition}

\subsection{Computational Tools Across Genera}

\begin{lemma}[Genus-Graded Koszul Complex Resolution]
For genus-graded Koszul $\mathcal{A}$, the minimal resolution of $\mathbb{k}$ at genus $g$ is:
$$\cdots \to \mathcal{A}^{(g)} \otimes (\mathcal{A}^!)_{(2)}^{(g)} \to \mathcal{A}^{(g)} \otimes (\mathcal{A}^!)_{(1)}^{(g)} \to \mathcal{A}^{(g)} \to \mathbb{k}$$
where $(\mathcal{A}^!)_{(n)}^{(g)}$ is the degree $n$ part of $\mathcal{A}^!$ at genus $g$, with modular corrections from period integrals.
\end{lemma}

\subsection{Physical Interpretation Across Genera}

In physics, genus-graded Koszul duality appears as:
\begin{itemize}
\item Electric-magnetic duality with genus corrections (abelian case)
\item Open-closed string duality with modular forms (topological strings)  
\item Holographic duality with genus expansion (AdS/CFT)
\item Mirror symmetry with period integrals (A-model/B-model)
\item String amplitudes with genus-graded corrections
\end{itemize}

\subsection{Genus-Graded Maurer-Cartan Elements and Twisting}

\begin{theorem}[Genus-Graded MC Elements Parametrize Deformations]
For a genus-graded chiral algebra $\mathcal{A} = \bigoplus_{g \geq 0} \mathcal{A}^{(g)}$ and its bar complex $\bar{B}(\mathcal{A})$:

\textbf{1. Genus-Graded Maurer-Cartan Equation:}
$$\alpha^{(g)} \in \barBgeom^{(g)}(\mathcal{A}), \quad d^{(g)}\alpha^{(g)} + \frac{1}{2}[\alpha^{(g)}, \alpha^{(g)}] = 0$$
with modular corrections from period integrals.

\textbf{2. Genus-Graded Twisting:}
Each MC element $\alpha^{(g)}$ yields a twisted differential:
$$d_{\alpha^{(g)}}^{(g)} = d^{(g)} + [\alpha^{(g)}, -]$$
with $(d_{\alpha^{(g)}}^{(g)})^2 = 0$ and modular covariance.

\textbf{3. Genus-Graded Deformation:}
MC elements correspond to first-order deformations of $\mathcal{A}^{(g)}$:
$$\mu_{\alpha^{(g)}}^{(g)}(a \otimes b) = \mu^{(g)}(a \otimes b) + \langle \alpha^{(g)}, a \otimes b \rangle$$
with genus corrections.

\textbf{4. Geometric Interpretation Across Genera:}
On configuration spaces, MC elements are:
\begin{itemize}
\item Closed 1-forms on $\overline{C}_2^{(g)}(\Sigma_g)$ with prescribed residues and period integrals
\item Flat connections on the punctured configuration space with modular structure
\item Solutions to the classical Yang-Baxter equation with genus corrections
\end{itemize}

\textbf{5. Genus-Graded Moduli Space:}
$$\mathcal{M}_{\text{MC}}^{(g)}(\mathcal{A}) = \{\text{MC elements at genus } g\}/\text{gauge equivalence}$$
parametrizes deformations of the chiral algebra structure at each genus.
\end{theorem}

\subsection{Koszul Duality at Higher Genus: The Tower Structure}
\label{app:koszul_higher_genus}

The genus 0 Koszul duality:
$$\Omega C_{\bullet}^{(0)}(\mathcal{A}) \simeq \mathcal{A}$$
extends to all genera by the modular operad structure.

\subsubsection{The Genus $g$ Statement}

For each $g \geq 0$, there is a duality:
$$\Omega^{(g)} C_{\bullet}^{(g)}(\mathcal{A}) \simeq \mathcal{A}^{(g)}$$
where:
\begin{itemize}
\item $\Omega^{(g)}$ is the genus $g$ cobar construction
\item $\mathcal{A}^{(g)}$ is the genus $g$ component of $\mathcal{A}$
\end{itemize}

\subsubsection{Compatibility}

The genus stratification satisfies:
$$\partial: C_{\bullet}^{(g)} \to C_{\bullet}^{(g-1)}$$
(boundary/degeneration maps) compatible with:
$$\iota: \mathcal{A}^{(g-1)} \to \mathcal{A}^{(g)}$$
(restriction maps).

This gives a \textbf{tower of Koszul dualities}:
\begin{center}
\begin{tikzcd}[column sep=small]
\cdots \arrow[r] & C_{\bullet}^{(2)}(\mathcal{A}) \arrow[r] \arrow[d, "\Omega^{(2)}"] & 
C_{\bullet}^{(1)}(\mathcal{A}) \arrow[r] \arrow[d, "\Omega^{(1)}"] & 
C_{\bullet}^{(0)}(\mathcal{A}) \arrow[d, "\Omega^{(0)}"] \\
\cdots \arrow[r] & \mathcal{A}^{(2)} \arrow[r] & 
\mathcal{A}^{(1)} \arrow[r] & 
\mathcal{A}^{(0)}
\end{tikzcd}
\end{center}

\subsubsection{The Limit}

Taking the inverse limit:
$$\mathcal{A}_{\text{complete}} = \varprojlim_g \mathcal{A}^{(g)}$$
gives the \textbf{completed chiral algebra}, encoding all genus contributions.

\subsubsection{Modular Invariance}

At each genus $g$, the duality respects the action of the mapping class group $\Gamma_g = \operatorname{MCG}(\Sigma_g)$:
$$\Omega^{(g)}(\sigma^* C_{\bullet}^{(g)}(\mathcal{A})) \simeq \sigma^* \mathcal{A}^{(g)}$$
for $\sigma \in \Gamma_g$.

This ensures that genus $g$ quantum corrections are modular-invariant.


\end{document}