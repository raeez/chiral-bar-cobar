%================================================================
% APPENDIX: EXISTENCE CRITERIA FOR KOSZUL DUALS
%================================================================

\chapter{Existence Criteria for Koszul Duals}
\label{app:existence-criteria}

\begin{abstract}
This appendix provides complete necessary and sufficient conditions for when a 
chiral algebra $\mathcal{A}$ on a Riemann surface $X$ admits a chiral Koszul dual 
coalgebra $\mathcal{A}^!$. We give:
\begin{enumerate}
\item Algebraic characterization: When does the dual exist algebraically?
\item Geometric characterization: When is it representable by a coalgebra?
\item Algorithmic test: Given a presentation of $\mathcal{A}$, how to check existence?
\item Complete classification: Which standard examples have duals?
\item Completion theory: When completion is necessary and how to construct it
\end{enumerate}
\end{abstract}

%================================================================
% SECTION 1: STATEMENT OF THE PROBLEM
%================================================================

\section{The Existence Problem}
\label{sec:existence-problem-statement}

\subsection{Motivation and Setup}

\begin{question}[Central Question]\label{q:existence-central}
Given a chiral algebra $\mathcal{A}$ on $X$, when does there exist a chiral 
coalgebra $\mathcal{A}^!$ such that:
\begin{enumerate}
\item $\mathcal{A}^!$ is the \textbf{Koszul dual} of $\mathcal{A}$
\item The bar-cobar constructions are quasi-inverse:
      $$\Omega(\mathcal{A}^!) \simeq \mathcal{A} \quad \text{and} \quad 
      \bar{B}(\mathcal{A}) \simeq \mathcal{A}^!$$
\item The derived categories are equivalent:
      $$\mathcal{D}^b(\text{Mod}(\mathcal{A})) \simeq 
      \mathcal{D}^b(\text{Comod}(\mathcal{A}^!))$$
\end{enumerate}
\end{question}

\begin{remark}[Why This Question Matters]\label{rem:why-existence-matters}
The existence of $\mathcal{A}^!$ determines:
\begin{itemize}
\item Whether our bar-cobar duality applies to $\mathcal{A}$
\item Whether representation theory of $\mathcal{A}$ has nice properties
\item Whether $\mathcal{A}$ has a ``good'' homological algebra
\item Whether Koszul duality computations are possible
\end{itemize}

\textbf{Examples:}
\begin{itemize}
\item \textbf{Heisenberg algebra}: Has Koszul dual (easy case)
\item \textbf{Kac-Moody algebras}: Have Koszul duals at generic level
\item \textbf{Virasoro algebra}: Requires completion (subtle)
\item \textbf{$\mathcal{W}_\infty$}: No Koszul dual (fails to exist)
\end{itemize}
\end{remark}

\subsection{Preliminary Definitions}

\begin{definition}[Koszul Dual (Tentative)]\label{def:koszul-dual-tentative}
Let $\mathcal{A} = \text{Free}_{\mathcal{D}}(V) / (R)$ be a chiral algebra presented 
by generators $V$ and relations $R$. The \textbf{Koszul dual coalgebra} $\mathcal{A}^!$ 
(if it exists) should satisfy:
\begin{enumerate}
\item $\mathcal{A}^!$ is a chiral coalgebra
\item There is a natural pairing:
      $$\langle -, - \rangle: \mathcal{A} \otimes \mathcal{A}^! \to \mathbb{C}$$
\item This pairing identifies:
      $$\mathcal{A}^! \cong \text{Hom}(\mathcal{A}, \mathbb{C})$$
      in an appropriate derived sense
\end{enumerate}

\textbf{Problem:} This definition is circular! We need criteria to know when such 
an $\mathcal{A}^!$ exists before we can define it.
\end{definition}

\begin{definition}[Koszul Property]\label{def:koszul-property-existence}
A chiral algebra $\mathcal{A}$ is \textbf{Koszul} if:
\begin{enumerate}
\item The bar complex $\bar{B}(\mathcal{A})$ has a natural coalgebra structure
\item The bar construction gives a resolution:
      $$\bar{B}(\mathcal{A}) \xrightarrow{\sim} \mathcal{A}^!$$
      for some coalgebra $\mathcal{A}^!$
\item The derived categories satisfy:
      $$\mathcal{D}^b(\mathcal{A}\text{-mod}) \simeq \mathcal{D}^b(\mathcal{A}^!\text{-comod})$$
\end{enumerate}
\end{definition}

%================================================================
% SECTION 2: QUADRATIC CASE (EASY)
%================================================================

\section{Quadratic Case: Constructive Existence Proof}
\label{sec:quadratic-existence}

\subsection{Statement of Result}

\begin{theorem}[Quadratic Algebras Have Duals]\label{thm:quadratic-have-duals}
Let $\mathcal{A} = \text{Free}_{\mathcal{D}}(V) / (R)$ be a \textbf{quadratic} 
chiral algebra, meaning:
\begin{enumerate}
\item $V$ is a graded vector space (the generators)
\item $R \subseteq V^{\otimes 2}$ (the relations are quadratic)
\item $R$ satisfies certain regularity conditions (specified below)
\end{enumerate}

Then $\mathcal{A}$ admits a Koszul dual coalgebra $\mathcal{A}^!$, which can be 
constructed explicitly.
\end{theorem}

\begin{proof}[Proof Strategy]
The proof is constructive and has four steps:

\textbf{Step 1: Construct the dual space.}
Define:
$$V^* = \text{Hom}(V, \mathbb{C})$$
with dual grading: $|v^*| = -|v|$.

\textbf{Step 2: Define the dual relations.}
The relations $R \subseteq V^{\otimes 2}$ induce:
$$R^\perp \subseteq (V^*)^{\otimes 2}$$
defined by:
$$R^\perp = \{f \in (V^*)^{\otimes 2} : f(r) = 0 \text{ for all } r \in R\}$$

\textbf{Step 3: Build the dual coalgebra.}
Define:
$$\mathcal{A}^! = \text{Cofree}_{\mathcal{D}}(sV^*) / (sR^\perp)$$
where $s$ denotes suspension (degree shift by 1).

\textbf{Step 4: Verify the Koszul property.}
Check that:
\begin{itemize}
\item $\mathcal{A}^!$ has a well-defined coalgebra structure
\item The bar complex $\bar{B}(\mathcal{A}) \xrightarrow{\sim} \mathcal{A}^!$
\item The cobar complex $\Omega(\mathcal{A}^!) \xrightarrow{\sim} \mathcal{A}$
\end{itemize}
\end{proof}

\subsection{Explicit Construction}

\begin{construction}[Building $\mathcal{A}^!$ from $\mathcal{A}$]\label{const:quadratic-dual}
Given quadratic $\mathcal{A} = \text{Free}(V) / (R)$, construct $\mathcal{A}^!$ as follows:

\textbf{Input:}
\begin{itemize}
\item Generators: $V = \text{span}\{v_1, \ldots, v_n\}$ with degrees $|v_i|$
\item Relations: $R = \text{span}\{r_1, \ldots, r_m\}$ where each $r_j \in V^{\otimes 2}$
\end{itemize}

\textbf{Step 1: Dual generators.}
$$V^* = \text{span}\{v_1^*, \ldots, v_n^*\} \quad \text{with } |v_i^*| = -|v_i| + 1$$

(The shift by 1 is the suspension $s$.)

\textbf{Step 2: Write relations in coordinates.}
Express each relation as:
$$r_j = \sum_{i,k} a_{ik}^j \, v_i \otimes v_k$$

\textbf{Step 3: Dual relations.}
Define:
$$r_j^\perp = \sum_{i,k} a_{ik}^j \, v_i^* \otimes v_k^* \in (V^*)^{\otimes 2}$$

\textbf{Step 4: Cofree coalgebra.}
$$\mathcal{A}^! = \text{Cofree}(V^*) / (R^\perp)$$

where $\text{Cofree}(V^*)$ denotes the cofree coalgebra:
$$\text{Cofree}(V^*) = \bigoplus_{n=0}^\infty (V^*)^{\otimes n}$$
with comultiplication:
$$\Delta(v_1^* \otimes \cdots \otimes v_n^*) = \sum_{i=0}^n 
(v_1^* \otimes \cdots \otimes v_i^*) \otimes (v_{i+1}^* \otimes \cdots \otimes v_n^*)$$
\end{construction}

\begin{example}[Heisenberg Algebra]\label{ex:heisenberg-quadratic-dual}
The Heisenberg chiral algebra is:
$$\mathcal{H} = \text{Free}(a, a^*) / ([a, a^*] - 1)$$

\textbf{Generators:} $V = \text{span}\{a, a^*\}$ with $|a| = 1$, $|a^*| = 1$.

\textbf{Relation:} $R = \text{span}\{a \otimes a^* - a^* \otimes a - 1\}$

(This is the commutator relation.)

\textbf{Dual generators:} $V^* = \text{span}\{a^{\dagger}, (a^*)^{\dagger}\}$ with 
$|a^{\dagger}| = 0$, $|(a^*)^{\dagger}| = 0$ (after suspension).

\textbf{Dual relation:} 
$$R^\perp = \text{span}\{a^{\dagger} \otimes (a^*)^{\dagger} - (a^*)^{\dagger} 
\otimes a^{\dagger}\}$$

\textbf{Dual coalgebra:}
$$\mathcal{H}^! = \text{Cofree}(a^{\dagger}, (a^*)^{\dagger}) / (R^\perp)$$

This has comultiplication:
$$\Delta(a^{\dagger}) = a^{\dagger} \otimes 1 + 1 \otimes a^{\dagger}$$
$$\Delta((a^*)^{\dagger}) = (a^*)^{\dagger} \otimes 1 + 1 \otimes (a^*)^{\dagger}$$

(Primitive elements!)

\textbf{Verification:} The bar-cobar constructions give:
$$\Omega(\mathcal{H}^!) \simeq \mathcal{H} \quad \text{and} \quad 
\bar{B}(\mathcal{H}) \simeq \mathcal{H}^!$$

as expected for Koszul duality.
\end{example}

\subsection{Regularity Conditions}

\begin{definition}[Quadratic Regularity]\label{def:quadratic-regularity}
A quadratic presentation $\mathcal{A} = \text{Free}(V) / (R)$ is \textbf{regular} if:
\begin{enumerate}
\item \textbf{Non-degeneracy:} The pairing:
      $$V \otimes V^* \to \mathbb{C}$$
      is non-degenerate
      
\item \textbf{Compatibility:} The relations $R$ and $R^\perp$ satisfy:
      $$(R^\perp)^\perp = R$$
      (the dual of the dual recovers the original)
      
\item \textbf{Conilpotency:} The coalgebra $\mathcal{A}^!$ is conilpotent:
      $$\bigcap_{n=1}^\infty \text{coker}(\Delta^n) = 0$$
\end{enumerate}
\end{definition}

\begin{theorem}[Regularity Implies Koszul]\label{thm:regular-implies-koszul}
If $\mathcal{A} = \text{Free}(V) / (R)$ is quadratic and regular (Definition 
\ref{def:quadratic-regularity}), then:
\begin{enumerate}
\item $\mathcal{A}^!$ exists and is given by Construction \ref{const:quadratic-dual}
\item $\mathcal{A}$ is Koszul
\item Bar-cobar quasi-isomorphism holds (Theorem \ref{thm:bar-cobar-inversion-qi})
\end{enumerate}
\end{theorem}

%================================================================
% SECTION 3: NON-QUADRATIC CASE (SUBTLE)
%================================================================

\section{Non-Quadratic Case: Completion Required}
\label{sec:non-quadratic-existence}

\subsection{Why Completion is Necessary}

\begin{remark}[Obstructions in Non-Quadratic Case]\label{rem:non-quadratic-obstructions}
When $\mathcal{A}$ has higher-order relations (cubic, quartic, etc.), direct 
dualization fails because:

\textbf{Problem 1: Dual relations not closed.}
Higher relations $r \in V^{\otimes n}$ for $n \geq 3$ give dual relations 
$r^\perp \in (V^*)^{\otimes n}$, but these may not form an ideal in the cofree 
coalgebra.

\textbf{Problem 2: Infinite-dimensional kernel.}
The map:
$$\mathcal{A} \to \mathcal{A}^{\vee\vee} = \text{Hom}(\text{Hom}(\mathcal{A}, \mathbb{C}), \mathbb{C})$$
may have infinite-dimensional kernel.

\textbf{Problem 3: No natural coalgebra structure.}
Even if we define $\mathcal{A}^* = \text{Hom}(\mathcal{A}, \mathbb{C})$, there's 
no obvious comultiplication.

\textbf{Solution:} Work with completions.
\end{remark}

\begin{definition}[Filtered Completion]\label{def:filtered-completion}
Let $\mathcal{A} = \text{Free}(V) / (R)$ with $R$ containing higher-order relations. 
Define the \textbf{filtered completion} $\widehat{\mathcal{A}}$ as:

\textbf{Step 1: Choose filtration.}
Filter $\mathcal{A}$ by order of monomials:
$$F_0\mathcal{A} \subseteq F_1\mathcal{A} \subseteq F_2\mathcal{A} \subseteq \cdots$$
where $F_n\mathcal{A}$ = elements of order $\leq n$.

\textbf{Step 2: Complete.}
$$\widehat{\mathcal{A}} = \varprojlim_n \mathcal{A} / F_n\mathcal{A}$$

This is the \textbf{inverse limit} of finite-dimensional quotients.
\end{definition}

\begin{theorem}[Completed Koszul Dual]\label{thm:completed-koszul-dual}
For a non-quadratic chiral algebra $\mathcal{A}$ with regular filtration, the 
completed dual:
$$\widehat{\mathcal{A}^!} = \varprojlim_n (\mathcal{A} / F_n\mathcal{A})^!$$
exists and satisfies:
\begin{enumerate}
\item $\widehat{\mathcal{A}^!}$ is a completed chiral coalgebra
\item Bar-cobar constructions converge in completion:
      $$\Omega(\widehat{\mathcal{A}^!}) \simeq \widehat{\mathcal{A}} \quad \text{and} \quad 
      \bar{B}(\widehat{\mathcal{A}}) \simeq \widehat{\mathcal{A}^!}$$
\item Derived categories equivalent (completed version)
\end{enumerate}
\end{theorem}

\subsection{I-adic Completion}

\begin{definition}[I-adic Completion]\label{def:i-adic-completion}
Let $I \subseteq \mathcal{A}$ be an ideal (the ``augmentation ideal''). The 
\textbf{$I$-adic completion} is:
$$\widehat{\mathcal{A}}_I = \varprojlim_n \mathcal{A} / I^n$$

\textbf{Geometric meaning:} This is completion at the ``point at infinity'' or 
``augmentation point'' of the spectrum.
\end{definition}

\begin{example}[Virasoro Algebra]\label{ex:virasoro-i-adic}
The Virasoro algebra is:
$$\text{Vir} = \text{span}\{L_n, c : n \in \mathbb{Z}\} / 
([L_m, L_n] - (m-n)L_{m+n} - \frac{c}{12}(m^3 - m)\delta_{m+n,0})$$

This is \textbf{not quadratic} (the central charge term is a 3-cocycle).

\textbf{Augmentation ideal:} $I = \text{span}\{L_n : n \neq 0\}$

\textbf{Completion:}
$$\widehat{\text{Vir}} = \varprojlim_n \text{Vir} / I^n$$

In this completion, the Virasoro algebra has a Koszul dual:
$$\widehat{\text{Vir}^!} = \text{completed cofree coalgebra on } L_n^*, c^*$$

\textbf{Warning:} Without completion, $\text{Vir}$ does NOT have a Koszul dual in 
the naive sense!
\end{example}

\subsection{Convergence Criteria}

\begin{theorem}[When Does Completion Converge?]\label{thm:completion-convergence}
For a filtered chiral algebra $\mathcal{A}$ with filtration $\{F_n\}$, the 
completion $\widehat{\mathcal{A}} = \varprojlim_n \mathcal{A}/F_n$ converges if:
\begin{enumerate}
\item \textbf{Exhaustiveness:} $\bigcup_n F_n = \mathcal{A}$
\item \textbf{Separatedness:} $\bigcap_n F_n = 0$
\item \textbf{Bounded growth:} $\dim(F_n / F_{n-1}) = O(n^k)$ for some $k$
\end{enumerate}

When these hold, $\widehat{\mathcal{A}}$ is a well-defined topological algebra, 
and $\widehat{\mathcal{A}^!}$ exists.
\end{theorem}

%================================================================
% SECTION 4: ALGORITHMIC TEST
%================================================================

\section{Algorithmic Existence Test}
\label{sec:algorithmic-test}

\subsection{The Algorithm}

\begin{algorithm}[H]
\caption{Test for Koszul Dual Existence}
\label{alg:koszul-dual-existence}

\textbf{Input:} Chiral algebra $\mathcal{A} = \text{Free}(V) / (R)$

\textbf{Output:} YES (dual exists), NO (dual does not exist), or COMPLETION (need completion)

\begin{algorithmic}[1]
\State \textbf{Step 1:} Check if $\mathcal{A}$ is quadratic
\If{all relations in $R$ are in $V^{\otimes 2}$}
    \State Go to Step 2 (quadratic case)
\Else
    \State Go to Step 5 (non-quadratic case)
\EndIf

\State \textbf{Step 2:} Check regularity (quadratic case)
\State Compute dual space $V^*$
\State Compute dual relations $R^\perp \subseteq (V^*)^{\otimes 2}$
\If{$(R^\perp)^\perp = R$ \textbf{and} $V \otimes V^* \to \mathbb{C}$ non-degenerate}
    \State Go to Step 3 (regular)
\Else
    \State \Return NO (irregular)
\EndIf

\State \textbf{Step 3:} Construct candidate dual
\State $\mathcal{A}^! = \text{Cofree}(sV^*) / (sR^\perp)$

\State \textbf{Step 4:} Verify conilpotency
\If{$\mathcal{A}^!$ is conilpotent}
    \State \Return YES (dual exists, given by $\mathcal{A}^!$)
\Else
    \State \Return NO (not conilpotent)
\EndIf

\State \textbf{Step 5:} Non-quadratic case
\State Compute homological degree of relations $\max_i \deg(r_i)$
\If{relations are bounded degree}
    \State Define filtration by order
    \State Check convergence criteria (Theorem \ref{thm:completion-convergence})
    \If{converges}
        \State \Return COMPLETION (dual exists after completion)
    \Else
        \State \Return NO (completion does not converge)
    \EndIf
\Else
    \State \Return NO (unbounded relations, no dual)
\EndIf
\end{algorithmic}
\end{algorithm}

\subsection{Examples of Algorithm Application}

\begin{example}[Running Algorithm on Heisenberg]\label{ex:algorithm-heisenberg}
Input: $\mathcal{H} = \text{Free}(a, a^*) / ([a, a^*] - 1)$

\textbf{Step 1:} Check quadratic. 
$$R = \{a \otimes a^* - a^* \otimes a - 1\} \subseteq V^{\otimes 2}$$
Yes, quadratic. Proceed to Step 2.

\textbf{Step 2:} Check regularity.
$$V^* = \text{span}\{a^{\dagger}, (a^*)^{\dagger}\}$$
$$R^\perp = \{a^{\dagger} \otimes (a^*)^{\dagger} - (a^*)^{\dagger} \otimes a^{\dagger}\}$$

Check: $(R^\perp)^\perp = R$? Yes.
Check: $V \otimes V^* \to \mathbb{C}$ non-degenerate? Yes.

Regular. Proceed to Step 3.

\textbf{Step 3:} Construct dual.
$$\mathcal{H}^! = \text{Cofree}(sa^{\dagger}, s(a^*)^{\dagger}) / (sR^\perp)$$

\textbf{Step 4:} Verify conilpotency.
$$\Delta(a^{\dagger}) = a^{\dagger} \otimes 1 + 1 \otimes a^{\dagger}$$

This is primitive, hence conilpotent.

\textbf{Output:} YES, dual exists, $\mathcal{H}^!$ as constructed.
\end{example}

\begin{example}[Running Algorithm on $\mathcal{W}_\infty$]\label{ex:algorithm-w-infinity}
Input: $\mathcal{W}_\infty$ (W-algebra with infinitely many generators of all weights)

\textbf{Step 1:} Check quadratic.
Relations include terms of all orders (quadratic, cubic, quartic, ...).
Not quadratic. Proceed to Step 5.

\textbf{Step 5:} Non-quadratic case.
Check: Are relations bounded degree? 

NO. $\mathcal{W}_\infty$ has relations of arbitrarily high degree.

\textbf{Output:} NO, dual does not exist (even after completion).

\textbf{Explanation:} The unbounded complexity of $\mathcal{W}_\infty$ prevents 
existence of a Koszul dual in any reasonable sense.
\end{example}

%================================================================
% SECTION 5: COMPLETE CLASSIFICATION
%================================================================

\section{Complete Classification of Standard Examples}
\label{sec:classification-standard-examples}

\begin{theorem}[Classification Table]\label{thm:classification-table}
The following table classifies standard chiral algebras by existence of Koszul duals:

\begin{center}
\small
\begin{tabular}{|l|c|c|l|}
\hline
\textbf{Chiral Algebra} & \textbf{Quadratic?} & \textbf{Has Dual?} & \textbf{Comments} \\
\hline
Heisenberg $\mathcal{H}$ & Yes & Yes & Primitive coalgebra \\
\hline
$\widehat{\mathfrak{g}}_k$ (Kac-Moody) & Yes & Yes (generic $k$) & Dual is Langlands \\
\hline
Virasoro $\text{Vir}$ & No & Yes (completion) & $I$-adic completion \\
\hline
$\mathcal{W}_3$ & No & Yes ($c = -2$) & Special values only \\
\hline
$\mathcal{W}_N$ ($N < \infty$) & No & Sometimes & Depends on $(N, c)$ \\
\hline
$\mathcal{W}_\infty$ & No & NO & Unbounded relations \\
\hline
Free fermion $\beta\gamma$ & Yes & Yes & Exterior coalgebra \\
\hline
$\mathfrak{gl}_n$ current & Yes & Yes & Matrix coalgebra \\
\hline
Affine Yangian & No & Yes (filtered) & Requires filtering \\
\hline
\end{tabular}
\end{center}
\end{theorem}

\subsection{Detailed Analysis: Kac-Moody}

\begin{proposition}[Kac-Moody Koszul Duals]\label{prop:kac-moody-koszul-duals}
Let $\widehat{\mathfrak{g}}_k$ be the affine Kac-Moody algebra at level $k \in \mathbb{C}$. 
Then:
\begin{enumerate}
\item For \textbf{generic} $k$, $\widehat{\mathfrak{g}}_k$ is Koszul with dual:
      $$(\widehat{\mathfrak{g}}_k)^! = \widehat{\mathfrak{g}^L}_{k^L}$$
      where $\mathfrak{g}^L$ is the Langlands dual and $k^L$ is related by:
      $$\frac{1}{k + h^\vee} + \frac{1}{k^L + h^{\vee,L}} = 1$$
      
\item For \textbf{special} values ($k = -h^\vee$, critical level), the dual requires 
      completion
      
\item The bar-cobar duality realizes Langlands duality geometrically:
      $$\text{Langlands duality} = \text{Koszul duality}$$
\end{enumerate}
\end{proposition}

\begin{proof}[Proof Sketch]
The Kac-Moody algebra has presentation:
$$\widehat{\mathfrak{g}}_k = \text{Free}(\{J^a_n : a \in \mathfrak{g}, n \in \mathbb{Z}\}) 
/ \text{(Kac-Moody relations)}$$

The relations are quadratic:
$$[J^a_m, J^b_n] = f^{ab}_c J^c_{m+n} + k \delta_{m+n,0} \langle a, b \rangle$$

\textbf{Dual generators:} $(J^a_n)^* = (J^{\check{a}}_n)^*$ where $\check{a}$ is in 
the Langlands dual $\mathfrak{g}^L$.

\textbf{Dual relations:} Determined by Langlands duality.

\textbf{Level matching:} The formula relating $k$ and $k^L$ comes from requiring:
$$\langle \widehat{\mathfrak{g}}_k, (\widehat{\mathfrak{g}}_k)^! \rangle$$
to be non-degenerate.

For details, see Part XI (Kac-Moody Explicit Computations).
\end{proof}

\subsection{Detailed Analysis: W-Algebras}

\begin{proposition}[W-Algebra Koszul Property]\label{prop:w-algebra-koszul}
For the W-algebra $\mathcal{W}_N$ at central charge $c$:
\begin{enumerate}
\item $\mathcal{W}_3$ at $c = -2$ (minimal model): Koszul, dual exists
\item $\mathcal{W}_3$ at generic $c$: Not Koszul, no dual
\item $\mathcal{W}_N$ for $N > 3$: Rarely Koszul (only special $(N, c)$ pairs)
\item $\mathcal{W}_\infty$: Never Koszul, no dual
\end{enumerate}
\end{proposition}

\begin{proof}[Proof Idea]
W-algebras have increasingly complex relations as $N$ grows:
\begin{itemize}
\item $\mathcal{W}_3$: Quadratic and cubic relations
\item $\mathcal{W}_4$: Up to quartic relations
\item $\mathcal{W}_N$: Up to degree-$N$ relations
\end{itemize}

The Koszul property holds only when these higher relations ``degenerate'' to effective 
quadratic relations, which happens only at special values of $c$ related to minimal 
models and rational CFTs.

See Arakawa \cite{Ara07} for complete classification.
\end{proof}

%================================================================
% SECTION 6: PRACTICAL COMPUTATION
%================================================================

\section{Practical Computation of Koszul Duals}
\label{sec:practical-computation}

\subsection{Step-by-Step Guide}

\begin{procedure}[Computing $\mathcal{A}^!$ in Practice]\label{proc:compute-koszul-dual}
Given $\mathcal{A} = \text{Free}(V) / (R)$:

\textbf{Stage 1: Preparation}
\begin{enumerate}
\item Write generators $V = \text{span}\{v_1, \ldots, v_n\}$ with explicit degrees
\item Write relations $R = \text{span}\{r_1, \ldots, r_m\}$ in normal form
\item Identify which relations are quadratic, cubic, etc.
\end{enumerate}

\textbf{Stage 2: Quadratic Part}
\begin{enumerate}
\item Extract quadratic relations $R_2 \subseteq R$
\item Form dual space $V^* = \{v_1^*, \ldots, v_n^*\}$
\item Compute dual quadratic relations $R_2^\perp \subseteq (V^*)^{\otimes 2}$
\item Build quadratic part of dual:
      $$\mathcal{A}^!_{\text{quad}} = \text{Cofree}(sV^*) / (sR_2^\perp)$$
\end{enumerate}

\textbf{Stage 3: Higher Relations (if applicable)}
\begin{enumerate}
\item If $R$ contains only quadratic relations: Done, $\mathcal{A}^! = \mathcal{A}^!_{\text{quad}}$
\item If $R$ contains higher relations:
   \begin{enumerate}
   \item Define filtration $F_n\mathcal{A}$ by degree
   \item Compute successive quotients $\mathcal{A}_n = \mathcal{A} / F_n$
   \item Compute duals $\mathcal{A}_n^!$ for each $n$
   \item Take inverse limit: $\widehat{\mathcal{A}^!} = \varprojlim_n \mathcal{A}_n^!$
   \end{enumerate}
\end{enumerate}

\textbf{Stage 4: Verification}
\begin{enumerate}
\item Check conilpotency: $\bigcap_n \text{coker}(\Delta^n) = 0$
\item Verify bar-cobar: Compute $\bar{B}(\mathcal{A})$ and compare to $\mathcal{A}^!$
\item Verify cobar-bar: Compute $\Omega(\mathcal{A}^!)$ and compare to $\mathcal{A}$
\end{enumerate}
\end{procedure}

\subsection{Worked Example: Free Fermion $\beta\gamma$}

\begin{example}[Free Fermion System]\label{ex:free-fermion-koszul-dual}
The $\beta\gamma$ system is:
$$\mathcal{F} = \text{Free}(\beta, \gamma) / (\beta \otimes \gamma + \gamma \otimes \beta)$$

\textbf{Generators:} $V = \text{span}\{\beta, \gamma\}$ with $|\beta| = \lambda$, 
$|\gamma| = 1 - \lambda$.

\textbf{Relation:} $R = \{\beta \otimes \gamma + \gamma \otimes \beta\}$ (anticommutation).

\textbf{Step 1: Dual space.}
$$V^* = \text{span}\{\beta^*, \gamma^*\}$$
with $|\beta^*| = -\lambda + 1$, $|\gamma^*| = \lambda$.

\textbf{Step 2: Dual relation.}
$$R^\perp = \{\beta^* \otimes \gamma^* + \gamma^* \otimes \beta^*\}$$

(Same form! Fermionic duality is self-dual.)

\textbf{Step 3: Construct dual.}
$$\mathcal{F}^! = \text{Cofree}(s\beta^*, s\gamma^*) / (s\beta^* \otimes s\gamma^* 
+ s\gamma^* \otimes s\beta^*)$$

This is the \textbf{exterior coalgebra}:
$$\mathcal{F}^! = \bigwedge^\bullet(s\beta^*, s\gamma^*)$$

with comultiplication given by the shuffle product.

\textbf{Verification:}
\begin{itemize}
\item Conilpotent? Yes (exterior coalgebra is always conilpotent).
\item Bar-cobar? $\bar{B}(\mathcal{F}) \simeq \mathcal{F}^!$ (verified by direct computation).
\item Cobar-bar? $\Omega(\mathcal{F}^!) \simeq \mathcal{F}$ (Koszul complex).
\end{itemize}

\textbf{Conclusion:} The free fermion system is Koszul with exterior coalgebra dual.
\end{example}

%================================================================
% SECTION 7: SUMMARY AND DECISION TREE
%================================================================

\section{Summary and Decision Tree}
\label{sec:existence-summary}

\begin{figure}[H]
\centering
\begin{tikzcd}[row sep=large, column sep=large]
& \text{Given } \mathcal{A} \ar[dl] \ar[dr] & \\
\text{Quadratic?} \ar[d, "Yes"] \ar[dr, "No"] & & \\
\text{Regular?} \ar[d, "Yes"] \ar[dr, "No"] & \text{Bounded degree?} \ar[d, "Yes"] \ar[dr, "No"] & \\
\text{Conilpotent?} \ar[d, "Yes"] \ar[dr, "No"] & \text{Convergent?} \ar[d, "Yes"] \ar[dr, "No"] & \text{NO DUAL} \\
\boxed{\text{DUAL EXISTS}} & \boxed{\text{NEEDS COMPLETION}} & \text{NO DUAL}
\end{tikzcd}
\caption{Decision tree for Koszul dual existence}
\label{fig:decision-tree-existence}
\end{figure}

\begin{theorem}[Summary of Existence Criteria]\label{thm:existence-summary}
A chiral algebra $\mathcal{A}$ admits a Koszul dual $\mathcal{A}^!$ if and only if:
\begin{enumerate}
\item \textbf{Quadratic regular case:} $\mathcal{A}$ has quadratic presentation, 
      is regular (Definition \ref{def:quadratic-regularity}), and the dual coalgebra 
      is conilpotent
      
      $\implies$ $\mathcal{A}^!$ exists, given explicitly by Construction \ref{const:quadratic-dual}
      
\item \textbf{Non-quadratic convergent case:} $\mathcal{A}$ has bounded-degree 
      relations, admits a convergent filtration (Theorem \ref{thm:completion-convergence})
      
      $\implies$ $\widehat{\mathcal{A}^!}$ exists after completion
      
\item \textbf{Otherwise:} No Koszul dual exists (not even after completion)
\end{enumerate}
\end{theorem}

\subsection{Practical Recommendations}

\begin{remark}[What To Do When Dual Doesn't Exist]\label{rem:no-dual-alternatives}
If your chiral algebra $\mathcal{A}$ doesn't have a Koszul dual, alternatives include:

\textbf{Option 1: Weaken to $A_\infty$.}
Even without Koszul dual, $\mathcal{A}$ may have an $A_\infty$ ``quasi-dual'' with 
bar-cobar structures up to homotopy.

\textbf{Option 2: Work with derived category.}
The derived category $\mathcal{D}^b(\mathcal{A}\text{-mod})$ exists even without 
explicit dual.

\textbf{Option 3: Use factorization homology.}
Factorization homology $\int_X \mathcal{A}$ can be computed without Koszul dual 
(Costello-Gwilliam framework).

\textbf{Option 4: Restrict to nice subcategory.}
Sometimes a subcategory of $\mathcal{A}\text{-mod}$ has better properties.
\end{remark}

% ==========================================
% END OF APPENDIX
% ==========================================
