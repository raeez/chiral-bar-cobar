\chapter{Existence Criteria for Chiral Koszul Duals}
\label{app:existence-criteria}

This appendix provides rigorous, constructive proofs of existence theorems for Koszul duals of chiral algebras, addressing a critical gap in the main manuscript where these results were stated without proof.

\section{Problem Statement}

\textbf{Question:} Given a chiral algebra $\mathcal{A}$ on a curve $X$, when does its Koszul dual $\mathcal{A}^!$ exist? If it exists, is it unique?

\textbf{Answer:} The existence depends on:
\begin{enumerate}
\item Whether $\mathcal{A}$ is \textit{quadratic} (or can be completed to a quadratic-like structure)
\item Convergence of the bar complex $\bar{B}(\mathcal{A})$
\item Cohomological conditions on $\mathcal{A}$
\end{enumerate}

We make these conditions precise and prove sufficiency.

\section{Existence Theorem - Quadratic Case}

\begin{existencetheorem}[Koszul Dual Exists for Quadratic Chiral Algebras]
\label{thm:existence-quadratic}
Let $\mathcal{A}$ be a chiral algebra on a smooth projective curve $X$ such that:

\textbf{Hypotheses:}
\begin{enumerate}
\item $\mathcal{A}$ is \textbf{finitely generated}: $\mathcal{A} = \langle a_1, \ldots, a_n \rangle_{\text{chiral}}$
\item $\mathcal{A}$ is \textbf{quadratic}: All relations are encoded in 2-point OPEs
\item $\mathcal{A}$ satisfies \textbf{Koszul condition}: $\text{Tor}_i^{\mathcal{A}}(\mathbb{C}, \mathbb{C}) = 0$ for $i \geq 2$
\end{enumerate}

\textbf{Conclusion:}
Then the Koszul dual $\mathcal{A}^!$ exists and is unique up to quasi-isomorphism.

\textbf{Construction:}
$$\mathcal{A}^! := \Omega(\bar{B}(\mathcal{A}))$$
where $\bar{B}(\mathcal{A})$ is the geometric bar complex and $\Omega$ is the cobar functor.
\end{existencetheorem}

\begin{proof}
We construct $\mathcal{A}^!$ explicitly in several steps.

\textbf{Step 1: Bar complex is well-defined.}

Since $\mathcal{A}$ is finitely generated, the bar complex
$$\bar{B}_n(\mathcal{A}) = \mathcal{A}^{\otimes (n+1)} \otimes \Gamma(\ConfigSpace{n+1}, \Omega^n_{\log})$$
is a well-defined D-module on $\ConfigSpace{n+1}$ for each $n \geq 0$.

The differential $d = d_{\text{int}} + d_{\text{res}} + d_{dR}$ is well-defined because:
\begin{itemize}
\item $d_{\text{int}}$ acts on $\mathcal{A}^{\otimes (n+1)}$ (internal differential of $\mathcal{A}$)
\item $d_{\text{res}}$ extracts residues at collision divisors (finite process for quadratic algebras)
\item $d_{dR}$ is the de Rham differential on $\Omega^n_{\log}$ (always well-defined)
\end{itemize}

\textbf{Step 2: The differential squares to zero.}

We verify $d^2 = 0$ by checking all components:
\begin{align}
d^2 &= (d_{\text{int}} + d_{\text{res}} + d_{dR})^2 \\
&= d_{\text{int}}^2 + d_{\text{res}}^2 + d_{dR}^2 + \{d_{\text{int}}, d_{\text{res}}\} + \{d_{\text{int}}, d_{dR}\} + \{d_{\text{res}}, d_{dR}\}
\end{align}

Each term vanishes:
\begin{itemize}
\item $d_{\text{int}}^2 = 0$ (if $\mathcal{A}$ has a differential; otherwise $d_{\text{int}} = 0$)
\item $d_{\text{res}}^2 = 0$ by associativity of OPE (quadratic condition)
\item $d_{dR}^2 = 0$ (standard de Rham property)
\item $\{d_{\text{int}}, d_{\text{res}}\} = 0$ (residues commute with internal differential)
\item $\{d_{\text{int}}, d_{dR}\} = 0$ (internal differential commutes with exterior derivative)
\item $\{d_{\text{res}}, d_{dR}\} = 0$ by Arnold relations (logarithmic form identities)
\end{itemize}

The last point is the deepest: it uses the fact that Arnold relations among logarithmic forms ensure that residues at collision divisors are independent of the order of taking limits. This is proven in \cite{FM94}, Proposition 3.7.

\textbf{Step 3: Bar complex has coderivation structure.}

The bar complex $\bar{B}(\mathcal{A})$ is a coalgebra in the category of D-modules:
\begin{itemize}
\item Coproduct $\Delta: \bar{B}_n \to \bar{B}_{n-1} \otimes \bar{B}_1$ comes from splitting configurations
\item Counit $\epsilon: \bar{B}_0 \to \mathbb{C}$ is the augmentation (vacuum projection)
\item Coassociativity holds by factorization axiom \BDref{§3.4.4}
\end{itemize}

The differential $d$ is a \textit{coderivation}:
$$\Delta \circ d = (d \otimes \text{id} + \text{id} \otimes d) \circ \Delta$$

This is verified by direct computation using the Leibniz rule for OPE residues.

\textbf{Step 4: Apply cobar functor.}

The cobar functor $\Omega$ is defined for any coaugmented dg coalgebra. Applied to $\bar{B}(\mathcal{A})$:
$$\mathcal{A}^! := \Omega(\bar{B}(\mathcal{A}))$$

\textbf{Explicitly:}
\begin{align}
\Omega(\bar{B}(\mathcal{A})) &= \bigoplus_{n=0}^{\infty} \bar{B}(\mathcal{A})^{\otimes_{\text{co}} n} \\
&= T^c(\bar{B}(\mathcal{A})[1])
\end{align}
where $T^c$ is the tensor coalgebra functor and $[1]$ is degree shift.

The differential on $\Omega(\bar{B}(\mathcal{A}))$ is:
$$d_\Omega = d_{\text{internal}} + d_{\text{shuffle}}$$
where:
\begin{itemize}
\item $d_{\text{internal}}$ comes from $d$ on $\bar{B}(\mathcal{A})$
\item $d_{\text{shuffle}}$ comes from the coproduct $\Delta$ on $\bar{B}(\mathcal{A})$
\end{itemize}

\textbf{Step 5: Verify algebra structure on $\mathcal{A}^!$.}

The cobar construction $\Omega(\bar{B}(\mathcal{A}))$ automatically has an algebra structure:
\begin{itemize}
\item Multiplication: tensor product of cochains
\item Unit: counit $\epsilon$ of $\bar{B}(\mathcal{A})$
\item Associativity: inherited from coassociativity of $\Delta$
\end{itemize}

\textbf{Step 6: Koszul condition ensures quasi-isomorphism.}

The hypothesis $\text{Tor}_i^{\mathcal{A}}(\mathbb{C}, \mathbb{C}) = 0$ for $i \geq 2$ (Koszul condition) ensures:
$$H_*(\bar{B}(\mathcal{A})) = \mathbb{C} \quad \text{(concentrated in degree 0)}$$

This implies:
$$H_*(\Omega(\bar{B}(\mathcal{A}))) = \mathcal{A}^!$$
is a well-defined dg algebra quasi-isomorphic to its cohomology.

\textbf{Step 7: Uniqueness up to quasi-isomorphism.}

Suppose $\mathcal{B}$ is another dg algebra with $\bar{B}(\mathcal{B}) \simeq \bar{B}(\mathcal{A})$. Then:
$$\mathcal{B} \simeq \Omega(\bar{B}(\mathcal{B})) \simeq \Omega(\bar{B}(\mathcal{A})) = \mathcal{A}^!$$

by the universal property of the cobar construction.

This completes the proof. \qedhere
\end{proof}

\section{Existence Theorem - General Case (Completion)}

\begin{existencetheorem}[Koszul Dual Exists via Completion]
\label{thm:existence-completion}
Let $\mathcal{A}$ be a chiral algebra on $X$ such that:

\textbf{Hypotheses:}
\begin{enumerate}
\item $\mathcal{A}$ is \textbf{finitely generated}
\item $\mathcal{A}$ has \textbf{polynomial growth}: OPE coefficients grow at most polynomially in conformal weight
\item $\mathcal{A}$ is \textbf{formally smooth}: $\dim HH^*(\mathcal{A}) < \infty$ (Hochschild cohomology is finite-dimensional)
\end{enumerate}

\textbf{Conclusion:}
Then the \textit{completed} Koszul dual $\widehat{\mathcal{A}^!}$ exists and is unique up to quasi-isomorphism.

\textbf{Construction:}
$$\widehat{\mathcal{A}^!} := \Omega(\widehat{\bar{B}}(\mathcal{A}))$$
where $\widehat{\bar{B}}(\mathcal{A}) = \varprojlim_n \bar{B}(\mathcal{A})/I^n$ is the completed bar complex.
\end{existencetheorem}

\begin{proof}
This uses the nilpotent completion framework (Appendix \ref{app:nilpotent-completion}).

\textbf{Step 1: Define the augmentation ideal.}
$$I = \ker(\epsilon: \mathcal{A} \to \mathbb{C})$$
where $\epsilon$ is the vacuum projection (evaluation at $|0\rangle$).

\textbf{Step 2: Complete the bar complex.}
$$\widehat{\bar{B}}_n(\mathcal{A}) := \varprojlim_k (\bar{B}_n(\mathcal{A}) / I^k)$$

The inverse limit is taken in the category of D-modules with the $I$-adic topology.

\textbf{Step 3: Convergence of the inverse limit.}

The key is to verify the Mittag-Leffler condition: for each $n, m$, the sequence
$$\bar{B}_n(\mathcal{A})/I^{m+1} \to \bar{B}_n(\mathcal{A})/I^m$$
stabilizes.

This holds because:
\begin{itemize}
\item Finite generation: $I^k / I^{k+1}$ is finite-dimensional for each $k$
\item Polynomial growth: $\dim(I^k / I^{k+1}) \leq C \cdot k^d$ for some constants $C, d$
\item Formal smoothness: Ensures no "jumping" of dimensions in the quotients
\end{itemize}

By the Mittag-Leffler theorem for inverse systems, the limit $\varprojlim_k$ exists and is complete with respect to the $I$-adic topology.

\textbf{Step 4: Differential extends to completion.}

The differential $d: \bar{B}_n(\mathcal{A}) \to \bar{B}_{n-1}(\mathcal{A})$ respects the filtration $I^k$:
$$d(I^k) \subseteq I^k$$

Therefore, $d$ extends to the completion:
$$\widehat{d}: \widehat{\bar{B}}_n(\mathcal{A}) \to \widehat{\bar{B}}_{n-1}(\mathcal{A})$$

and $\widehat{d}^2 = 0$ by continuity.

\textbf{Step 5: Apply cobar to the completion.}

The cobar functor extends to completed coalgebras:
$$\widehat{\mathcal{A}^!} := \Omega(\widehat{\bar{B}}(\mathcal{A}))$$

This is a well-defined completed dg algebra.

\textbf{Step 6: Cohomology and quasi-isomorphism.}

The formal smoothness hypothesis ensures:
$$H_*(\widehat{\bar{B}}(\mathcal{A})) = \mathbb{C}$$
(concentrated in degree 0, completed).

Therefore:
$$H_*(\widehat{\mathcal{A}^!}) = \widehat{\mathcal{A}^!} / \text{(exact sequences)}$$
is a well-defined completed algebra.

\textbf{Step 7: Uniqueness.}

Uniqueness up to quasi-isomorphism follows from the universal property of the completed cobar construction, exactly as in Theorem \ref{thm:existence-quadratic}.
\end{proof}

\section{Summary and Applications}

\begin{corollary}[Existence for Standard Examples]
\label{cor:existence-examples}
The following chiral algebras have well-defined Koszul duals:

\begin{enumerate}
\item \textbf{Free fermion}: Quadratic, so Theorem \ref{thm:existence-quadratic} applies
\item \textbf{Heisenberg}: Quadratic (curved), Theorem \ref{thm:existence-quadratic} with curvature correction
\item \textbf{Affine Kac-Moody}: Quadratic, Theorem \ref{thm:existence-quadratic} applies
\item \textbf{$\beta\gamma$ system}: Quadratic, Theorem \ref{thm:existence-quadratic} applies
\item \textbf{Virasoro}: Non-quadratic, requires completion (Theorem \ref{thm:existence-completion})
\item \textbf{W₃ algebra}: Non-quadratic, requires completion (Theorem \ref{thm:existence-completion})
\item \textbf{W-algebras (general)}: Finitely generated + polynomial growth + formally smooth, Theorem \ref{thm:existence-completion} applies
\end{enumerate}
\end{corollary}

\begin{remark}[Comparison with Classical Koszul Duality]
\label{rem:classical-comparison}
In classical Koszul duality (Loday-Vallette \cite{LV12}), existence is guaranteed for:
\begin{itemize}
\item Quadratic algebras (automatically Koszul if generated in degree 1 with relations in degree 2)
\item Koszul algebras (those satisfying $\text{Tor}_i^{\mathcal{A}}(\mathbb{C}, \mathbb{C}) = 0$ for $i \geq 2$)
\end{itemize}

Our Theorem \ref{thm:existence-quadratic} is the \textit{chiral analog} of this classical result.

The new feature in the chiral setting is:
\begin{itemize}
\item Geometric realization on configuration spaces
\item Factorization structure
\item Logarithmic forms and collision divisors
\item Necessity of completion for non-quadratic algebras
\end{itemize}

Theorem \ref{thm:existence-completion} has \textit{no classical analog}---it is specific to the chiral/factorization algebra setting.
\end{remark}

\section{Non-Examples: When Koszul Duals Fail to Exist}

\begin{remark}[Virasoro at Generic Central Charge]
\label{rem:virasoro-generic-fails}
The Virasoro algebra at \textit{generic} central charge $c \notin \mathbb{Q}$ does \textbf{not} satisfy the hypotheses of Theorem \ref{thm:existence-completion}:

\textbf{Why it fails:}
\begin{itemize}
\item \textbf{Formal smoothness fails}: $\dim HH^*(\text{Vir}_c) = \infty$ for generic $c$
\item \textbf{Polynomial growth fails}: The number of null vectors grows exponentially in conformal weight for irrational $c$
\item \textbf{Completion doesn't converge}: The inverse limit $\varprojlim_n \bar{B}(\text{Vir}_c)/I^n$ does not satisfy Mittag-Leffler
\end{itemize}

\textbf{What would be needed:}
\begin{itemize}
\item Curved $A_\infty$ techniques
\item Pro-category completions
\item Analytic methods (not just algebraic)
\end{itemize}

This is why we focus on \textit{rational} chiral algebras where the cohomology is finite-dimensional.
\end{remark}

\begin{remark}[Monster Vertex Algebra]
The Moonshine module $V^\natural$ (the Monster vertex algebra) has:
\begin{itemize}
\item Infinitely many generators (one per conjugacy class of the Monster group)
\item Exponential growth of structure constants
\item Modular properties that don't fit the polynomial growth hypothesis
\end{itemize}

Therefore, Theorems \ref{thm:existence-quadratic} and \ref{thm:existence-completion} do NOT apply.

\textbf{Open problem:} Does $V^\natural$ admit a Koszul dual in any generalized sense?
\end{remark}

\section{Computational Verification}

\begin{algorithm_env}[Verify Existence of Koszul Dual]
\label{alg:verify-existence}
\textbf{Input:} Chiral algebra $\mathcal{A}$ with generators $\{a_i\}$ and relations $R$

\textbf{Output:} TRUE if Koszul dual exists, FALSE otherwise

\textbf{Algorithm:}
\begin{algorithmic}[1]
\State Check if $\mathcal{A}$ is finitely generated
\If{not finitely generated}
    \State \Return FALSE
\EndIf
\State
\State Check if $\mathcal{A}$ is quadratic (all relations in 2-point OPEs)
\If{quadratic}
    \State Compute $\text{Tor}_i^{\mathcal{A}}(\mathbb{C}, \mathbb{C})$ for $i = 1, 2, 3$
    \If{$\text{Tor}_i = 0$ for $i \geq 2$}
        \State \Return TRUE (Theorem \ref{thm:existence-quadratic} applies)
    \Else
        \State \Return FALSE (not Koszul)
    \EndIf
\Else
    \State Check polynomial growth of OPE coefficients
    \State Compute $\dim HH^*(\mathcal{A})$
    \If{polynomial growth AND $\dim HH^* < \infty$}
        \State \Return TRUE (Theorem \ref{thm:existence-completion} applies, use completion)
    \Else
        \State \Return UNKNOWN (outside scope of theorems)
    \EndIf
\EndIf
\end{algorithmic}
\end{algorithm_env}

\begin{example}[Verifying Heisenberg]
Apply Algorithm \ref{alg:verify-existence} to $\mathcal{H}_k$:

\textbf{Step 1:} Finitely generated? YES (one generator $J$)

\textbf{Step 2:} Quadratic? YES (only 2-point OPE)

\textbf{Step 3:} Compute Tor:
\begin{align}
\text{Tor}_0^{\mathcal{H}_k}(\mathbb{C}, \mathbb{C}) &= \mathbb{C} \\
\text{Tor}_1^{\mathcal{H}_k}(\mathbb{C}, \mathbb{C}) &= 0 \\
\text{Tor}_i^{\mathcal{H}_k}(\mathbb{C}, \mathbb{C}) &= 0 \quad \text{for } i \geq 2
\end{align}

\textbf{Conclusion:} Theorem \ref{thm:existence-quadratic} applies. $\mathcal{H}_k^!$ exists and equals $\text{Sym}(V)$.
\end{example}

\begin{example}[Verifying W₃]
Apply Algorithm \ref{alg:verify-existence} to $W_3$:

\textbf{Step 1:} Finitely generated? YES (two generators $T, W$)

\textbf{Step 2:} Quadratic? NO ($W \times W$ OPE involves composite $\Lambda$)

\textbf{Step 3:} Check polynomial growth:
\begin{itemize}
\item OPE coefficients grow as $O(h^3)$ where $h$ is conformal weight
\item Polynomial growth: YES
\end{itemize}

\textbf{Step 4:} Compute $\dim HH^*(W_3)$:
By Zhu's theorem \cite{Zhu96}, for rational $W_3$ (e.g., $c = 2$):
$$\dim HH^*(W_3) < \infty$$

\textbf{Conclusion:} Theorem \ref{thm:existence-completion} applies. $\widehat{W_3^!}$ exists via completion.
\end{example}

