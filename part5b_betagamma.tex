\chapter{Complete Example: The $\beta\gamma$ System}

\section{Setup and Conventions}

The $\beta\gamma$ system is the simplest nontrivial chiral algebra.

\subsection{Algebraic Structure}

Fields: $\beta(z)$ of conformal weight $h_\beta = 1 - \lambda$, $\gamma(z)$ of weight $h_\gamma = \lambda$.

OPE:
$$\beta(z)\gamma(w) = \frac{1}{z-w} + \text{regular}$$
$$\beta(z)\beta(w) = \text{regular}, \quad \gamma(z)\gamma(w) = \text{regular}$$

Stress tensor:
$$T = -\lambda(\beta\partial\gamma) + (1-\lambda)(\partial\beta\gamma)$$

\section{Bar Complex Computation}

\subsection{Degree by Degree Analysis}

\begin{theorem}[Complete Bar Complex]
The bar complex of $\beta\gamma$ through degree 5:

\textbf{Degree 0}: $\bar{B}^0 = \mathbb{C}|0\rangle$ (vacuum)

\textbf{Degree 1}: $\bar{B}^1 = V_\beta \oplus V_\gamma$ where
$$V_\beta = \text{span}\{\beta_{-n-h_\beta}|0\rangle : n \geq 0\}$$
$$V_\gamma = \text{span}\{\gamma_{-n-h_\gamma}|0\rangle : n \geq 0\}$$

\textbf{Degree 2}: 
\begin{align}
\bar{B}^2 = &(V_\beta \otimes V_\beta) \oplus (V_\gamma \otimes V_\gamma) \\
&\oplus (V_\beta \otimes V_\gamma) \oplus (V_\gamma \otimes V_\beta) \\
&\oplus V_{\partial\beta} \oplus V_{\partial\gamma}
\end{align}

The differential $d: \bar{B}^2 \to \bar{B}^1$:
\begin{align}
d(\beta \otimes \beta) &= 0 \text{ (no pole in OPE)} \\
d(\gamma \otimes \gamma) &= 0 \\
d(\beta \otimes \gamma) &= \text{Res}_{z_1=z_2}\left[\frac{dz_1}{z_1-z_2}\right] \cdot 1 = 1 \\
d(\gamma \otimes \beta) &= -1 \\
d(\partial\beta) &= 0, \quad d(\partial\gamma) = 0
\end{align}

\textbf{Degree 3}: Dimension = 27
Components include:
\begin{itemize}
\item $(V_\beta)^{\otimes 3}$: 1-dimensional
\item $(V_\beta)^{\otimes 2} \otimes V_\gamma$: 3 orderings
\item $V_\beta \otimes (V_\gamma)^{\otimes 2}$: 3 orderings
\item $(V_\gamma)^{\otimes 3}$: 1-dimensional
\item Derivative terms
\end{itemize}

Key differential:
$$d(\beta_1 \otimes \beta_2 \otimes \gamma_3) = \beta_1 \otimes 1 - 1 \otimes \beta_2$$

\textbf{Growth Formula}:
$$\dim(\bar{B}^n) = 2 \cdot 3^{n-1} \text{ for } n \geq 1$$
\end{theorem}

\begin{proof}
By induction on degree. The factor of 2 comes from choosing $\beta$ or $\gamma$ as leading term. The factor $3^{n-1}$ from choosing $\beta$, $\gamma$, or derivative at each subsequent position.
\end{proof}

\subsection{Cohomology Calculation}

\begin{theorem}[Bar Cohomology of $\beta\gamma$]
$$H^n(\bar{B}(\beta\gamma)) = \begin{cases}
\mathbb{C} & n = 0 \\
\mathbb{C} & n = 1 \\
\mathbb{C}^2 & n = 2 \\
\vdots
\end{cases}$$
The cohomology is concentrated in finite degrees when $\lambda$ is generic.
\end{theorem}

\begin{proof}
We compute kernel and image at each degree:

\textbf{Degree 0}: $H^0 = \mathbb{C}$ (vacuum).

\textbf{Degree 1}: 
$$\ker(d^1) = V_\beta \oplus V_\gamma$$
$$\text{im}(d^2) = \mathbb{C} \cdot (\beta - \gamma)$$
$$H^1 = (V_\beta \oplus V_\gamma)/\mathbb{C}(\beta - \gamma) \cong \mathbb{C}$$

\textbf{Degree 2}: Similar analysis using explicit bases.
\end{proof}

\section{Koszul Dual}

\subsection{Dual Algebra Structure}

\begin{theorem}[Koszul Dual of $\beta\gamma$]
The Koszul dual is the $\beta'\gamma'$ system with:
\begin{itemize}
\item Opposite conformal weights: $h_{\beta'} = \lambda$, $h_{\gamma'} = 1 - \lambda$
\item Same OPE structure
\item Twisted by parity if $\lambda \in \mathbb{Z}$
\end{itemize}
\end{theorem}

\subsection{Verification of Duality}

\begin{proposition}
The pairing
$$\langle \cdot, \cdot \rangle: \bar{B}(\beta\gamma) \otimes \bar{B}(\beta'\gamma') \to \mathbb{C}$$
defined by configuration space integration is perfect.
\end{proposition}

\section{Special Cases}

\subsection{Free Fermions ($\lambda = 0$ or $1$)}

When $\lambda = 1$:
$$\{\beta(z), \gamma(w)\} = \delta(z-w)$$
The system becomes fermionic.

\begin{theorem}[Fermionic Bar Complex]
$$\bar{B}(\text{fermions}) \simeq \Lambda^*[\xi, \eta]$$
exterior algebra on two generators.
\end{theorem}

\subsection{Symplectic Bosons ($\lambda = 1/2$)}

At $\lambda = 1/2$, both fields have weight $1/2$:
$$T = \frac{1}{2}(\partial\beta\gamma - \beta\partial\gamma)$$

Special properties:
\begin{itemize}
\item Logarithmic OPE with stress tensor
\item Non-semisimple representation theory
\item Appears in logarithmic CFT
\end{itemize}

\section{Geometric Realization}

\subsection{Configuration Space Picture}

The bar complex elements are:
$$\omega_{n,m} \in \Gamma(C_{n+m+1}(X), (\beta^{\boxtimes n} \otimes \gamma^{\boxtimes m}) \otimes \Omega^*_{\log})$$

Explicit form:
$$\omega_{n,m} = \beta(z_1) \cdots \beta(z_n) \gamma(w_1) \cdots \gamma(w_m) \prod_{i<j} \eta_{ij}$$

\subsection{Residue Computation}

The differential extracts:
$$d(\omega_{n,m}) = \sum_{i,j} \text{Res}_{z_i = w_j}[\omega_{n,m}] = \sum_{i,j} \omega_{n-1,m-1}|_{z_i = w_j}$$

This realizes the algebraic bar differential geometrically.

%===================================================================================
% PATCH 041: BETA-GAMMA SYSTEM - COMPLETE ANALYSIS
%===================================================================================

\section{Beta-Gamma Systems: Complete Analysis}
\label{sec:beta-gamma-complete-analysis}

\subsection{Physical Motivation}

\begin{motivation}[Witten: Ghosts in String Theory]
In the covariant quantization of bosonic string theory, the BRST procedure introduces 
ghost fields:
\begin{itemize}
\item $b(z)$: Anti-commuting ghost of conformal weight $\lambda = 2$
\item $c(z)$: Anti-commuting ghost of conformal weight $1-\lambda = -1$
\end{itemize}

\textbf{General $\beta$-$\gamma$ system:} For any $\lambda \in \mathbb{C}$:
\begin{itemize}
\item $\beta(z)$: Field of weight $\lambda$
\item $\gamma(z)$: Field of weight $1-\lambda$
\end{itemize}

Statistics: Fermionic if $\lambda \in \mathbb{Z} + 1/2$, Bosonic if $\lambda \in \mathbb{Z}$.
\end{motivation}

\subsection{Geometric Realization}

\begin{construction}[Geometric $\beta$-$\gamma$ System]
\label{const:geometric-beta-gamma}
On a curve $X$, the $\beta$-$\gamma$ system with parameter $\lambda$ is geometrically:
\begin{align}
\beta &\in \Gamma(X, K_X^\lambda \otimes \mathcal{L})\\
\gamma &\in \Gamma(X, K_X^{1-\lambda} \otimes \mathcal{L}^*)
\end{align}

where $K_X$ is the canonical bundle and $\mathcal{L}$ is an auxiliary line bundle.

\textbf{Special cases:}
\begin{enumerate}
\item $\lambda = 1$: $\beta \in \Gamma(K_X)$ (differentials), $\gamma \in \Gamma(\mathcal{O}_X)$ (functions)
\item $\lambda = 2$: $\beta \in \Gamma(K_X^2)$ (quadratic differentials), $\gamma \in \Gamma(K_X^{-1})$ (vector fields)
\item $\lambda = 1/2$: Fermions (spin structures required)
\end{enumerate}
\end{construction}

\subsection{Complete OPE Structure}

\begin{definition}[Defining OPE for $\beta$-$\gamma$]
\label{def:beta-gamma-ope-complete}
The fundamental OPE is:
$$\beta(z)\gamma(w) \sim \frac{1}{z-w}$$

All other OPEs are regular:
\begin{align}
\beta(z)\beta(w) &\sim 0\\
\gamma(z)\gamma(w) &\sim 0
\end{align}
\end{definition}

\subsection{Mode Expansions and Commutation Relations}

\begin{proposition}[Mode Algebra]
\label{prop:beta-gamma-modes}
Expand in modes:
\begin{align}
\beta(z) &= \sum_{n \in \mathbb{Z}} \beta_n z^{-n-\lambda}\\
\gamma(z) &= \sum_{n \in \mathbb{Z}} \gamma_n z^{-n-(1-\lambda)}
\end{align}

The (anti-)commutation relations are:
$$[\beta_m, \gamma_n]_\pm = \delta_{m+n,0}$$

where $[\cdot,\cdot]_\pm$ is commutator for bosonic, anti-commutator for fermionic.
\end{proposition}

\subsection{Stress-Energy Tensor}

\begin{theorem}[Stress Tensor for $\beta$-$\gamma$]
\label{thm:beta-gamma-stress}
The stress-energy tensor is:
$$T^{\beta\gamma}(z) = \lambda :\beta(z)\partial\gamma(z): + (1-\lambda):\partial\beta(z)\gamma(z):$$

This generates the Virasoro algebra with central charge:
$$c_{\beta\gamma} = -2(6\lambda^2 - 6\lambda + 1)$$
\end{theorem}

\begin{computation}[Central Charges for Special Cases]
\label{comp:beta-gamma-central-charges}
\begin{itemize}
\item $\lambda = 1$: $c = -2(6-6+1) = -2$ 
\item $\lambda = 2$: $c = -2(24-12+1) = -26$ (string theory $bc$ ghosts!)
\item $\lambda = 1/2$: $c = -2(6/4 - 3 + 1) = 1/2$ (fermions)
\item $\lambda = 0$: $c = -2$ (symplectic bosons)
\end{itemize}

\textbf{Note:} Negative central charges are allowed for non-unitary theories (ghosts).
\end{computation}

\subsection{Koszul Dual Structure}

\begin{theorem}[Bar Complex of $\beta$-$\gamma$ System]
\label{thm:beta-gamma-bar}
The bar complex of the $\beta$-$\gamma$ system is:
$$\bar{B}^n = \left(\text{Free}[\beta, \gamma]\right)^{\otimes (n+1)} \otimes \Omega^n(\overline{C}_{n+1}(X))$$

The differential is:
$$d = \sum_{i<j} \text{Res}_{z_i=z_j}\left[\beta_i(z_i)\gamma_j(z_j) \cdot \eta_{ij}\right]$$

where $\eta_{ij} = \frac{dz_i}{z_i-z_j}$ are logarithmic forms.
\end{theorem}

\subsection{Role in BRST and Wakimoto}

\begin{remark}[Connection to Wakimoto]
The Wakimoto free field realization uses $\beta$-$\gamma$ systems extensively:
$$\mathcal{M}_{\text{Wak}} = \text{Free}[\phi_i] \otimes \bigotimes_{\alpha \in \Delta_+} \text{Free}[\beta_\alpha, \gamma_\alpha]$$

Each root $\alpha$ contributes a $\beta$-$\gamma$ system. These are the building blocks 
for the free field realization of affine Kac-Moody and W-algebras.
\end{remark}

\subsection{Universal Property}

\begin{theorem}[Universal Property of $\beta$-$\gamma$]
\label{thm:beta-gamma-universal}
The $\beta$-$\gamma$ system is the \textbf{free vertex algebra} generated by two fields 
$\beta, \gamma$ of weights $\lambda, 1-\lambda$ with the single relation:
$$\beta(z)\gamma(w) \sim \frac{1}{z-w}$$

Universal property: For any vertex algebra $V$ with fields $\beta', \gamma'$ satisfying 
this OPE, there exists a unique homomorphism:
$$\text{Free}[\beta, \gamma] \to V$$
sending $\beta \mapsto \beta'$, $\gamma \mapsto \gamma'$.
\end{theorem}

\subsection{Summary}

\begin{summary}[Four Perspectives on $\beta$-$\gamma$]
\textbf{Witten:} Ghost fields in string theory, BRST quantization

\textbf{Kontsevich:} Geometric realization as sections of bundles

\textbf{Serre:} All composite operators computed explicitly

\textbf{Grothendieck:} Universal free field, functoriality
\end{summary}

