\chapter{Complete Example: The $\beta\gamma$ System}

\section{Setup and Conventions}

The $\beta\gamma$ system is the simplest nontrivial chiral algebra.

\subsection{Algebraic Structure}

Fields: $\beta(z)$ of conformal weight $h_\beta = 1 - \lambda$, $\gamma(z)$ of weight $h_\gamma = \lambda$.

OPE:
$$\beta(z)\gamma(w) = \frac{1}{z-w} + \text{regular}$$
$$\beta(z)\beta(w) = \text{regular}, \quad \gamma(z)\gamma(w) = \text{regular}$$

Stress tensor:
$$T = -\lambda(\beta\partial\gamma) + (1-\lambda)(\partial\beta\gamma)$$

\section{Bar Complex Computation}

\subsection{Degree by Degree Analysis}

\begin{theorem}[Complete Bar Complex]
The bar complex of $\beta\gamma$ through degree 5:

\textbf{Degree 0}: $\bar{B}^0 = \mathbb{C}|0\rangle$ (vacuum)

\textbf{Degree 1}: $\bar{B}^1 = V_\beta \oplus V_\gamma$ where
$$V_\beta = \text{span}\{\beta_{-n-h_\beta}|0\rangle : n \geq 0\}$$
$$V_\gamma = \text{span}\{\gamma_{-n-h_\gamma}|0\rangle : n \geq 0\}$$

\textbf{Degree 2}: 
\begin{align}
\bar{B}^2 = &(V_\beta \otimes V_\beta) \oplus (V_\gamma \otimes V_\gamma) \\
&\oplus (V_\beta \otimes V_\gamma) \oplus (V_\gamma \otimes V_\beta) \\
&\oplus V_{\partial\beta} \oplus V_{\partial\gamma}
\end{align}

The differential $d: \bar{B}^2 \to \bar{B}^1$:
\begin{align}
d(\beta \otimes \beta) &= 0 \text{ (no pole in OPE)} \\
d(\gamma \otimes \gamma) &= 0 \\
d(\beta \otimes \gamma) &= \text{Res}_{z_1=z_2}\left[\frac{dz_1}{z_1-z_2}\right] \cdot 1 = 1 \\
d(\gamma \otimes \beta) &= -1 \\
d(\partial\beta) &= 0, \quad d(\partial\gamma) = 0
\end{align}

\textbf{Degree 3}: Dimension = 27
Components include:
\begin{itemize}
\item $(V_\beta)^{\otimes 3}$: 1-dimensional
\item $(V_\beta)^{\otimes 2} \otimes V_\gamma$: 3 orderings
\item $V_\beta \otimes (V_\gamma)^{\otimes 2}$: 3 orderings
\item $(V_\gamma)^{\otimes 3}$: 1-dimensional
\item Derivative terms
\end{itemize}

Key differential:
$$d(\beta_1 \otimes \beta_2 \otimes \gamma_3) = \beta_1 \otimes 1 - 1 \otimes \beta_2$$

\textbf{Growth Formula}:
$$\dim(\bar{B}^n) = 2 \cdot 3^{n-1} \text{ for } n \geq 1$$
\end{theorem}

\begin{proof}
By induction on degree. The factor of 2 comes from choosing $\beta$ or $\gamma$ as leading term. The factor $3^{n-1}$ from choosing $\beta$, $\gamma$, or derivative at each subsequent position.
\end{proof}

\subsection{Cohomology Calculation}

\begin{theorem}[Bar Cohomology of $\beta\gamma$]
$$H^n(\bar{B}(\beta\gamma)) = \begin{cases}
\mathbb{C} & n = 0 \\
\mathbb{C} & n = 1 \\
\mathbb{C}^2 & n = 2 \\
\vdots
\end{cases}$$
The cohomology is concentrated in finite degrees when $\lambda$ is generic.
\end{theorem}

\begin{proof}
We compute kernel and image at each degree:

\textbf{Degree 0}: $H^0 = \mathbb{C}$ (vacuum).

\textbf{Degree 1}: 
$$\ker(d^1) = V_\beta \oplus V_\gamma$$
$$\text{im}(d^2) = \mathbb{C} \cdot (\beta - \gamma)$$
$$H^1 = (V_\beta \oplus V_\gamma)/\mathbb{C}(\beta - \gamma) \cong \mathbb{C}$$

\textbf{Degree 2}: Similar analysis using explicit bases.
\end{proof}

\section{Koszul Dual}

\subsection{Dual Algebra Structure}

\begin{theorem}[Koszul Dual of $\beta\gamma$]
The Koszul dual is the $\beta'\gamma'$ system with:
\begin{itemize}
\item Opposite conformal weights: $h_{\beta'} = \lambda$, $h_{\gamma'} = 1 - \lambda$
\item Same OPE structure
\item Twisted by parity if $\lambda \in \mathbb{Z}$
\end{itemize}
\end{theorem}

\subsection{Verification of Duality}

\begin{proposition}
The pairing
$$\langle \cdot, \cdot \rangle: \bar{B}(\beta\gamma) \otimes \bar{B}(\beta'\gamma') \to \mathbb{C}$$
defined by configuration space integration is perfect.
\end{proposition}

\section{Special Cases}

\subsection{Free Fermions ($\lambda = 0$ or $1$)}

When $\lambda = 1$:
$$\{\beta(z), \gamma(w)\} = \delta(z-w)$$
The system becomes fermionic.

\begin{theorem}[Fermionic Bar Complex]
$$\bar{B}(\text{fermions}) \simeq \Lambda^*[\xi, \eta]$$
exterior algebra on two generators.
\end{theorem}

\subsection{Symplectic Bosons ($\lambda = 1/2$)}

At $\lambda = 1/2$, both fields have weight $1/2$:
$$T = \frac{1}{2}(\partial\beta\gamma - \beta\partial\gamma)$$

Special properties:
\begin{itemize}
\item Logarithmic OPE with stress tensor
\item Non-semisimple representation theory
\item Appears in logarithmic CFT
\end{itemize}

\section{Geometric Realization}

\subsection{Configuration Space Picture}

The bar complex elements are:
$$\omega_{n,m} \in \Gamma(C_{n+m+1}(X), (\beta^{\boxtimes n} \otimes \gamma^{\boxtimes m}) \otimes \Omega^*_{\log})$$

Explicit form:
$$\omega_{n,m} = \beta(z_1) \cdots \beta(z_n) \gamma(w_1) \cdots \gamma(w_m) \prod_{i<j} \eta_{ij}$$

\subsection{Residue Computation}

The differential extracts:
$$d(\omega_{n,m}) = \sum_{i,j} \text{Res}_{z_i = w_j}[\omega_{n,m}] = \sum_{i,j} \omega_{n-1,m-1}|_{z_i = w_j}$$

This realizes the algebraic bar differential geometrically.