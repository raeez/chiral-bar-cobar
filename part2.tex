
\section{Configuration Spaces and Logarithmic Forms}

% \begin{tcolorbox}[colback=blue!5!white,colframe=blue!75!black,title=Section Overview]
% \textbf{Key Questions:}
% \begin{itemize}
% \item How do we compactify configuration spaces to capture collision data?
% \item What are logarithmic forms and why do they appear?
% \item How do Arnold relations encode higher operadic structure?
% \end{itemize}

% \textbf{Main Results:}
% \begin{itemize}
% \item Theorem \ref{thm:FM-compactification}: Fulton-MacPherson compactification
% \item Proposition \ref{prop:log-forms}: Logarithmic forms and residues
% \item Example \ref{ex:config-P1}: Explicit computation for $\mathbb{P}^1$
% \end{itemize}
% \end{tcolorbox}
 
\subsection{The Relative Perspective}

Following Grothendieck's philosophy of relative algebraic geometry, we work systematically in families:

\begin{definition}[Relative Bar Complex]
For a family of chiral algebras $\mathcal{A} \to S$ parametrized by a base $S$, the relative bar complex
\[
\chirbar_{S/\text{rel}}(\mathcal{A}) \to S
\]
lives over the relative configuration space $\Conf{\bullet}(X \times S/S)$.
\end{definition}

\begin{theorem}[Base Change]
The geometric bar construction commutes with base change:
\[
f^*\chirbar_S(\mathcal{A}) \cong \chirbar_{S'}(f^*\mathcal{A})
\]
for any morphism $f: S' \to S$.
\end{theorem}

This relative viewpoint reveals:
\begin{itemize}
\item Deformation theory: Families over $\text{Spec}(\mathbb{C}[\epsilon]/\epsilon^2)$
\item Moduli spaces: Universal families over $\mathcal{M}_{\text{ChirAlg}}$
\item Quantum groups: Families over $\text{Spec}(\mathbb{C}[[h]])$ with $h \to 0$ classical limit
\end{itemize}

\subsection{Configuration Spaces of Curves}
 
We now introduce the geometric spaces that will support our bar complexes. Throughout this section, $X$ denotes a smooth algebraic curve over $\C$ of dimension 1.
 
\begin{definition}[Configuration Space]
For a smooth algebraic curve $X$ over $\C$, the configuration space of $n$ distinct ordered points is:
\[
C_n(X) = \{(x_1, \ldots, x_n) \in X^n \mid x_i \neq x_j \text{ for all } i \neq j\}
\]
This is a smooth quasi-projective variety of dimension $n \cdot \dim X = n$ when $\dim X = 1$.

\begin{notation}
Throughout this paper:
\begin{itemize}
\item $C_n(X)$ denotes the open configuration space
\item $\overline{C_n(X)}$ denotes its Fulton-MacPherson compactification  
\item $\partial\overline{C_n(X)} = \overline{C_n(X)} \setminus C_n(X)$ denotes the boundary divisor
\end{itemize}
\end{notation}

\end{definition}
 
\begin{proposition}[Fundamental Group]
The fundamental group $\pi_1(C_n(X))$ is the pure braid group $P_n(X)$ on $n$ strands over $X$. For $X = \C$, this is the kernel of $B_n \to S_n$ where $B_n$ is the Artin braid group with generators $\sigma_i$ ($i = 1, \ldots, n-1$) and relations:
\begin{align}
\sigma_i\sigma_j &= \sigma_j\sigma_i \quad \text{if } |i-j| > 1 \\
\sigma_i\sigma_{i+1}\sigma_i &= \sigma_{i+1}\sigma_i\sigma_{i+1} \quad \text{(braid relations)}
\end{align}
\end{proposition}
 
The configuration space $C_n(X)$ is highly non-compact, with "points at infinity" corresponding to various collision patterns. The Fulton-MacPherson compactification provides a canonical way to add these points:
 
\subsection{The Fulton-MacPherson Compactification}
 
\begin{theorem}[Fulton-MacPherson Compactification \cite{FM94}]\label{thm:FM}
There exists a smooth compactification $\overline{C_n(X)}$ with a natural stratification by combinatorial type. More precisely, we have
a functorial compactification
\[j: C_n(X) \hookrightarrow \overline{C_n(X)}\]
where $\overline{C_n(X)}$ is obtained by iterated blow-ups along diagonals.

The compactification has the following properties:
\begin{enumerate}
\item The complement $D = \barC_n(X) \setminus C_n(X)$ is a normal crossing divisor
\item Boundary strata are indexed by nested partitions of $\{1, \ldots, n\}$ (equivalently, by rooted trees with $n$ leaves)
\item For each stratum $D_\pi$ corresponding to partition $\pi = \{B_1, \ldots, B_k\}$:
\[
D_\pi \cong \barC_k(X) \times \prod_{i=1}^k \barC_{|B_i|}(\C)
\]
where the first factor records the positions of "bubbles" and the product records configurations within each bubble
\item The compactification is functorial for smooth morphisms and open embeddings of curves
\end{enumerate}
\end{theorem}
 
\begin{proof}[Construction Sketch]
The compactification is obtained by a sequence of blow-ups:
\begin{enumerate}
\item Start with $X^n$
\item Blow up the smallest diagonal $\Delta_n = \{x_1 = \cdots = x_n\}$
\item Blow up the proper transforms of all partial diagonals $\Delta_I = \{x_i = x_j : i,j \in I\}$ in order of decreasing codimension
\item The exceptional divisors encode:
\begin{itemize}
\item Which points collide (given by the partition)
\item Relative rates of approach (radial coordinates in the blow-up)
\item Relative angles of approach (angular coordinates)
\end{itemize}
\end{enumerate}
 
The key insight is that the blow-up process naturally records the "speed" and "direction" of collisions, not just which points collide. The normal crossing property follows from the careful ordering of blow-ups, ensuring transversality at each step.
\end{proof}
 
\begin{example}[Configuration Space of $\mathbb{P}^1$]\label{ex:config-P1}
For $X = \mathbb{P}^1$, we compute $\overline{C}_3(\mathbb{P}^1)$ explicitly:
\begin{enumerate}
\item The open configuration space: $C_3(\mathbb{P}^1) = \{(z_1, z_2, z_3) \in (\mathbb{P}^1)^3 : z_i \neq z_j\}$

\item Use $\text{PSL}_2(\mathbb{C})$ to fix $(z_1, z_2, z_3) = (0, 1, \lambda)$ with $\lambda \in \mathbb{C} \setminus \{0,1\}$

\item The compactification adds three divisors:
   \begin{itemize}
   \item $D_{12}$: $\lambda \to 0$ (collision of $z_1, z_2$)  
   \item $D_{23}$: $\lambda \to 1$ (collision of $z_2, z_3$)
   \item $D_{13}$: $\lambda \to \infty$ (collision of $z_1, z_3$)
   \end{itemize}

\item Result: $\overline{C}_3(\mathbb{P}^1) \cong \mathbb{P}^1$ with three marked points

\item The logarithmic forms:
   $$\eta_{12} = d\log\lambda, \quad \eta_{23} = d\log(1-\lambda), \quad \eta_{13} = -d\log\lambda$$
   
\item Key relation (Arnold relation):
   $$\eta_{12} + \eta_{23} + \eta_{13} = d\log\lambda + d\log(1-\lambda) - d\log\lambda = d\log(1-\lambda) \neq 0$$
   
   But when pulled back to any 2-dimensional stratum:
   $$\eta_{12} + \eta_{23} + \eta_{13}|_{\text{boundary}} = 0$$
\end{enumerate}

This exemplifies how configuration spaces encode both local (OPE) and global (monodromy) data.
\end{example}
 
\subsection{Logarithmic Differential Forms}

\begin{remark}[Why Logarithmic Forms?]
The appearance of logarithmic forms is not accidental but inevitable: they are the unique meromorphic 1-forms with prescribed residues at collision divisors. When operators collide in conformal field theory, the singularity structure is captured precisely by forms like $d\log(z_i - z_j)$. To make these forms single-valued requires choice. These choices encode precisely the monodromy data that will later appear in our $A_\infty$ relations. The branch cuts we choose are not arbitrary conventions but encode genuine topological information about the configuration space.
\end{remark}


\begin{definition}[Branch Cut Convention - Rigorous]
For each pair $(i,j)$ with $i < j$, we fix a branch of $\log(z_i - z_j)$ as follows:
\begin{enumerate}
\item Choose a basepoint $* \in C_n(X)$
\item For intuition: think of this as choosing a reference configuration where all points are well-separated

\item For each loop $\gamma$ based at $*$, define the monodromy $M_\gamma: \mathbb{C} \to \mathbb{C}$
\item The monodromy measures how our chosen branch of the logarithm changes as points wind around each other

\item Fix the branch by requiring $M_\gamma = \text{id}$ for contractible loops
\item This is equivalent to choosing a trivialization of the local system of logarithms over the universal cover
\item For concreteness on $X = \mathbb{C}$, we use the principal branch: $-\pi < \text{Im}(\log(z_i - z_j)) \leq \pi$

\item This determines $\log(z_i - z_j)$ up to a constant, which we fix by continuity from the basepoint
\item The constant is normalized so that $\log(1) = 0$
\end{enumerate}
The resulting logarithmic forms are single-valued on the universal cover $\widetilde{C_n(X)}$.
\end{definition}

\begin{remark}[Monodromy Consistency] The choice of branch cuts must be compatible with the factorization structure of the chiral algebra. Specifically, for any three points $z_i, z_j, z_k$, the monodromy around the total diagonal satisfies:
$$M_{ijk} = M_{ij} \circ M_{jk} \circ M_{ki}$$
This ensures the Arnold relations lift consistently to the universal cover.
\end{remark}

\begin{definition}[Logarithmic Forms with Poles]
The sheaf of logarithmic $p$-forms on $\overline{C}_n(X)$ is the subsheaf of meromorphic forms:
$$\Omega^p_{\overline{C}_n(X)}(\log D) = \{p\text{-forms } \omega : \omega \text{ and } d\omega \text{ have at most simple poles along } D\}$$

In local coordinates $(u_1,\ldots,u_n,\epsilon_{ij},\theta_{ij})_{i<j}$ near a boundary stratum:
$$\Omega^p_{\overline{C}_n(X)}(\log D) = \bigoplus_{I \subset \{(i,j): i<j\}} \Omega^{p-|I|}_{smooth} \wedge \bigwedge_{(i,j) \in I} d\log\epsilon_{ij}$$
\end{definition}

\begin{proposition}[Logarithmic Form Properties]
The forms $\eta_{ij} = d\log(z_i - z_j)$ satisfy:
\begin{enumerate}
\item $\eta_{ji} = -\eta_{ij}$ (antisymmetry)
\item Near $D_{ij}$: $\eta_{ij} = d\log\epsilon_{ij} + id\theta_{ij} + O(\epsilon_{ij})$
\item $\text{Res}_{D_{ij}}[\eta_{ij}] = 1$ (normalization)
\item $d\eta_{ij} = 0$ away from higher codimension strata
\item The residue map $\text{Res}_{D_{ij}}: \Omega^p(\log D) \to \Omega^{p-1}(D_{ij})$ is well-defined
\end{enumerate}
\end{proposition}

Near a boundary divisor $D_{ij}$ where points $x_i \to x_j$ collide, we use blow-up coordinates:
 
\begin{definition}[Blow-up Coordinates]\label{def:blowup}
Near $D_{ij} \subset \barC_n(X)$, introduce coordinates:
\begin{align}
u_{ij} &= \frac{x_i + x_j}{2} \quad \text{(center of collision)} \\
\epsilon_{ij} &= |x_i - x_j| \quad \text{(separation, serves as normal coordinate to } D_{ij}) \\
\theta_{ij} &= \arg(x_i - x_j) \quad \text{(angle of approach)}
\end{align}
In these coordinates:
\begin{align}
x_i &= u_{ij} + \frac{\epsilon_{ij}}{2}e^{i\theta_{ij}} \\
x_j &= u_{ij} - \frac{\epsilon_{ij}}{2}e^{i\theta_{ij}}
\end{align}
\end{definition}

\begin{proposition}[Explicit Local Charts for $\overline{C}_n(X)$]\label{prop:local-charts}
Near a boundary divisor $D_{ij}$ where $z_i \to z_j$, introduce local coordinates:
\begin{align}
w &= z_j \quad \text{(center of collision)} \\
\epsilon &= z_i - z_j \quad \text{(separation, goes to 0)} \\
\zeta_k &= \frac{z_k - z_j}{z_i - z_j} \quad \text{for } k \neq i,j
\end{align}

The compactification replaces $\epsilon \to 0$ with a $\mathbb{P}^1$ of ``directions of approach.''
The logarithmic form becomes:
$$\eta_{ij} = d\log \epsilon = \frac{d\epsilon}{\epsilon}$$
having a simple pole along $D_{ij} = \{\epsilon = 0\}$.

This construction is:
\begin{itemize}
\item \textbf{Canonical}: Independent of choices (uses only the complex structure)
\item \textbf{Functorial}: Natural with respect to curve morphisms
\item \textbf{Minimal}: The unique smooth compactification with normal crossing divisors
\end{itemize}
\end{proposition}
 
The basic logarithmic 1-forms that will appear throughout our constructions are:
 
\begin{definition}[Basic Logarithmic Forms]
For distinct indices $i, j \in \{1, \ldots, n\}$, define:
\[
\eta_{ij} = d\log(x_i - x_j) = \frac{dx_i - dx_j}{x_i - x_j}
\]
These forms have simple poles along $D_{ij}$ and are regular elsewhere.
\end{definition}
 
\begin{proposition}[Properties of $\eta_{ij}$]\label{prop:eta}
The forms $\eta_{ij}$ satisfy:
\begin{enumerate}
\item Antisymmetry: $\eta_{ji} = -\eta_{ij}$
\item Blow-up expansion: Near $D_{ij}$,
\[
\eta_{ij} = d\log \epsilon_{ij} + id\theta_{ij} + \text{(regular terms)}
\]
\item Residue: $\Res_{D_{ij}} \eta_{ij} = 1$ (normalized by our convention)
\item Closure: $d\eta_{ij} = 0$ away from higher codimension strata
\end{enumerate}
\end{proposition}
 
\begin{proof}
(1) is immediate from the definition. For (2), compute in blow-up coordinates:
\[
x_i - x_j = \epsilon_{ij} e^{i\theta_{ij}}
\]
Therefore $d\log(x_i - x_j) = d\log(\epsilon_{ij} e^{i\theta_{ij}}) = d\log \epsilon_{ij} + id\theta_{ij}$.
 
For (3), the residue extracts the coefficient of $d\log \epsilon_{ij}$, which is 1 by our computation.
 
For (4), since $\eta_{ij}$ is locally $d$ of a function away from other collision divisors, we have $d\eta_{ij} = d^2\log(x_i - x_j) = 0$.
\end{proof}
 
\subsection{The Orlik-Solomon Algebra}
 
The logarithmic forms $\eta_{ij}$ generate a differential graded algebra with remarkable properties:

\subsubsection{Three-term relation}
\begin{theorem}[Arnold Relations - Rigorous]
For any triple of distinct indices $i, j, k \in \{1,\ldots,n\}$:
$$\eta_{ij} \wedge \eta_{jk} + \eta_{jk} \wedge \eta_{ki} + \eta_{ki} \wedge \eta_{ij} = 0$$
\end{theorem}

\begin{proof}[Complete Proof]
We work on the universal cover to avoid branch issues. Define:
$$\omega = \eta_{ij} + \eta_{jk} + \eta_{ki} = d\log((z_i - z_j)(z_j - z_k)(z_k - z_i))$$

Since $\omega = df$ for a single-valued function $f$ on the universal cover, we have $d\omega = 0$.

Computing explicitly:
\begin{align}
d\omega &= d\eta_{ij} + d\eta_{jk} + d\eta_{ki}\\
&= 0 \text{ away from higher codimension}
\end{align}

At the codimension-2 stratum $D_{ijk}$ where all three points collide, we use residue calculus:
$$\text{Res}_{D_{ijk}}[\eta_{ij} \wedge \eta_{jk}] = \lim_{(z_i,z_j,z_k) \to (z,z,z)} \left[\frac{dz_i - dz_j}{z_i - z_j} \wedge \frac{dz_j - dz_k}{z_j - z_k}\right]$$

In blow-up coordinates with $z_i = z + \epsilon_1 e^{i\theta_1}$, $z_j = z$, $z_k = z + \epsilon_2 e^{i\theta_2}$:
$$\eta_{ij} \wedge \eta_{jk} = d\log\epsilon_1 \wedge d\log\epsilon_2 + \text{(angular terms)}$$

The sum of all three terms gives zero by symmetry under $S_3$ action.
\end{proof}
 
\begin{theorem}[Cohomology via Orlik-Solomon]
For $X = \C$, the cohomology of $\barC_n(\C)$ is:
\[
H^*(\barC_n(\C)) \cong \text{OS}(A_{n-1})
\]
where $\text{OS}(A_{n-1})$ is the Orlik-Solomon algebra of the braid arrangement $A_{n-1}$. The Poincaré polynomial is:
\[
\sum_{k=0}^{n-1} \dim H^k(\barC_n(\C)) \cdot t^k = \prod_{i=1}^{n-1}(1 + it)
\]
\end{theorem}
 
\subsection{No-Broken-Circuit Bases}
 
For explicit computations, we need concrete bases for the cohomology:
 
\begin{definition}[Broken Circuit]
Fix a total order on pairs $(i, j)$ with $i < j$ (we use lexicographic order). A \emph{broken circuit} is a set obtained by removing the minimal element from a circuit (minimal dependent set) in the graphical matroid on $K_n$.
\end{definition}
 
\begin{definition}[NBC Basis]
A \emph{no-broken-circuit (NBC)} set is a collection of pairs that contains no broken circuit. These correspond bijectively to:
\begin{itemize}
\item Acyclic directed graphs on $[n]$ (forests)
\item Independent sets in the graphical matroid
\item Monomials in $\eta_{ij}$ that don't vanish by Arnold relations
\end{itemize}
\end{definition}
 
\begin{theorem}[NBC Basis Theorem]\label{thm:NBC}
The NBC sets provide a basis for $H^*(\barC_n(X))$. More precisely, if $F$ is an NBC forest with edges $E(F) = \{(i_1, j_1), \ldots, (i_k, j_k)\}$, then:
\[
\omega_F = \eta_{i_1j_1} \wedge \cdots \wedge \eta_{i_kj_k}
\]
forms a basis element of $H^k(\barC_n(X))$.
\end{theorem}
 
\begin{example}[NBC Basis for $n = 4$]\label{ex:NBC4}
For $\barC_4(X)$, using the lexicographic order on pairs, the NBC basis consists of:
\begin{itemize}
\item Degree 0: $1$
\item Degree 1: $\eta_{12}, \eta_{13}, \eta_{14}, \eta_{23}, \eta_{24}, \eta_{34}$ (6 elements)
\item Degree 2: $\eta_{12} \wedge \eta_{34}, \eta_{13} \wedge \eta_{24}, \eta_{14} \wedge \eta_{23}$, plus 8 other terms (11 total)
\item Degree 3: $\eta_{12} \wedge \eta_{23} \wedge \eta_{34}$ and 5 other spanning trees (6 total)
\end{itemize}
Total: $1 + 6 + 11 + 6 = 24 = 4!$ basis elements, confirming $\dim H^*(\barC_4(\C)) = 4!$.
\end{example}
 
This completes our foundational setup. We have established:
\begin{itemize}
\item The operadic framework for describing algebraic structures with complete categorical precision
\item The Com-Lie Koszul duality as our prototypical example with full proofs
\item The geometric spaces (configuration spaces) where our constructions live
\item The differential forms (logarithmic forms) that encode the structure
\end{itemize}
 
These ingredients will now be combined in subsequent sections to construct the geometric bar complex for chiral algebras.
 
\section{Configuration Spaces, Factorization and Higher Genus}
 
\subsection{The Ran Space and Chiral Operations}

\begin{definition}[D-module Category - Precise]
We work with the category $\text{D-mod}_{rh}(X)$ of regular holonomic D-modules on $X$. 
These are D-modules $\mathcal{M}$ satisfying:
\begin{enumerate}
\item Finite presentation: locally finitely generated over $\mathcal{D}_X$
\item Regular singularities: characteristic variety is Lagrangian
\item Holonomicity: $\text{dim}(\text{Char}(\mathcal{M})) = \text{dim}(X)$
\end{enumerate}
This category has:
\begin{itemize}
\item Six functors: $f^*, f_*, f^!, f_!, \otimes^L, \mathcal{RHom}$
\item Riemann-Hilbert correspondence with perverse sheaves
\item Well-defined maximal extension $j_*j^*$ for $j: U \hookrightarrow X$ open
\end{itemize}
\end{definition}

We now introduce the fundamental geometric object underlying chiral algebras---the Ran space---which 
encodes the idea of ``finite subsets with multiplicities'' of a curve. Following Beilinson-Drinfeld 
\cite{BD04}, we work with the following precise categorical framework.
 
\begin{definition}[Ran Space via Categorical Colimit]\label{def:ran-precise}
Let $X$ be a smooth algebraic curve over $\mathbb{C}$. The \emph{Ran space} of $X$ is the ind-scheme 
defined as the colimit:
\[
\text{Ran}(X) = \underset{I \in \text{FinSet}^{\text{surj,op}}}{\text{colim}} \, X^I
\]
where:
\begin{itemize}
\item $\text{FinSet}^{\text{surj}}$ is the category of finite sets with surjections as morphisms
\item For a surjection $\phi: I \twoheadrightarrow J$, the induced map $X^J \to X^I$ is the diagonal 
embedding on fibers $\phi^{-1}(j)$
\item The colimit is taken in the category of ind-schemes with the Zariski topology
\end{itemize}
Explicitly, a point in $\text{Ran}(X)$ is a finite collection of points in $X$ with multiplicities,
represented as $\sum_{i=1}^n m_i[x_i]$ where $x_i \in X$ are distinct and $m_i \in \mathbb{Z}_{>0}$.
\end{definition}
 
\begin{remark}[Set-Theoretic Description]
The underlying set of $\text{Ran}(X)$ can be identified with the free commutative monoid on the 
underlying set of $X$, but the scheme structure is more subtle and encodes the deformation theory
of point configurations.
\end{remark}
 
The Ran space carries a fundamental monoidal structure encoding disjoint union:
 
\begin{definition}[Factorization Structure]\label{def:factorization}
\textbf{Critical Warning:} The naive definition 
$$\mathcal{M} \otimes^{\text{ch}} \mathcal{N} = \Delta_! \left( \rho_1^* \mathcal{M} \otimes^! \rho_2^* \mathcal{N} \right)$$
\textbf{FAILS} because the union map $\Delta: \text{Ran}(X) \times \text{Ran}(X) \to \text{Ran}(X)$ is \textbf{not proper}, 
so $\Delta_!$ is undefined. The correct framework uses factorization algebras.
\end{definition}

\begin{definition}[Factorization Algebra - Correct Framework]\label{def:fact-algebra-correct}
A \emph{factorization algebra} $\mathcal{F}$ on $X$ consists of:
\begin{enumerate}
\item A quasi-coherent $\mathcal{D}$-module $\mathcal{F}_S$ for each finite set $S \subset X$
\item For disjoint $S_1, S_2$, a factorization isomorphism:
   $$\mu_{S_1,S_2}: \mathcal{F}_{S_1} \boxtimes \mathcal{F}_{S_2} \xrightarrow{\sim} \mathcal{F}_{S_1 \sqcup S_2}$$
\item These satisfy:
   \begin{itemize}
   \item \textbf{Associativity:} For disjoint $S_1, S_2, S_3$:
   \begin{center}
   \begin{tikzcd}
   \mathcal{F}_{S_1} \boxtimes \mathcal{F}_{S_2} \boxtimes \mathcal{F}_{S_3} 
   \arrow[r, "\mu_{S_1,S_2} \boxtimes \text{id}"] 
   \arrow[d, "\text{id} \boxtimes \mu_{S_2,S_3}"'] &
   \mathcal{F}_{S_1 \sqcup S_2} \boxtimes \mathcal{F}_{S_3} 
   \arrow[d, "\mu_{S_1 \sqcup S_2, S_3}"] \\
   \mathcal{F}_{S_1} \boxtimes \mathcal{F}_{S_2 \sqcup S_3} 
   \arrow[r, "\mu_{S_1, S_2 \sqcup S_3}"'] &
   \mathcal{F}_{S_1 \sqcup S_2 \sqcup S_3}
   \end{tikzcd}
   \end{center}
   \item \textbf{Commutativity:} $\mu_{S_2,S_1} = \sigma_{S_1,S_2} \circ \mu_{S_1,S_2}$ where $\sigma$ is the swap
   \item \textbf{Unit:} $\mathcal{F}_\emptyset = \mathbb{C}$ with canonical isomorphisms $\mathcal{F}_S \cong \mathbb{C} \boxtimes \mathcal{F}_S$
   \end{itemize}
\end{enumerate}
\end{definition}

\begin{remark}[Geometric Insight à la Kontsevich]
Factorization algebras encode the principle of \emph{locality} in quantum field theory: the observables 
on disjoint regions combine independently. The factorization isomorphisms are the mathematical incarnation 
of the physical statement that ``spacelike separated observables commute.'' This philosophy, emphasized by 
Kontsevich and developed by Costello-Gwilliam, views quantum field theory as assigning algebraic structures 
to spacetime in a locally determined way.
\end{remark}

\begin{theorem}[Chiral Algebras as Factorization Algebras]\label{thm:chiral-as-fact}
Every chiral algebra $\mathcal{A}$ on $X$ determines a factorization algebra $\mathcal{F}_\mathcal{A}$ where:
\begin{itemize}
\item $\mathcal{F}_\mathcal{A}(S) = \mathcal{A}^{\boxtimes S}$ for finite $S \subset X$
\item The factorization structure comes from the chiral multiplication
\item This defines a fully faithful functor $\text{ChirAlg}(X) \to \text{FactAlg}(X)$
\end{itemize}
\end{theorem}

\begin{proof}[Proof following Beilinson-Drinfeld]
The key observation is that chiral multiplication provides exactly the factorization isomorphisms needed.
The Jacobi identity for chiral algebras translates to associativity of factorization. The technical 
issue with properness is avoided because we work fiberwise over finite sets rather than globally on Ran space.
\end{proof}
% Add precise D-module structure


\begin{theorem}[Factorization Monoidal Structure - CORRECTED]\label{thm:fact-monoidal-corrected}
The category $\text{FactAlg}(X)$ of factorization algebras (NOT all D-modules on Ran space) forms a symmetric monoidal 
category with:
\begin{enumerate}
\item Tensor product: $(\mathcal{F} \otimes_{\text{fact}} \mathcal{G})(S) = \bigoplus_{S_1 \sqcup S_2 = S} \mathcal{F}(S_1) \otimes \mathcal{G}(S_2)$
\item Unit: The vacuum factorization algebra $\mathbb{1}$ with $\mathbb{1}(S) = \begin{cases} \mathbb{C} & S = \emptyset \\ 0 & \text{otherwise} \end{cases}$
\item Associativity isomorphism satisfying the pentagon axiom
\item Braiding isomorphism induced by the symmetric group action
\end{enumerate}

Moreover, there is a fully faithful embedding:
$$\text{ChirAlg}(X) \hookrightarrow \text{FactAlg}(X)$$
sending a chiral algebra $\mathcal{A}$ to its associated factorization algebra $\mathcal{F}_{\mathcal{A}}$.
\end{theorem}

\begin{proof}[Proof Sketch following Beilinson-Drinfeld and Ayala-Francis]
The key insight is that factorization algebras form a \emph{lax} symmetric monoidal category, which becomes 
strict when we pass to the homotopy category. The Day convolution is well-defined because we take colimits 
over finite decompositions, avoiding the properness issues with the naive approach.

The pentagon and hexagon axioms follow from the corresponding properties of finite set unions. The 
symmetric monoidal structure is compatible with the embedding from chiral algebras, making this the 
correct categorical framework for studying chiral algebras.
\end{proof}

\textbf{Underlying D-modules:} A collection $\{\mathcal{A}_n\}_{n \geq 0}$ where each $\mathcal{A}_n$ is a quasi-coherent $\mathcal{D}_{X^n}$-module, meaning:
\begin{itemize}
\item $\mathcal{A}_n$ is a sheaf of modules over the sheaf of differential operators $\mathcal{D}_{X^n}$
\item The action satisfies the Leibniz rule: $\partial(fs) = (\partial f)s + f(\partial s)$ for local functions $f$ and sections $s$
\item $\mathcal{A}_n$ is quasi-coherent as an $\mathcal{O}_{X^n}$-module
\end{itemize}

\subsection{Elliptic Configuration Spaces and Theta Functions}

\subsubsection{The Genus 1 Realm: Elliptic Curves as Quotients}

For genus 1, we work with elliptic curves $E_\tau = \mathbb{C}/(\mathbb{Z} + \tau\mathbb{Z})$ where $\tau \in \mathfrak{h}$ lies in the upper half-plane. The configuration space has a fundamentally different character from genus 0:

\begin{definition}[Elliptic Configuration Space]
For an elliptic curve $E_\tau$, the configuration space of $n$ points is:
\[
C_n(E_\tau) = \{(z_1, \ldots, z_n) \in E_\tau^n \mid z_i \neq z_j \text{ mod } \Lambda_\tau\}
\]
where $\Lambda_\tau = \mathbb{Z} + \tau\mathbb{Z}$ is the period lattice.
\end{definition}

\begin{theorem}[Elliptic Compactification]
The compactification $\overline{C_n(E_\tau)}$ is constructed via:
\begin{enumerate}
\item \textbf{Local blow-ups}: Near collision points, use elliptic blow-up coordinates
\item \textbf{Global structure}: The compactified space admits a stratification by \emph{stable elliptic graphs}
\item \textbf{Modular invariance}: Under $SL_2(\mathbb{Z})$ action on $\tau$, the construction is equivariant
\end{enumerate}
\end{theorem}

\begin{proof}[Construction]
Near a collision point $z_i \to z_j$ on $E_\tau$, introduce elliptic blow-up coordinates:
\begin{align}
\epsilon_{ij} &= |z_i - z_j|_{E_\tau} \quad \text{(elliptic distance)} \\
\theta_{ij} &= \arg(z_i - z_j) \quad \text{(angular parameter)} \\
u_{ij} &= \frac{z_i + z_j}{2} \quad \text{(center on } E_\tau)
\end{align}

The key difference from genus 0: the elliptic distance involves the Weierstrass $\sigma$-function:
\[
|z_i - z_j|_{E_\tau} = |\sigma(z_i - z_j; \tau)|e^{-\eta(\tau)\text{Im}(z_i - z_j)^2/\text{Im}(\tau)}
\]
where $\eta(\tau)$ is the Dedekind eta function.
\end{proof}

\subsubsection{Theta Functions as Building Blocks}

The logarithmic forms on elliptic curves are replaced by forms built from theta functions:

\begin{definition}[Elliptic Logarithmic Forms]
On $\overline{C_n(E_\tau)}$, define the elliptic analogs of $\eta_{ij}$:
\[
\eta_{ij}^{(1)} = d\log\theta_1\left(\frac{z_i - z_j}{2\pi i}; \tau\right) + \text{regularization}
\]
where $\theta_1(z; \tau) = -i\sum_{n \in \mathbb{Z}}(-1)^n q^{(n-1/2)^2}e^{i(2n-1)z}$ with $q = e^{i\pi\tau}$.
\end{definition}

\begin{proposition}[Elliptic Arnold Relations]
The elliptic logarithmic forms satisfy modified Arnold relations:
\[
\eta_{ij}^{(1)} \wedge \eta_{jk}^{(1)} + \eta_{jk}^{(1)} \wedge \eta_{ki}^{(1)} + \eta_{ki}^{(1)} \wedge \eta_{ij}^{(1)} = 2\pi i \omega_\tau
\]
where $\omega_\tau = \frac{dz \wedge d\bar{z}}{2i\text{Im}(\tau)}$ is the volume form on $E_\tau$.
\end{proposition}

The non-vanishing right-hand side encodes the central extension that appears at genus 1!

\subsection{Higher Genus Configuration Spaces}

\subsubsection{Hyperbolic Surfaces and Teichmüller Theory}

For genus $g \geq 2$, the underlying curve $\Sigma_g$ admits a hyperbolic metric. The configuration spaces inherit rich geometric structure:

\begin{definition}[Higher Genus Configuration]
For a compact Riemann surface $\Sigma_g$ of genus $g \geq 2$:
\[
C_n(\Sigma_g) = \{(p_1, \ldots, p_n) \in \Sigma_g^n \mid p_i \neq p_j\}/\text{Aut}(\Sigma_g)
\]
The compactification $\overline{C_n(\Sigma_g)}$ involves:
\begin{itemize}
\item Stable curves with marked points
\item Deligne-Mumford compactification techniques
\item Intersection with the moduli space $\overline{\mathcal{M}}_{g,n}$
\end{itemize}
\end{definition}

\begin{theorem}[Period Integrals and Bar Differential]
On $\overline{C_n(\Sigma_g)}$, the bar differential decomposes:
\[
d_{\text{bar}}^{(g)} = d_{\text{local}} + d_{\text{global}} + d_{\text{quantum}}
\]
where:
\begin{enumerate}
\item $d_{\text{local}}$: Standard residues at collision divisors (genus 0 contribution)
\item $d_{\text{global}}$: Period integrals over homology cycles of $\Sigma_g$
\item $d_{\text{quantum}}$: Corrections from the moduli space $\mathcal{M}_g$
\end{enumerate}
\end{theorem}

\begin{proof}[Sketch]
The decomposition follows from the Leray spectral sequence for the fibration:
\[
\overline{C_n(\Sigma_g)} \to \overline{\mathcal{M}}_{g,n} \to \overline{\mathcal{M}}_g
\]

Each term contributes differently:
\begin{itemize}
\item Local: Fiberwise residues give the standard chiral multiplication
\item Global: Integration over the $2g$ cycles of $H_1(\Sigma_g, \mathbb{Z})$
\item Quantum: Contributions from varying complex structure
\end{itemize}
\end{proof}

\subsection{Convergence of Configuration Space Integrals}

\begin{definition}[Convergent Chiral Algebra]
A chiral algebra $\mathcal{A}$ is \emph{convergent} if for all $n$ and all $\phi_i \in \mathcal{A}$:
$$\int_{\ConfigSpace{n}} |\phi_1(z_1) \cdots \phi_n(z_n)|^2 \prod_{i<j} |z_i - z_j|^{2\alpha_{ij}} < \infty$$
for appropriate regularization exponents $\alpha_{ij} > 0$.
\end{definition}

\begin{theorem}[Convergence Criterion]
The bar complex $\barBgeom(\mathcal{A})$ is well-defined if:
\begin{enumerate}
\item $\mathcal{A}$ has bounded conformal weights: $h_i \leq h_{\max} < \infty$
\item The OPE has polynomial growth: $|C_{ij}^{k,n}| \leq C(1 + n)^N$
\item The genus satisfies: $g \leq g_{\max}$ (for higher genus)
\end{enumerate}
\end{theorem}

\begin{proof}
Near collision divisors $D_{ij}$, the integrand behaves as:
$$|\phi_i(z_i)\phi_j(z_j)|^2 \sim \frac{1}{|z_i - z_j|^{2(h_i + h_j - h_{\min})}}$$

The logarithmic form contributes:
$$|d\log(z_i - z_j)|^2 = \frac{|dz_i - dz_j|^2}{|z_i - z_j|^2}$$

The integral converges if:
$$\int_{\epsilon < |z_i - z_j| < 1} \frac{d^2z_i d^2z_j}{|z_i - z_j|^{2(h_i + h_j - h_{\min} + 1)}} < \infty$$

Using polar coordinates around collision: $z_i - z_j = re^{i\theta}$:
$$\int_\epsilon^1 \frac{r \, dr}{r^{2(h_i + h_j - h_{\min} + 1)}} = \int_\epsilon^1 r^{1 - 2(h_i + h_j - h_{\min} + 1)} dr$$

This converges if:
$$2 - 2(h_i + h_j - h_{\min} + 1) > -1 \iff h_i + h_j - h_{\min} < \frac{3}{2}$$

For unitary theories with $h_{\min} \geq 0$, this is satisfied when weights are bounded.
\end{proof}

\begin{remark}[Regularization]
When convergence fails, we use:
\begin{itemize}
\item Analytic continuation in dimensions
\item Point-splitting regularization
\item Pauli-Villars regularization for quantum corrections
\end{itemize}
\end{remark}

\subsection{Orientation Conventions for Configuration Spaces}

\begin{definition}[Oriented Configuration Space]
The configuration space $C_n(X)$ inherits an orientation from $X^n$ via:
$$\text{or}(C_n(X)) = \text{or}(X)^{\otimes n} / S_n$$
where we quotient by the symmetric group action.
\end{definition}

\begin{definition}[Orientation of Compactification]
The Fulton-MacPherson compactification $\ConfigSpace{n}$ is oriented by:
\begin{enumerate}
\item Choose orientation on $C_n(X)$ as above
\item At each blow-up, use the standard orientation on exceptional divisors
\item The boundary $\partial\ConfigSpace{n} = D$ inherits the outward normal orientation
\end{enumerate}
\end{definition}

\begin{lemma}[Orientation Compatibility]
For the stratification of $\partial\ConfigSpace{n}$:
$$\partial\ConfigSpace{n} = \bigcup_{I \subset \{1,\ldots,n\}, |I| \geq 2} D_I$$
The orientations satisfy:
$$\text{or}(\partial D_I) = (-1)^{\text{codim}(D_I)} \text{or}(D_I)$$
\end{lemma}

\begin{proof}
We proceed by induction on codimension.

\textbf{Codimension 1}: $D_{ij}$ has orientation from the normal bundle:
$$\text{or}(D_{ij}) = \text{or}(N_{D_{ij}}) \wedge \text{or}(\ConfigSpace{n-1})$$
where $N_{D_{ij}}$ is oriented by $d\epsilon_{ij}$ (radial coordinate).

\textbf{Codimension 2}: At $D_{ijk} = D_{ij} \cap D_{jk}$:
$$\text{or}(D_{ijk}) = \text{or}(N_{D_{ij}}) \wedge \text{or}(N_{D_{jk}|D_{ij}}) \wedge \text{or}(\ConfigSpace{n-2})$$

The key sign:
$$\text{or}(D_{ijk})|_{D_{ij} \to D_{ijk}} = -\text{or}(D_{ijk})|_{D_{jk} \to D_{ijk}}$$

This ensures Stokes' theorem holds:
$$\int_{\partial D_{ij}} \omega = \sum_{k} \epsilon_k \int_{D_{ijk}} \omega$$
with appropriate signs $\epsilon_k = \pm 1$.
\end{proof}

\begin{theorem}[Stokes on Configuration Spaces]
For $\omega \in \Omega^{n-1}(\ConfigSpace{n})$:
$$\int_{\ConfigSpace{n}} d\omega = \int_{\partial\ConfigSpace{n}} \omega = \sum_{I} \epsilon_I \int_{D_I} \omega$$
where $\epsilon_I$ is determined by the orientation convention.
\end{theorem}

% Original numbering continues
\begin{enumerate}
\item[(1)] A collection $\{\mathcal{A}_n\}_{n \geq 0}$ of quasi-coherent D-modules on $X^n$, equivariant under 
the symmetric group $S_n$ action

\item For each pair $(i,j)$ with $1 \leq i < j \leq m+n$, a \emph{chiral multiplication map}:
\[
\mu_{ij}: j_{ij*}j_{ij}^* \left(\mathcal{A}_m \boxtimes \mathcal{A}_n\right) \to \Delta_{*}\mathcal{A}_{m+n-1}
\]
where:
\begin{itemize}
\item $j_{ij}: U_{ij} \hookrightarrow X^m \times X^n$ is the inclusion of the open subset where the 
$i$-th coordinate of the first factor differs from the $j$-th coordinate of the second
\item $\Delta: X \hookrightarrow X^{m+n-1}$ is the small diagonal embedding
\item The extension $j_{ij*}j_{ij}^*$ is the maximal extension functor for D-modules
\end{itemize}
 
\item \emph{Factorization isomorphisms}: For disjoint finite sets $I, J$,
\[
\phi_{I,J}: \mathcal{A}_{I \sqcup J} \xrightarrow{\sim} \mathcal{A}_I \boxtimes \mathcal{A}_J
\]
compatible with the symmetric group actions
 
\item These data satisfy:
\begin{itemize}
\item \emph{Associativity}: For any triple collision, the diagram
\[
\begin{tikzcd}
j_{123*}j_{123}^*(\mathcal{A}_k \boxtimes \mathcal{A}_\ell \boxtimes \mathcal{A}_m) 
\arrow[r, "\mu_{12} \boxtimes \text{id}"] \arrow[d, "\text{id} \boxtimes \mu_{23}"'] &
j_{23*}j_{23}^*(\mathcal{A}_{k+\ell-1} \boxtimes \mathcal{A}_m) \arrow[d, "\mu_{(12)3}"] \\
j_{12*}j_{12}^*(\mathcal{A}_k \boxtimes \mathcal{A}_{\ell+m-1}) \arrow[r, "\mu_{1(23)}"'] &
\mathcal{A}_{k+\ell+m-2}
\end{tikzcd}
\]
commutes up to coherent isomorphism satisfying higher coherence conditions
 
\item \emph{Unit}: $\mathcal{A}_0 = \mathbb{C}$ with $\mathcal{A}_1$ acting as identity under composition
 
\item \emph{Compatibility}: The factorization isomorphisms are compatible with the chiral multiplication
in the sense that appropriate diagrams commute
\end{itemize}
\end{enumerate}
 
\begin{remark}[Physical Interpretation]
In physics, $\mathcal{A}_n$ represents the space of $n$-point correlation functions. The condition 
$j_{ij*}j_{ij}^*$ implements locality (operators are defined away from coincident points), while 
$\mu_{ij}$ encodes the operator product expansion when two operators collide. The factorization 
isomorphisms express the clustering principle of quantum field theory.
\end{remark}

\begin{remark}[Geometric Intuition] The chiral algebra structure encodes how local operators merge when brought together. The condition $j_{ij*}j_{ij}^*$ implements the principle that operators are well-defined away from coincident points, while the multiplication $\mu_{ij}$ captures what happens at collision. This is the mathematical formalization of the operator product expansion in conformal field theory, where:
\begin{itemize}
\item The domain $U_{ij}$ represents configurations with separated operators
\item The codomain $\mathcal{A}_{m+n-1}$ represents the merged configuration  
\item The map $\mu_{ij}$ encodes the singular part of the correlation function
\end{itemize}
\end{remark}

\subsection{The Chiral Endomorphism Operad}
 
For any D-module $\mathcal{M}$ on $X$, we construct the operad controlling chiral algebra structures:
 
\begin{definition}[Chiral Endomorphisms - Precise]\label{def:chiral-endo}
The \emph{chiral endomorphism operad} of a D-module $\mathcal{M}$ on $X$ is defined by:
\[
\text{End}_{\mathcal{M}}^{\text{ch}}(n) = \text{Hom}_{\mathcal{D}(X^n)}\left(j_*j^*\mathcal{M}^{\boxtimes n}, \Delta_*\mathcal{M}\right)
\]
where:
\begin{itemize}
\item $j: C_n(X) \hookrightarrow X^n$ is the inclusion of the configuration space
\item $\Delta: X \hookrightarrow X^n$ is the small diagonal
\item The morphisms are taken in the derived category of D-modules
\end{itemize}
\end{definition}
 
\begin{proposition}[Operadic Structure]
$\text{End}_{\mathcal{M}}^{\text{ch}}$ forms an operad in the category of D-modules with:
\begin{enumerate}
\item Composition: For $f \in \text{End}_{\mathcal{M}}^{\text{ch}}(k)$ and $g_i \in \text{End}_{\mathcal{M}}^{\text{ch}}(n_i)$,
\[
f \circ (g_1, \ldots, g_k) = f \circ \left(\Delta_{n_1,\ldots,n_k}^* (g_1 \boxtimes \cdots \boxtimes g_k)\right)
\]
where $\Delta_{n_1,\ldots,n_k}: X^{n_1 + \cdots + n_k} \to X^k \times X^{n_1} \times \cdots \times X^{n_k}$
 
\item Unit: The identity map $\text{id}_{\mathcal{M}} \in \text{End}_{\mathcal{M}}^{\text{ch}}(1)$
 
\item The composition satisfies associativity up to coherent isomorphism
\end{enumerate}
\end{proposition}
 
\begin{proof}
Associativity follows from the functoriality of the diagonal embeddings. Consider the diagram:
\[
X^{n_1 + \cdots + n_k} \xrightarrow{\Delta_{n_1,\ldots,n_k}} X^k \times \prod_i X^{n_i} 
\xrightarrow{\text{id} \times \prod_i \Delta_{m_{i1},\ldots}} X^k \times \prod_i \prod_j X^{m_{ij}}
\]
The two ways of composing correspond to different factorizations of the total diagonal, which are 
canonically isomorphic. The coherence follows from the coherence theorem for operads.
\end{proof}
 
\begin{theorem}[Chiral Algebras as Algebra Objects]
A chiral algebra structure on $\mathcal{M}$ is equivalent to an algebra structure over the operad 
$\text{End}_{\mathcal{M}}^{\text{ch}}$ in the symmetric monoidal category of D-modules. Moreover, this 
equivalence is functorial and preserves quasi-isomorphisms.
\end{theorem}


 \section{Chain-Level Constructions and Simplicial Models}
 
\subsection{NBC Bases and Computational Optimality}
 
The no-broken-circuit (NBC) basis provides the computationally optimal choice for the Orlik-Solomon algebra.
 
\begin{definition}[NBC Basis]
For the configuration space $C_n(X)$, an NBC basis element corresponds to a forest $F$ on vertices $\{1,\ldots,n\}$ with edges $(i,j)$ where $i < j$, such that $F$ contains no broken circuit.
\end{definition}
 
\begin{theorem}[NBC Basis Optimality]
The NBC basis satisfies:
\begin{enumerate}
\item Each basis element is $\eta_F = \bigwedge_{(i,j) \in F} \eta_{ij}$
\item The differential has matrix entries in $\{0, \pm 1\}$ only
\item No cancellations occur in computing $d^2 = 0$
\item $|\text{NBC forests on $n$ vertices}| = \dim H^*(C_n(\mathbb{C}))$
\end{enumerate}
\end{theorem}

\begin{proof}
We proceed by induction on $n$. For $n = 2$, the single NBC element is $\eta_{12}$ with $d\eta_{12} = 0$.
 
For the inductive step, consider the fibration
\[
C_n(\mathbb{C}) \to C_{n-1}(\mathbb{C}) \times \mathbb{C}
\]
given by forgetting the $n$-th point. The NBC basis respects this fibration:
\begin{itemize}
\item NBC forests on $n$ vertices without edge to vertex $n$ pull back from $C_{n-1}(\mathbb{C})$
\item NBC forests with edges to vertex $n$ correspond to adding non-circuit-completing edges
\end{itemize}
 
The differential preserves the NBC property because contracting an edge in an NBC forest cannot create a circuit. Matrix entries are $\pm 1$ from the Koszul sign rule. The count follows from the recurrence
\[
f(n) = n \cdot f(n-1)
\]
which yields the explicit formula:
\[
|\text{NBC}(n)| = n! = \dim H^*(\overline{C}_n(\mathbb{C}))
\]

matching the Poincaré polynomial of $C_n(\mathbb{C})$.
\end{proof}

\begin{proposition}[NBC Sparsity Analysis]\label{prop:nbc-sparsity}
For the geometric bar complex, the differential has at most $O(n^3)$ non-zero entries due to weight constraints.
\end{proposition}

\begin{proof}
Consider NBC forests $F_1, F_2$ on $n$ vertices. A non-zero differential $\langle dF_1, F_2 \rangle$ requires:
\begin{enumerate}
\item $F_2$ obtained from $F_1$ by contracting one edge $(i,j)$
\item The weight condition $h_{\phi_i} + h_{\phi_j} = h_{\phi_k} + 1$ for some resulting field $\phi_k$
\end{enumerate}

For a chiral algebra with $r$ generators of weights $\{h_1, \ldots, h_r\}$:
- Each vertex can be labeled by one of $r$ generators
- Weight-preserving collisions form a sparse $r \times r$ matrix $M_{ij}$
- $M_{ij} \neq 0$ only if $h_i + h_j \in \{h_k + 1 : k = 1, \ldots, r\}$

The sparsity factor is:
$\rho = \frac{|\{(i,j,k) : h_i + h_j = h_k + 1\}|}{r^3} \leq \frac{r^2}{r^3} = \frac{1}{r}$

Total non-zero entries: $\leq n \cdot \binom{n-1}{2} \cdot \rho \cdot |\text{NBC}(n)| = O(n^3)$ after sparsity.
\end{proof}

\begin{theorem}[Presentation Independence - REFINED]\label{thm:presentation-independence}
   The geometric bar complex satisfies:
   \begin{enumerate}
   \item \textbf{Functoriality:} A morphism $\phi: \mathcal{A}_1 \to \mathcal{A}_2$ induces 
   $\bar{B}^{\text{ch}}(\phi): \bar{B}^{\text{ch}}(\mathcal{A}_1) \to \bar{B}^{\text{ch}}(\mathcal{A}_2)$
   
   \item \textbf{Quasi-isomorphism invariance:} If $\phi$ is a quasi-isomorphism, so is $\bar{B}^{\text{ch}}(\phi)$
   
   \item \textbf{Presentation independence within equivalence class:} Two presentations 
   $\mathcal{A} = \text{Free}^{\text{ch}}(V_1)/R_1 = \text{Free}^{\text{ch}}(V_2)/R_2$ 
   yield quasi-isomorphic bar complexes if and only if:
      \begin{itemize}
      \item Conformal weights are preserved modulo integers
      \item Relations differ only by Jacobi identity consequences
      \item Only tautological generators/relations are added/removed
      \end{itemize}
      
   \item \textbf{Criticality obstruction:} Different weight assignments satisfying different criticality 
   conditions yield non-quasi-isomorphic complexes
   \end{enumerate}
   \end{theorem}
   
   \begin{proof}[Proof via Universal Property]
   Rather than comparing specific presentations, we characterize when presentations yield isomorphic 
   objects in the derived category.
   
   \textbf{Key observation:} The geometric bar complex depends on:
   \begin{enumerate}
   \item The conformal weights of generators (determines residue contributions)
   \item The OPE structure (determines factorization differential)  
   \item The relations modulo Jacobi identity (determines boundaries)
   \end{enumerate}
   
   Two presentations yield the same complex if and only if these three data match.
   \end{proof}
   
   \begin{remark}[The Prism Reveals Non-Invariance]
   The criticality obstruction shows that our ``prism'' is sensitive to the ``wavelength'' of generators:
   \begin{itemize}
   \item Different conformal weights = different wavelengths
   \item The residue pairing acts as a ``filter'' selecting compatible wavelengths
   \item Only when $h_i + h_j = h_k + 1$ does the ``light'' pass through
   \item Different presentations with different weights yield different ``spectra''
   \end{itemize}
   
   This is not a bug but a feature: the geometric bar complex detects the conformal dimension, which is 
   essential data in CFT that purely algebraic constructions might miss.
   \end{remark}
   
\begin{lemma}[Arnold Relations on Boundary]\label{lem:arnold-boundary}
The Arnold relations extend continuously to $\partial \overline{C}_n(X)$.
\end{lemma}

\begin{proof}
Near a boundary stratum $D_I$ where points in $I \subset \{1,\ldots,n\}$ collide, use coordinates:
- $u = \frac{1}{|I|}\sum_{i \in I} z_i$ (center of mass)
- $\epsilon_{ij} = |z_i - z_j|$ for $i,j \in I$
- $\theta_{ij} = \arg(z_i - z_j)$

The logarithmic forms become:
$\eta_{ij} = d\log \epsilon_{ij} + id\theta_{ij} + O(\epsilon_{ij})$

For any triple $i,j,k \in I$:
$\eta_{ij} \wedge \eta_{jk} + \eta_{jk} \wedge \eta_{ki} + \eta_{ki} \wedge \eta_{ij} = d\log \epsilon_{ij} \wedge d\log \epsilon_{jk} + \text{cyclic} + O(\epsilon)$

The leading term vanishes by the classical Arnold relation for the configuration space of the bubble. The $O(\epsilon)$ terms vanish in the limit $\epsilon \to 0$, establishing continuity.
\end{proof}

\subsection{Permutohedral Tiling and Cell Complex}
 
\begin{theorem}[Permutohedral Cell Complex]
The real configuration space $C_n(\mathbb{R})$ admits a CW decomposition where:
\begin{enumerate}
\item Cells $C_\pi$ correspond to ordered partitions $\pi = B_1 < B_2 < \cdots < B_k$ of $[n]$
\item $\dim C_\pi = n - k$
\item $\partial C_\pi = \bigcup_{i} C_{\pi_i}$ where $\pi_i$ merges blocks $B_i$ and $B_{i+1}$
\item The cellular cochain complex computes $H^*(C_n(\mathbb{R}))$
\end{enumerate}
\end{theorem} 
\begin{proof}
We construct the cell decomposition explicitly. Points in $C_\pi$ have configuration type
\[
x_{B_1} < x_{B_2} < \cdots < x_{B_k}
\]
where $x_{B_i}$ denotes the common position of points in block $B_i$. The dimension formula follows from counting degrees of freedom: $k$ positions minus 1 for translation invariance gives $k-1$, but we need $n-1$ total dimensions, so the cell has dimension $n-k$.
 
The boundary formula follows from approaching configurations where adjacent blocks merge. The cellular differential
\[
\delta: C^{n-k}(\pi) \to \bigoplus_{\pi \to \pi'} C^{n-k+1}(\pi')
\]
corresponds exactly to the operadic differential in the bar complex of the commutative operad.
\end{proof}
 
\section{Computational Complexity and Algorithms}
 
\subsection{Complexity Analysis}

\begin{remark}[Practical Implementation]
While the theoretical bounds appear daunting,
the actual computation benefits from massive sparsity. In practice, most residues vanish
by weight or dimension considerations, reducing the effective complexity by several orders
of magnitude. For $n \leq 10$, computations are feasible on standard hardware.
\end{remark}

\begin{theorem}[Complexity Bounds - Rigorous]
For the geometric bar complex in dimension $n$:
\begin{enumerate}
\item NBC basis size: $B(n) = n! \cdot \text{Cat}(n-1) = O((4n)^n/n^{3/2})$
\item Differential computation: $O(n^3)$ operations
\item Storage: $O(n \cdot B(n))$ sparse representation
\item Verification of $d^2=0$: $O(n^5)$ operations
\end{enumerate}
\end{theorem}

\begin{proof}[Derivation]
\textbf{NBC count:} Satisfies recurrence $B(n) = \sum_{k=1}^{n-1} \binom{n-1}{k-1} B(k)B(n-k)$.
This generates shifted Catalan numbers: $B(n) = n! \cdot \text{Cat}(n-1)$.
Using $\text{Cat}(m) \sim \frac{4^m}{m^{3/2}\sqrt{\pi}}$ gives the bound.

\textbf{Differential:} Each NBC forest has $\leq n-1$ edges. 
Computing residue per edge: $O(n)$ for weight matching.
Total per basis element: $O(n^2)$.
With $B(n)$ elements: seemingly $O(n^2 \cdot B(n))$, but sparsity reduces to $O(n^3)$ nonzero entries.

\textbf{Verification:} Compose differential twice on $O(B(n))$ elements, each taking $O(n^3)$ operations.
\end{proof}

\begin{theorem}[Spectral Sequence Convergence]\label{thm:spectral-convergence}
For curved Koszul pairs $(\mathcal{A}_1, \mathcal{A}_2)$ with filtrations $F_\bullet$, the spectral sequence:
$E_1^{p,q} = H^{p+q}(\text{gr}_p \bar{B}^{\text{ch}}(\mathcal{A}_1)) \Rightarrow H^{p+q}(\bar{B}^{\text{ch}}(\mathcal{A}_1))$
converges strongly.
\end{theorem}

\begin{proof}
Strong convergence requires:
\begin{enumerate}
\item \textbf{Boundedness}: For each total degree $n$, only finitely many $(p,q)$ with $p+q=n$ contribute.

This follows from the filtration $F_p\bar{B}^{\text{ch}}$ having $F_p = 0$ for $p < 0$ and $F_p\bar{B}^n = \bar{B}^n$ for $p \gg n$.

\item \textbf{Completeness}: $\bar{B}^{\text{ch}} = \lim_{\leftarrow} \bar{B}^{\text{ch}}/F_p$.

The geometric bar complex consists of sections over $\overline{C}_{n+1}(X)$ with logarithmic poles. The filtration by pole order along collision divisors is complete in the $\mathcal{D}$-module category.

\item \textbf{Hausdorff property}: $\bigcap_p F_p = 0$.

Elements in all $F_p$ would have poles of arbitrary order, impossible for meromorphic sections.
\end{enumerate}

The differentials $d_r: E_r^{p,q} \to E_r^{p+r,q-r+1}$ are induced by higher residues at deeper collision strata, converging by dimensional reasons.
\end{proof}

\subsubsection{Efficient Residue Computation}
 
\begin{algorithm}
\caption{Optimized Residue Evaluation}
\label{alg:residue-evaluation}
\begin{algorithmic}[1]
\Require Fields $\phi_i(z)$ with weights $h_i$
\Ensure Sum of residue contributions
\State \textbf{Input:} $\phi_1(z_1) \otimes \cdots \otimes \phi_n(z_n) \otimes \omega$
\For{each collision divisor $D_{ij}$}
    \State Check weight condition: $h_i + h_j - h_k = 1$ for some $k$
    \If{condition satisfied}
        \State Extract OPE coefficient $C^k_{ij}$
        \State Replace $\phi_i \otimes \phi_j$ with $\phi_k$
        \State Remove factor $\eta_{ij}$ from $\omega$
        \State Add sign from Koszul rule
    \EndIf
\EndFor
\State \textbf{Output:} Sum of residue contributions
\end{algorithmic}
\end{algorithm}

 
\begin{proposition}[Algorithm Correctness]
The above algorithm computes residues with complexity $O(n^2 \cdot T_{\text{OPE}})$ where $T_{\text{OPE}}$ is the time to look up an OPE coefficient.
\end{proposition}
 
\begin{proof}
Correctness follows from the residue formula in Theorem 6.4. We only get nonzero contributions when the weight condition is satisfied, corresponding to simple poles. The algorithm checks all $\binom{n}{2}$ pairs, each in time $T_{\text{OPE}}$.
\end{proof}
 