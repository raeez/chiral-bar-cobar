\subsubsection{Bar Complex Computation for $\mathcal{W}_3$ Algebra}

\begin{example}[$\mathcal{W}_3$ Bar Complex]\label{ex:w3-bar}
For $\mathcal{W}_3$ (the $\mathfrak{sl}_3$ principal W-algebra):

\textbf{Generators:} $T$ (spin 2), $W$ (spin 3)

\textbf{Bar Complex Dimensions:}
\begin{align}
\dim \bar{B}^0 &= 1 \,\, (\text{vacuum}) \\
\dim \bar{B}^1 &= 2 \,\, (\text{generators}) \\
\dim \bar{B}^2 &= 5 \,\, (\text{computed via OPE}) \\
\dim \bar{B}^3 &= 14 \,\, (\text{growth controlled by } \mathbb{P}^2 \text{ cohomology})
\end{align}

\textbf{Geometric Interpretation:} The bar complex computes $H^*(\mathrm{Maps}(X, \mathbb{P}^2))$.
\end{example}

\subsubsection{Critical Level Phenomena}

\begin{definition}[Critical Level]\label{def:critical}
The critical level is $k = -h^\vee$ where $h^\vee$ is the dual Coxeter number. At this level:
\begin{itemize}
\item The Sugawara construction fails (denominator vanishes)
\item The center becomes large (Feigin-Frenkel center)
\item Connection to geometric Langlands emerges
\end{itemize}
\end{definition}

\begin{theorem}[Feigin-Frenkel Center]\label{thm:ff-center}
At critical level, the center of $\widehat{\mathfrak{g}}_{-h^\vee}$ is:
\[
Z(\widehat{\mathfrak{g}}_{-h^\vee}) \cong \mathrm{Fun}(\mathrm{Op}_{\mathfrak{g}^\vee}(X))
\]
functions on the space of $\mathfrak{g}^\vee$-opers on $X$.
\end{theorem}

\begin{remark}[Opers and Connections]
An oper is a special kind of connection:
\[
\nabla = \partial + p_{-1} + \text{regular terms}
\]
where $p_{-1}$ is a principal nilpotent element. These parametrize geometric solutions 
to the KZ equations.
\end{remark}

\subsubsection{Chiral Coalgebra Structure for $\beta\gamma$}

\begin{theorem}[$\beta\gamma$ Bar Complex Coalgebra]\label{thm:bg-bar-coalg}
The bar complex $\bar{B}^{\text{ch}}(\beta\gamma)$ has chiral coalgebra structure:
\begin{enumerate}
\item \textbf{Comultiplication:} Elements decompose as:
\[
\Delta(\beta_{i_1} \cdots \beta_{i_p} \gamma_{j_1} \cdots \gamma_{j_q} \partial^k) = 
\sum_{\substack{I_\beta \sqcup I'_\beta = \{i_1,\ldots,i_p\} \\ I_\gamma \sqcup I'_\gamma = \{j_1,\ldots,j_q\}}} 
\beta_{I_\beta}\gamma_{I_\gamma}\partial^{k_1} \otimes \beta_{I'_\beta}\gamma_{I'_\gamma}\partial^{k_2}
\]
respecting normal ordering: $\beta$'s to the left of $\gamma$'s.

\item \textbf{Growth Formula:} The dimension growth $\dim(\bar{B}^n) = 2 \cdot 3^{n-1}$ reflects:
\begin{itemize}
\item Factor of 2: Choice of leading term ($\beta$ or $\gamma$)
\item Factor of $3^{n-1}$: Each additional point can be $\beta$, $\gamma$, or derivative
\end{itemize}

\item \textbf{Coassociativity:} Follows from the factorization property of configuration spaces:
\[
\overline{C}_{n}(X) \xrightarrow{\text{forget}} \overline{C}_{n-1}(X) \times X
\]
\end{enumerate}
\end{theorem}

\begin{proof}[Kontsevich-style Construction]
The coalgebra structure emerges from considering correlation functions on punctured curves.

\textbf{Step 1: Propagator Expansion.} The $\beta\gamma$ propagator:
\[
\langle \beta(z)\gamma(w) \rangle = \frac{1}{z-w}
\]
defines a distribution on $C_2(X) = X \times X \setminus \Delta$.

\textbf{Step 2: Feynman Graphs.} Higher correlations factor through tree graphs:
\[
\langle \beta(z_1)\gamma(z_2)\beta(z_3)\gamma(z_4) \rangle = 
\sum_{\text{pairings}} \prod_{\text{edges}} \frac{1}{z_i - z_j}
\]

\textbf{Step 3: Compactification.} The Fulton-MacPherson compactification $\overline{C}_n(X)$ 
regularizes these distributions, with the coalgebra structure encoding how correlators 
factorize when points collide.
\end{proof}

\subsection{The Prism Principle in Action}

\begin{example}[Structure Coefficients via Residues]
Consider a chiral algebra with generators $\phi_i$ and OPE:
$$\phi_i(z) \phi_j(w) = \sum_k \frac{C_{ij}^k \phi_k(w)}{(z-w)^{h_i + h_j - h_k}} + \cdots$$

The geometric bar complex extracts these coefficients:
$$\text{Res}_{D_{ij}}[\phi_i \otimes \phi_j \otimes \eta_{ij}] = \sum_k C_{ij}^k \phi_k$$

This is the ``spectral decomposition'' --- each residue reveals one ``color'' (structure coefficient) 
of the algebraic ``composite light.'' The collection of all residues provides complete information about 
the chiral algebra structure.
\end{example}

\begin{remark}[Lurie's Higher Algebra Perspective]
Following Lurie \cite{HA}, we can understand the geometric bar complex through the theory of 
$\mathbb{E}_n$-algebras:

\begin{itemize}
\item Chiral algebras are ``$\mathbb{E}_2$-algebras with holomorphic structure''
\item The little 2-disks operad $\mathbb{E}_2$ has spaces $\mathbb{E}_2(n) \simeq \text{Conf}_n(\mathbb{C})$
\item The bar complex computes Hochschild homology in the $\mathbb{E}_2$ setting
\item Holomorphic structure forces logarithmic poles at boundaries
\end{itemize}

This explains why configuration spaces appear: they \emph{are} the operad governing 2d algebraic structures.
\end{remark}

\subsection{The Ayala-Francis Perspective}

\begin{theorem}[Factorization Homology = Bar Complex]\label{thm:fact-homology}
For a chiral algebra $\mathcal{A}$ on $X$, there is a canonical equivalence:
$$\int_X \mathcal{A} \simeq C_{\bullet}^{\text{ch}}(\mathcal{A})$$
where the left side is Ayala-Francis factorization homology and the right side is our geometric bar complex 
(viewed as chains rather than cochains).
\end{theorem}

\begin{proof}[Proof Sketch]
Both sides compute the same derived functor:
\begin{itemize}
\item Factorization homology: derived tensor product $\mathcal{A} \otimes^L_{\text{Disk}(X)} \text{pt}$
\item Bar complex: derived Hom $\text{RHom}_{\mathcal{A}\text{-mod}}(k, k)$
\end{itemize}
These are related by Koszul duality for $\mathbb{E}_2$-algebras.
\end{proof}

\begin{remark}[Gaitsgory's Insight]
Dennis Gaitsgory observed that chiral homology can be computed by the ``semi-infinite cohomology'' 
of the corresponding vertex algebra. Our geometric bar complex provides the explicit realization:
\begin{itemize}
\item Semi-infinite = configuration spaces (infinite-dimensional but locally finite)
\item Cohomology = differential forms with logarithmic poles
\item The bar differential = BRST operator in physics
\end{itemize}
\end{remark}

\subsection{Why Logarithmic Forms?}

\begin{proposition}[Forced by Conformal Invariance]
The appearance of logarithmic forms $\eta_{ij} = d\log(z_i - z_j)$ is not a choice but forced by:
\begin{enumerate}
\item \textbf{Conformal invariance:} Under $z \mapsto f(z)$, we need $\eta_{ij} \mapsto \eta_{ij}$
\item \textbf{Single-valuedness:} Around collision divisors, forms must have logarithmic singularities
\item \textbf{Residue theorem:} Only logarithmic forms give well-defined residues
\end{enumerate}
\end{proposition}

\begin{convention}[Signs from Trees]
For the bar differential on decorated trees, we use the following sign convention:
\begin{enumerate}
\item Label edges by depth-first traversal starting from the root
\item For contracting edge $e$ connecting vertices with operations $p_1, p_2$ of degrees $|p_1|, |p_2|$:
\item The sign is $(-1)^{\epsilon(e)}$ where:
$$\epsilon(e) = \sum_{e' < e} |p_{s(e')}| + |p_1| + 1$$
where $s(e')$ is the source vertex of edge $e'$ and the sum is over edges preceding $e$ in the ordering.
\item The extra $+1$ comes from the suspension in the bar construction.
\end{enumerate}

% Add missing verification
To verify $d^2 = 0$ for this sign convention, consider a tree with three vertices and two edges $e_1, e_2$. The two ways to contract both edges give:
\begin{itemize}
\item Contract $e_1$ then $e_2$: sign is $(-1)^{\epsilon(e_1)} \cdot (-1)^{\epsilon'(e_2)}$
\item Contract $e_2$ then $e_1$: sign is $(-1)^{\epsilon(e_2)} \cdot (-1)^{\epsilon'(e_1)}$
\end{itemize}
where $\epsilon'$ accounts for the change in edge labeling after the first contraction. A detailed calculation shows these contributions cancel:
$$(-1)^{\epsilon(e_1) + \epsilon'(e_2)} + (-1)^{\epsilon(e_2) + \epsilon'(e_1)} = 0$$
This generalizes to all trees by induction on the number of edges.

This ensures $d^2 = 0$ by a careful analysis of double contractions.
\end{convention}

\begin{lemma}[Sign Consistency for Bar Differential]
The sign convention above ensures that for any pair of edges $e_1, e_2$ in a tree, the signs arising from contracting $e_1$ then $e_2$ versus contracting $e_2$ then $e_1$ differ by exactly $(-1)$, ensuring $d^2 = 0$.
\end{lemma}

\begin{proof}
Consider the four-vertex tree with edges $e_1$ connecting vertices with operations $p_1, p_2$ and edge $e_2$ connecting vertices with operations $p_3, p_4$. The sign from contracting $e_1$ then $e_2$ is:
$$(-1)^{\epsilon(e_1)} \cdot (-1)^{\epsilon'(e_2)}$$
where $\epsilon'(e_2)$ accounts for the change in edge ordering after contracting $e_1$. A direct computation shows this equals $-1$ times the sign from contracting $e_2$ then $e_1$.
\end{proof}

For an augmented operad $P$ with augmentation $\epsilon: P \to I$, we construct...

\begin{definition}[Cobar Construction]
Dually, for a coaugmented cooperad $C$ with coaugmentation $\eta : \mathbb{I} \to C$, the cobar construction $\Omega(C)$ is the free operad on the desuspension $s^{-1}\bar{C}$ (where $\bar{C} = \text{coker}(\eta)$) with differential induced by the cooperad comultiplication.
\end{definition}
 
\begin{theorem}[Bar-Cobar Adjunction]
There is an adjunction:
\[
\barB : \text{Operads} \rightleftarrows \text{Cooperads}^{\text{op}} : \Omega
\]
Moreover, if $P$ is Koszul (defined below in Section 3.1), then the unit and counit are quasi-isomorphisms, establishing an equivalence of homotopy categories.
\end{theorem}
 
\subsection{Partition Complexes and the Commutative Operad}
 
For the commutative operad $\Com$, the bar construction admits a beautiful combinatorial model via partition lattices:
 
\begin{definition}[Partition Lattice]
The partition lattice $\Pi_n$ is the poset of all partitions of $\{1, 2, \ldots, n\}$, ordered by refinement: $\pi \leq \sigma$ if every block of $\pi$ is contained in some block of $\sigma$. The proper part $\barPi_n = \Pi_n \setminus \{\hat{0}, \hat{1}\}$ excludes the minimum (discrete partition) and maximum (trivial partition).
\end{definition}
 
\begin{theorem}[Partition Complex Structure]\label{thm:partition}
The bar complex $\barB(\Com)(n)$ is quasi-isomorphic to the reduced chain complex $\tilde{C}_*(\barPi_n)$ of the proper part of the partition lattice $\Pi_n$. More precisely:
\[
\barB(\Com)(n) \simeq s^{n-2}\tilde{C}_{n-2}(\barPi_n) \otimes \sgn_n
\]
where $\sgn_n$ is the sign representation of $S_n$.
\end{theorem}
 
\begin{proof}
Elements of $\Com^{\circ k}(n)$ (the $k$-fold composition) correspond to ways of iteratively partitioning $n$ elements through $k$ levels. The simplicial structure is:
\begin{itemize}
\item Face maps compose adjacent levels of partitioning (coarsening)
\item Degeneracy maps repeat a level (refinement followed by immediate coarsening)
\end{itemize}
 
After normalization (removing degeneracies), we obtain chains on $\barPi_n$. The dimension shift and sign representation arise from the suspension in the bar construction and the need for $S_n$-equivariance.
 
The key observation is that $\barPi_n$ has the homology of a wedge of $(n-1)!$ spheres of dimension $n-2$, with the $S_n$-action on the top homology given by the Lie representation tensored with the sign. This follows from the classical results of Björner-Wachs \cite{BW93} and Stanley \cite{Sta97}, who computed:
\[
\tilde{H}_{n-2}(\barPi_n) \cong \Lie(n) \otimes \sgn_n \text{ as } S_n\text{-representations}
\]
and $\tilde{H}_k(\barPi_n) = 0$ for $k \neq n-2$.
\end{proof}
\begin{remark}[Simplicial Model - Precise Construction]
The simplicial bar for $\Com$ literally consists of chains of refinements $\pi_0 \leq \pi_1 \leq \cdots \leq \pi_k$ in $\Pi_n$. This is the nerve of the poset $\Pi_n$, and the identification with the cooperad structure follows from taking normalized chains.
\end{remark}
 
\section{Com-Lie Koszul Duality from First Principles}
 
\subsection{Quadratic Operads and Koszul Duality}
 
We now specialize to quadratic operads, which admit a particularly refined duality theory:
 
\begin{definition}[Quadratic Operad]
A quadratic operad has the form $P = \Free(E)/(R)$ where:
\begin{itemize}
\item $E$ is a collection of generating operations concentrated in arity 2
\item $R \subset \Free(E)(3)$ consists of quadratic relations (involving exactly two compositions)
\item $\Free$ denotes the free operad functor
\item $(R)$ denotes the operadic ideal generated by $R$
\end{itemize}
\end{definition}
 
\begin{definition}[Koszul Dual Cooperad]
The Koszul dual cooperad $P^!$ is the maximal sub-cooperad of the cofree cooperad $T^c(s^{-1}E^\vee)$ cogenerated by the orthogonal relations $R^\perp \subset (s^{-1}E^\vee)^{\otimes 2}$, where the orthogonality is with respect to the natural pairing induced by evaluation.
\end{definition}
 
\begin{definition}[Koszul Operad]
An operad $P$ is \emph{Koszul} if the canonical map $\Omega(P^!) \to P$ is a quasi-isomorphism. Equivalently, the Koszul complex $K_\bullet(P) = P^! \circ P$ with differential induced by the cooperad and operad structures is acyclic in positive degrees.
\end{definition}
 
\subsection{Derivation of Com-Lie Duality}
 
We now prove the fundamental duality between the commutative and Lie operads:
 
\begin{theorem}[Com-Lie Koszul Duality]\label{thm:com-lie}
We have canonical isomorphisms of cooperads:
\[
\Com^! \cong \text{co}\Lie \quad \text{and} \quad \Lie^! \cong \text{co}\Com
\]
Moreover, both $\Com$ and $\Lie$ are Koszul operads with quasi-isomorphisms:
\[
\Omega(\text{co}\Lie) \xrightarrow{\sim} \Com, \quad \Omega(\text{co}\Com) \xrightarrow{\sim} \Lie
\]
\end{theorem}
 
\begin{proof}[Proof via Partition Lattices]
By Theorem \ref{thm:partition}, $\barB(\Com)(n) \simeq s^{n-2}\tilde{C}_{n-2}(\barPi_n) \otimes \sgn_n$.
 
Classical results of Björner-Wachs \cite{BW93} and Stanley \cite{Sta97} establish that the reduced homology of $\barPi_n$ is:
\begin{itemize}
\item The complex $\tilde{C}_*(\barPi_n)$ has homology concentrated in degree $n-2$
\item The $S_n$-representation on $\tilde{H}_{n-2}(\barPi_n)$ decomposes as $\Lie(n) \otimes \sgn_n$ where $\Lie(n)$ is the Lie representation
\item $\tilde{H}_k(\barPi_n) = 0$ for $k \neq n-2$
\end{itemize}
 
The key observation is that $\barPi_n$ has the homology of a wedge of $(n-1)!$ spheres of dimension $n-2$, with the $S_n$-action on the top homology given by the Lie representation tensored with the sign.

To see why this yields Com-Lie duality, observe that the bar construction gives:
$$\barB(\Com)(n) \simeq s^{n-2}\tilde{C}_{n-2}(\barPi_n) \otimes \sgn_n$$
Taking homology and using that $\barPi_n$ is $(n-3)$-connected:
$$H_*(\barB(\Com)(n)) \simeq s^{n-2}\Lie(n) \otimes \sgn_n \otimes \sgn_n = s^{n-2}\Lie(n)$$
Since this is concentrated in a single degree, the bar complex is formal and we obtain:
$$\barB(\Com) \simeq \text{co}\Lie[1]$$
as required.
 
Since the bar complex has homology concentrated in a single degree, it follows that:
\[
H_*(\barB(\Com)) \cong \text{co}\Lie[1]
\]
where the shift accounts for the suspension. Applying $\Omega$ yields $\Omega(\text{co}\Lie) \simeq \Com$.
 
The dual statement $\Lie^! \cong \text{co}\Com$ follows by Schur-Weyl duality, using the characterization of $\Lie$ as the primitive part of the tensor coalgebra.
\end{proof}
 
\begin{proof}[Alternative Proof via Generating Series]
The Poincaré series of the operads satisfy:
\begin{align}
P_{\Com}(x) &= e^x - 1 \\
P_{\Lie}(x) &= -\log(1 - x)
\end{align}
These are compositional inverses: $P_{\Lie}(-P_{\Com}(-x)) = x$. This functional equation characterizes Koszul dual pairs, providing an independent verification of the duality.
\end{proof}
 
\subsection{The Quadratic Dual and Orthogonality}
 
For explicit computations, we need the quadratic presentations:
 
\begin{proposition}[Quadratic Presentations]
The operads $\Com$ and $\Lie$ have quadratic presentations:
\begin{align}
\Com &= \Free(\mu)/(R_{\Com}) \text{ where } R_{\Com} = \langle \mu_{12,3} - \mu_{1,23}, \mu_{12} - \mu_{21} \rangle \\
\Lie &= \Free(\ell)/(R_{\Lie}) \text{ where } R_{\Lie} = \langle \ell_{12,3} + \ell_{23,1} + \ell_{31,2}, \ell_{12} + \ell_{21} \rangle
\end{align}
where subscripts denote inputs, and composition is denoted by adjacency. Here $\mu_{12,3}$ means $\mu \circ_1 \mu$ and $\mu_{1,23}$ means $\mu \circ_2 \mu$.
\end{proposition}
 
\begin{proposition}[Orthogonality]\label{prop:orthogonal}
Under the natural pairing between $\Free(\mu)(3)$ and $\Free(\ell^*)(3)$ induced by $\langle \mu, \ell^* \rangle = 1$, we have:
\[
R_{\Com} \perp R_{\Lie}
\]
This orthogonality is the concrete manifestation of Koszul duality.
\end{proposition}
 
\begin{proof}
We compute the pairing explicitly. The spaces have bases:
\begin{align}
\Free(\mu)(3) &= \text{span}\{\mu_{12,3}, \mu_{1,23}, \mu_{13,2}, \mu_{2,13}, \mu_{23,1}, \mu_{3,12}\} \\
\Free(\ell^*)(3) &= \text{span}\{\ell^*_{12,3}, \ell^*_{1,23}, \text{etc.}\}
\end{align}
 
The pairing $\langle \mu_{ij,k}, \ell^*_{pq,r} \rangle = 1$ if the tree structures match and $0$ otherwise. Computing:
\begin{align}
\langle \mu_{12,3} - \mu_{1,23}, \ell^*_{12,3} + \ell^*_{23,1} + \ell^*_{31,2} \rangle &= 1 + 0 + 0 - 0 - 0 - 1 = 0 \\
\langle \mu_{12,3} - \mu_{1,23}, \ell^*_{13,2} + \ell^*_{32,1} + \ell^*_{21,3} \rangle &= 0 - 1 + 0 + 0 + 1 + 0 = 0
\end{align}
Similar computations for all pairs verify the orthogonality.
\end{proof}
 
\section{Configuration Spaces and Logarithmic Forms}
 
\subsection{The Relative Perspective}

Following Grothendieck's philosophy of relative algebraic geometry, we work systematically in families:

\begin{definition}[Relative Bar Complex]
For a family of chiral algebras $\mathcal{A} \to S$ parametrized by a base $S$, the relative bar complex
\[
\chirbar_{S/\text{rel}}(\mathcal{A}) \to S
\]
lives over the relative configuration space $\Conf{\bullet}(X \times S/S)$.
\end{definition}

\begin{theorem}[Base Change]
The geometric bar construction commutes with base change:
\[
f^*\chirbar_S(\mathcal{A}) \cong \chirbar_{S'}(f^*\mathcal{A})
\]
for any morphism $f: S' \to S$.
\end{theorem}

This relative viewpoint reveals:
\begin{itemize}
\item Deformation theory: Families over $\text{Spec}(\mathbb{C}[\epsilon]/\epsilon^2)$
\item Moduli spaces: Universal families over $\mathcal{M}_{\text{ChirAlg}}$
\item Quantum groups: Families over $\text{Spec}(\mathbb{C}[[h]])$ with $h \to 0$ classical limit
\end{itemize}

\subsection{Configuration Spaces of Curves}
 
We now introduce the geometric spaces that will support our bar complexes. Throughout this section, $X$ denotes a smooth algebraic curve over $\C$ of dimension 1.
 
\begin{definition}[Configuration Space]
For a smooth algebraic curve $X$ over $\C$, the configuration space of $n$ distinct ordered points is:
\[
C_n(X) = \{(x_1, \ldots, x_n) \in X^n \mid x_i \neq x_j \text{ for all } i \neq j\}
\]
This is a smooth quasi-projective variety of dimension $n \cdot \dim X = n$ when $\dim X = 1$.

\begin{notation}
Throughout this paper:
\begin{itemize}
\item $C_n(X)$ denotes the open configuration space
\item $\overline{C_n(X)}$ denotes its Fulton-MacPherson compactification  
\item $\partial\overline{C_n(X)} = \overline{C_n(X)} \setminus C_n(X)$ denotes the boundary divisor
\end{itemize}
\end{notation}

\end{definition}
 
\begin{proposition}[Fundamental Group]
The fundamental group $\pi_1(C_n(X))$ is the pure braid group $P_n(X)$ on $n$ strands over $X$. For $X = \C$, this is the kernel of $B_n \to S_n$ where $B_n$ is the Artin braid group with generators $\sigma_i$ ($i = 1, \ldots, n-1$) and relations:
\begin{align}
\sigma_i\sigma_j &= \sigma_j\sigma_i \quad \text{if } |i-j| > 1 \\
\sigma_i\sigma_{i+1}\sigma_i &= \sigma_{i+1}\sigma_i\sigma_{i+1} \quad \text{(braid relations)}
\end{align}
\end{proposition}
 
The configuration space $C_n(X)$ is highly non-compact, with "points at infinity" corresponding to various collision patterns. The Fulton-MacPherson compactification provides a canonical way to add these points:
 
\subsection{The Fulton-MacPherson Compactification}
 
\begin{theorem}[Fulton-MacPherson Compactification \cite{FM94}]\label{thm:FM}
There exists a smooth compactification $\overline{C_n(X)}$ with a natural stratification by combinatorial type. More precisely, we have
a functorial compactification
\[j: C_n(X) \hookrightarrow \overline{C_n(X)}\]
where $\overline{C_n(X)}$ is obtained by iterated blow-ups along diagonals.

The compactification has the following properties:
\begin{enumerate}
\item The complement $D = \barC_n(X) \setminus C_n(X)$ is a normal crossing divisor
\item Boundary strata are indexed by nested partitions of $\{1, \ldots, n\}$ (equivalently, by rooted trees with $n$ leaves)
\item For each stratum $D_\pi$ corresponding to partition $\pi = \{B_1, \ldots, B_k\}$:
\[
D_\pi \cong \barC_k(X) \times \prod_{i=1}^k \barC_{|B_i|}(\C)
\]
where the first factor records the positions of "bubbles" and the product records configurations within each bubble
\item The compactification is functorial for smooth morphisms and open embeddings of curves
\end{enumerate}
\end{theorem}
 
\begin{proof}[Construction Sketch]
The compactification is obtained by a sequence of blow-ups:
\begin{enumerate}
\item Start with $X^n$
\item Blow up the smallest diagonal $\Delta_n = \{x_1 = \cdots = x_n\}$
\item Blow up the proper transforms of all partial diagonals $\Delta_I = \{x_i = x_j : i,j \in I\}$ in order of decreasing codimension
\item The exceptional divisors encode:
\begin{itemize}
\item Which points collide (given by the partition)
\item Relative rates of approach (radial coordinates in the blow-up)
\item Relative angles of approach (angular coordinates)
\end{itemize}
\end{enumerate}
 
The key insight is that the blow-up process naturally records the "speed" and "direction" of collisions, not just which points collide. The normal crossing property follows from the careful ordering of blow-ups, ensuring transversality at each step.
\end{proof}
 
\begin{example}[Three Points on $\mathbb{P}^1$]
For $\barC_3(\mathbb{P}^1)$, using projective invariance to fix three points, we get $\barC_3(\mathbb{P}^1) = \{\text{point}\}$. For $\barC_4(\mathbb{P}^1) \cong \mathbb{P}^1$, the boundary consists of three points corresponding to the three ways pairs can collide: $(12)(34)$, $(13)(24)$, $(14)(23)$.
\end{example}
 
\subsection{Logarithmic Differential Forms}

\begin{remark}[Why Logarithmic Forms?]
The appearance of logarithmic forms is not accidental but inevitable: they are the unique meromorphic 1-forms with prescribed residues at collision divisors. When operators collide in conformal field theory, the singularity structure is captured precisely by forms like $d\log(z_i - z_j)$. To make these forms single-valued requires choice. These choices encode precisely the monodromy data that will later appear in our $A_\infty$ relations. The branch cuts we choose are not arbitrary conventions but encode genuine topological information about the configuration space.
\end{remark}


\begin{definition}[Branch Cut Convention - Rigorous]
For each pair $(i,j)$ with $i < j$, we fix a branch of $\log(z_i - z_j)$ as follows:
\begin{enumerate}
\item Choose a basepoint $* \in C_n(X)$
\item For intuition: think of this as choosing a reference configuration where all points are well-separated

\item For each loop $\gamma$ based at $*$, define the monodromy $M_\gamma: \mathbb{C} \to \mathbb{C}$
\item The monodromy measures how our chosen branch of the logarithm changes as points wind around each other

\item Fix the branch by requiring $M_\gamma = \text{id}$ for contractible loops
\item This is equivalent to choosing a trivialization of the local system of logarithms over the universal cover
\item For concreteness on $X = \mathbb{C}$, we use the principal branch: $-\pi < \text{Im}(\log(z_i - z_j)) \leq \pi$

\item This determines $\log(z_i - z_j)$ up to a constant, which we fix by continuity from the basepoint
\item The constant is normalized so that $\log(1) = 0$
\end{enumerate}
The resulting logarithmic forms are single-valued on the universal cover $\widetilde{C_n(X)}$.
\end{definition}

\begin{remark}[Monodromy Consistency] The choice of branch cuts must be compatible with the factorization structure of the chiral algebra. Specifically, for any three points $z_i, z_j, z_k$, the monodromy around the total diagonal satisfies:
$$M_{ijk} = M_{ij} \circ M_{jk} \circ M_{ki}$$
This ensures the Arnold relations lift consistently to the universal cover.
\end{remark}

\begin{definition}[Logarithmic Forms with Poles]
The sheaf of logarithmic $p$-forms on $\overline{C}_n(X)$ is the subsheaf of meromorphic forms:
$$\Omega^p_{\overline{C}_n(X)}(\log D) = \{p\text{-forms } \omega : \omega \text{ and } d\omega \text{ have at most simple poles along } D\}$$

In local coordinates $(u_1,\ldots,u_n,\epsilon_{ij},\theta_{ij})_{i<j}$ near a boundary stratum:
$$\Omega^p_{\overline{C}_n(X)}(\log D) = \bigoplus_{I \subset \{(i,j): i<j\}} \Omega^{p-|I|}_{smooth} \wedge \bigwedge_{(i,j) \in I} d\log\epsilon_{ij}$$
\end{definition}

\begin{proposition}[Logarithmic Form Properties]
The forms $\eta_{ij} = d\log(z_i - z_j)$ satisfy:
\begin{enumerate}
\item $\eta_{ji} = -\eta_{ij}$ (antisymmetry)
\item Near $D_{ij}$: $\eta_{ij} = d\log\epsilon_{ij} + id\theta_{ij} + O(\epsilon_{ij})$
\item $\text{Res}_{D_{ij}}[\eta_{ij}] = 1$ (normalization)
\item $d\eta_{ij} = 0$ away from higher codimension strata
\item The residue map $\text{Res}_{D_{ij}}: \Omega^p(\log D) \to \Omega^{p-1}(D_{ij})$ is well-defined
\end{enumerate}
\end{proposition}

Near a boundary divisor $D_{ij}$ where points $x_i \to x_j$ collide, we use blow-up coordinates:
 
\begin{definition}[Blow-up Coordinates]\label{def:blowup}
Near $D_{ij} \subset \barC_n(X)$, introduce coordinates:
\begin{align}
u_{ij} &= \frac{x_i + x_j}{2} \quad \text{(center of collision)} \\
\epsilon_{ij} &= |x_i - x_j| \quad \text{(separation, serves as normal coordinate to } D_{ij}) \\
\theta_{ij} &= \arg(x_i - x_j) \quad \text{(angle of approach)}
\end{align}
In these coordinates:
\begin{align}
x_i &= u_{ij} + \frac{\epsilon_{ij}}{2}e^{i\theta_{ij}} \\
x_j &= u_{ij} - \frac{\epsilon_{ij}}{2}e^{i\theta_{ij}}
\end{align}
\end{definition}
 
The basic logarithmic 1-forms that will appear throughout our constructions are:
 
\begin{definition}[Basic Logarithmic Forms]
For distinct indices $i, j \in \{1, \ldots, n\}$, define:
\[
\eta_{ij} = d\log(x_i - x_j) = \frac{dx_i - dx_j}{x_i - x_j}
\]
These forms have simple poles along $D_{ij}$ and are regular elsewhere.
\end{definition}
 
\begin{proposition}[Properties of $\eta_{ij}$]\label{prop:eta}
The forms $\eta_{ij}$ satisfy:
\begin{enumerate}
\item Antisymmetry: $\eta_{ji} = -\eta_{ij}$
\item Blow-up expansion: Near $D_{ij}$,
\[
\eta_{ij} = d\log \epsilon_{ij} + id\theta_{ij} + \text{(regular terms)}
\]
\item Residue: $\Res_{D_{ij}} \eta_{ij} = 1$ (normalized by our convention)
\item Closure: $d\eta_{ij} = 0$ away from higher codimension strata
\end{enumerate}
\end{proposition}
 
\begin{proof}
(1) is immediate from the definition. For (2), compute in blow-up coordinates:
\[
x_i - x_j = \epsilon_{ij} e^{i\theta_{ij}}
\]
Therefore $d\log(x_i - x_j) = d\log(\epsilon_{ij} e^{i\theta_{ij}}) = d\log \epsilon_{ij} + id\theta_{ij}$.
 
For (3), the residue extracts the coefficient of $d\log \epsilon_{ij}$, which is 1 by our computation.
 
For (4), since $\eta_{ij}$ is locally $d$ of a function away from other collision divisors, we have $d\eta_{ij} = d^2\log(x_i - x_j) = 0$.
\end{proof}
 
\subsection{The Orlik-Solomon Algebra}
 
The logarithmic forms $\eta_{ij}$ generate a differential graded algebra with remarkable properties:

\subsubsection{Three-term relation}
\begin{theorem}[Arnold Relations - Rigorous]
For any triple of distinct indices $i, j, k \in \{1,\ldots,n\}$:
$$\eta_{ij} \wedge \eta_{jk} + \eta_{jk} \wedge \eta_{ki} + \eta_{ki} \wedge \eta_{ij} = 0$$
\end{theorem}

\begin{proof}[Complete Proof]
We work on the universal cover to avoid branch issues. Define:
$$\omega = \eta_{ij} + \eta_{jk} + \eta_{ki} = d\log((z_i - z_j)(z_j - z_k)(z_k - z_i))$$

Since $\omega = df$ for a single-valued function $f$ on the universal cover, we have $d\omega = 0$.

Computing explicitly:
\begin{align}
d\omega &= d\eta_{ij} + d\eta_{jk} + d\eta_{ki}\\
&= 0 \text{ away from higher codimension}
\end{align}

At the codimension-2 stratum $D_{ijk}$ where all three points collide, we use residue calculus:
$$\text{Res}_{D_{ijk}}[\eta_{ij} \wedge \eta_{jk}] = \lim_{(z_i,z_j,z_k) \to (z,z,z)} \left[\frac{dz_i - dz_j}{z_i - z_j} \wedge \frac{dz_j - dz_k}{z_j - z_k}\right]$$

In blow-up coordinates with $z_i = z + \epsilon_1 e^{i\theta_1}$, $z_j = z$, $z_k = z + \epsilon_2 e^{i\theta_2}$:
$$\eta_{ij} \wedge \eta_{jk} = d\log\epsilon_1 \wedge d\log\epsilon_2 + \text{(angular terms)}$$

The sum of all three terms gives zero by symmetry under $S_3$ action.
\end{proof}
 
\begin{theorem}[Cohomology via Orlik-Solomon]
For $X = \C$, the cohomology of $\barC_n(\C)$ is:
\[
H^*(\barC_n(\C)) \cong \text{OS}(A_{n-1})
\]
where $\text{OS}(A_{n-1})$ is the Orlik-Solomon algebra of the braid arrangement $A_{n-1}$. The Poincaré polynomial is:
\[
\sum_{k=0}^{n-1} \dim H^k(\barC_n(\C)) \cdot t^k = \prod_{i=1}^{n-1}(1 + it)
\]
\end{theorem}
 
\subsection{No-Broken-Circuit Bases}
 
For explicit computations, we need concrete bases for the cohomology:
 
\begin{definition}[Broken Circuit]
Fix a total order on pairs $(i, j)$ with $i < j$ (we use lexicographic order). A \emph{broken circuit} is a set obtained by removing the minimal element from a circuit (minimal dependent set) in the graphical matroid on $K_n$.
\end{definition}
 
\begin{definition}[NBC Basis]
A \emph{no-broken-circuit (NBC)} set is a collection of pairs that contains no broken circuit. These correspond bijectively to:
\begin{itemize}
\item Acyclic directed graphs on $[n]$ (forests)
\item Independent sets in the graphical matroid
\item Monomials in $\eta_{ij}$ that don't vanish by Arnold relations
\end{itemize}
\end{definition}
 
\begin{theorem}[NBC Basis Theorem]\label{thm:NBC}
The NBC sets provide a basis for $H^*(\barC_n(X))$. More precisely, if $F$ is an NBC forest with edges $E(F) = \{(i_1, j_1), \ldots, (i_k, j_k)\}$, then:
\[
\omega_F = \eta_{i_1j_1} \wedge \cdots \wedge \eta_{i_kj_k}
\]
forms a basis element of $H^k(\barC_n(X))$.
\end{theorem}
 
\begin{example}[NBC Basis for $n = 4$]\label{ex:NBC4}
For $\barC_4(X)$, using the lexicographic order on pairs, the NBC basis consists of:
\begin{itemize}
\item Degree 0: $1$
\item Degree 1: $\eta_{12}, \eta_{13}, \eta_{14}, \eta_{23}, \eta_{24}, \eta_{34}$ (6 elements)
\item Degree 2: $\eta_{12} \wedge \eta_{34}, \eta_{13} \wedge \eta_{24}, \eta_{14} \wedge \eta_{23}$, plus 8 other terms (11 total)
\item Degree 3: $\eta_{12} \wedge \eta_{23} \wedge \eta_{34}$ and 5 other spanning trees (6 total)
\end{itemize}
Total: $1 + 6 + 11 + 6 = 24 = 4!$ basis elements, confirming $\dim H^*(\barC_4(\C)) = 4!$.
\end{example}
 
This completes our foundational setup. We have established:
\begin{itemize}
\item The operadic framework for describing algebraic structures with complete categorical precision
\item The Com-Lie Koszul duality as our prototypical example with full proofs
\item The geometric spaces (configuration spaces) where our constructions live
\item The differential forms (logarithmic forms) that encode the structure
\end{itemize}
 
These ingredients will now be combined in subsequent sections to construct the geometric bar complex for chiral algebras.
 
\section{Chiral Algebras and Factorization}
 
\subsection{The Ran Space and Chiral Operations}

\begin{definition}[D-module Category - Precise]
We work with the category $\text{D-mod}_{rh}(X)$ of regular holonomic D-modules on $X$. 
These are D-modules $\mathcal{M}$ satisfying:
\begin{enumerate}
\item Finite presentation: locally finitely generated over $\mathcal{D}_X$
\item Regular singularities: characteristic variety is Lagrangian
\item Holonomicity: $\text{dim}(\text{Char}(\mathcal{M})) = \text{dim}(X)$
\end{enumerate}
This category has:
\begin{itemize}
\item Six functors: $f^*, f_*, f^!, f_!, \otimes^L, \mathcal{RHom}$
\item Riemann-Hilbert correspondence with perverse sheaves
\item Well-defined maximal extension $j_*j^*$ for $j: U \hookrightarrow X$ open
\end{itemize}
\end{definition}

We now introduce the fundamental geometric object underlying chiral algebras---the Ran space---which 
encodes the idea of ``finite subsets with multiplicities'' of a curve. Following Beilinson-Drinfeld 
\cite{BD04}, we work with the following precise categorical framework.
 
\begin{definition}[Ran Space via Categorical Colimit]\label{def:ran-precise}
Let $X$ be a smooth algebraic curve over $\mathbb{C}$. The \emph{Ran space} of $X$ is the ind-scheme 
defined as the colimit:
\[
\text{Ran}(X) = \underset{I \in \text{FinSet}^{\text{surj,op}}}{\text{colim}} \, X^I
\]
where:
\begin{itemize}
\item $\text{FinSet}^{\text{surj}}$ is the category of finite sets with surjections as morphisms
\item For a surjection $\phi: I \twoheadrightarrow J$, the induced map $X^J \to X^I$ is the diagonal 
embedding on fibers $\phi^{-1}(j)$
\item The colimit is taken in the category of ind-schemes with the Zariski topology
\end{itemize}
Explicitly, a point in $\text{Ran}(X)$ is a finite collection of points in $X$ with multiplicities,
represented as $\sum_{i=1}^n m_i[x_i]$ where $x_i \in X$ are distinct and $m_i \in \mathbb{Z}_{>0}$.
\end{definition}
 
\begin{remark}[Set-Theoretic Description]
The underlying set of $\text{Ran}(X)$ can be identified with the free commutative monoid on the 
underlying set of $X$, but the scheme structure is more subtle and encodes the deformation theory
of point configurations.
\end{remark}
 
The Ran space carries a fundamental monoidal structure encoding disjoint union:
 
\begin{definition}[Factorization Structure]\label{def:factorization}
\textbf{Critical Warning:} The naive definition 
$$\mathcal{M} \otimes^{\text{ch}} \mathcal{N} = \Delta_! \left( \rho_1^* \mathcal{M} \otimes^! \rho_2^* \mathcal{N} \right)$$
\textbf{FAILS} because the union map $\Delta: \text{Ran}(X) \times \text{Ran}(X) \to \text{Ran}(X)$ is \textbf{not proper}, 
so $\Delta_!$ is undefined. The correct framework uses factorization algebras.
\end{definition}

\begin{definition}[Factorization Algebra - Correct Framework]\label{def:fact-algebra-correct}
A \emph{factorization algebra} $\mathcal{F}$ on $X$ consists of:
\begin{enumerate}
\item A quasi-coherent $\mathcal{D}$-module $\mathcal{F}_S$ for each finite set $S \subset X$
\item For disjoint $S_1, S_2$, a factorization isomorphism:
   $$\mu_{S_1,S_2}: \mathcal{F}_{S_1} \boxtimes \mathcal{F}_{S_2} \xrightarrow{\sim} \mathcal{F}_{S_1 \sqcup S_2}$$
\item These satisfy:
   \begin{itemize}
   \item \textbf{Associativity:} For disjoint $S_1, S_2, S_3$:
   \begin{center}
   \begin{tikzcd}
   \mathcal{F}_{S_1} \boxtimes \mathcal{F}_{S_2} \boxtimes \mathcal{F}_{S_3} 
   \arrow[r, "\mu_{S_1,S_2} \boxtimes \text{id}"] 
   \arrow[d, "\text{id} \boxtimes \mu_{S_2,S_3}"'] &
   \mathcal{F}_{S_1 \sqcup S_2} \boxtimes \mathcal{F}_{S_3} 
   \arrow[d, "\mu_{S_1 \sqcup S_2, S_3}"] \\
   \mathcal{F}_{S_1} \boxtimes \mathcal{F}_{S_2 \sqcup S_3} 
   \arrow[r, "\mu_{S_1, S_2 \sqcup S_3}"'] &
   \mathcal{F}_{S_1 \sqcup S_2 \sqcup S_3}
   \end{tikzcd}
   \end{center}
   \item \textbf{Commutativity:} $\mu_{S_2,S_1} = \sigma_{S_1,S_2} \circ \mu_{S_1,S_2}$ where $\sigma$ is the swap
   \item \textbf{Unit:} $\mathcal{F}_\emptyset = \mathbb{C}$ with canonical isomorphisms $\mathcal{F}_S \cong \mathbb{C} \boxtimes \mathcal{F}_S$
   \end{itemize}
\end{enumerate}
\end{definition}

\begin{remark}[Geometric Insight à la Kontsevich]
Factorization algebras encode the principle of \emph{locality} in quantum field theory: the observables 
on disjoint regions combine independently. The factorization isomorphisms are the mathematical incarnation 
of the physical statement that ``spacelike separated observables commute.'' This philosophy, emphasized by 
Kontsevich and developed by Costello-Gwilliam, views quantum field theory as assigning algebraic structures 
to spacetime in a locally determined way.
\end{remark}

\begin{theorem}[Chiral Algebras as Factorization Algebras]\label{thm:chiral-as-fact}
Every chiral algebra $\mathcal{A}$ on $X$ determines a factorization algebra $\mathcal{F}_\mathcal{A}$ where:
\begin{itemize}
\item $\mathcal{F}_\mathcal{A}(S) = \mathcal{A}^{\boxtimes S}$ for finite $S \subset X$
\item The factorization structure comes from the chiral multiplication
\item This defines a fully faithful functor $\text{ChirAlg}(X) \to \text{FactAlg}(X)$
\end{itemize}
\end{theorem}

\begin{proof}[Proof following Beilinson-Drinfeld]
The key observation is that chiral multiplication provides exactly the factorization isomorphisms needed.
The Jacobi identity for chiral algebras translates to associativity of factorization. The technical 
issue with properness is avoided because we work fiberwise over finite sets rather than globally on Ran space.
\end{proof}
% Add precise D-module structure


\begin{theorem}[Factorization Monoidal Structure - CORRECTED]\label{thm:fact-monoidal-corrected}
The category $\text{FactAlg}(X)$ of factorization algebras (NOT all D-modules on Ran space) forms a symmetric monoidal 
category with:
\begin{enumerate}
\item Tensor product: $(\mathcal{F} \otimes_{\text{fact}} \mathcal{G})(S) = \bigoplus_{S_1 \sqcup S_2 = S} \mathcal{F}(S_1) \otimes \mathcal{G}(S_2)$
\item Unit: The vacuum factorization algebra $\mathbb{1}$ with $\mathbb{1}(S) = \begin{cases} \mathbb{C} & S = \emptyset \\ 0 & \text{otherwise} \end{cases}$
\item Associativity isomorphism satisfying the pentagon axiom
\item Braiding isomorphism induced by the symmetric group action
\end{enumerate}

Moreover, there is a fully faithful embedding:
$$\text{ChirAlg}(X) \hookrightarrow \text{FactAlg}(X)$$
sending a chiral algebra $\mathcal{A}$ to its associated factorization algebra $\mathcal{F}_{\mathcal{A}}$.
\end{theorem}

\begin{proof}[Proof Sketch following Beilinson-Drinfeld and Ayala-Francis]
The key insight is that factorization algebras form a \emph{lax} symmetric monoidal category, which becomes 
strict when we pass to the homotopy category. The Day convolution is well-defined because we take colimits 
over finite decompositions, avoiding the properness issues with the naive approach.

The pentagon and hexagon axioms follow from the corresponding properties of finite set unions. The 
symmetric monoidal structure is compatible with the embedding from chiral algebras, making this the 
correct categorical framework for studying chiral algebras.
\end{proof}

\textbf{Underlying D-modules:} A collection $\{\mathcal{A}_n\}_{n \geq 0}$ where each $\mathcal{A}_n$ is a quasi-coherent $\mathcal{D}_{X^n}$-module, meaning:
\begin{itemize}
\item $\mathcal{A}_n$ is a sheaf of modules over the sheaf of differential operators $\mathcal{D}_{X^n}$
\item The action satisfies the Leibniz rule: $\partial(fs) = (\partial f)s + f(\partial s)$ for local functions $f$ and sections $s$
\item $\mathcal{A}_n$ is quasi-coherent as an $\mathcal{O}_{X^n}$-module
\end{itemize}

\subsection{Elliptic Configuration Spaces and Theta Functions}

\subsubsection{The Genus 1 Realm: Elliptic Curves as Quotients}

For genus 1, we work with elliptic curves $E_\tau = \mathbb{C}/(\mathbb{Z} + \tau\mathbb{Z})$ where $\tau \in \mathfrak{h}$ lies in the upper half-plane. The configuration space has a fundamentally different character from genus 0:

\begin{definition}[Elliptic Configuration Space]
For an elliptic curve $E_\tau$, the configuration space of $n$ points is:
\[
C_n(E_\tau) = \{(z_1, \ldots, z_n) \in E_\tau^n \mid z_i \neq z_j \text{ mod } \Lambda_\tau\}
\]
where $\Lambda_\tau = \mathbb{Z} + \tau\mathbb{Z}$ is the period lattice.
\end{definition}

\begin{theorem}[Elliptic Compactification]
The compactification $\overline{C_n(E_\tau)}$ is constructed via:
\begin{enumerate}
\item \textbf{Local blow-ups}: Near collision points, use elliptic blow-up coordinates
\item \textbf{Global structure}: The compactified space admits a stratification by \emph{stable elliptic graphs}
\item \textbf{Modular invariance}: Under $SL_2(\mathbb{Z})$ action on $\tau$, the construction is equivariant
\end{enumerate}
\end{theorem}

\begin{proof}[Construction]
Near a collision point $z_i \to z_j$ on $E_\tau$, introduce elliptic blow-up coordinates:
\begin{align}
\epsilon_{ij} &= |z_i - z_j|_{E_\tau} \quad \text{(elliptic distance)} \\
\theta_{ij} &= \arg(z_i - z_j) \quad \text{(angular parameter)} \\
u_{ij} &= \frac{z_i + z_j}{2} \quad \text{(center on } E_\tau)
\end{align}

The key difference from genus 0: the elliptic distance involves the Weierstrass $\sigma$-function:
\[
|z_i - z_j|_{E_\tau} = |\sigma(z_i - z_j; \tau)|e^{-\eta(\tau)\text{Im}(z_i - z_j)^2/\text{Im}(\tau)}
\]
where $\eta(\tau)$ is the Dedekind eta function.
\end{proof}

\subsubsection{Theta Functions as Building Blocks}

The logarithmic forms on elliptic curves are replaced by forms built from theta functions:

\begin{definition}[Elliptic Logarithmic Forms]
On $\overline{C_n(E_\tau)}$, define the elliptic analogs of $\eta_{ij}$:
\[
\eta_{ij}^{(1)} = d\log\theta_1\left(\frac{z_i - z_j}{2\pi i}; \tau\right) + \text{regularization}
\]
where $\theta_1(z; \tau) = -i\sum_{n \in \mathbb{Z}}(-1)^n q^{(n-1/2)^2}e^{i(2n-1)z}$ with $q = e^{i\pi\tau}$.
\end{definition}

\begin{proposition}[Elliptic Arnold Relations]
The elliptic logarithmic forms satisfy modified Arnold relations:
\[
\eta_{ij}^{(1)} \wedge \eta_{jk}^{(1)} + \eta_{jk}^{(1)} \wedge \eta_{ki}^{(1)} + \eta_{ki}^{(1)} \wedge \eta_{ij}^{(1)} = 2\pi i \omega_\tau
\]
where $\omega_\tau = \frac{dz \wedge d\bar{z}}{2i\text{Im}(\tau)}$ is the volume form on $E_\tau$.
\end{proposition}

The non-vanishing right-hand side encodes the central extension that appears at genus 1!

\subsection{Higher Genus Configuration Spaces}

\subsubsection{Hyperbolic Surfaces and Teichmüller Theory}

For genus $g \geq 2$, the underlying curve $\Sigma_g$ admits a hyperbolic metric. The configuration spaces inherit rich geometric structure:

\begin{definition}[Higher Genus Configuration]
For a compact Riemann surface $\Sigma_g$ of genus $g \geq 2$:
\[
C_n(\Sigma_g) = \{(p_1, \ldots, p_n) \in \Sigma_g^n \mid p_i \neq p_j\}/\text{Aut}(\Sigma_g)
\]
The compactification $\overline{C_n(\Sigma_g)}$ involves:
\begin{itemize}
\item Stable curves with marked points
\item Deligne-Mumford compactification techniques
\item Intersection with the moduli space $\overline{\mathcal{M}}_{g,n}$
\end{itemize}
\end{definition}

\begin{theorem}[Period Integrals and Bar Differential]
On $\overline{C_n(\Sigma_g)}$, the bar differential decomposes:
\[
d_{\text{bar}}^{(g)} = d_{\text{local}} + d_{\text{global}} + d_{\text{quantum}}
\]
where:
\begin{enumerate}
\item $d_{\text{local}}$: Standard residues at collision divisors (genus 0 contribution)
\item $d_{\text{global}}$: Period integrals over homology cycles of $\Sigma_g$
\item $d_{\text{quantum}}$: Corrections from the moduli space $\mathcal{M}_g$
\end{enumerate}
\end{theorem}

\begin{proof}[Sketch]
The decomposition follows from the Leray spectral sequence for the fibration:
\[
\overline{C_n(\Sigma_g)} \to \overline{\mathcal{M}}_{g,n} \to \overline{\mathcal{M}}_g
\]

Each term contributes differently:
\begin{itemize}
\item Local: Fiberwise residues give the standard chiral multiplication
\item Global: Integration over the $2g$ cycles of $H_1(\Sigma_g, \mathbb{Z})$
\item Quantum: Contributions from varying complex structure
\end{itemize}
\end{proof}

