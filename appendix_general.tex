\appendix
\chapter{Geometric Dictionary}

\textbf{Reading Guide:} This dictionary should be read as a Rosetta Stone between three languages:
\begin{itemize}
\item \textbf{Physical:} The language of conformal field theory and operator products
\item \textbf{Algebraic:} The language of operads and homological algebra  
\item \textbf{Geometric:} The language of configuration spaces and residues
\end{itemize}
Each entry represents a precise mathematical correspondence, not merely an analogy.


This dictionary translates between algebraic structures in chiral algebras and geometric features of configuration spaces:

\begin{center}
\begin{tabular}{|l|l|}
\hline
\textbf{Algebraic Structure} & \textbf{Geometric Realization} \\
\hline
Chiral multiplication & Residues at collision divisors \\
Central extensions & Curved $A_\infty$ structures \\
Conformal weights & Pole orders in residue extraction \\
Normal ordering & NBC basis choice \\
BRST cohomology & Spectral sequence pages \\
Operator product expansion & Logarithmic form singularities \\
Jacobi identity & Arnold-Orlik-Solomon relations \\
Module categories & D-module pushforward \\
Koszul duality & Orthogonality under residue pairing \\
Vertex operators & Sections over configuration spaces \\
Screening charges & Exact forms modulo boundaries \\
Conformal blocks & Flat sections of connections \\
\hline
\end{tabular}
\end{center}

\begin{remark}[Reading the Dictionary]
This correspondence is not merely a formal analogy but reflects deep mathematical structure. Each entry represents a precise functor or natural transformation between categories. For instance, the correspondence "Chiral multiplication $\leftrightarrow$ Residues at collision divisors" is the content of Theorem \ref{thm:residue-formula}, establishing that the multiplication map factors through the residue homomorphism. Similarly, "Central extensions $\leftrightarrow$ Curved $A_\infty$ structures" reflects Theorem \ref{thm:heisenberg-bar}, showing how the failure of strict associativity due to central charges is precisely captured by the curvature term $m_0$.
\end{remark}


 
\chapter{Sign Conventions}
 
We collect our sign conventions for reference:
\begin{itemize}
\item Logarithmic forms: $\eta_{ij} = d\log(z_i - z_j) = \frac{dz_i - dz_j}{z_i - z_j}$
\item Transposition: $\eta_{ji} = -\eta_{ij}$
\item Residues: $\text{Res}_{z_i=z_j}[\eta_{ij}] = 1$
\item Fermionic permutation: $\psi_i\psi_j = -\psi_j\psi_i$
\item Koszul sign rule: Moving degree $p$ past degree $q$ introduces $(-1)^{pq}$
\item Differential grading: $\deg(d) = 1$, $\deg(\eta_{ij}) = 1$
\item Suspension: $s$ has degree $1$, desuspension $s^{-1}$ has degree $-1$
\end{itemize}
 
\chapter{Complete OPE Tables}
 
\begin{center}
\begin{tabular}{|c|c|c|}
\hline
Field 1 & Field 2 & OPE \\
\hline
$\psi(z)$ & $\psi(w)$ & $(z-w)^{-1}$ \\
$J(z)$ & $J(w)$ & $k(z-w)^{-2}$ \\
$\beta(z)$ & $\gamma(w)$ & $(z-w)^{-1}$ \\
$\gamma(z)$ & $\beta(w)$ & $-(z-w)^{-1}$ \\
$b(z)$ & $c(w)$ & $(z-w)^{-1}$ \\
$T(z)$ & $T(w)$ & $\frac{c/2}{(z-w)^4} + \frac{2T(w)}{(z-w)^2} + \frac{\partial T(w)}{z-w}$ \\
$W^{(s)}(z)$ & $W^{(t)}(w)$ & $\sum_u \frac{C^u_{st} W^{(u)}(w)}{(z-w)^{s+t-u}}$ \\
$e^\alpha(z)$ & $e^\beta(w)$ & $(z-w)^{(\alpha,\beta)} e^{\alpha+\beta}(w)$ \\
\hline
\end{tabular}
\end{center}
 
\chapter{Arnold Relations for Small $n$}
 
Complete list of Arnold relations for logarithmic forms:
 
\textbf{$n = 3$:}
\[
\eta_{12} \wedge \eta_{23} + \eta_{23} \wedge \eta_{31} + \eta_{31} \wedge \eta_{12} = 0
\]
 
\textbf{$n = 4$ (4-term relation):}
\[
\eta_{12} \wedge \eta_{34} - \eta_{13} \wedge \eta_{24} + \eta_{14} \wedge \eta_{23} = 0
\]
 
\textbf{$n = 5$ (10 independent relations):}
\begin{align}
&\eta_{12} \wedge \eta_{23} \wedge \eta_{45} + \text{cyclic} = 0 \\
&\eta_{12} \wedge \eta_{34} \wedge \eta_{35} - \eta_{13} \wedge \eta_{24} \wedge \eta_{35} + \cdots = 0
\end{align}
 
\textbf{General $n$:} The relations form the kernel of
\[
\bigwedge^k \mathbb{C}^{\binom{n}{2}} \to H^k(C_n(\mathbb{C}))
\]
with dimension $\binom{n}{2} - \prod_{i=1}^{n-1}(1 + i)$ for the kernel.

% ================================================================
% PATCH 013: CURVED A_INFTY FORMULAS
% ================================================================

\chapter{Curved $A_\infty$ Relations: Complete Formulas}
\label{app:curved-ainfty-formulas}

For reference, we collect the complete curved $A_\infty$ relations. An $A_\infty$ algebra 
$(\mathcal{A}, \{m_k\}_{k \geq 0}, \mu_0)$ satisfies:

\textbf{$n = 0$:} (Curvature is a cycle)
$$m_1(\mu_0) = 0$$

\textbf{$n = 1$:} (Failure of strict nilpotence)
$$m_1^2 = m_2(\mu_0 \otimes \text{id}) + m_2(\text{id} \otimes \mu_0)$$

\textbf{$n = 2$:} (Associativity with curvature corrections)
\begin{multline}
m_1 m_2 - m_2(m_1 \otimes \text{id}) - m_2(\text{id} \otimes m_1) + m_3(\mu_0 \otimes \text{id} 
\otimes \text{id}) \\
+ m_3(\text{id} \otimes \mu_0 \otimes \text{id}) + m_3(\text{id} \otimes \text{id} \otimes \mu_0) = 0
\end{multline}

\textbf{$n = 3$:} (Higher coherences)
$$\sum_{\substack{i+j+\ell = 4 \\ j \geq 1}} (-1)^{i+j\ell} m_{i+1+\ell}(\text{id}^{\otimes i} 
\otimes m_j \otimes \text{id}^{\otimes \ell}) = 0$$
including terms with $\mu_0$ inserted.

\textbf{General formula:}
$$\sum_{\substack{i+j+\ell=n+1 \\ i,\ell \geq 0, j \geq 1}} (-1)^{i+j\ell} m_{i+1+\ell}
(\text{id}^{\otimes i} \otimes m_j \otimes \text{id}^{\otimes \ell}) = 0$$

\textbf{Special case - Central curvature:}
If $\mu_0 \in Z(\mathcal{A})$, then:
$$m_1^2 = 0 \quad \text{and} \quad m_2(\mu_0 \otimes a) = m_2(a \otimes \mu_0) 
= \mu_0 \cdot m_1(a) = 0$$

This simplifies all relations! 
